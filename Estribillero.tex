%%%%%% rcsid = @(#)$Id: sample-sb.tex,v 1.23 2010-04-12 18:04:11 rathc Exp $
%%%%%%
%%
%%      ===============================
%%      Sample Songbook (sample-sb.tex)
%%      ===============================
%%
%%      Version 4.5, 30 April, 2010
%%
%%      Copyright 1992--2010 Christopher Rath <christopher@rath.ca>
%%
%%      This package is free software; you can redistribute it and/or
%%      modify it under the terms of version 2.1 of the GNU Lesser
%%	General Public License as published by the Free Software 
%%	Foundation.
%%
%%      This package is distributed in the hope that it will be
%%      useful, but WITHOUT ANY WARRANTY; without even the implied
%%      warranty of MERCHANTABILITY or FITNESS FOR A PARTICULAR
%%      PURPOSE.  See the GNU Lesser General Public License for more
%%      details.
%%
%%      This file contains a subset of the songbook we distribute
%%      at our church.  To the best of my knowledge, all of the lyrics
%%      contained herein are freely distributable.  This file has been
%%      provided as a sample of what can be produced by the chordbk,
%%      wordbk, and overhead LaTeX styles.
%%
%%      NEEDED:  The fancyhdr LaTeX style is required to properly
%%              format this file.  If you don't have that then comment
%%              out the commands in the preamble which deal with the
%%              fancyhdr style.
%%
%%%%%%
%%%%%%
%%
%%      1. Chord notation.  Within this songbook the following
%%         conventions have been adopted:
%%
%%              "Minor" is entered as "m";
%%                      e.g. Cm7 for C minor 7th.
%%              "Major" is entered as "M";
%%                      e.g. CM7 for C major 7th.
%%
%%%%%%
%%%%%%
%%      ============
%%      Bibliography
%%      ============
%%
%%    
%%
%%%%%%
%%%%%%

%%%%%%%%%%%%%%%%%%%%%%%%%%%%%%%%%%%%%%%%%%%%%%%%%%%%%%%%%%
%%%%%%%%%%%%%%%%%%%%%%%%%%%%%%%%%%%%%%%%%%%%%%%%%%%%%%%%%%
%%                                                      %%
%%           P R E A M B L E   B E G I N S              %%
%%                                                      %%
%%%%%%%%%%%%%%%%%%%%%%%%%%%%%%%%%%%%%%%%%%%%%%%%%%%%%%%%%%
%%%%%%%%%%%%%%%%%%%%%%%%%%%%%%%%%%%%%%%%%%%%%%%%%%%%%%%%%%

\documentclass[12pt, spanish]{book}
\usepackage[T1]{fontenc}
\usepackage[latin9]{inputenc}
\usepackage{babel}
\usepackage{mypiano}
\usepackage{gchords}
\usepackage{latexsym,fancyhdr}
\usepackage[unicode=true,pdfusetitle, bookmarks=true,bookmarksnumbered=false,bookmarksopen=false, 
 breaklinks=false,pdfborder={0 0 1},backref=false,colorlinks=false]
 {hyperref}
\usepackage[chordbk]{songbook}                  %% Words & Chords edition.
%%\usepackage[compactallsongs,chordbk]{songbook}    %% Words & Chords edition.
%%\usepackage[wordbk]{songbook}                 %% Words Only edition.
%%\usepackage[overhead]{songbook}               %% Overhead Transparency edition.



  \renewcommand{\SBChorusTag}{Coro:}
%%%
% Revision Date and Release Date definitions.
%
%       \RelDate - The last time this songbook was released.  Set this
%                  date each time a new release/update of the songbook
%                  is generated.
%       \RevDate - The last time a particular song was revised in any
%                  way.  This command will be renewed inside every
%                  song.
%%%
\newcommand{\RelDate}{26~Marzo,~2014}
\newcommand{\RevDate}{\today}

%%%
% C.C.L.I. license number definition; for copyright licensing info.
% One of these macros will be manually inserted into the {CpyRt}
% parameter of the {song} environment.
%
%       \CCLInumber - The actual copyright license number.  Don't
%               insert this command in the {CpyRt} parameter, use one
%               of the others.
%       \CCLIed - Indicates a song falls under our CCLI license.
%       \NotCCLIed - Indicates a song doesn't fall under our CCLI
%               license.  Public Domain songs fall into this category.
%       \PGranted - We have received specific permission from the
%               copyright holder to use this song.
%       \PPending - We are in the process of obtaining permission to
%               use this song.
%%%
\newcommand{\CCLInumber}{Your CCLI Number}
\newcommand{\CCLIed}{{\CpyRtInfoFont (CCLI \CCLInumber)}}
\newcommand{\NotCCLIed}{\relax}
\newcommand{\PGranted}{\relax}
\newcommand{\PPending}{{\CpyRtInfoFont (Permission Pending)}}

%%%
% Title page information.
%%%
\title{Cuaderno de Himnos Tradicionales y Contemporáneos}
\author{Javatlacati}
\date{\'Ultima Revisión:  \RevDate}

%%%
% Redefine fonts from SongBook style that I don't like.
%%%
\font\myTinySF=cmss8 at 8pt
\renewcommand{\CpyRtInfoFont}{\tiny\myTinySF}

%%%
% Define fonts to use in the headers and footers of the songbook.
%%%
\newcommand{\LHeadFont}{\normalsize}            % = cmr12  at 12pt
\newcommand{\CHeadFont}{\normalsize\rm}         % = cmr12  at 12pt
\newcommand{\RHeadFont}{\normalsize}            % = cmr12  at 12pt
\newcommand{\LFootFont}{\scriptsize}            % = cmr8   at  8pt
\newcommand{\CFootFont}{\tiny\myTinySF}         % = cmss8  at  8pt
\newcommand{\RFootFont}{\scriptsize}            % = cmr8   at  8pt

%%%
% Turn on and define fancy page heading/footing definition.
%%%
\pagestyle{fancy}

\ifChordBk
  % It's a words & chords songbook...
  \addtolength{\headwidth}{\marginparsep}
  \addtolength{\headwidth}{\marginparwidth}
  \renewcommand{\headrulewidth}{0.4pt}
  \renewcommand{\footrulewidth}{0.4pt}
  \fancyhead[LE,RO]{\LHeadFont\emph{\leftmark\/}\SBContinueMark}
  \fancyhead[CE,CO]{\CHeadFont\thepage}
  \fancyhead[RE,LO]{\RHeadFont\RelDate}
\else\ifOverhead
  % It's an overhead...
  \renewcommand{\footrulewidth}{0pt}
  \renewcommand{\headrulewidth}{0pt}
  \fancyhead[LE,RO]{}
  \fancyhead[CE,CO]{}
  \fancyhead[RE,LO]{}
\else\ifWordBk
  % It's a words only songbook...
  \addtolength{\headwidth}{\marginparsep}
  \addtolength{\headwidth}{\marginparwidth}
  \renewcommand{\headrulewidth}{0.4pt}
  \renewcommand{\footrulewidth}{0.4pt}
  \fancyhead[LE,RO]{\LHeadFont Estribillero}
  \fancyhead[CE,CO]{\CHeadFont\thepage}
  \fancyhead[RE,LO]{\RHeadFont\RelDate}
\fi\fi\fi

\fancyfoot[LE,RO]{\LFootFont Transcripciones}
\ifSongEject
  \fancyfoot[CE,CO]{\CFootFont \RevDate}
\else
  \fancyfoot[CE,CO]{\CFootFont}
\fi
\fancyfoot[RE,LO]{\RFootFont Todo el material son transcripciones personales.}

%%%
% Turn on/off index-file generation.  Uncomment the \makeindex line to
% turn index generation on;  comment it out to turn index generation
% off.
%%%
\makeTitleIndex         %% Title and First Line Index.
\makeTitleContents      %% Table of Contents.
\makeKeyIndex           %% Index of song by key.
\makeArtistIndex	%% Index of song by artist.

%%%%%%%%%%%%%%%%%%%%%%%%%%%%%%%%%%%%%%%%%%%%%%%%%%%%%%%%%%
%%%%%%%%%%%%%%%%%%%%%%%%%%%%%%%%%%%%%%%%%%%%%%%%%%%%%%%%%%
%%                                                      %%
%%           D O C U M E N T   B E G I N S              %%
%%                                                      %%
%%%%%%%%%%%%%%%%%%%%%%%%%%%%%%%%%%%%%%%%%%%%%%%%%%%%%%%%%%
%%%%%%%%%%%%%%%%%%%%%%%%%%%%%%%%%%%%%%%%%%%%%%%%%%%%%%%%%%
\begin{document}

%%%
% Uncomment "\maketitle" statement to make a title page.
%%%
\maketitle

\mainmatter
\ifWordBk
  \twocolumn
\fi

\include{EstribilleroToc.pdf}

%%%
% Songbook begins.
%%%
\begin{song}{Abba Padre}{C}
  {} %copyright \SBPubDom
  {Marco Barrientos}
  {Romanos 8:15} %pasaje
  {\href{http://open.spotify.com/track/0yj0zBaa7Ckn6ZMQPmCmfF}{Escuchar}} %\NotCCLIed

%  \renewcommand{\RevDate}{February~11,~1993}
  \SBRef{No puedo parar de alabarte}{2006}%fuente \#

  \begin{SBVerse}
\Ch{D}{Una} vez más

me a\Ch{Bm}{cer}co a Tí

con \Ch{Em}{li}bertad

en adora\Ch{A}{ció}n
  \end{SBVerse}
  \begin{SBVerse}
Tú e\Ch{D}{res} mi Dios

tu \Ch{Bm}{hi}jo soy

mi \Ch{Em}{co}munión contigo

es una \Ch{A}{dul}ce bendición
  \end{SBVerse}

  \begin{SBChorus}
// ¡\Ch{D}{A}bba Pa\Ch{G}{dre}! ¡\Ch{D}{A}bba Pa\Ch{G}{dre}!

Es\Ch{D}{tar}\Ch{A/C#}{} conti\Ch{Bm}{go} \Ch{D/A}{es} una \Ch{Bm}{dul}ce bendi\Ch{A}{ción}

¡\Ch{D}{A}bba Pa\Ch{G}{dre}! te \Ch{F#m}{a}mo Se\Ch{Bm}{ñor}

\Ch{G}{quie}ro estar en \Ch{D}{co}munión

\Ch{Em}{quie}ro es\Ch{A}{tar} con\Ch{D}{ti}go. //
  \end{SBChorus}
\ifChordBk
\begin{SBOpGroup}
    Acordes:
%\keyboardtwooctaves[Do][Fso][Ao]
% \upchord{
\keyboard[Do][Fso][Ao]\D\keyboard[Do][Go][Bo]\G
%}{Re} Mayor

% \upchord{

%}{Sol} Mayor

 \upchord{\keyboard[Do][Fso][Bo]\Bm}{Si} Menor
  \end{SBOpGroup}
\fi
\end{song}

\begin{song}{A Cristo solo a Cristo}{G}
  {} %copyright \SBPubDom
  {Marcos Witt}
  {} %pasaje
  {} %\NotCCLIed

%  \renewcommand{\RevDate}{February~11,~1993}
%  \SBRef{No puedo parar de alabarte}{2006}%fuente \#

  \begin{SBVerse}
A Cristo solo a Cristo, yo exaltaré

A Cristo solo a Cristo, yo adoraré
  \end{SBVerse}
  \begin{SBVerse}
Porque Él me ha dado vida eterna
Porque Él me ha dado el poder
Porque Él me ha dado la victoria
Él es mi Rey.
a Cristo he proclamado Rey
  \end{SBVerse}
\ifChordBk
\begin{SBOpGroup}
    Acordes:
%\keyboardtwooctaves[Do][Fso][Ao]
 \upchord{\keyboard[Do][Fso][Ao]\D}{Re} Mayor
  \end{SBOpGroup}
\fi
\end{song}

\begin{song}{A Dios el Padre Celestial}{G}
  {} %copyright \SBPubDom
  {Thomas Ken, Genevan Psalter, 1551, atr. a Louis Bourgeois}
  {} %pasaje
  {} %\NotCCLIed

%  \renewcommand{\RevDate}{February~11,~1993}
%  \SBRef{No puedo parar de alabarte}{2006}%fuente \#

  \begin{SBVerse}
A Dios el Padre Celestial,
Al Hijo nuestro Redentor.
Y al Eternal Consolador,
unidos todos alabad.
Amén.
  \end{SBVerse}
\ifChordBk
\begin{SBOpGroup}
    Acordes:
%\keyboardtwooctaves[Do][Fso][Ao]
 \upchord{\keyboard[Do][Fso][Ao]\D}{Re} Mayor
  \end{SBOpGroup}
\fi
\end{song}


\begin{song}{Admirable}{G}
  {} %copyright \SBPubDom
  {}
  {} %pasaje
  {} %\NotCCLIed

%  \renewcommand{\RevDate}{February~11,~1993}
%  \SBRef{No puedo parar de alabarte}{2006}%fuente \#

  \begin{SBVerse}
Con poder y autoridad nuestro Dios venció a la
muerte
Sobre el trono celestial siempre reinará.
Sentado en majestad suyo es el reino por los
Siglos y por la eternidad su luz de gloria brillará.
  \end{SBVerse}

\begin{SBChorus}
Admirable, consejero mi Dios consolador,
Eres digno de alabanza, Príncipe de paz.
  \end{SBChorus}
\ifChordBk
\begin{SBOpGroup}
    Acordes:
%\keyboardtwooctaves[Do][Fso][Ao]
 \upchord{\keyboard[Do][Fso][Ao]\D}{Re} Mayor
  \end{SBOpGroup}
\fi
\end{song}

\begin{song}{A Nuestro Padre Dios}{G}
  {} %copyright \SBPubDom
  {Anónimo en Thesaurus Musicus 1744}
  {} %pasaje
  {} %\NotCCLIed

%  \renewcommand{\RevDate}{February~11,~1993}
%  \SBRef{No puedo parar de alabarte}{2006}%fuente \#

  \begin{SBVerse}
A nuestro Padre Dios
Alcemos nuestra voz
¡Gloria a Él!
Tal fué su amor que dió
Al hijo que murió,
En quien confío yo;
¡Gloria a Él!
  \end{SBVerse}

  \begin{SBVerse}
A nuestro Salvador
Demos con fe loor
½Gloria a Él!
Su sangre derramó
Con ella me lavó,
Y el cielo me abrió
½Gloria a Él!
  \end{SBVerse}

  \begin{SBVerse}
Espíritu de Dios,
Elevo a Ti mi voz;
½Gloria a Ti!
Con celestial fulgor
Me muestras el amor
De Cristo mi Señor
½Gloria a Ti!
  \end{SBVerse}

  \begin{SBVerse}
Con gozo y amor,
Cantemos con fervor
Al Trino Dios.
En la eternidad
Mora la Trinidad;
½Por siempre alabad
Al Trino Dios!
  \end{SBVerse}

\ifChordBk
\begin{SBOpGroup}
    Acordes:
%\keyboardtwooctaves[Do][Fso][Ao]
 \upchord{\keyboard[Do][Fso][Ao]\D}{Re} Mayor
  \end{SBOpGroup}
\fi
\end{song}
%\WBPageBrk %para forzar salto de pagina

\begin{song}{Garment Of Praise}{C}
  {}
  {}
  {Isaiah~61:3}
  {\NotCCLIed}

  \renewcommand{\RevDate}{February~11,~1993}
  %\SBRef{}{}
  \FLineIdx{I have put on my garment of praise}

  \begin{SBOpGroup}
    I have \Ch{C}{put} on my garment of \Ch{G7}{praise,}
    
    I have \Ch{G7}{put} on my garment of \Ch{C}{prai}\Ch{C7}{se}
    
    And the \Ch{F}{spirit} of heaviness, is \Ch{C}{gone} from \Ch{Am}{me.}
    
    I have \Ch{C}{put} on my \Ch{G7}{gar}ment of \Ch{C}{praise!} \ChX{[}{}\ChX{F}{} \ChX{C}{}\ChX{]}{}
  \end{SBOpGroup}

  \begin{SBExtraKeys}{
    \STitle{Garment Of Praise}{D}

    \begin{SBOpGroup}
      I have \Ch{D}{put} on my garment of \Ch{A7}{praise,}
      
      I have \Ch{A7}{put} on my garment of \Ch{D}{prai}\Ch{D7}{se}
      
      And the \Ch{G}{spirit} of heaviness, is \Ch{D}{gone} from \Ch{Bm}{me.}
      
      I have \Ch{D}{put} on my \Ch{A7}{gar}ment of \Ch{D}{praise!} \Ch{[}{}\Ch{G}{} \Ch{D}{}\Ch{]}{}
    \end{SBOpGroup}
  }\end{SBExtraKeys}
\end{song}

\include{EstribilleroTdx.pdf}
\include{EstribilleroKdx.pdf}
\include{EstribilleroKdx.pdf}
%\begin{document}
\newcommand{\myTitleFont}{\Huge\myHugeSF}
\ifguitarra
\pdfbookmark[0]{Acordes~para~Guitarra}{acordesguitarra}
\lhead{\LHeadFont Acordes~para~Guitarra}
\chead{\CHeadFont ({\rm\thepage})}
\rhead{\RHeadFont\RelDate}
{\parindent 8pt
        {\myTitleFont --- Acodes para Guitarra ---}}\par
\vskip 20pt
\textbf{Acodes Mayores}

%\small{El s\'imbolo \# significa sostenido y {\flat}~significa~bemol}
\small
\upchord{\AGuitar}{La Mayor} \upchord{\BGuitar}{Si Mayor} \upchord{\CGuitar}{Do Mayor} \upchord{\DGuitar}{Re Mayor} \upchord{\EGuitar}{Mi Mayor} \upchord{\FGuitar}{Fa Mayor} \upchord{\GGuitar}{Sol Mayor}

\upchord{\AsGuitar}{A\#/$B\flat$ Mayor} \upchord{\CsGuitar}{C\#/$D\flat$ Mayor} \upchord{\DsGuitar}{D\#/$E\flat$ Mayor}  \upchord{\FsGuitar}{F\#/$G\flat$ Mayor} \upchord{\GsGuitar}{G\#/$A\flat$ Mayor} \upchord{\AsGuitar}{A\#/$B\flat$ Mayor}
\normalsize

\vskip 20pt
\textbf{Acodes Menores}

Estos acordes tienen las siguientes notaciones:
A-, Amin, Am, Aminor\break
\vskip 20pt

\small
\upchord{\AmGuitar}{La} Menor \upchord{\BmGuitar}{Si} Menor \upchord{\CmGuitar}{Do} Menor \upchord{\DmGuitar}{Re} Menor \upchord{\EmGuitar}{Mi} Menor \upchord{\FmGuitar}{Fa} Menor \upchord{\GmGuitar}{Sol} Menor

\upchord{\AsmGuitar}{\small{A\#/$B\flat$ Menor}} \upchord{\CsmGuitar}{\small{C\#/$D\flat$ Menor}} \upchord{\DsmGuitar}{\small{D\#/$E\flat$ Menor}}  \upchord{\FsmGuitar}{\small{F\#/$G\flat$ Menor}} \upchord{\GsmGuitar}{\small{G\#/$A\flat$ Menor}} \upchord{\AsmGuitar}{\small{A\#/$B\flat$ Menor}}
\normalsize

\vskip 20pt
\textbf{Acodes Mayores S\'eptima}

\upchord{\AsevenGuitar}{La} Mayor s\'eptima
\upchord{\BflatsevenGuitar}{Si} bemol Mayor s\'eptima
\upchord{\BsevenGuitar}{Si} Mayor s\'eptima
\upchord{\CsevenGuitar}{\small{Do Mayor s\'eptima}}
\vskip 20pt
\upchord{\CssevenGuitar}{\small{Do sostenidoMayor s\'eptima}}
\upchord{\DsevenGuitar}{\small{Re Mayor s\'eptima}}
\upchord{\EflatsevenGuitar}{\small{Mi bemol Mayor s\'eptima}}
\upchord{\EsevenGuitar}{\small{Mi Mayor s\'eptima}}
\upchord{\FsevenGuitar}{\small{Fa Mayor s\'eptima}}
\upchord{\FssevenGuitar}{\small{Fa sostenido Mayor s\'eptima}}
\vskip 20pt
\upchord{\GsevenGuitar}{\small{Sol Mayor s\'eptima}}
\upchord{\AflatsevenGuitar}{\qquad La bemol S\'eptima}

\textbf{Acodes Menores S\'eptima}

\small
\upchord{\AmsevenGuitar}{La} Menor s\'eptima
\upchord{\BmsevenGuitar}{Si} Menor s\'eptima
\upchord{\CmsevenGuitar}{Do} Menor s\'eptima
\upchord{\CsmsevenGuitar}{Do} Menor s\'eptima
\upchord{\DmsevenGuitar}{Re} Menor s\'eptima
\vskip 20pt
\upchord{\DsmsevenGuitar}{Re} Sostenido Menor s\'eptima
\upchord{\EmsevenGuitar}{Mi} Menor s\'eptima
\upchord{\EmseventrGuitar}{Mi} Menor s\'eptima
\upchord{\FmsevenGuitar}{Fa} Menor s\'eptima
\vskip 20pt
\upchord{\FsmsevenGuitar}{Fa sostenido} Menor s\'eptima
\upchord{\GmsevenGuitar}{Sol} Menor s\'eptima
\upchord{\GsmsevenGuitar}{Sol} Sostenido Menor s\'eptima
\upchord{\BflatmsevenGuitar}{Si bemol} Menor s\'eptima
\normalsize

\vskip 20pt
\textbf{Acodes Mayores Suspendido cuarta}
\vskip 25pt

\small
\upchord{\AsusGuitar}{La} Suspendida cuarta
\upchord{\BsusGuitar}{Si} Suspendida cuarta
\upchord{\CsusGuitar}{Do} Suspendida cuarta
\upchord{\DsusGuitar}{Re} Suspendida cuarta
\upchord{\EsusGuitar}{Mi} Suspendida cuarta
\upchord{\FsusGuitar}{Fa} Suspendida cuarta
\upchord{\GsusGuitar}{Sol} Suspendida cuarta

\upchord{\FssusGuitar}{Fa} sostenido Suspendida cuarta
\upchord{\GssusGuitar}{Sol sostenido} Suspendida cuarta
\normalsize

\vskip 20pt
\textbf{Acodes Mayor S\'eptima Aumentada}
\vskip 25pt

Estos acordes tienen las siguientes notaciones:
Amaj7, A+7, AM7, $A^{+7}$, $A^{M7}$, $A\Delta7$, $A^{\Delta7}$\break
\vskip 20pt

\small
\upchord{\AMajGuitar}{La} Maj
\upchord{\CMajGuitar}{Do} Maj
\upchord{\DMajGuitar}{Re} Maj
\upchord{\FMajGuitar}{Fa} Maj
\upchord{\GMajGuitar}{Sol} Maj
\upchord{\AflatMajGuitar}{Ab} Maj
\upchord{\AsMajGuitar}{A\#} Maj
\upchord{\EflatMajGuitar}{Eb} Maj
\normalsize

\vskip 20pt
\textbf{Acodes Suspendida 2}
\vskip 25pt

\small
\upchord{\AsustwoGuitar}{La} Suspendida 2
\upchord{\CsustwoGuitar}{Do} Suspendida 2
\normalsize

\vskip 20pt
\textbf{Acodes Suspendida 9}
\vskip 25pt

\small
\upchord{\DsusnineGuitar}{Re} Suspendida novena
\normalsize

\vskip 20pt
\textbf{Acodes Aumentada 2}
\vskip 25pt

\small
\upchord{\AtwoGuitar}{La} Aumentada 2
\upchord{\CtwoGuitar}{Do} Aumentada 2
\upchord{\DtwoGuitar}{Re} Aumentada 2
\upchord{\EflattwoGuitar}{Eb} Aumentada 2
\upchord{\FtwoGuitar}{Fa} Aumentada 2
\upchord{\GtwoGuitar}{Sol} Aumentada 2
\normalsize

\vskip 20pt
\textbf{Acodes Novena}
\vskip 25pt

\small
\upchord{\AnineGuitar}{La} Novena
\upchord{\CnineGuitar}{Do} Novena
\upchord{\FnineGuitar}{Fa} Novena
\upchord{\GnineGuitar}{Sol} Novena
\normalsize

\vskip 20pt
\textbf{Acodes trecena}
\vskip 25pt

\small
\upchord{\AsthirteenGuitar}{A\#} 13
\normalsize

\vskip 20pt
\textbf{Acodes Menor Onceava}
\vskip 25pt

\small
\upchord{\EmelevenGuitar}{Mi} Menor 11
\normalsize

\vskip 20pt
\textbf{Acodes Disminuidos}
\vskip 25pt

\small
\upchord{\GsdimGuitar}{Sol} sostenido disminuido
\normalsize


\vskip 20pt
\textbf{Acodes Con Bajo cambiado}

\small
\upchord{\AAsGuitar}{La Mayor bajo Bb} \hfill
\upchord{\ACsGuitar}{La Mayor bajo C\#} \hfill
\upchord{\ADGuitar}{La Mayor bajo D} \hfill
\upchord{\AEGuitar}{La Mayor bajo E} \hfill\null\break
\vskip 20pt
\upchord{\AGGuitar}{La Mayor bajo G} \hfill
\upchord{\AsevenCsGuitar}{La} S\'eptima bajo C\# \hfill
\upchord{\AmCGuitar}{La Menor bajo C} \hfill
\upchord{\AmCsGuitar}{La Menor bajo C\#} \hfill\null\break
\vskip 20pt
\upchord{\AmFGuitar}{La Menor bajo F} \hfill
\upchord{\AmFsGuitar}{La Menor bajo F\#} \hfill
\upchord{\AmGGuitar}{La Menor bajo G} \hfill
\upchord{\AmGsGuitar}{La Menor bajo G\#} \hfill\null\break
\vskip 20pt
\upchord{\AmsevenFGuitar}{La Menor S\'eptima bajo F} \hfill
\upchord{\AflatFGuitar}{La bemol Mayor bajo F} \hfill
\upchord{\BDsGuitar}{Si Mayor bajo D\#} \hfill\null\break
\vskip 20pt
\upchord{\BflatFGuitar}{Si bemol} Mayor bajo F \hfill
\upchord{\BsevenDsGuitar}{Si S\'eptima bajo D\#} \hfill
\upchord{\BsevenFsGuitar}{Si S\'eptima bajo F\#} \hfill
\upchord{\CEGuitar}{Do} Mayor bajo E \hfill\null\break
\vskip 20pt
\upchord{\CGGuitar}{Do} Mayor bajo G \hfill
\upchord{\CmAflatGuitar}{Do} Menor bajo Ab \hfill
\upchord{\DflatFGuitar}{Re bemol bajo F} \hfill
\upchord{\DAGuitar}{Re Mayor bajo A} \hfill\null\break
\vskip 20pt
\upchord{\DCGuitar}{Re Mayor bajo C} \hfill
\upchord{\DEGuitar}{Re Mayor bajo E} \hfill
\upchord{\DmCGuitar}{Re Menor bajo C} \hfill
\upchord{\DmFGuitar}{Re Menor bajo F} \hfill\null\break
\vskip 20pt
\upchord{\DmsevenGGuitar}{Re Menor S\'eptima bajo G} \hfill
\upchord{\DFsGuitar}{Re Mayor bajo F\#} \hfill
\upchord{\DsevenFsGuitar}{Re S\'eptima bajo F\#} \hfill
\upchord{\EflatFGuitar}{Mi bemol bajo F} \hfill\null\break
\vskip 20pt
\upchord{\EDGuitar}{Mi Mayor Bajo D} \hfill
\upchord{\EFsGuitar}{Mi Mayor Bajo F\#} \hfill
\upchord{\EGGuitar}{Mi Mayor Bajo G} \hfill
\upchord{\EGsGuitar}{Mi Mayor Bajo G\#} \hfill\null\break
\vskip 20pt
\upchord{\EmDGuitar}{Mi Menor Bajo D} \hfill
\upchord{\EmGGuitar}{Mi Menor Bajo G} \hfill
\upchord{\FAGuitar}{Fa Mayor Bajo A} \hfill
\upchord{\GflatBflatGuitar}{Sol bemol} Bajo Si bemol \hfill\null\break
\vskip 20pt
\upchord{\GBGuitar}{Sol Mayor Bajo B} \hfill
\upchord{\GDGuitar}{Sol Mayor Bajo D} \hfill
\upchord{\GEGuitar}{Sol Mayor Bajo E} \hfill
\upchord{\GFGuitar}{Sol Mayor Bajo F} \hfill\null\break
\vskip 20pt
\upchord{\GnineDGuitar}{Sol} Mayor Novena Bajo D \hfill
\upchord{\GmBflatGuitar}{Sol Menor Bajo Bb} \hfill
\upchord{\AflatCGuitar}{La bemol Bajo C} \hfill\null\break
\vskip 20pt
\upchord{\AflatEflatGuitar}{La bemol Bajo Eb} \hfill
\upchord{\BmajsevenGGuitar}{Si} + bajo G \hfill\null\break
\vskip 20pt
\upchord{\BflatCGuitar}{Si bemol} bajo C \hfill
\upchord{\BflatDGuitar}{Si bemol} bajo D \hfill
\upchord{\BflatGGuitar}{Si bemol} bajo G \hfill\null\break
\vskip 20pt
\upchord{\DsusFsGuitar}{Re} Suspendida cuarta bajo F\# \hfill
\upchord{\CMajtwoDGuitar}{\qquad Do} maj7 aumentada 2 bajo D \hfill
\upchord{\FtwoAGuitar}{Fa}+2 Mayor bajo A \hfill\null\break
\vskip 20pt
\upchord{\BdimDGuitar}{Si} disminuido bajo D \hfill\null\break
\normalsize

\vskip 20pt
\textbf{Acodes mayor disminuida quinta}

\small
\upchord{\EdimfiveGuitar}{Mi} dim5
\normalsize

\vskip 20pt
\textbf{Acodes semidisminuidos}

\small
\upchord{\BmsevenbfiveGuitar}{Si} Menor s\'eptima semidisminuido
\upchord{\EdimfiveGuitar}{Mi} Menor s\'eptima semidisminuido
\normalsize

\vskip 20pt
\textbf{Acodes diada arm\'onica}

\small
\upchord{\EfiveGuitar}{Mi} 5
\normalsize

\vskip 20pt
\textbf{Acodes 13 suspendida cuarta}

\small
\upchord{\CsusthirteenGuitar}{Do} 13 suspendida cuarta
\normalsize

\clearpage
\fi

\ifpiano
\pdfbookmark[0]{Acordes~para~Piano}{acordespiano}
\lhead{\LHeadFont Acordes~para~Piano}
{\parindent 8pt
        {\myTitleFont --- Acordes para Piano ---}}\par
\vskip 20pt
\textbf{Acodes Mayores}
\vskip 25pt

%\small{El s\'imbolo \# significa sostenido y {\flat}~significa~bemol}
\small
\upchord{\APiano}{\qquad La Mayor} \qquad\qquad \upchord{\BPiano}{Si Mayor} \qquad\qquad \upchord{\CPiano}{\qquad Do Mayor} \qquad\qquad \upchord{\DPiano}{\qquad Re Mayor} \hfill \break
\vskip 25pt
\upchord{\EPiano}{\qquad Mi Mayor} \qquad\qquad  \upchord{\FPiano}{\qquad Fa Mayor} \qquad\qquad \upchord{\GPiano}{\qquad Sol Mayor}
\vskip 25pt
\upchord{\AsPiano}{A\#/$B\flat$ Mayor} \qquad\qquad \upchord{\CsPiano}{C\#/$D\flat$ Mayor} \qquad\qquad \upchord{\DsPiano}{D\#/$E\flat$ Mayor} \qquad\qquad \upchord{\FsPiano}{F\#/$G\flat$ Mayor} \hfill \break
\vskip 25pt
\upchord{\GsPiano}{G\#/$A\flat$ Mayor} \qquad\qquad \upchord{\AsPiano}{A\#/$B\flat$ Mayor}
\normalsize

\textbf{Acodes Menores}
\vskip 25pt

\small
\upchord{\AmPiano}{\qquad La} Menor \qquad\qquad \upchord{\BmPiano}{\qquad Si} Menor \qquad\qquad \upchord{\CmPiano}{\qquad Do} Menor \qquad\qquad \upchord{\DmPiano}{\qquad Re} Menor \hfill \break
\vskip 25pt
\upchord{\EmPiano}{\qquad Mi} Menor \qquad\qquad \upchord{\FmPiano}{\qquad Fa} Menor \qquad\qquad \upchord{\GmPiano}{\qquad Sol} Menor
\vskip 25pt
\upchord{\AsmPiano}{\small{A\#/$B\flat$ Menor}}  \qquad\qquad  \upchord{\CsmPiano}{C\#/$D\flat$ Menor}  \qquad\qquad  \upchord{\DsmPiano}{D\#/$E\flat$ Menor} \hfill \break
\vskip 25pt
\upchord{\FsmPiano}{F\#/$G\flat$ Menor} \qquad\qquad \upchord{\GsmPiano}{G\#/$A\flat$ Menor}  \qquad\qquad  \upchord{\AsmPiano}{A\#/$B\flat$ Menor}
\normalsize

\clearpage
%\vskip 20pt
\textbf{Acodes Mayores S\'eptima}
\vskip 25pt

\small
\upchord{\AsevenPiano}{La Mayor s\'eptima} \hfill
\upchord{\BsevenPiano}{Si Mayor s\'eptima} \hfill
\upchord{\CsevenPiano}{Do Mayor s\'eptima} \hfill\null\break
\vskip 25pt
\upchord{\DsevenPiano}{Re Mayor s\'eptima} \hfill
\upchord{\EsevenPiano}{Mi Mayor s\'eptima} \hfill
\upchord{\FsevenPiano}{Fa Mayor s\'eptima} \hfill\null\break
\vskip 25pt
\upchord{\GsevenPiano}{Sol Mayor s\'eptima} \hfill
\upchord{\BflatsevenPiano}{Si} bemol Mayor s\'eptima \hfill\null\break
\vskip 25pt
\upchord{\EflatsevenPiano}{Mi bemol Mayor s\'eptima} \hfill
\upchord{\FssevenPiano}{Fa sostenido Mayor s\'eptima} \hfill
\upchord{\AflatsevenPiano}{Sol sostenido Mayor s\'eptima} \hfill\null\break
\normalsize
\vskip 20pt

\textbf{Acodes Menores S\'eptima}
\vskip 25pt

\small
\upchord{\AmsevenPiano}{La} Menor s\'eptima \hfill
\upchord{\BmsevenPiano}{Si} Menor s\'eptima \hfill
\upchord{\CmsevenPiano}{Do} Menor s\'eptima \hfill\null\break
\vskip 25pt
\upchord{\DmsevenPiano}{Re} Menor s\'eptima \hfill
\upchord{\EmsevenPiano}{Mi} Menor s\'eptima \hfill
\upchord{\FmsevenPiano}{Fa} Menor s\'eptima \hfill\null\break
\vskip 25pt
\upchord{\GmsevenPiano}{Sol} Menor s\'eptima \hfill
\upchord{\BflatmsevenPiano}{Si bemol} Menor s\'eptima \hfill
\upchord{\FsmsevenPiano}{Fa sostenido} Menor s\'eptima \hfill\null\break
\vskip 25pt
\upchord{\CssevenPiano}{Do sostenido} Mayor s\'eptima \hfill
\upchord{\DsmsevenPiano}{Re} Sostenido Menor s\'eptima \hfill
\upchord{\GsmsevenPiano}{Sol} Sostenido Menor s\'eptima \hfill\null\break
\normalsize

\vskip 20pt

\textbf{Acodes Mayores Suspendido cuarta}
\vskip 25pt

\small
\upchord{\AsusPiano}{La} Suspendida cuarta \hfill
\upchord{\BsusPiano}{Si} Suspendida cuarta \hfill
\upchord{\CsusPiano}{Do} Suspendida cuarta \hfill\null\break
\vskip 25pt
\upchord{\DsusPiano}{Re} Suspendida cuarta \hfill
\upchord{\EsusPiano}{Mi} Suspendida cuarta \hfill
\upchord{\FsusPiano}{Fa} Suspendida cuarta \hfill\null\break
\vskip 25pt
\upchord{\FssusPiano}{Fa sostenido} Suspendida cuarta \hfill
\upchord{\GsusPiano}{Sol} Suspendida cuarta \hfill
\upchord{\GssusPiano}{Sol sostenido} Suspendida cuarta \hfill\null\break
\normalsize

\vskip 20pt
\textbf{Acodes Mayor S\'eptima Aumentada}
\vskip 25pt

Estos acordes tienen las siguientes notaciones:
Amaj7, A+7, AM7, $A^{+7}$, $A^{M7}$, $A\Delta7$, $A^{\Delta7}$\break
\vskip 20pt

\small
\upchord{\AMajPiano}{La} Maj \hfill
\upchord{\CMajPiano}{Do} Maj \hfill
\upchord{\DMajPiano}{Re} Maj \hfill\null\break
\vskip 20pt
\upchord{\FMajPiano}{Fa} Maj \hfill
\upchord{\GMajPiano}{Sol} Maj \hfill
\upchord{\AsMajPiano}{La\#/Sib} Maj \hfill\null\break
\vskip 20pt
\upchord{\EflatMajPiano}{Re\#/Mib} Maj \hfill\null\break
\normalsize

\vskip 20pt
\textbf{Acodes Suspendida 2}
\vskip 25pt

\small
\upchord{\AsustwoPiano}{La} Suspendida 2
\upchord{\CsustwoPiano}{Do} Suspendida 2
\normalsize

\vskip 20pt
\textbf{Acodes Suspendida 9}
\vskip 25pt

\small
\upchord{\DsusninePiano}{Re} Suspendida novena
\normalsize

\vskip 20pt
\textbf{Acodes Aumentada 2}
\vskip 25pt

\small
\upchord{\AtwoPiano}{La} Aumentada 2 \hfill
\upchord{\CtwoPiano}{Do} Aumentada 2 \hfill
\upchord{\DtwoPiano}{Re} Aumentada 2 \hfill\null\break
\vskip 20pt
\upchord{\EflattwoPiano}{Re b} Aumentada 2 \hfill
\upchord{\FtwoPiano}{Fa} Aumentada 2 \hfill
\upchord{\GtwoPiano}{Sol} Aumentada 2 \hfill\null\break
\normalsize


\vskip 20pt
\textbf{Acodes Novena}
\vskip 25pt

\small
\upchord{\AninePiano}{La} Novena \hfill
\upchord{\CninePiano}{Do} Novena \hfill
\upchord{\FninePiano}{Fa} Novena \hfill\null\break
\vskip 20pt
\upchord{\GninePiano}{Sol} Novena \hfill
\normalsize

\vskip 20pt
\textbf{Acodes trecena}
\vskip 25pt

\small
\upchord{\AsthirteenPiano}{A\#} 13
\normalsize

\vskip 20pt
\textbf{Acodes Menor Onceava}
\vskip 25pt

\small
\upchord{\EmelevenPiano}{Mi} Menor 11
\normalsize
\clearpage

%\vskip 20pt
\textbf{Acodes Disminuidos}
\vskip 25pt

\small
\upchord{\GsdimPiano}{Sol} sostenido disminuido
\normalsize


\vskip 20pt
\textbf{Acodes Con Bajo cambiado}
\vskip 25pt

En el caso del piano, es com\'un que la parte del bajo se toque octavado en la mano izquierda y en la derecha el acorde normal, aqu\'i se pone la transposici\'on para ayudar a encontrar la melod\'ia.
\vskip 25pt
\small
\upchord{\AAsPiano}{La Mayor bajo Bb} \hfill
\upchord{\ACsPiano}{La Mayor bajo C\#} \hfill\null\break
\vskip 20pt
\upchord{\ADPiano}{La Mayor bajo D} \hfill
\upchord{\AEPiano}{La} Mayor bajo E \hfill\null\break
\vskip 20pt
\upchord{\AGPiano}{La Mayor bajo G} \hfill
\upchord{\AsevenCsPiano}{La} Mayor S\'eptima bajo C\# \hfill\null\break
\vskip 20pt
\upchord{\AmCPiano}{La Menor bajo C} \hfill
\upchord{\AmCsPiano}{La Menor bajo C\#} \hfill\null\break
\vskip 20pt
\upchord{\AmFPiano}{La Menor bajo F} \hfill
\upchord{\AmFsPiano}{La Menor bajo F\#} \hfill
\upchord{\AmGPiano}{La Menor bajo G} \hfill\null\break
\vskip 20pt
\upchord{\AmGsPiano}{La Menor bajo G\#} \hfill
\upchord{\AmsevenFPiano}{La Menor S\'eptima bajo F} \hfill
\upchord{\AflatFPiano}{La bemol Mayor bajo F} \hfill\null\break
\vskip 20pt
\upchord{\BDsPiano}{Si Mayor bajo D\#} \hfill
\upchord{\BflatFPiano}{Si bemol} Mayor bajo F \hfill
\upchord{\BsevenDsPiano}{Si S\'eptima bajo D$\#$} \hfill\null\break
\vskip 20pt
\upchord{\BsevenFsPiano}{Si S\'eptima bajo F$\#$} \hfill\null\break
\vskip 20pt
\upchord{\CEPiano}{Do Mayor bajo E} \hfill
\upchord{\CmAflatPiano}{Do Menor bajo Ab} \hfill\null\break
\vskip 20pt
\upchord{\CGPiano}{Do Mayor bajo G} \hfill
\upchord{\DAPiano}{Re Mayor bajo A} \hfill
\upchord{\DCPiano}{Re Mayor bajo C} \hfill\null\break
\vskip 20pt
\upchord{\DmCPiano}{Re Menor bajo C} \hfill
\upchord{\DmFPiano}{Re Menor bajo F} \hfill
\upchord{\DmsevenGPiano}{Re Menor S\'eptima bajo G} \hfill\null\break
\vskip 20pt
\upchord{\DFsPiano}{Re Mayor bajo F\#} \hfill
\upchord{\DsevenFsPiano}{Re S\'eptima bajo F\#} \hfill
\upchord{\EflatFPiano}{Mi bemol bajo F} \hfill\null\break
\vskip 20pt
\upchord{\EDPiano}{\qquad Mi} Mayor con bajo D \hfill
\upchord{\EFsPiano}{\qquad Mi} Mayor con bajo F\# \hfill
\upchord{\EGPiano}{\qquad Mi} Mayor con bajo G \hfill\null\break
\vskip 20pt
\upchord{\EGsPiano}{\qquad Mi} Mayor con bajo G\# \hfill
\upchord{\EmDPiano}{\qquad Mi} Menor Bajo D \hfill
\upchord{\EmGPiano}{\qquad Mi} Menor Bajo G \hfill\null\break
\vskip 20pt
\upchord{\FAPiano}{Fa} Mayor bajo A \hfill
\upchord{\GflatBflatPiano}{Sol bemol Mayor bajo Si Bemol} \hfill
\upchord{\GDPiano}{Sol Mayor Bajo D} \hfill\null\break
\vskip 20pt
\upchord{\GEPiano}{Sol Mayor Bajo E} \hfill %long
\upchord{\GFPiano}{Sol Mayor Bajo F} \hfill\null\break
\vskip 20pt
\upchord{\GBPiano}{\qquad\qquad Sol}  Mayor Bajo B \hfill
\upchord{\GnineDPiano}{Sol} Mayor Novena Bajo D \hfill\null\break %long
\vskip 20pt
\upchord{\GmBflatPiano}{Sol Menor Bajo Bb} \hfill
\upchord{\AflatCPiano}{La bemol Bajo C} \hfill
\upchord{\AflatEflatPiano}{La bemol Bajo Eb} \hfill\null\break
\vskip 20pt
\upchord{\BflatCPiano}{Si bemol} bajo C \hfill
\upchord{\BflatDPiano}{Si bemol} bajo D \hfill\null\break
\vskip 20pt
\upchord{\BflatGPiano}{Si bemol} bajo G \hfill
\upchord{\BmajsevenGPiano}{Si} + bajo G \hfill
\upchord{\DsusFsPiano}{Re Suspendida} cuarta bajo F\# \hfill\null\break
\vskip 20pt
\upchord{\CMajtwoDPiano}{\qquad Do}+7+2 bajo D \hfill
\upchord{\FtwoAPiano}{Fa}+2 Mayor bajo A \hfill\null\break
\vskip 20pt
\upchord{\BdimDPiano}{Si} disminuido bajo D \hfill\null\break
\vskip 20pt
\normalsize

\clearpage
%\vskip 20pt
\textbf{Acodes mayor disminuida quinta}
\vskip 25pt

\small
\upchord{\EdimfivePiano}{Mi} dim5
\normalsize

\vskip 20pt
\textbf{Acodes medio disminuido s\'eptima}
\vskip 25pt

\small
\upchord{\BmsevenbfivePiano}{Si} medio disminuido s\'eptima
\normalsize

\vskip 20pt
\textbf{Acodes diada arm\'onica}
\vskip 25pt

\small
\upchord{\EfivePiano}{Mi} 5
\normalsize

\vskip 20pt
\textbf{Acodes 13 suspendida cuarta}
\vskip 25pt

\small
\upchord{\CsusthirteenPiano}{Do} 13 suspendida cuarta
\normalsize

\clearpage
\fi
%\end{document}
%\bye
\end{document}
\bye
%
%%%
% Document ends.
%%%
