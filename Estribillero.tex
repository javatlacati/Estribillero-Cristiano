%%%%%% rcsid = @(#)$Id: sample-sb.tex,v 1.23 2010-04-12 18:04:11 rathc Exp $
%%%%%%
%%
%%      ===============================
%%      Sample Songbook (sample-sb.tex)
%%      ===============================
%%
%%      Version 4.5, 30 April, 2010
%%
%%      Copyright 1992--2010 Christopher Rath <christopher@rath.ca>
%%
%%      This package is free software; you can redistribute it and/or
%%      modify it under the terms of version 2.1 of the GNU Lesser
%%	General Public License as published by the Free Software 
%%	Foundation.
%%
%%      This package is distributed in the hope that it will be
%%      useful, but WITHOUT ANY WARRANTY; without even the implied
%%      warranty of MERCHANTABILITY or FITNESS FOR A PARTICULAR
%%      PURPOSE.  See the GNU Lesser General Public License for more
%%      details.
%%
%%      This file contains a subset of the songbook we distribute
%%      at our church.  To the best of my knowledge, all of the lyrics
%%      contained herein are freely distributable.  This file has been
%%      provided as a sample of what can be produced by the chordbk,
%%      wordbk, and overhead LaTeX styles.
%%
%%      NEEDED:  The fancyhdr LaTeX style is required to properly
%%              format this file.  If you don't have that then comment
%%              out the commands in the preamble which deal with the
%%              fancyhdr style.
%%
%%%%%%
%%%%%%
%%
%%      1. Chord notation.  Within this songbook the following
%%         conventions have been adopted:
%%
%%              "Minor" is entered as "m";
%%                      e.g. Cm7 for C minor 7th.
%%              "Major" is entered as "M";
%%                      e.g. CM7 for C major 7th.
%%
%%%%%%
%%%%%%
%%      ============
%%      Bibliography
%%      ============
%%
%%    
%%
%%%%%%
%%%%%%

%%%%%%%%%%%%%%%%%%%%%%%%%%%%%%%%%%%%%%%%%%%%%%%%%%%%%%%%%%
%%%%%%%%%%%%%%%%%%%%%%%%%%%%%%%%%%%%%%%%%%%%%%%%%%%%%%%%%%
%%                                                      %%
%%           P R E A M B L E   B E G I N S              %%
%%                                                      %%
%%%%%%%%%%%%%%%%%%%%%%%%%%%%%%%%%%%%%%%%%%%%%%%%%%%%%%%%%%
%%%%%%%%%%%%%%%%%%%%%%%%%%%%%%%%%%%%%%%%%%%%%%%%%%%%%%%%%%

\documentclass[12pt, spanish]{book}
\usepackage[T1]{fontenc}
\usepackage[latin9]{inputenc}
\usepackage{babel}
\usepackage{mypiano}
\usepackage{gchords}
\usepackage{latexsym,fancyhdr}
\usepackage[unicode=true,pdfusetitle, bookmarks=true,bookmarksnumbered=false,bookmarksopen=false, 
 breaklinks=false,pdfborder={0 0 1},backref=false,colorlinks=false]
 {hyperref}
\usepackage[chordbk]{songbook}                  %% Words & Chords edition.
%%\usepackage[compactallsongs,chordbk]{songbook}    %% Words & Chords edition.
%%\usepackage[wordbk]{songbook}                 %% Words Only edition.
%%\usepackage[overhead]{songbook}               %% Overhead Transparency edition.



  \renewcommand{\SBChorusTag}{Coro:}
%%%
% Revision Date and Release Date definitions.
%
%       \RelDate - The last time this songbook was released.  Set this
%                  date each time a new release/update of the songbook
%                  is generated.
%       \RevDate - The last time a particular song was revised in any
%                  way.  This command will be renewed inside every
%                  song.
%%%
\newcommand{\RelDate}{26~Marzo,~2014}
\newcommand{\RevDate}{\today}

%%%
% C.C.L.I. license number definition; for copyright licensing info.
% One of these macros will be manually inserted into the {CpyRt}
% parameter of the {song} environment.
%
%       \CCLInumber - The actual copyright license number.  Don't
%               insert this command in the {CpyRt} parameter, use one
%               of the others.
%       \CCLIed - Indicates a song falls under our CCLI license.
%       \NotCCLIed - Indicates a song doesn't fall under our CCLI
%               license.  Public Domain songs fall into this category.
%       \PGranted - We have received specific permission from the
%               copyright holder to use this song.
%       \PPending - We are in the process of obtaining permission to
%               use this song.
%%%
\newcommand{\CCLInumber}{Your CCLI Number}
\newcommand{\CCLIed}{{\CpyRtInfoFont (CCLI \CCLInumber)}}
\newcommand{\NotCCLIed}{\relax}
\newcommand{\PGranted}{\relax}
\newcommand{\PPending}{{\CpyRtInfoFont (Permission Pending)}}

%%%
% Title page information.
%%%
\title{Cuaderno de Himnos Tradicionales y Contemporáneos}
\author{Javatlacati}
\date{\'Ultima Revisión:  \RevDate}

%%%
% Redefine fonts from SongBook style that I don't like.
%%%
\font\myTinySF=cmss8 at 8pt
\renewcommand{\CpyRtInfoFont}{\tiny\myTinySF}

%%%
% Define fonts to use in the headers and footers of the songbook.
%%%
\newcommand{\LHeadFont}{\normalsize}            % = cmr12  at 12pt
\newcommand{\CHeadFont}{\normalsize\rm}         % = cmr12  at 12pt
\newcommand{\RHeadFont}{\normalsize}            % = cmr12  at 12pt
\newcommand{\LFootFont}{\scriptsize}            % = cmr8   at  8pt
\newcommand{\CFootFont}{\tiny\myTinySF}         % = cmss8  at  8pt
\newcommand{\RFootFont}{\scriptsize}            % = cmr8   at  8pt

%%%
% Turn on and define fancy page heading/footing definition.
%%%
\pagestyle{fancy}

\ifChordBk
  % It's a words & chords songbook...
  \addtolength{\headwidth}{\marginparsep}
  \addtolength{\headwidth}{\marginparwidth}
  \renewcommand{\headrulewidth}{0.4pt}
  \renewcommand{\footrulewidth}{0.4pt}
  \fancyhead[LE,RO]{\LHeadFont\emph{\leftmark\/}\SBContinueMark}
  \fancyhead[CE,CO]{\CHeadFont\thepage}
  \fancyhead[RE,LO]{\RHeadFont\RelDate}
\else\ifOverhead
  % It's an overhead...
  \renewcommand{\footrulewidth}{0pt}
  \renewcommand{\headrulewidth}{0pt}
  \fancyhead[LE,RO]{}
  \fancyhead[CE,CO]{}
  \fancyhead[RE,LO]{}
\else\ifWordBk
  % It's a words only songbook...
  \addtolength{\headwidth}{\marginparsep}
  \addtolength{\headwidth}{\marginparwidth}
  \renewcommand{\headrulewidth}{0.4pt}
  \renewcommand{\footrulewidth}{0.4pt}
  \fancyhead[LE,RO]{\LHeadFont Estribillero}
  \fancyhead[CE,CO]{\CHeadFont\thepage}
  \fancyhead[RE,LO]{\RHeadFont\RelDate}
\fi\fi\fi

\fancyfoot[LE,RO]{\LFootFont Transcripciones}
\ifSongEject
  \fancyfoot[CE,CO]{\CFootFont \RevDate}
\else
  \fancyfoot[CE,CO]{\CFootFont}
\fi
\fancyfoot[RE,LO]{\RFootFont Todo el material son transcripciones personales.}

%%%
% Turn on/off index-file generation.  Uncomment the \makeindex line to
% turn index generation on;  comment it out to turn index generation
% off.
%%%
\makeTitleIndex         %% Title and First Line Index.
\makeTitleContents      %% Table of Contents.
\makeKeyIndex           %% Index of song by key.
\makeArtistIndex	%% Index of song by artist.

%%%%%%%%%%%%%%%%%%%%%%%%%%%%%%%%%%%%%%%%%%%%%%%%%%%%%%%%%%
%%%%%%%%%%%%%%%%%%%%%%%%%%%%%%%%%%%%%%%%%%%%%%%%%%%%%%%%%%
%%                                                      %%
%%           D O C U M E N T   B E G I N S              %%
%%                                                      %%
%%%%%%%%%%%%%%%%%%%%%%%%%%%%%%%%%%%%%%%%%%%%%%%%%%%%%%%%%%
%%%%%%%%%%%%%%%%%%%%%%%%%%%%%%%%%%%%%%%%%%%%%%%%%%%%%%%%%%
\begin{document}

%%%
% Uncomment "\maketitle" statement to make a title page.
%%%
\maketitle

\mainmatter
\ifWordBk
  \twocolumn
\fi

%%%%%% rcsid = @(#)$Id: sampleToc.tex,v 1.18 2010-04-12 18:04:32 rathc Exp $
%%%%%%
%%
%%      ========================================
%%      Sample Table Of Contents (sampleToc.tex)
%%      ========================================
%%
%%      Version 4.5, 30 April, 2010
%%
%%      Copyright 1992--2010 Christopher Rath <christopher@rath.ca>
%%
%%      This package is free software; you can redistribute it and/or
%%      modify it under the terms of version 2.1 of the GNU Lesser 
%%	General Public License as published by the Free Software 
%%	Foundation.
%%
%%      This package is distributed in the hope that it will be
%%      useful, but WITHOUT ANY WARRANTY; without even the implied
%%      warranty of MERCHANTABILITY or FITNESS FOR A PARTICULAR
%%      PURPOSE.  See the GNU Lesser General Public License for more
%%      details.
%%
%%      This file is provided as a template for Table of Contents
%%      generation.
%%
%%%%%%
%%%%%%

%%%%%%%%%%%%%%%%%%%%%%%%%%%%%%%%%%%%%%%%%%%%%%%%%%%%%%%%%%
%%%%%%%%%%%%%%%%%%%%%%%%%%%%%%%%%%%%%%%%%%%%%%%%%%%%%%%%%%
%%                                                      %%
%%           P R E A M B L E   B E G I N S              %%
%%                                                      %%
%%%%%%%%%%%%%%%%%%%%%%%%%%%%%%%%%%%%%%%%%%%%%%%%%%%%%%%%%%
%%%%%%%%%%%%%%%%%%%%%%%%%%%%%%%%%%%%%%%%%%%%%%%%%%%%%%%%%%

\documentclass[12pt,twocolumn]{book}
\usepackage{latexsym,fancyhdr}
\usepackage[wordbk]{songbook}

%%%
% Revision Date and Release Date definitions.
%
%       \RelDate - The last time this songbook was released.
%       \RevDate - The last time this file was revised in any way.
%%%
\newcommand{\RelDate}{30 May'96}
\newcommand{\RevDate}{\today}

%%%
% Redefine fonts from SongBook style that I don't like, and define
% any extra fonts I require.
%%%
\font\myTinySF=cmss8    at  8pt
\font\myHugeSF=cmssbx10 at 25pt
\renewcommand{\CpyRtInfoFont}{\tiny\myTinySF}
\newcommand{\myTitleFont}{\Huge\myHugeSF}
\newcommand{\mySubTitleFont}{\large\sf}

%%%
% Define fonts to use in the headers and footers of the songbook.
%%%
\newcommand{\LHeadFont}{\normalsize}            % = cmr12  at 12pt
\newcommand{\CHeadFont}{\normalsize\rm}         % = cmr12  at 12pt
\newcommand{\RHeadFont}{\normalsize}            % = cmr12  at 12pt
\newcommand{\LFootFont}{\scriptsize}            % = cmr8   at  8pt
\newcommand{\CFootFont}{\tiny\myTinySF}         % = cmss8  at  8pt
\newcommand{\RFootFont}{\scriptsize}            % = cmr8   at  8pt

%%%
% Turn on and define fancy page heading/footing definition.
%%%
\pagestyle{fancy}

\addtolength{\headwidth}{\marginparsep}
\addtolength{\headwidth}{\marginparwidth}
\renewcommand{\footrulewidth}{0.4pt}
\lhead{\LHeadFont A Church Songbook}
       \chead{\CHeadFont Tabla~De~Contenidos({\rm\thepage})}
       \rhead{\RHeadFont\RelDate}

\lfoot{\LFootFont Property of a Church}
       \cfoot{\CFootFont Last Revised:  \RevDate}
       \rfoot{\RFootFont Material used by permission.}

%%%
% Index entries command definition.
%%%
\renewcommand{\item}{\par\hangindent=40pt}
\renewcommand{\subitem}{\par\hangindent=40pt \hspace*{20pt}}
\renewcommand{\subsubitem}{\par\hangindent=40pt \hspace*{30pt}}


%%%%%%%%%%%%%%%%%%%%%%%%%%%%%%%%%%%%%%%%%%%%%%%%%%%%%%%%%%
%%%%%%%%%%%%%%%%%%%%%%%%%%%%%%%%%%%%%%%%%%%%%%%%%%%%%%%%%%
%%                                                      %%
%%           D O C U M E N T   B E G I N S              %%
%%                                                      %%
%%%%%%%%%%%%%%%%%%%%%%%%%%%%%%%%%%%%%%%%%%%%%%%%%%%%%%%%%%
%%%%%%%%%%%%%%%%%%%%%%%%%%%%%%%%%%%%%%%%%%%%%%%%%%%%%%%%%%
\begin{document}

%%%
% Begin the Table Of Contents.
%%%
{\parindent 10pt
  {\myTitleFont --- Contenidos ---}}\par
\vskip 30pt

\input{Estribilero.toc}

\end{document}
\bye
%
%%%
% Document ends.
%%%


%%%
% Songbook begins.
%%%
\begin{song}{Abba Padre}{C}
  {} %copyright \SBPubDom
  {Marco Barrientos}
  {Romanos 8:15} %pasaje
  {\href{http://open.spotify.com/track/0yj0zBaa7Ckn6ZMQPmCmfF}{Escuchar}} %\NotCCLIed

%  \renewcommand{\RevDate}{February~11,~1993}
  \SBRef{No puedo parar de alabarte}{2006}%fuente \#

  \begin{SBVerse}
\Ch{D}{Una} vez más

me a\Ch{Bm}{cer}co a Tí

con \Ch{Em}{li}bertad

en adora\Ch{A}{ció}n
  \end{SBVerse}
  \begin{SBVerse}
Tú e\Ch{D}{res} mi Dios

tu \Ch{Bm}{hi}jo soy

mi \Ch{Em}{co}munión contigo

es una \Ch{A}{dul}ce bendición
  \end{SBVerse}

  \begin{SBChorus}
// ¡\Ch{D}{A}bba Pa\Ch{G}{dre}! ¡\Ch{D}{A}bba Pa\Ch{G}{dre}!

Es\Ch{D}{tar}\Ch{A/C#}{} conti\Ch{Bm}{go} \Ch{D/A}{es} una \Ch{Bm}{dul}ce bendi\Ch{A}{ción}

¡\Ch{D}{A}bba Pa\Ch{G}{dre}! te \Ch{F#m}{a}mo Se\Ch{Bm}{ñor}

\Ch{G}{quie}ro estar en \Ch{D}{co}munión

\Ch{Em}{quie}ro es\Ch{A}{tar} con\Ch{D}{ti}go. //
  \end{SBChorus}
\ifChordBk
\begin{SBOpGroup}
    Acordes:
%\keyboardtwooctaves[Do][Fso][Ao]
% \upchord{
\keyboard[Do][Fso][Ao]\D\keyboard[Do][Go][Bo]\G
%}{Re} Mayor

% \upchord{

%}{Sol} Mayor

 \upchord{\keyboard[Do][Fso][Bo]\Bm}{Si} Menor
  \end{SBOpGroup}
\fi
\end{song}

\begin{song}{A Cristo solo a Cristo}{G}
  {} %copyright \SBPubDom
  {Marcos Witt}
  {} %pasaje
  {} %\NotCCLIed

%  \renewcommand{\RevDate}{February~11,~1993}
%  \SBRef{No puedo parar de alabarte}{2006}%fuente \#

  \begin{SBVerse}
A Cristo solo a Cristo, yo exaltaré

A Cristo solo a Cristo, yo adoraré
  \end{SBVerse}
  \begin{SBVerse}
Porque Él me ha dado vida eterna
Porque Él me ha dado el poder
Porque Él me ha dado la victoria
Él es mi Rey.
a Cristo he proclamado Rey
  \end{SBVerse}
\ifChordBk
\begin{SBOpGroup}
    Acordes:
%\keyboardtwooctaves[Do][Fso][Ao]
 \upchord{\keyboard[Do][Fso][Ao]\D}{Re} Mayor
  \end{SBOpGroup}
\fi
\end{song}

\begin{song}{A Dios el Padre Celestial}{G}
  {} %copyright \SBPubDom
  {Thomas Ken, Genevan Psalter, 1551, atr. a Louis Bourgeois}
  {} %pasaje
  {} %\NotCCLIed

%  \renewcommand{\RevDate}{February~11,~1993}
%  \SBRef{No puedo parar de alabarte}{2006}%fuente \#

  \begin{SBVerse}
A Dios el Padre Celestial,
Al Hijo nuestro Redentor.
Y al Eternal Consolador,
unidos todos alabad.
Amén.
  \end{SBVerse}
\ifChordBk
\begin{SBOpGroup}
    Acordes:
%\keyboardtwooctaves[Do][Fso][Ao]
 \upchord{\keyboard[Do][Fso][Ao]\D}{Re} Mayor
  \end{SBOpGroup}
\fi
\end{song}


\begin{song}{Admirable}{G}
  {} %copyright \SBPubDom
  {}
  {} %pasaje
  {} %\NotCCLIed

%  \renewcommand{\RevDate}{February~11,~1993}
%  \SBRef{No puedo parar de alabarte}{2006}%fuente \#

  \begin{SBVerse}
Con poder y autoridad nuestro Dios venció a la
muerte
Sobre el trono celestial siempre reinará.
Sentado en majestad suyo es el reino por los
Siglos y por la eternidad su luz de gloria brillará.
  \end{SBVerse}

\begin{SBChorus}
Admirable, consejero mi Dios consolador,
Eres digno de alabanza, Príncipe de paz.
  \end{SBChorus}
\ifChordBk
\begin{SBOpGroup}
    Acordes:
%\keyboardtwooctaves[Do][Fso][Ao]
 \upchord{\keyboard[Do][Fso][Ao]\D}{Re} Mayor
  \end{SBOpGroup}
\fi
\end{song}

\begin{song}{A Nuestro Padre Dios}{G}
  {} %copyright \SBPubDom
  {Anónimo en Thesaurus Musicus 1744}
  {} %pasaje
  {} %\NotCCLIed

%  \renewcommand{\RevDate}{February~11,~1993}
%  \SBRef{No puedo parar de alabarte}{2006}%fuente \#

  \begin{SBVerse}
A nuestro Padre Dios
Alcemos nuestra voz
¡Gloria a Él!
Tal fué su amor que dió
Al hijo que murió,
En quien confío yo;
¡Gloria a Él!
  \end{SBVerse}

  \begin{SBVerse}
A nuestro Salvador
Demos con fe loor
½Gloria a Él!
Su sangre derramó
Con ella me lavó,
Y el cielo me abrió
½Gloria a Él!
  \end{SBVerse}

  \begin{SBVerse}
Espíritu de Dios,
Elevo a Ti mi voz;
½Gloria a Ti!
Con celestial fulgor
Me muestras el amor
De Cristo mi Señor
½Gloria a Ti!
  \end{SBVerse}

  \begin{SBVerse}
Con gozo y amor,
Cantemos con fervor
Al Trino Dios.
En la eternidad
Mora la Trinidad;
½Por siempre alabad
Al Trino Dios!
  \end{SBVerse}

\ifChordBk
\begin{SBOpGroup}
    Acordes:
%\keyboardtwooctaves[Do][Fso][Ao]
 \upchord{\keyboard[Do][Fso][Ao]\D}{Re} Mayor
  \end{SBOpGroup}
\fi
\end{song}
%\WBPageBrk %para forzar salto de pagina

\begin{song}{Garment Of Praise}{C}
  {}
  {}
  {Isaiah~61:3}
  {\NotCCLIed}

  \renewcommand{\RevDate}{February~11,~1993}
  %\SBRef{}{}
  \FLineIdx{I have put on my garment of praise}

  \begin{SBOpGroup}
    I have \Ch{C}{put} on my garment of \Ch{G7}{praise,}
    
    I have \Ch{G7}{put} on my garment of \Ch{C}{prai}\Ch{C7}{se}
    
    And the \Ch{F}{spirit} of heaviness, is \Ch{C}{gone} from \Ch{Am}{me.}
    
    I have \Ch{C}{put} on my \Ch{G7}{gar}ment of \Ch{C}{praise!} \ChX{[}{}\ChX{F}{} \ChX{C}{}\ChX{]}{}
  \end{SBOpGroup}

  \begin{SBExtraKeys}{
    \STitle{Garment Of Praise}{D}

    \begin{SBOpGroup}
      I have \Ch{D}{put} on my garment of \Ch{A7}{praise,}
      
      I have \Ch{A7}{put} on my garment of \Ch{D}{prai}\Ch{D7}{se}
      
      And the \Ch{G}{spirit} of heaviness, is \Ch{D}{gone} from \Ch{Bm}{me.}
      
      I have \Ch{D}{put} on my \Ch{A7}{gar}ment of \Ch{D}{praise!} \Ch{[}{}\Ch{G}{} \Ch{D}{}\Ch{]}{}
    \end{SBOpGroup}
  }\end{SBExtraKeys}
\end{song}

%%%%%% rcsid = @(#)$Id: sampleTdx.tex,v 1.18 2010-04-12 18:04:31 rathc Exp $
%%%%%%
%%
%%      ===============================================
%%      Sample Title & First Line Index (sampleTdx.tex)
%%      ===============================================
%%
%%      Version 4.5, 30 April, 2010
%%
%%      Copyright 1992--2010 Christopher Rath <christopher@rath.ca>
%%
%%      This package is free software; you can redistribute it and/or
%%      modify it under the terms of version 2.1 of the GNU Lesser 
%%	General Public License as published by the Free Software 
%%	Foundation.
%%
%%      This package is distributed in the hope that it will be
%%      useful, but WITHOUT ANY WARRANTY; without even the implied
%%      warranty of MERCHANTABILITY or FITNESS FOR A PARTICULAR
%%      PURPOSE.  See the GNU Lesser General Public License for more
%%      details.
%%
%%      This file is provided as a template for Title and First Line
%%      Index generation.
%%
%%%%%%
%%%%%%

%%%%%%%%%%%%%%%%%%%%%%%%%%%%%%%%%%%%%%%%%%%%%%%%%%%%%%%%%%
%%%%%%%%%%%%%%%%%%%%%%%%%%%%%%%%%%%%%%%%%%%%%%%%%%%%%%%%%%
%%                                                      %%
%%           P R E A M B L E   B E G I N S              %%
%%                                                      %%
%%%%%%%%%%%%%%%%%%%%%%%%%%%%%%%%%%%%%%%%%%%%%%%%%%%%%%%%%%
%%%%%%%%%%%%%%%%%%%%%%%%%%%%%%%%%%%%%%%%%%%%%%%%%%%%%%%%%%

%\documentclass[12pt,twocolumn,spanish]{book}
%\usepackage{latexsym,fancyhdr}
%\usepackage[wordbk]{songbook}


%%%
% Revision Date and Release Date definitions.
%
%       \RelDate - The last time this songbook was released.
%       \RevDate - The last time this file was revised in any way.
%%%
%\newcommand{\RelDate}{30 May'96}
%\newcommand{\RevDate}{\today}

%%%
% Redefine fonts from SongBook style that I don't like, and define
% any extra fonts I require.
%%%
\font\myTinySF=cmss8    at  8pt
\font\myHugeSF=cmssbx10 at 25pt
\renewcommand{\CpyRtInfoFont}{\tiny\myTinySF}
%\newcommand{\myTitleFont}{\Huge\myHugeSF}
%\newcommand{\mySubTitleFont}{\large\sf}

%%%
% Define fonts to use in the headers and footers of the songbook.
%%%
%\newcommand{\LHeadFont}{\normalsize}            % = cmr12  at 12pt
%\newcommand{\CHeadFont}{\normalsize\rm}         % = cmr12  at 12pt
%\newcommand{\RHeadFont}{\normalsize}            % = cmr12  at 12pt
%\newcommand{\LFootFont}{\scriptsize}            % = cmr8   at  8pt
%\newcommand{\CFootFont}{\tiny\myTinySF}         % = cmss8  at  8pt
%\newcommand{\RFootFont}{\scriptsize}            % = cmr8   at  8pt

%%%
% Turn on and define fancy page heading/footing definition.
%%%
\pagestyle{fancy}

\addtolength{\headwidth}{\marginparsep}
\addtolength{\headwidth}{\marginparwidth}
\renewcommand{\footrulewidth}{0.4pt}
\lhead{\LHeadFont A Church Songbook}
       \chead{\CHeadFont \I'ndice~por~T\'itulo({\rm\thepage})}
       \rhead{\RHeadFont\RelDate}

\lfoot{\LFootFont Property of a Church}
       \cfoot{\CFootFont Last Revised:  \RevDate}
       \rfoot{\RFootFont Material used by permission.}

%%%
% Index entries command definition.
%%%
\renewcommand{\item}{\par\hangindent=40pt}
\renewcommand{\subitem}{\par\hangindent=40pt \hspace*{20pt}}
\renewcommand{\subsubitem}{\par\hangindent=40pt \hspace*{30pt}}


%%%%%%%%%%%%%%%%%%%%%%%%%%%%%%%%%%%%%%%%%%%%%%%%%%%%%%%%%%
%%%%%%%%%%%%%%%%%%%%%%%%%%%%%%%%%%%%%%%%%%%%%%%%%%%%%%%%%%
%%                                                      %%
%%           D O C U M E N T   B E G I N S              %%
%%                                                      %%
%%%%%%%%%%%%%%%%%%%%%%%%%%%%%%%%%%%%%%%%%%%%%%%%%%%%%%%%%%
%%%%%%%%%%%%%%%%%%%%%%%%%%%%%%%%%%%%%%%%%%%%%%%%%%%%%%%%%%
%\begin{document}

%%%
% Begin the Index.
%%%
{\parindent 8pt
  {\myTitleFont --- Title Index ---}}\par
\vskip 5pt
{\parindent 20pt
  {\mySubTitleFont --- with first lines in italic ---}}
\vskip 20pt

%%%%%% rcsid = @(#)$Id: sample-sb.tex,v 1.23 2010-04-12 18:04:11 rathc Exp $
%%%%%%
%%
%%      ===============================
%%      Sample Songbook (sample-sb.tex)
%%      ===============================
%%
%%      Version 4.5, 30 April, 2010
%%
%%      Copyright 1992--2010 Christopher Rath <christopher@rath.ca>
%%
%%      This package is free software; you can redistribute it and/or
%%      modify it under the terms of version 2.1 of the GNU Lesser
%%	General Public License as published by the Free Software 
%%	Foundation.
%%
%%      This package is distributed in the hope that it will be
%%      useful, but WITHOUT ANY WARRANTY; without even the implied
%%      warranty of MERCHANTABILITY or FITNESS FOR A PARTICULAR
%%      PURPOSE.  See the GNU Lesser General Public License for more
%%      details.
%%
%%      This file contains a subset of the songbook we distribute
%%      at our church.  To the best of my knowledge, all of the lyrics
%%      contained herein are freely distributable.  This file has been
%%      provided as a sample of what can be produced by the chordbk,
%%      wordbk, and overhead LaTeX styles.
%%
%%      NEEDED:  The fancyhdr LaTeX style is required to properly
%%              format this file.  If you don't have that then comment
%%              out the commands in the preamble which deal with the
%%              fancyhdr style.
%%
%%%%%%
%%%%%%
%%
%%      1. Chord notation.  Within this songbook the following
%%         conventions have been adopted:
%%
%%              "Minor" is entered as "m";
%%                      e.g. Cm7 for C minor 7th.
%%              "Major" is entered as "M";
%%                      e.g. CM7 for C major 7th.
%%
%%%%%%
%%%%%%
%%      ============
%%      Bibliography
%%      ============
%%
%%    
%%
%%%%%%
%%%%%%

%%%%%%%%%%%%%%%%%%%%%%%%%%%%%%%%%%%%%%%%%%%%%%%%%%%%%%%%%%
%%%%%%%%%%%%%%%%%%%%%%%%%%%%%%%%%%%%%%%%%%%%%%%%%%%%%%%%%%
%%                                                      %%
%%           P R E A M B L E   B E G I N S              %%
%%                                                      %%
%%%%%%%%%%%%%%%%%%%%%%%%%%%%%%%%%%%%%%%%%%%%%%%%%%%%%%%%%%
%%%%%%%%%%%%%%%%%%%%%%%%%%%%%%%%%%%%%%%%%%%%%%%%%%%%%%%%%%

\documentclass[12pt, spanish, titlepage]{book}
\usepackage[T1]{fontenc}
\usepackage[latin9]{inputenc}
\usepackage{babel}
\usepackage{mypiano}
\usepackage{gchords}
\usepackage{latexsym,fancyhdr}
\usepackage{imakeidx}
\usepackage[unicode=true,pdfusetitle, bookmarks=true,bookmarksnumbered=false,bookmarksopen=false,
    breaklinks=false,pdfborder={0 0 1},backref=false,colorlinks=true]{hyperref}
\usepackage[chordbk]{songbook} %% Words & Chords edition.
%%\usepackage[compactallsongs,chordbk]{songbook}    %% Words & Chords edition.
%%\usepackage[wordbk]{songbook}                 %% Words Only edition.
%%\usepackage[overhead]{songbook}               %% Overhead Transparency edition.


% genera acordes de guitarra
\newif\ifguitarra

%genera acordes de piano
\newif\ifpiano

\guitarratrue
\pianotrue



\renewcommand{\SBChorusTag}{Coro:}
\renewcommand{\SBBridgeTag}{Puente:}
\newcommand{\myTitleFont}{\Huge\myHugeSF}
\newcommand{\mySubTitleFont}{\large\sf}
%%%
% Revision Date and Release Date definitions.
%
%       \RelDate - The last time this songbook was released.  Set this
%                  date each time a new release/update of the songbook
%                  is generated.
%       \RevDate - The last time a particular song was revised in any
%                  way.  This command will be renewed inside every
%                  song.
%%%
\newcommand{\RelDate}{26~Marzo,~2014}
\newcommand{\RevDate}{\today}

%%%
% C.C.L.I. license number definition; for copyright licensing info.
% One of these macros will be manually inserted into the {CpyRt}
% parameter of the {song} environment.
%
%       \CCLInumber - The actual copyright license number.  Don't
%               insert this command in the {CpyRt} parameter, use one
%               of the others.
%       \CCLIed - Indicates a song falls under our CCLI license.
%       \NotCCLIed - Indicates a song doesn't fall under our CCLI
%               license.  Public Domain songs fall into this category.
%       \PGranted - We have received specific permission from the
%               copyright holder to use this song.
%       \PPending - We are in the process of obtaining permission to
%               use this song.
%%%
\newcommand{\CCLInumber}{Your CCLI Number}
\newcommand{\CCLIed}{{\CpyRtInfoFont (CCLI \CCLInumber)}}
\newcommand{\NotCCLIed}{\relax}
\newcommand{\PGranted}{\relax}
\newcommand{\PPending}{{\CpyRtInfoFont (Permission Pending)}}

% comandos para pintar acordes de guitarra
\newcommand{\A}{\chord{t}{x,n,p2,p2,p2,n}{A}}
\newcommand{\Aseven}{\chord{t}{x,n,p2,n,p2,n}{A7}}
\newcommand{\AsevenMaj}{\chord{t}{x,n,p2,p1,p2,n}{A7+}}
\newcommand{\Am}{\chord{t}{x,n,p2,p2,p1,n}{Am}}
\newcommand{\Amseven}{\chord{t}{x,n,p2,n,p1,n}{Am7}}
\newcommand{\ACs}{\chord{t}{x,p3,p2,p2,p2,n}{A/C\#}}

\newcommand{\As}{\chord{t}{p1,p1,p3,p3,p3,p1}{A\#}}
\newcommand{\Bflat}{\chord{t}{p1,p1,p3,p3,p3,p1}{Bb}}

\newcommand{\B}{\chord{t}{x,bf1p2,f2p4,f3p4,f4p4,f1p2}{B}}
\newcommand{\Bseven}{\chord{t}{x,f1p2,p4,f1p2,p4,f1p2,}{B7}}
\newcommand{\BsevenBasDs}{\chord{t}{x,x,p1,p2,n,p2}{B7/D\#}}
\newcommand{\Bm}{\chord{t}{x,p2,p4,p4,p3,p2}{Bm}}
\newcommand{\Bmseven}{\chord{t}{x,p2,p4,p2,p3,p2}{Bm7}}
\newcommand{\BmseveN}{\chord{t}{x,p2,p4,p3,p3,p2}{Bm7+}}
\newcommand{\BmsevenA}{\chord{t}{x,n,p4,p4,p3,n}{Bm/A}}

\newcommand{\C}{\chord{t}{x,p3,n,p2,p1,n}{C}}
\newcommand{\Cseven}{\chord{t}{x,p3,p3,p2,p1,n}{C7}}
\newcommand{\Cm}{\chord{t}{x,p3,p1,n,p1,p3}{Cm}}
%\newcommand{\Cmseven}{\chord{t}{x,p3,p1,n,p1,p3}{Cm}}
\newcommand{\CE}{\chord{t}{o,p3,n,p2,p1,n}{C/E}}

\newcommand{\Cs}{\chord{4}{n,n,p2,p2,p2,n}{C\#}}
\newcommand{\Csm}{\chord{t}{p4,p4,p6,p6,p5,p4}{C\#m}}
\newcommand{\CssevenLight}{\chord{t}{x,p4,p3,p4,p2,x}{C\#7}}

\newcommand{\D}{\chord{t}{x,x,n,p2,p3,p2}{D}}
\newcommand{\Dseven}{\chord{t}{x,x,n,p2,p1,p2}{D7}}
\newcommand{\DseveN}{\chord{t}{x,x,n,p2,p2,p2}{D7+}}
\newcommand{\Dm}{\chord{t}{x,x,n,p2,p3,p1}{Dm}}
\newcommand{\Dmseven}{\chord{t}{n,n,n,p2,p1,p1}{Dm7}}
\newcommand{\DmsevenG}{\chord{t}{p3,n,n,p2,p1,p1}{Dm7/G}}
\newcommand{\Dsix}{\chord{t}{x,x,n,p2,n,p2}{D6}}
\newcommand{\DmBasB}{\chord{t}{x,p2,p3,p2,p3,x}{Dm/B}}
\newcommand{\DA}{\chord{t}{x,o,n,p2,p3,p2}{D/A}}
\newcommand{\DFs}{\chord{t}{p2,n,n,p2,p3,p2}{D/F\#}}

\newcommand{\Ds}{\chord{t}{n,n,p1,p3,p4,p3}{D\#}}
\newcommand{\Eflat}{\chord{t}{n,n,p1,p3,p4,p3}{E$\flat$}}

\newcommand{\E}{\chord{t}{n,p2,p2,p1,n,n}{E}}
\newcommand{\Eseven}{\chord{t}{n,p2,p2,p1,p3,n}{E7}}
\newcommand{\EseveN}{\chord{t}{n,p2,p2,p4,p3,p4}{E7}}
\newcommand{\Em}{\chord{t}{n,p2,p2,n,n,n}{Em}}
\newcommand{\EsevenFour}{\chord{t}{n,p2,p2,p4,p3,p5}{E7,11}}
\newcommand{\EseveNNine}{\chord{t}{n,f1p2,f1p2,p4,p3,f1p2,}{E79}}

\newcommand{\F}{\chord{t}{p1,p3,p3,p2,p1,p1}{F}}

\newcommand{\Fs}{\chord{t}{p2,p4,p4,p3,p2,p2}{F\#}}
\newcommand{\Fsm}{\chord{t}{f1p2,p4,p4,f1p2,f1p2,f1p2,}{F\#m}}
\newcommand{\FsminLight}{\chord{t}{x,x,f3p4,f1p2,f1p2,f1p2,}{F\#m}}
\newcommand{\FsminBasSeveN}{\chord{t}{x,x,f3p3,f1p2,f1p2,f1p2,}{F\#m/E\#}}
\newcommand{\FsminBasSeven}{\chord{t}{x,x,f2p2,f1p2,f1p2,f1p2,}{F\#m/E}}
\newcommand{\FsminSeven}{\chord{t}{f1p2,p4,p4,f1p2,p5,f1p2,}{F\#7m}}

\newcommand{\G}{\chord{t}{p3,p2,n,n,n,p2}{G}}
\newcommand{\Gseven}{\chord{t}{p3,p2,n,n,n,p1}{G7}}
\newcommand{\GB}{\chord{t}{n,p2,n,n,p3,n}{G/B}}
\newcommand{\GD}{\chord{t}{x,p2,p2,n,p3,p3}{G/D}} %verificar porque parece m�s ien G/E o G/A o Em7algo
\newcommand{\Gnine}{\chord{t}{p3,p2,n,p0,p2,p2}{G9}}
\newcommand{\Gm}{\chord{t}{f1p3,p5,p5,f1p3,f1p3,f1p3,}{Gm}}
\newcommand{\Gmseven}{\chord{t}{f1p3,p5,p3,f1p3,f1p3,f1p3,}{Gm7}}

\newcommand{\Gs}{\chord{3}{x,x,p4,p3,p2,p2,}{G\#}}
\newcommand{\Aflat}{\chord{3}{x,x,p4,p3,p2,p2,}{A$\flat$}}
\newcommand{\Gsmseven}{\chord{t}{f2p4,x,f4p4,f4p4,f4p4,f4p4,}{G\#7}}
\newcommand{\Gssus}{\chord{t}{p4,p6,p6,p6,p4,p4}{Gsus4}}

% comandos para pintar acordes de piano
\newcommand{\APiano}{\keyboardf[Ao][Cso][Eo]}
\newcommand{\AsevenPiano}{\keyboardf[Ao][Cso][Eo][Go]}
\newcommand{\AmPiano}{\keyboardf[Ao][Co][Eo]}
\newcommand{\AmsevenPiano}{\keyboardf[Ao][Co][Eo][Go]}
\newcommand{\ACsPiano}{\keyboard[Cso][Eo][Ao]}

\newcommand{\AsPiano}{\keyboard[Do][Fo][Aso]}
\newcommand{\BflatPiano}{\keyboard[Do][Fo][Aso]}

\newcommand{\BPiano}{\keyboard[Dso][Fso][Bo]}
\newcommand{\BsevenPiano}{\keyboard[Dso][Fso][Bo][Ao]}
\newcommand{\BmPiano}{\keyboard[Do][Fso][Bo]}
\newcommand{\BmsevenPiano}{\keyboard[Do][Fso][Bo][Ao]}

\newcommand{\CPiano}{\keyboard[Co][Eo][Go]}
\newcommand{\CsevenPiano}{\keyboard[Co][Eo][Go][Aso]}
\newcommand{\CmPiano}{\keyboard[Co][Dso][Go]}

\newcommand{\CsPiano}{\keyboard[Cso][Fo][Gso]}
\newcommand{\CsmPiano}{\keyboard[Cso][Eo][Gso]}

\newcommand{\DPiano}{\keyboard[Do][Fso][Ao]}
\newcommand{\DsevenPiano}{\keyboard[Do][Fso][Ao][Co]}
\newcommand{\DmPiano}{\keyboard[Do][Fo][Ao]}
\newcommand{\DmsevenPiano}{\keyboard[Do][Fo][Ao][Co]}
\newcommand{\DAPiano}{\keyboardf[Ao][Do][Fso]}
\newcommand{\DFsPiano}{\keyboardf[Fso][Ao][Do]}

\newcommand{\DsPiano}{\keyboard[Dso][Go][Aso]}
\newcommand{\EflatPiano}{\keyboard[Dso][Go][Aso]}

\newcommand{\EPiano}{\keyboard[Eo][Gso][Bo]}
\newcommand{\EsevenPiano}{\keyboard[Eo][Gso][Bo][Do]}
\newcommand{\EmPiano}{\keyboard[Eo][Go][Bo]}
\newcommand{\EmsevenPiano}{\keyboard[Eo][Go][Bo][Do]}

\newcommand{\FPiano}{\keyboard[Co][Fo][Ao]}

\newcommand{\FsPiano}{\keyboard[Cso][Fso][Aso]}
\newcommand{\FsmPiano}{\keyboard[Cso][Fso][Ao]}
\newcommand{\FsmsevenPiano}{\keyboard[Cso][Fso][Ao][Eo]}

\newcommand{\GPiano}{\keyboard[Do][Go][Bo]}
\newcommand{\GsevenPiano}{\keyboard[Do][Fo][Go][Bo]}
\newcommand{\GBPiano}{\keyboardtwooctaves[Bo][Dt][Gt]}
\newcommand{\GDPiano}{\keyboard[Do][Go][Bo]}
\newcommand{\GmPiano}{\keyboard[Do][Go][Aso]}
\newcommand{\GmsevenPiano}{\keyboard[Do][Go][Aso][Fo]}

\newcommand{\GsPiano}{\keyboard[Dso][Gso][Co]}
\newcommand{\AflatPiano}{\keyboard[Dso][Gso][Co]}

%%%
% Title page information.
%%%
\title{Cuaderno de Himnos Tradicionales y Contempor�neos}
\author{Ruslan L\'opez}
\date{\'Ultima Revisi\'on:  \RevDate}

%%%
% Redefine fonts from SongBook style that I don't like.
%%%
\font\myTinySF=cmss8 at 8pt
\renewcommand{\CpyRtInfoFont}{\tiny\myTinySF}

%%%
% Define fonts to use in the headers and footers of the songbook.
%%%
\newcommand{\LHeadFont}{\normalsize}            % = cmr12  at 12pt
\newcommand{\CHeadFont}{\normalsize\rm}         % = cmr12  at 12pt
\newcommand{\RHeadFont}{\normalsize}            % = cmr12  at 12pt
\newcommand{\LFootFont}{\scriptsize}            % = cmr8   at  8pt
\newcommand{\CFootFont}{\tiny\myTinySF}         % = cmss8  at  8pt
\newcommand{\RFootFont}{\scriptsize}            % = cmr8   at  8pt

%%%
% Turn on and define fancy page heading/footing definition.
%%%
\pagestyle{fancy}

\ifChordBk
% It's a words & chords songbook...
\addtolength{\headwidth}{\marginparsep}
\addtolength{\headwidth}{\marginparwidth}
\renewcommand{\headrulewidth}{0.4pt}
\renewcommand{\footrulewidth}{0.4pt}
\fancyhead[LE,RO]{\LHeadFont\emph{\leftmark\/}\SBContinueMark}
\fancyhead[CE,CO]{\CHeadFont\thepage}
\fancyhead[RE,LO]{\RHeadFont\RelDate}
\else\ifOverhead
% It's an overhead...
\renewcommand{\footrulewidth}{0pt}
\renewcommand{\headrulewidth}{0pt}
\fancyhead[LE,RO]{}
\fancyhead[CE,CO]{}
\fancyhead[RE,LO]{}
\else\ifWordBk
% It's a words only songbook...
\addtolength{\headwidth}{\marginparsep}
\addtolength{\headwidth}{\marginparwidth}
\renewcommand{\headrulewidth}{0.4pt}
\renewcommand{\footrulewidth}{0.4pt}
\fancyhead[LE,RO]{\LHeadFont Estribillero}
\fancyhead[CE,CO]{\CHeadFont\thepage}
\fancyhead[RE,LO]{\RHeadFont\RelDate}
\fi\fi\fi

\fancyfoot[LE,RO]{\LFootFont Transcripciones}
\ifSongEject
\fancyfoot[CE,CO]{\CFootFont \RevDate}
\else
\fancyfoot[CE,CO]{\CFootFont}
\fi
\fancyfoot[RE,LO]{\RFootFont Todo el material son transcripciones personales.}

%%%
% Turn on/off index-file generation.  Uncomment the \makeindex line to
% turn index generation on;  comment it out to turn index generation
% off.
%%%
\makeTitleIndex         %% Title and First Line Index.
\makeTitleContents      %% Table of Contents.
\makeKeyIndex           %% Index of song by key.
\makeArtistIndex        %% Index of song by artist.
\makeindex

%%%%%%%%%%%%%%%%%%%%%%%%%%%%%%%%%%%%%%%%%%%%%%%%%%%%%%%%%%
%%%%%%%%%%%%%%%%%%%%%%%%%%%%%%%%%%%%%%%%%%%%%%%%%%%%%%%%%%
%%                                                      %%
%%           D O C U M E N T   B E G I N S              %%
%%                                                      %%
%%%%%%%%%%%%%%%%%%%%%%%%%%%%%%%%%%%%%%%%%%%%%%%%%%%%%%%%%%
%%%%%%%%%%%%%%%%%%%%%%%%%%%%%%%%%%%%%%%%%%%%%%%%%%%%%%%%%%
\begin{document}

%%%
% Uncomment "\maketitle" statement to make a title page.
%%%
    \maketitle
    %\mainmatter
    \ifWordBk
    \twocolumn
    \fi
%%%
% Songbook begins.
%%%
    \begin{song}{Abba Padre}{D}
    {} %copyright \SBPubDom
    {Marco Barrientos}
    {Romanos 8:15} %pasaje
    {\href{http://open.spotify.com/track/0yj0zBaa7Ckn6ZMQPmCmfF}{Escuchar}} %\NotCCLIed

%        \SBRef{No puedo parar de alabarte}{2006}%fuente \#
        \FLineIdx{Una Vez m�s}

        \begin{SBOptional}
            \Ch{D}{Una} vez m�s

            me a\Ch{Bm}{cer}co a T�

            con \Ch{Em}{li}bertad

            en adora\Ch{A}{ci\'o}n
        \end{SBOptional}

        \begin{SBOptional}
            T\'u e\Ch{D}{res} mi Dios

            tu \Ch{Bm}{hi}jo soy

            mi \Ch{Em}{co}muni\'on contigo

            es una \Ch{A}{dul}ce bendici\'on
        \end{SBOptional}

        \begin{SBChorus}
            // �\Ch{D}{A}bba Pa\Ch{G}{dre}! �\Ch{D}{A}bba Pa\Ch{G}{dre}!

            Es\Ch{D}{tar}\Ch{A/C#}{} conti\Ch{Bm}{go}

            \Ch{D/A}{es} una \Ch{Bm}{dul}ce bendi\Ch{A}{ci\'on}

            �\Ch{D}{A}bba Pa\Ch{G}{dre}! te \Ch{F#m}{a}mo Se\Ch{Bm}{�or}

            \Ch{G}{quie}ro estar en \Ch{D}{co}muni\'on

            \Ch{Em}{quie}ro es\Ch{A}{tar} con\Ch{D}{ti}go. //
        \end{SBChorus}
        \ifChordBk
        \begin{SBOpGroup}
            Acordes: \break

%\keyboardtwooctaves[Do][Fso][Ao]
            \upchord{\DPiano\D}{\qquad Re} Mayor \qquad\qquad
            \upchord{\BmPiano\Bm}{\qquad Si} Menor \hfill \break
            \upchord{\EmPiano\Em}{\qquad Mi} Menor\qquad\qquad
            \upchord{\APiano\A}{\qquad La} Mayor \hfill \break
            \upchord{\GPiano\G}{\qquad Sol} Mayor \qquad\qquad
            \upchord{\FsmPiano\Fsm}{\qquad Fa\#} Menor \hfill \break
            \upchord{\ACsPiano\ACs}{\qquad La} Mayor con bajo C\# \qquad\qquad
            \upchord{\DAPiano\DA}{\qquad Re} Mayor con Bajo La \qquad\qquad
        \end{SBOpGroup}
        \fi

    \end{song}

    \begin{song}{A Cristo solo a Cristo}{G}
    {Proyecto AA} %copyright \SBPubDom
    {Marcos Witt}
    {Hechos 4:12} %pasaje
    {\href{https://open.spotify.com/track/31LsSTJ8P7HZWu6KVpY9Fz}{Escuchar}} %\NotCCLIed

%  \renewcommand{\RevDate}{February~11,~1993}
%  \SBRef{No puedo parar de alabarte}{2006}%fuente \#

        \begin{SBOptional}
            \Ch{G}{A} Cristo \Ch{G7}{so}lo a \Ch{C}{C}risto\Ch{Bm}{~}, \Ch{Am}{yo} exalta\Ch{D}{r\'e}

            \Ch{G}{A} Cristo \Ch{G7}{so}lo a \Ch{C}{C}risto\Ch{Bm}{~}, \Ch{Am}{yo} adora\Ch{D}{r\'e}
        \end{SBOptional}

        \begin{SBOptional}
            \Ch{Am}{Por}que \'El me ha dado vida e\Ch{D}{ter}na

            \Ch{Am}{Por}que \'El me ha dado el po\Ch{D}{der}

            \Ch{Am}{Por}que \'El me ha dado la vic\Ch{D}{to}ria

            \Ch{D}{�l} \Ch{Em}{es} \Ch{D/F\#}{mi} \Ch{G}{Rey}.

            a \Ch{C}{Cris}to he \Ch{Bm}{pro}cla\Ch{Am}{ma}\Ch{D}{do} \Ch{G}{Rey}
        \end{SBOptional}
        \ifChordBk
        \begin{SBOpGroup}
            Acordes:
            \upchord{\GPiano\G}{\qquad Sol} Mayor \qquad\qquad
            \upchord{\GsevenPiano\Gseven}{\qquad Sol} Mayor S\'eptima \hfill \break
            \upchord{\CPiano\C}{\qquad Do} Mayor \qquad\qquad
            \upchord{\BmPiano\Bm}{\qquad Si} Menor \hfill \break
            \upchord{\AmPiano\Am}{\qquad La} Menor \qquad\qquad
            \upchord{\DPiano\D}{\qquad Re} Mayor \hfill \break
            \upchord{\EmPiano\Em}{\qquad Mi} Menor \qquad\qquad
            \upchord{\DFsPiano\DFs}{\qquad Re} Mayor con bajo F\# \hfill \break
        \end{SBOpGroup}
        \fi
    \end{song}

    \begin{song}{Doxolog\'ia}{G}
    {} %copyright \SBPubDom
    {Thomas Ken, Genevan Psalter, 1551, atr. a Louis Bourgeois}
    {Juan 17:22} %pasaje
    {\href{https://open.spotify.com/intl-es/track/24I9oe6qBsY3JHvL6g4yTT}{Escuchar}} %\NotCCLIed

        \FLineIdx{A Dios el Padre Celestial}

        \begin{SBVerse}
            \Ch{G}{A} Dios \Ch{D}{el} \Ch{Em}{Pa}\Ch{Bm}{dre} \Ch{Em}{Ce}\Ch{D}{les}\Ch{G}{tial},

            Al Hijo \Ch{D}{nues}\Ch{Em}{tro} \Ch{C}{Re}\Ch{G}{den}\Ch{D}{tor}.

            \Ch{Em}{Y} al \Ch{D}{E}\Ch{G}{ter}\Ch{D}{nal} \Ch{G}{Con}\Ch{C}{so}\Ch{Am7}{la}\Ch{G}{dor},

            uni\Ch{Em}{dos} \Ch{D}{to}\Ch{Am}{dos} \Ch{G/B}{a}\Ch{D}{la}\Ch{G}{bad}.

            \Ch{C}{A}\Ch{G}{m\'en}.
        \end{SBVerse}
        \ifChordBk
        \begin{SBOpGroup}
            Acordes:
            \upchord{\GPiano\G}{\qquad Sol} Mayor \qquad\qquad
            \upchord{\DPiano\D}{Re} Mayor \hfill \break
            \upchord{\EmPiano\Em}{\qquad Mi} Menor \qquad\qquad
            \upchord{\BmPiano\Bm}{\qquad Si} Menor \hfill \break
            \upchord{\CPiano\C}{\qquad Do} Mayor \qquad\qquad
            \upchord{\AmsevenPiano\Amseven}{\qquad La} Menor S\'eptima \hfill \break
            \upchord{\AmPiano\Am}{\qquad La} Menor \qquad\qquad
            \upchord{\GBPiano\GB}{\qquad Sol} con bajo B \hfill \break
        \end{SBOpGroup}
        \fi
    \end{song}


    \begin{song}{Admirable}{Dm7}
    {} %copyright \SBPubDom
    {Danilo Montero}
    {Apocalipsis 1:18} %pasaje
    {\href{https://open.spotify.com/intl-es/track/3EigVRcP0VQ5MhAUkCfZX8}{Escuchar}} %\NotCCLIed

%  \renewcommand{\RevDate}{February~11,~1993}
%  \SBRef{No puedo parar de alabarte}{2006}%fuente \#
        \FLineIdx{Con poder y autoridad}
        \begin{SBOpGroup}
            \Ch{Dm7}{Con} poder y au\Ch{Bb}{to}ridad \Ch{F}{nues}tro Dios ven\Ch{C}{ci\'o} a la
            muerte

            \Ch{Dm7}{So}bre el trono \Ch{Bb}{ce}lestial \Ch{Am7}{siem}pre reina\Ch{Dm7}{r\'a}.

            \Ch{Bb}{Sen}tado en \Ch{C}{ma}jestad \Ch{Bb}{su}yo es el reino por los
            \Ch{F}{Si}\Ch{C}{glos} \Ch{Bb}{y} por la \Ch{C}{e}ternidad \Ch{Bb}{su} luz de \Ch{Gm7}{glo}ria brilla\Ch{Am7}{r\'a}.
        \end{SBOpGroup}

        \begin{SBChorus}
            Admi\Ch{C}{ra}\Ch{Dm7}{ble}, conse\Ch{C}{je}\Ch{Dm7}{ro} \Ch{Bb}{mi} Dios \Ch{C}{con}sola\Ch{Dm7}{dor},\Ch{Am7}{}

            Eres \Ch{C}{dig}\Ch{Dm7}{no} de a\Ch{C}{la}ban\Ch{Dm7}{za}, \Ch{Bb}{Pr�n}ci\Ch{C}{pe} de \Ch{Dm7}{paz}.
        \end{SBChorus}
        \ifChordBk
        \begin{SBOpGroup}
            Acordes:
            \upchord{\DmsevenPiano\Dmseven}{Re} menor s\'eptima \qquad\qquad
            \upchord{\BflatPiano\Bflat}{\qquad Si bemol} Mayor \hfill \break
            \upchord{\FPiano\F}{\qquad Fa} Mayor \qquad\qquad
            \upchord{\CPiano\C}{\qquad Do} Mayor \hfill \break
            \upchord{\AmsevenPiano\Amseven}{\qquad La} Menor S\'eptima \qquad\qquad
            \upchord{\GmsevenPiano\Gmseven}{\qquad Sol} Menor S\'eptima \hfill \break
        \end{SBOpGroup}
        \fi
    \end{song}

    \begin{song}{A Nuestro Padre Dios}{F}
    {} %copyright \SBPubDom
    {An\'onimo en Thesaurus Musicus 1744}
    {Juan 3:16} %pasaje
    {} %\NotCCLIed

%  \renewcommand{\RevDate}{February~11,~1993}
%  \SBRef{No puedo parar de alabarte}{2006}%fuente \#

        \begin{SBVerse}
            \Ch{F}{A} \Ch{Dm}{nues}\Ch{Gm}{tro} \Ch{C}{Pa}\Ch{Dm7}{dre} \Ch{C}{Dios}

            \Ch{F}{Al}\Ch{Dm}{ce}\Ch{Gm}{mos} \Ch{F}{nues}\Ch{C7}{tra} \Ch{Dm}{voz}

            �\Ch{Gm}{Glo}\Ch{F}{ria} \Ch{C}{a} \Ch{F}{\'El}!

            Tal \Ch{Am}{fu�} su a\Ch{F}{mor} que di\'o

            \Ch{C7}{Al} hijo que \Ch{F}{mu}\Ch{C}{ri\'o,}

            \Ch{F}{En} \Ch{Bb}{quien} con\Ch{F}{f�o} yo;

            �\Ch{Bb}{Glo}\Ch{F}{ria} \Ch{C7}{a} \Ch{F}{\'El}!
        \end{SBVerse}

        \begin{SBVerse}
            \Ch{F}{A} \Ch{Dm}{nues}\Ch{Gm}{tro} \Ch{C}{Sal}\Ch{Dm}{va}\Ch{C}{dor}

            \Ch{F}{De}\Ch{Dm}{mos} \Ch{Gm}{con} \Ch{F}{fe} \Ch{C7}{lo}\Ch{Dm}{or}

            �\Ch{Gm}{Glo}ria \Ch{C}{a} \Ch{F}{\'El}!

            Su \Ch{Am}{san}gre \Ch{F}{de}rram\'o

            \Ch{C7}{Con} ella me \Ch{F}{la}\Ch{C}{v\'o,}

            \Ch{F}{Y} el \Ch{Bb}{cie}lo \Ch{F}{me} abri\'o

            �\Ch{Bb}{Glo}\Ch{F}{ria} \Ch{C7}{a} \Ch{F}{\'El}!
        \end{SBVerse}

        \begin{SBVerse}
            \Ch{F}{Es}\Ch{Dm}{p�}\Ch{Gm}{ri}\Ch{C}{tu} \Ch{Dm7}{de} \Ch{C}{Dios},

            \Ch{F}{E}\Ch{Dm}{le}\Ch{Gm}{vo} a Ti \Ch{C7}{mi} \Ch{Dm}{voz};

            �\Ch{Gm}{Glo}\Ch{F}{ria} \Ch{C}{a} \Ch{F}{Ti}!

            Con \Ch{Am}{ce}les\Ch{F}{tial} fulgor

            \Ch{C7}{Me} muestras el \Ch{F}{a}\Ch{C}{mor}

            \Ch{F}{De} \Ch{Bb}{Cris}to \Ch{F}{mi} Se�or

            �\Ch{Bb}{Glo}\Ch{F}{ria} \Ch{C7}{a} \Ch{F}{Ti}!
        \end{SBVerse}

        \begin{SBVerse}
            \Ch{F}{Con} \Ch{Dm}{go}\Ch{Gm}{zo} y \Ch{Dm}{a}\Ch{C}{mor},

            \Ch{F}{Can}\Ch{Dm}{te}\Ch{Gm}{mos} con \Ch{C7}{fer}\Ch{Dm}{vor}

            \Ch{Gm}{Al} \Ch{F}{Tri}\Ch{C}{no} \Ch{F}{Dios}.

            En \Ch{Am}{la} e\Ch{F}{ter}nidad

            \Ch{C7}{Mo}ra la Tri\Ch{F}{ni}\Ch{C}{dad};

            �\Ch{F}{Por} \Ch{Bb}{siem}pre \Ch{F}{a}labad

            \Ch{Bb}{Al} \Ch{F}{Tri}\Ch{C7}{no} \Ch{F}{Dios}!
        \end{SBVerse}

        \ifChordBk
        \begin{SBOpGroup}
            Acordes:
            \upchord{\FPiano\F}{\qquad Fa} Mayor \qquad\qquad
            \upchord{\DmPiano\Dm}{\qquad Re} menor  \hfill \break
            \upchord{\GmPiano\Gm}{\qquad Sol} Menor \qquad\qquad
            \upchord{\CPiano\C}{\qquad Do} Mayor \hfill \break
            \upchord{\DmsevenPiano\Dmseven}{\qquad Re} menor s\'eptima \qquad\qquad
            \upchord{\CsevenPiano\Cseven}{\qquad Do} Mayor s\'eptima \hfill \break
            \upchord{\AmPiano\Am}{\qquad La} Menor \qquad\qquad
            \upchord{\BflatPiano\Bflat}{\qquad Si bemol} Mayor \hfill \break
        \end{SBOpGroup}
        \fi
    \end{song}


    \begin{song}{Adonai}{Em}
    {} %copyright \SBPubDom
    {Marcos Witt}
    {} %pasaje
    {} %\NotCCLIed

%  \renewcommand{\RevDate}{February~11,~1993}
%  \SBRef{No puedo parar de alabarte}{2006}%fuente \#

        \begin{SBChorus}
            \Ch{Em}{Oh} Ado\Ch{B7}{nai}. Oh Ado\Ch{Em}{nai}.
            \Ch{C}{Dios} \Ch{D}{del} Uni\Ch{Em}{ver}so, Se\Ch{B7}{�or} de la Crea\Ch{Em}{ci�n}.
        \end{SBChorus}

        \begin{SBVerse}
            Los \Ch{D}{cie}los cuentan tu \Ch{G}{glo}ria,
            tus \Ch{D}{hi}jos hoy te a\Ch{G}{do}ran,
            por \Ch{B7}{to}das \Ch{Em}{tus} mara\Ch{Am}{vi}llas, Ado\Ch{B7}{nai}.
        \end{SBVerse}

        \ifChordBk
        \begin{SBOpGroup}
            Acordes:
            \upchord{\EmPiano\Em}{\qquad Mi} Menor \qquad\qquad
            \upchord{\BsevenPiano\Bseven}{\qquad Si} S\'eptima \hfill \break
            \upchord{\CPiano\C}{\qquad Do} Mayor \qquad\qquad
            \upchord{\DPiano\D}{Re} Mayor \hfill \break
            \upchord{\GPiano\G}{\qquad Sol} Mayor \qquad\qquad
            \upchord{\AmPiano\Am}{\qquad La} Menor \hfill \break
        \end{SBOpGroup}
        \fi
    \end{song}

    \begin{song}{Ahora Que Estoy Contigo}{A}
    {\SBPubDom} %copyright \SBPubDom
    {}
    {} %pasaje
    {} %\NotCCLIed

        \begin{SBVerse}
            \Ch{A}{A}hora que estoy con\Ch{D}{ti}go en tus \Ch{A}{bra}zos de amor,

            \Ch{F\#m}{pue}do escuchar de \Ch{D}{T�} y de m� un \Ch{E7}{la}tido

            \Ch{A}{el} tuyo dice ``siempre te \Ch{D}{a}mar�'', el m�o ``\Ch{A}{te} adorar�'',

            que \Ch{F\#m}{dul}ce comu\Ch{D}{ni�n} estar con\Ch{E7}{ti}go.

            \Ch{D}{A}hora que estoy conti\Ch{A}{go}, ante tus \Ch{G}{pies}, oh Padre, yo
            me rin\Ch{D}{do},

            y en \Ch{Dm7}{mi} necesidad nada te \Ch{Em}{pe}dir�, solo te a\Ch{G}{do}rar�
        \end{SBVerse}

        \ifChordBk
        \begin{SBOpGroup}
            Acordes:
            \upchord{\APiano\A}{\qquad La} Mayor \qquad\qquad
            \upchord{\DPiano\D}{\qquad Re} Mayor \hfill \break
            \upchord{\FsmPiano\Fsm}{\qquad Fa\#} Menor \qquad\qquad
            \upchord{\EsevenPiano\Eseven}{\qquad Mi} S\'eptima \hfill \break
            \upchord{\DmsevenPiano\Dmseven}{Re} menor s\'eptima \qquad\qquad
            \upchord{\EmPiano\Em}{\qquad Mi} Menor \hfill \break
            \upchord{\GPiano\G}{\qquad Sol} Mayor \qquad\qquad
        \end{SBOpGroup}
        \fi
    \end{song}

    \begin{song}{Aleluya}{G}
    {} %copyright \SBPubDom
    {Jes�s Adri�n Romero}
    {} %pasaje
    {} %\NotCCLIed

        \SBIntro[N]{\Ch{G}{~} \Ch{Dm7/G}{~} \Ch{C}{~}}

        \FLineIdx{}

        \begin{SBOpGroup}
            En el cielo y en la tierra te alabamos OH Se\Ch{G}{�or}.

            Eres \Ch{G}{dig}no de alabanza y de suprema adoraci�n.

            Te proclamamos Se�or. Te proclamamos Se\Ch{G}{�or}.
        \end{SBOpGroup}

        \begin{SBChorus}
            Ale\Ch{G}{lu}ya, Alelu\Ch{Em7}{ya}, Ale\Ch{C2}{lu}ya \Ch{Am7}{al} Se\Ch{Dsus4}{�or}.
            Ale\Ch{G}{lu}ya, Alelu\Ch{G}{ya}, Aleluya al Se\Ch{G}{�or}.
        \end{SBChorus}

        \begin{SBOpGroup}
            \Ch{G}{En} la cruz por m� te diste para darme liber\Ch{G}{tad}.

            De la tumba resurgiste y en tu trono ahora es\Ch{G}{t�s}.

            OH Jes�s te proclamamos Se�or.te proclamamos Se�or
        \end{SBOpGroup}

        \ifChordBk
        \begin{SBOpGroup}
            Acordes:
            \upchord{\GPiano\G}{\qquad Sol} Mayor \qquad\qquad
            \upchord{\DPiano\D}{Re} Mayor \hfill \break
            \upchord{\EmsevenPiano\Emseven}{\qquad Mi} Menor S\'eptima \qquad\qquad
        \end{SBOpGroup}
        \fi
    \end{song}


    \begin{song}{Al que me ci�e de poder}{E}
    {} %copyright \SBPubDom
    {Jes�s Adri�n Romero}
    {} %pasaje
    {} %\NotCCLIed

        \begin{SBChorus}
            Al que me ci�e de po\Ch{E}{der}

            a aqu�l que mi victoria \Ch{C\#m}{es}

            s�lo a �l alaba\Ch{A}{r�}, \Ch{B7}{~}s�lo a �l exalta\Ch{E}{r�}
        \end{SBChorus}

        \begin{SBOpGroup}
            De t� ser� mi alabanza
            en la congregaci�n
            cantar� y alabar�
            tu nombre Se�or
        \end{SBOpGroup}

        \ifChordBk
        \begin{SBOpGroup}
            Acordes:
            \upchord{\EPiano\E}{\qquad Mi} Mayor \qquad\qquad
            \upchord{\CsmPiano\Csm}{\qquad Do} Sostenido Menor \hfill \break
            \upchord{\APiano\A}{\qquad La} Mayor \qquad\qquad
            \upchord{\BsevenPiano\Bseven}{\qquad Si} S\'eptima \hfill \break
        \end{SBOpGroup}
        \fi
    \end{song}


    \begin{song}{Al Trono Majestuoso}{Eb}
    {} %copyright \SBPubDom
    {Aurelia \& Samuel sebastian Wesley}
    {} %pasaje
    {} %\NotCCLIed

        \begin{SBChorus}
            \Ch{Eb}{Al} trono majestuoso
            \Ch{Eb}{Del} Dios omnipoten\Ch{Eb}{te},
            Hu\Ch{Eb}{mil}des vuestra frente,
            naciones inclinad
            �l es el ser supre\Ch{Eb}{mo},
            Se�or de cuanto existe,
            y nada al fin \Ch{Eb}{res}iste
            Al grande Jeho\Ch{Eb}{v�}
        \end{SBChorus}

        \begin{SBOpGroup}
            Del polvo de la tierra
            formonos complacida
            su mano, y dionos vida
            su aliento creador.
            Y al vernos despu�s ciegos,
            en la maldad sumidos,
            Cual padre a hijos queridos
            Salud nos provey�.
        \end{SBOpGroup}

        \begin{SBOpGroup}
            La gratitud sincera
            nos dictar� canciones
            y en coro dulces sones
            al cielo subir�n
            con los celestes himnos
            arm�nica alianza
            formando, su alabanza
            doquier resonar�.
        \end{SBOpGroup}

        \begin{SBOpGroup}
            Se�or, a tu palabra
            los mundos obedecen,
            y del mortal perecen
            la ciencia y altivez.
            Tu amor y verdad solos
            en nada habr�n menguado,
            despu�s que hayan cesado
            los siglos de correr.
        \end{SBOpGroup}

        \ifChordBk
        \begin{SBOpGroup}
            Acordes:
            \upchord{\EflatPiano\Eflat}{\qquad Mi} bemol Mayor \qquad\qquad
        \end{SBOpGroup}
        \fi
    \end{song}

    \begin{song}{Alza tus ojos}{Am}
    {} %copyright \SBPubDom
    {Marcos Barrientos}
    {} %pasaje
    {} %\NotCCLIed


        \begin{SBOpGroup}
            \Ch{Am}{Al}za tus ojos y \Ch{F}{mi}ra, \Ch{C}{la} cosecha est� \Ch{E}{lis}ta

            el tiempo ha llegado la mies est� madura
            esfu�rzate y s� valiente lev�ntate y predica
            a todas las naciones que Cristo es la Vida
        \end{SBOpGroup}

        \begin{SBChorus}
            Y ser� llena la tierra de su gloria
            se cubrir� como las aguas cubren la mar
        \end{SBChorus}

        \begin{SBOpGroup}
            No, no hay otro nombre,
            dado a los hombres,
            Jesucristo es el Se�or
        \end{SBOpGroup}

        \ifChordBk
        \begin{SBOpGroup}
            Acordes:
            \upchord{\AmPiano\Am}{\qquad La} Menor \qquad\qquad
            \upchord{\FPiano\F}{\qquad Fa} Mayor \hfill \break
            \upchord{\CPiano\C}{\qquad Do} Mayor \qquad\qquad
            \upchord{\EPiano\E}{\qquad Mi} Mayor \hfill \break
            \upchord{\GPiano\G}{\qquad Sol} Mayor \qquad\qquad
            \upchord{\DmPiano\Dm}{\qquad Re} Menor \hfill \break
        \end{SBOpGroup}
        \fi
    \end{song}

    \begin{song}{Aqu� estoy}{A}
    {} %copyright \SBPubDom
    {Jes�s Adri�n Romero}
    {} %pasaje
    {} %\NotCCLIed


        \begin{SBOpGroup}
            A\Ch{A}{qu�} estoy, te ofrezco todo lo que soy
            Aqu� estoy, un sacrificio quiero ser
            Toma mi ser, mi vida entrego a T�
        \end{SBOpGroup}

        \begin{SBOpGroup}

            Porque T� eres mi Dios,
            Eres digno de adoraci�n
            Una ofrenda de amor ser�
            Para T�

        \end{SBOpGroup}

        \ifChordBk
        \begin{SBOpGroup}
            Acordes:
            \upchord{\APiano\A}{\qquad La} Mayor \qquad\qquad
        \end{SBOpGroup}
        \fi
    \end{song}

    \begin{song}{A sus pies}{Am}
    {} %copyright \SBPubDom
    {Jes�s Adri�n Romero}
    {} %pasaje
    {} %\NotCCLIed


        \begin{SBOpGroup}
            Cuando el mundo te inunda de fatalidad
            y te agobia la vida con su mucho af�n
            y se llena tu alma de preocupaci�n
            y se seca la fuente de tu coraz�n
        \end{SBOpGroup}

        \begin{SBChorus}
            Puedes sentarte a sus pies
            y de sus manos beber
            la plenitud que tu alma necesita
            puedes sentarte a sus pies
            y cada d�a tener
            una nueva canci�n y nueva vida
            a sus pies hay paz,
            gracia y bendici�n
            a sus pies tendr�s
            luz y direcci�n.
            la plenitud en �l
            nunca se agotar�
            puedes descansar en su presencia
        \end{SBChorus}

        \begin{SBOpGroup}
            Cuando quieras huir por que no puedes m�s
            por que s�lo te sientes entre los dem�s
            y no hay m�s en tus ojos brillo y emoci�n
            y se cierra tu boca por que no hay canci�n
        \end{SBOpGroup}



        \ifChordBk
        \begin{SBOpGroup}
            Acordes:
            \upchord{\AmPiano\Am}{\qquad La} Menor \qquad\qquad
        \end{SBOpGroup}
        \fi
    \end{song}

    \begin{song}{Bendice Nuestra Ofrenda}{G}
    {} %copyright \SBPubDom
    {D. Tinoco, G Franc}
    {} %pasaje
    {} %\NotCCLIed

        \begin{SBOpGroup}
            \Ch{G}{Ben}di\Ch{D}{ce} \Ch{Em}{Nues}\Ch{Bm}{tra} o\Ch{Em}{fren}\Ch{D}{da} oh \Ch{G}{Dios},

            Que pre\Ch{D}{sen}ta\Ch{Em}{mos} \Ch{C}{con} \Ch{G}{a}\Ch{D}{mor}.

            \Ch{Em}{Es} \Ch{D}{prue}\Ch{G}{ba} \Ch{D}{fiel} \Ch{G}{de} \Ch{C}{gra}\Ch{Am7}{ti}\Ch{G}{tud},

            Por tu \Ch{Em}{bon}\Ch{D}{dad} \Ch{Am}{en} ple\Ch{D}{ni}\Ch{G}{tud}.

            \Ch{C}{A}\Ch{G}{m�n}
        \end{SBOpGroup}

        \ifChordBk
        \begin{SBOpGroup}
            Acordes:
            \upchord{\GPiano\G}{\qquad Sol} Mayor \qquad\qquad
            \upchord{\DPiano\D}{\qquad Re} Mayor \hfill \break
            \upchord{\EmPiano\Em}{\qquad Mi} Menor \qquad\qquad
            \upchord{\BmPiano\Bm}{\qquad Si} Menor \hfill \break
            \upchord{\CPiano\C}{\qquad Do} Mayor \qquad\qquad
            \upchord{\AmsevenPiano\Amseven}{\qquad La} Menor S\'eptima \hfill \break
            \upchord{\AmPiano\Am}{\qquad La} Menor \qquad\qquad
            \upchord{\GBPiano\GB}{\qquad Sol} Mayor con bajo B \hfill \break
        \end{SBOpGroup}
        \fi
    \end{song}

    \begin{song}{Bueno es alabar}{G}
    {} %copyright \SBPubDom
    {Danilo Montero}
    {} %pasaje
    {\href{https://open.spotify.com/intl-es/track/6S90J1ENA9sgzfBDEHNyJ7}{Escuchar}}

        \begin{SBOpGroup}
            \Ch{G}{Bue}no es ala\Ch{C}{bar} �oh Se\Ch{D}{�or}!, tu \Ch{C}{nom}\Ch{D}{bre}

            \Ch{G}{dar}te honra, \Ch{C}{glo}ria y ho\Ch{D}{nor} por \Ch{C}{siem}\Ch{D}{pre},

            \Ch{G}{Bue}no es ala\Ch{C}{bar}te Je\Ch{D}{s�s}

            y go\Ch{Am7}{zar}me en tu po\Ch{D}{der}
        \end{SBOpGroup}

        \begin{SBChorus}
            Porque \Ch{G}{gran}de \Ch{C}{e}res \Ch{D}{T�}

            \Ch{G}{gran}des \Ch{C}{son} tus o\Ch{D}{bras},

            porque \Ch{G}{gran}de \Ch{C}{e}res \Ch{D}{T�}

            grande es tu a\Ch{Em}{mor}, grande es tu \Ch{D}{glo}ria
        \end{SBChorus}
        \ifChordBk
        \begin{SBOpGroup}
            Acordes:
            \upchord{\GPiano\G}{\qquad Sol} Mayor \qquad\qquad
            \upchord{\CPiano\C}{\qquad Do} Mayor  \hfill \break
            \upchord{\DPiano\D}{\qquad Re} Mayor \qquad\qquad
            \upchord{\AmsevenPiano\Amseven}{\qquad La} Menor S\'eptima \hfill \break
            \upchord{\EmPiano\Em}{\qquad Mi} Menor \qquad\qquad
        \end{SBOpGroup}
        \fi
    \end{song}

    \begin{song}{Cada d�a con Cristo}{D}
    {} %copyright \SBPubDom
    {}
    {Isa�as 26:3} %pasaje
    {\href{https://open.spotify.com/intl-es/track/0EtXTYDHwK81dnsloMPNha}{Escuchar}} %\NotCCLIed

        \begin{SBOpGroup}
            \Ch{D}{Ca}da d�a con Cristo me llena de perfecta \Ch{Em}{paz}

            \Ch{A7}{ca}da d�a con \Ch{Em}{Cris}to le amo \Ch{A7}{m�s} y m�s

            \Ch{D}{\'El} me salva y guarda y \Ch{D7}{s�} que pronto volve\Ch{G}{r�}

            y vi\Ch{Gm}{vir} con \Ch{D}{Cris}to m�s \Ch{A}{dul}ce cada d�a se\Ch{D}{r�}
        \end{SBOpGroup}

        \ifChordBk
        \begin{SBOpGroup}
            Acordes:
            \upchord{\DPiano\D}{\qquad Re} Mayor \qquad\qquad
            \upchord{\EmPiano\Em}{\qquad Mi} Menor \hfill \break
            \upchord{\AsevenPiano\Aseven}{\qquad La} Mayor S\'eptima \qquad\qquad
            \upchord{\DsevenPiano\Dseven}{\qquad Re} Mayor S\'eptima \hfill \break
            \upchord{\GPiano\G}{\qquad Sol} Mayor \qquad\qquad
            \upchord{\GmPiano\Gm}{\qquad Sol} Menor \hfill \break
            \upchord{\APiano\A}{\qquad La} Mayor \qquad\qquad
        \end{SBOpGroup}
        \fi
    \end{song}

    \begin{song}{Cada Ma�ana}{D}
    {} %copyright \SBPubDom
    {Jes�s Adri�n Romero}
    {} %pasaje
    {} %\NotCCLIed

        \begin{SBOpGroup}
            \Ch{D}{Ca}da ma�ana al despertar,
            y por la noche al descansar,
            agradezco tus bondades a mi vida
            por todo lo que me permites disfrutar
        \end{SBOpGroup}

        \begin{SBChorus}
            /// Ale-lu-u-ya ///
            // Agradecido estoy por tu bondad.//
            Agradecido estoy por tu bondad.
        \end{SBChorus}

        \ifChordBk
        \begin{SBOpGroup}
            Acordes:
            \upchord{\FsPiano\Fs}{\qquad Fa} Sostenido Mayor \qquad\qquad
            \upchord{\APiano\A}{\qquad La} Mayor \hfill \break
            \upchord{\DPiano\D}{\qquad Re} Mayor \qquad\qquad
            \upchord{\GPiano\G}{\qquad Sol} Mayor \hfill \break
            \upchord{\BmsevenPiano\Bmseven}{\qquad Si} Menor S\'eptima \qquad\qquad
            \upchord{\EmPiano\Em}{\qquad Mi} Menor \hfill \break
            \upchord{\DsevenPiano\Dseven}{\qquad Re} Mayor S\'eptima \qquad\qquad
        \end{SBOpGroup}
        \fi
    \end{song}

    \begin{song}{Cantar� de tu Amor}{F}
    {} %copyright \SBPubDom
    {Danilo Montero}
    {} %pasaje
    {\href{https://open.spotify.com/intl-es/track/7kzqc0u8SNQAU5Wca0WNKo}{Escuchar}} %\NotCCLIed

%        \SBRef{El aire de tu casa}{2005}%fuente \#

        \begin{SBOpGroup}
            \Ch{F}{Por} mucho tiempo bus\Ch{C/E}{qu�}

            \Ch{F}{u}na raz�n de vi\Ch{C/E}{vir}

            \Ch{F}{en} medio de \Ch{G}{mil} pre\Ch{Am7}{gun}tas

            \Ch{F}{tu} amor me respon\Ch{G}{di�}
        \end{SBOpGroup}


        \begin{SBOpGroup}
            \Ch{F}{A}hora veo la \Ch{C/E}{luz}

            \Ch{F}{y} ya no tengo te\Ch{C/E}{mor}

            \Ch{F}{tu} reino \Ch{G}{vi}no a mi \Ch{Am7}{vi}da

            \Ch{F}{y} ahora vivo para \Ch{G}{T�}
        \end{SBOpGroup}

        \begin{SBChorus}
            Canta\Ch{C}{r�} de tu a\Ch{G/B}{mor} rendi\Ch{Am7}{r�} mi cora\Ch{C}{z�n} ante \Ch{F}{t�}

            tu se\Ch{C}{r�s} mi pa\Ch{G/B}{si�n} y mis \Ch{Am7}{pa}sos se guia\Ch{C}{r�n} por tu \Ch{F}{voz}

            mi Je\Ch{Dm}{s�s} y mi \Ch{G}{Rey} de tu \Ch{F}{gran} amor canta\Ch{G}{r�}
        \end{SBChorus}

        \ifChordBk
        \begin{SBOpGroup}
            Acordes:
            \upchord{\FPiano\F}{\qquad Fa} Mayor \qquad\qquad
            \upchord{\CEPiano\CE}{\qquad Do} Mayor Bajo E  \hfill \break
            \upchord{\GPiano\G}{\qquad Sol} Mayor \qquad\qquad
            \upchord{\AmsevenPiano\Amseven}{\qquad La} Menor S\'eptima \hfill \break
            \upchord{\GBPiano\GB}{\qquad Sol} Mayor Bajo B \qquad\qquad\qquad\qquad\qquad
            \upchord{\DmPiano\Dm}{\qquad Re} Menor \hfill \break
            \upchord{\CPiano\C}{\qquad Do} Mayor \qquad\qquad
        \end{SBOpGroup}

        \fi
    \end{song}

    \begin{song}{Cara A Cara}{D}
    {Vidal Music} %copyright \SBPubDom
    {Marcos Vidal}
    {} %pasaje
    {} %\NotCCLIed


        \begin{SBChorus}
            Solamente una palabra,
            solamente una oraci�n,
            cuando llegue a Tu presencia, oh Se�or,
            no me importa en que lugar
            de la mesa me hagas sentar,
            o el color de mi corona, si la llego a ganar

            S�lo d�jame mirarte, cara a cara,
            y perderme como un ni�o en Tu mirada,
            y que pase mucho tiempo,
            y que nadie diga nada,
            por que estoy viendo al Maestro,
            cara a cara
        \end{SBChorus}

        \begin{SBOpGroup}
            Solamente una palabra,
            si es que a�n me queda voz,
            y si logro articularla en Tu presencia,
            no te quiero hacer preguntas, s�lo una petici�n,
            y si puede ser a solas, mucho mejor
        \end{SBOpGroup}

        \ifChordBk
        \begin{SBOpGroup}
            Acordes:
            \upchord{\DPiano\D}{\qquad Re} Mayor \qquad\qquad
        \end{SBOpGroup}
        \fi
    \end{song}


    \begin{song}{Cerca de T�}{G}
    {Vidal Music} %copyright \SBPubDom
    {Marcos Vidal}
    {} %pasaje
    {} %\NotCCLIed

        \begin{SBOpGroup}
            Si decidiera negar mi fe
            y no con ar nunca m�s en �l
            no tengo a donde ir, no tengo a donde ir
        \end{SBOpGroup}


        \begin{SBOpGroup}
            Si despreciare en mi coraz�n
            la santa gracia que me salv�
            no tengo a donde ir, no tengo a donde ir
        \end{SBOpGroup}

        \begin{SBOpGroup}
            Convencido estoy que sin tu amor se acabar�an mis
            fuerzas
            y sin T� mi coraz�n sediento se muere, se seca
        \end{SBOpGroup}

        \begin{SBOpGroup}
            Cerca de T�, yo quiero estar
            de tu presencia no me quiero alejar
        \end{SBOpGroup}

        \ifChordBk
        \begin{SBOpGroup}
            Acordes:
            \upchord{\DPiano\D}{\qquad Re} Mayor \qquad\qquad
        \end{SBOpGroup}
        \fi
    \end{song}

    \begin{song}{�C�mo Podr� Estar Triste?}{C}
    {} %copyright \SBPubDom
    {}
    {} %pasaje
    {} %\NotCCLIed

        \begin{SBOpGroup}
            C�mo podr� estar triste,
            c�mo entre sombras ir,
            c�mo sentirme solo
            y en el dolor vivir,
            siCristo es mi consuelo,
            mi amigo siempre el,
            // si aun las aves tienen
            Seguro asilo en �l //
        \end{SBOpGroup}

        \begin{SBChorus}
            Feliz cantando alegre
            yo vivo siempre aqu�;
            si El cuida de las aves
            cuidar� tambi�n de m�.
        \end{SBChorus}

        \begin{SBOpGroup}
            Nunca te desalientes,
            oigo al Se�or decir,
            y en su Palabra ado
            hago al dolor huir.
            A Cristo paso a paso
            yo sigo sin cesar,
            //y todas sus bondades
            me da sin limitar
        \end{SBOpGroup}

        \begin{SBOpGroup}
            Siempre que soy tentado
            o que en la sombra estoy,
            m�s cerca de �l camino
            y protegido voy.
            Si en mi la fe desmaya
            y caigo en la ansiedad
            //tan s�lo El me levanta,
            me da seguridad//
        \end{SBOpGroup}

        \ifChordBk
        \begin{SBOpGroup}
            Acordes:
            \upchord{\DPiano\D}{\qquad Re} Mayor \qquad\qquad
        \end{SBOpGroup}
        \fi
    \end{song}

    \begin{song}{Con C�nticos Se�or}{C}
    {} %copyright \SBPubDom
    {}
    {} %pasaje
    {} %\NotCCLIed

        \begin{SBOpGroup}

            Con c�nticos Se�or mi coraz�n y voz
            te adoran con fervor oh trino Santo Dios
            Tu mano paternal marc� mi senda aqu�
            mis pasos, cada cual, velados son por T�
            Innumerables son tus bienes y sin par
            que por tu compasi�n recibo sin cesar
            T� eres oh Se�or mi sumo todo bien
            mil lenguas tu amor cantando siempre est�n

        \end{SBOpGroup}

        \begin{SBChorus}

            En tu Mansi�n yo te ver�
            de t� perd�n fel�z tendr�


            En tu Mansi�n yo te ver�
            y galard�n fel�z tendr�

        \end{SBChorus}

        \ifChordBk
        \begin{SBOpGroup}
            Acordes:
            \upchord{\DPiano\D}{\qquad Re} Mayor \qquad\qquad
        \end{SBOpGroup}
        \fi
    \end{song}

    \begin{song}{Con Manos Vac�as}{E}
    {} %copyright \SBPubDom
    {Jes�s Adri�n Romero}
    {} %pasaje
    {} %\NotCCLIed

        \begin{SBOpGroup}
            \Ch{E}{Con} manos vac�as \Ch{A}{ven}go a T�

            \Ch{B}{no} tengo nada que dar\Ch{C\#m}{te}

            \Ch{F\#m}{no} hay nada de valor en \Ch{C\#m}{m�}

            \Ch{A}{no} puedo im\Ch{F\#m}{pre}sio\Ch{B}{nar}te
        \end{SBOpGroup}

        \begin{SBOpGroup}
            \Ch{E}{Te} puedo entregar mi \Ch{A}{co}raz�n,

            \Ch{B}{pe}ro est� quebran\Ch{C\#m}{ta}do

            \Ch{F\#m}{re}c�belo mi buen Pas\Ch{C\#m}{tor},

            \Ch{A}{Tu} puedes \Ch{F\#m}{res}tau\Ch{B}{rar}lo
        \end{SBOpGroup}


        \begin{SBChorus}
            \Ch{A}{Pon}go mi \Ch{B}{vi}da a tu servicio Se\Ch{C\#m}{�or}

            \Ch{A}{no} ser� \Ch{B}{mu}cho, pero la entrego \Ch{C\#m}{hoy}

            \Ch{A}{y} si mis \Ch{B}{ma}nos hoy vac�as es\Ch{C\#m}{t�n},

            \Ch{F\#m7}{pue}des lle\Ch{C\#m/G\#}{nar}las con tu \Ch{C\#m}{gran} po\Ch{B}{der} y a\Ch{A}{mor}.

            \Ch{B}{U}sa mis manos Se\Ch{C\#m}{�or}
        \end{SBChorus}

        \ifChordBk
        \begin{SBOpGroup}
            Acordes:
            \upchord{\EPiano\E}{\qquad Mi} Mayor \qquad\qquad
            \upchord{\APiano\A}{\qquad La} Mayor \hfill \break
            \upchord{\BPiano\B}{\qquad Si} Mayor \qquad\qquad
            \upchord{\CsmPiano\Csm}{\qquad Do sostenido} Menor \hfill \break
            \upchord{\FsmPiano\Fsm}{\qquad Fa sostenido} Menor \qquad\qquad
            \upchord{\GssusPiano\Gssus}{\qquad Sol sostenido} suspendida cuarta \hfill \break
            \upchord{\CsmPiano\Csm}{\qquad Do sostenido} Menor \qquad\qquad
            \upchord{\FsmsevenPiano\Fsmseven}{\qquad Fa sostenido} S\'eptima Menor \qquad\qquad
        \end{SBOpGroup}
        \fi

        \begin{SBExtraKeys}{
            \STitle{Con Manos Vac�as}{F}

            \begin{SBOpGroup}
                \Ch{F}{Con} manos vac�as \Ch{Bb}{ven}go a T� \Ch{C}{no} tengo nada que darte

                no hay nada de valor en m� \Ch{Bb}{no} puedo impresio\Ch{C}{nar}te
            \end{SBOpGroup}

            \begin{SBOpGroup}
                \Ch{F}{Te} puedo entregar mi \Ch{Bb}{co}raz�n,

                \Ch{C}{pe}ro est� quebrantado

                rec�belo mi buen Pastor,

                \Ch{Bb}{Tu} puedes restau\Ch{C}{rar}lo
            \end{SBOpGroup}

            \begin{SBChorus}
                \Ch{Bb}{Pon}go mi \Ch{C}{vi}da a tu servicio Se�or

                \Ch{Bb}{no} ser� \Ch{C}{mu}cho, pero la entrego hoy

                \Ch{Bb}{y} si mis \Ch{C}{ma}nos hoy vac�as est�n,

                puedes llenarlas con tu gran po\Ch{C}{der} y a\Ch{Bb}{mor}.

                \Ch{C}{U}sa mis manos Se�or
            \end{SBChorus}

            \ifChordBk
            \begin{SBOpGroup}
                Acordes:
                \upchord{\FPiano\F}{\qquad Fa} Mayor \qquad\qquad
                \upchord{\BflatPiano\Bflat}{\qquad Si bemol} Mayor \hfill \break
                \upchord{\CPiano\C}{\qquad Do} Mayor \qquad\qquad
            \end{SBOpGroup}
            \fi
        }\end{SBExtraKeys}
    \end{song}

    \begin{song}{Con mi Dios}{Em}
    {} %copyright \SBPubDom
    {}
    {} %pasaje
    {} %\NotCCLIed

        \begin{SBOpGroup}
            Con mi Dios yo saltar� los muros
            Con mi Dios Ej�rcitos derribar
        \end{SBOpGroup}

        \begin{SBOpGroup}
            �l adiestra mis manos para la batalla,
            puedo tomar con mis manos el arco de bronce
        \end{SBOpGroup}

        \begin{SBOpGroup}
            �l es escudo, la roca m�a,
            �l es la fuerza de mi salvaci�n
            Mi alto refugio, mi fortaleza
            �l es mi libertador
        \end{SBOpGroup}


        \ifChordBk
        \begin{SBOpGroup}
            Acordes:
            \upchord{\EmPiano\Em}{\qquad Mi} Menor \qquad\qquad
        \end{SBOpGroup}
        \fi
    \end{song}

    \begin{song}{Con Mis Labios}{D}
    {} %copyright \SBPubDom
    {}
    {} %pasaje
    {} %\NotCCLIed

        \begin{SBOpGroup}
            Con mis labios y mi vida
            te alabo Se�or, te alabo Se�or,
            con mis labios y mi vida te alabo bendito Se�or,
        \end{SBOpGroup}

        \begin{SBChorus}
            Porque T� has sido precioso para m�, precioso
            para m�,precioso para m�,
            Porque T� has sido precioso para m�, te alabo
            bendito Se�or
        \end{SBChorus}

        \ifChordBk
        \begin{SBOpGroup}
            Acordes:
            \upchord{\DPiano\D}{\qquad Re} Mayor \qquad\qquad
        \end{SBOpGroup}
        \fi
    \end{song}

    \begin{song}{Con Mis Manos Levantadas}{G}
    {} %copyright \SBPubDom
    {}
    {} %pasaje
    {} %\NotCCLIed

        \begin{SBOpGroup}
            Con mis manos levantadas hacia el cielo me
            Presento ante Ti hoy Se�or para recibir de Ti
            La fuerza y el poder para vivir junto a Ti.
        \end{SBOpGroup}

        \begin{SBChorus}
            Llenas hoy mi coraz�n con Tu presencia
            Llenas de alegr�a y paz todo mi ser
            De cualquier necesidad T� me responder�s
            Porque me amas, me amas
        \end{SBChorus}

        \ifChordBk
        \begin{SBOpGroup}
            Acordes:
            \upchord{\GPiano\G}{\qquad Sol} Mayor \qquad\qquad
        \end{SBOpGroup}
        \fi
    \end{song}

    \begin{song}{Cristo me Ayuda por �l a Vivir}{F}
    {} %copyright \SBPubDom
    {}
    {} %pasaje
    {} %\NotCCLIed

        \begin{SBOpGroup}
            Cristo me ayuda por �l a vivir
            Cristo me ayuda por �l a morir;
            Hasta que llegue su gloria a ver,
            Cada momento le entrego mi ser
        \end{SBOpGroup}

        \begin{SBChorus}
            Cada momento la vida me da,
            Cada momento conmigo �l est�;
            Hasta que llegue su gloria a ver,
            Cada momento le entrego ni s�r.
        \end{SBChorus}

        \begin{SBOpGroup}
            Siento pesares, muy cerca �l est�,
            Siento dolores, a livio me da;
            Tengo aflicciones, me muestra su amor;
            Cada momento me cuidas Se�or.
        \end{SBOpGroup}

        \begin{SBOpGroup}
            Tengo amarguras, o tengo temor
            Tengo tristezas, me inspira valor;
            Tengo conflictos o penas aqu�
            Cada momento te acuerdas de m�
        \end{SBOpGroup}

        \begin{SBOpGroup}
            Tengo aquezas, o d�bil estoy,
            Cristo de dice Tu amparo yo soy ;
            Cada momento en tinieblas o en luz,
            Siempre conmigo est� mi Jes�s.
        \end{SBOpGroup}

        \ifChordBk
        \begin{SBOpGroup}
            Acordes:
            \upchord{\FPiano\F}{\qquad Fa} Mayor \qquad\qquad
        \end{SBOpGroup}
        \fi
    \end{song}

    \begin{song}{Cristo t� me has amado}{G}
    {} %copyright \SBPubDom
    {}
    {} %pasaje
    {Ingrid Rosario} %\NotCCLIed

        \begin{SBOpGroup}
            Cristo t� me has amado
            Cristo nunca de t� me apartar�
            Del pecado T� me has rescatado
            Mis pies pusiste en la roca y yo s� que
        \end{SBOpGroup}

        \begin{SBChorus}
            Te amo, por siempre
            Aunque el mundo alrededor pueda cambiar
            Tu eres mi Salvador
            Yo te adorar� por la eternidad.
        \end{SBChorus}

        \ifChordBk
        \begin{SBOpGroup}
            Acordes:
            \upchord{\GPiano\G}{\qquad Sol} Mayor \qquad\qquad
        \end{SBOpGroup}
        \fi
    \end{song}

    \begin{song}{Cu�n bello es el Se�or}{D}
    {} %copyright \SBPubDom
    {}
    {} %pasaje
    {Ingrid Rosario} %\NotCCLIed

        \begin{SBOpGroup}
            //Cu�n bello es el Se�or
            Cu�n hermoso es el Se�or
            //Cu�n bello es el Se�or
            Hoy le quiero adorar//
        \end{SBOpGroup}

        \begin{SBChorus}
            La belleza de mi Se�or.
            Nunca se agotar�
            La hermosura de mi Se�or
            Siempre resplandecer�
        \end{SBChorus}

        \ifChordBk
        \begin{SBOpGroup}
            Acordes:
            \upchord{\DPiano\D}{\qquad Re} Mayor \qquad\qquad
        \end{SBOpGroup}
        \fi
    \end{song}

    \begin{song}{Dad a Dios Inmortal alabanza}{D}
    {} %copyright \SBPubDom
    {}
    {} %pasaje
    {} %\NotCCLIed

        \begin{SBOpGroup}
            Dad a \Ch{D}{Dios} inmortal ala\Ch{G/D}{ban}\Ch{D}{za};
            Su merced, su verdad \Ch{G}{nos} i\Ch{A}{nun}da;
            \Ch{D}{Es} su gracia en prodigios fe\Ch{G/D}{cun}\Ch{D}{da},
            Sus mer\Ch{C\#}{ce}des, humildes can\Ch{F\#m}{tad}
            �\Ch{D}{Al} Se\Ch{A7}{�or} de se�ores dad \Ch{D}{glo}ria,
            Rey de reyes, poder \Ch{A}{sin} se\Ch{D}{gun}do!
            Mori\Ch{A}{r�n} los se�ores del \Ch{D}{mun}do,
            mas su reino no a\Ch{A7}{ca}ba Ja\Ch{D}{m�s}.
        \end{SBOpGroup}

        \begin{SBOpGroup}
            Las naciones vi� en vicios sumidas
            Y sinti� compasi�n en su seno;
            De prodigios de gracia est� lleno;
            Sus mercedes, humildes cantad
            A su pueblo llev� por la mano
            A la tierra por �l prometida
            Por los siglos sin n le da vida
            y el pecado y la muerte caer�n.
        \end{SBOpGroup}

        \begin{SBOpGroup}
            A su Hijo envi� por salvarnos
            Del pecado y la muerte inherente:
            De prodigios de gracia es torrente,
            Sus mercedes, humildes cantad
            Por el mundo su mano nos lleva
            Y al celeste descanso nos gu�a;
            Su bondad vivir� eterno d�a,
            Cuando el mundo no exista ya m�s
        \end{SBOpGroup}

        \ifChordBk
        \begin{SBOpGroup}
            Acordes:
            \upchord{\DPiano\D}{\qquad Re} Mayor \qquad\qquad
            \upchord{\GDPiano\GD}{\qquad Sol} Mayor Bajo D \hfill \break
            \upchord{\APiano\A}{\qquad La} Mayor \qquad\qquad
            \upchord{\CsPiano\Cs}{\qquad Do sostenido} Mayor \hfill \break
            \upchord{\FsmPiano\Fsm}{\qquad Fa sostenido} Menor \qquad\qquad
            \upchord{\AsevenPiano\Aseven}{\qquad La} Mayor S\'eptima \hfill \break
        \end{SBOpGroup}
        \fi
    \end{song}

    \begin{song}{Damos Honor a Ti}{E}
    {} %copyright \SBPubDom
    {}
    {} %pasaje
    {} %\NotCCLIed

        \begin{SBOpGroup}
            Damos honor a Ti, damos honor a Ti.
            Creador de vida eres T�.
            Damos honor a Ti, damos honor a Ti.
            Porque no hay otro Dios como T�.
            Rey de reyes, Admirable.
            Eres el Principio y Fin.
            Soberano y sublime.
            Eres nuestro Salvador.

        \end{SBOpGroup}

        \ifChordBk
        \begin{SBOpGroup}
            Acordes:
            \upchord{\EPiano\E}{\qquad Mi} Mayor \qquad\qquad
        \end{SBOpGroup}
        \fi
    \end{song}

    \begin{song}{DeGloriaEnGloria}{D}
    {} %copyright \SBPubDom
    {}
    {} %pasaje
    {Marcos Witt} %\NotCCLIed

        \begin{SBOpGroup}
            De \Ch{D}{glo}ria en \Ch{A/C\#}{glo}ria te \Ch{Bm}{veo} \Ch{Am7}{~}
            cuanto m�s te conozco
            Quiero saber m�s de T�
        \end{SBOpGroup}

        \begin{SBOpGroup}
            Mi \Ch{D}{Dios} cuan buen alfarero
            Quebr�ntame transf�rmame mold�ame a tu imagen,
            Se�or
        \end{SBOpGroup}

        \begin{SBOpGroup}
            Quiero ser mas como Tu
            ver la vida como Tu
            Saturarme de tu esp�ritu
        \end{SBOpGroup}

        \SBEnd{y reflejar al mundo tu a\Ch{D}{mor}}

        \ifChordBk
        \begin{SBOpGroup}
            Acordes:
            \upchord{\DPiano\D}{\qquad Re} Mayor \qquad\qquad
            \upchord{\ACsPiano\ACs}{\qquad La} Mayor con bajo C\# \hfill \break
            \upchord{\BmPiano\Bm}{\qquad Si} Menor \qquad\qquad
            \upchord{\AmsevenPiano\Amseven}{\qquad La} Menor S\'eptima \hfill \break
        \end{SBOpGroup}
        \fi
    \end{song}

    \begin{song}{Delante de tu Trono}{G}
    {} %copyright \SBPubDom
    {}
    {} %pasaje
    {Marco Barrientos}

        \begin{SBOpGroup}
            // De\Ch{G}{lan}te de tu trono
            Se\Ch{Em}{�or} yo quiero estar
            \Ch{C}{pa}ra contemplar
            tu hermo\Ch{D}{su}ra y santidad //
        \end{SBOpGroup}

        \begin{SBOpGroup}
            Y de\Ch{G}{ci}\Ch{D}{rte}: Te \Ch{Em}{a}mo
            Y decir\Ch{G}{te}: Te a\Ch{Am}{do}\Ch{D}{ro}
            Y de\Ch{G}{cir}\Ch{D}{te}: Te \Ch{Em}{a}mo
            Y que eres todo \Ch{D}{pa}ra \Ch{G}{m�}.
        \end{SBOpGroup}

        \ifChordBk
        \begin{SBOpGroup}
            Acordes:
            \upchord{\GPiano\G}{\qquad Sol} Mayor \qquad\qquad
            \upchord{\EmPiano\Em}{\qquad Mi} Menor \hfill \break
            \upchord{\CPiano\C}{\qquad Do} Mayor \qquad\qquad
            \upchord{\DPiano\D}{\qquad Re} Mayor \hfill \break
            \upchord{\AmPiano\Am}{\qquad La} Menor \qquad\qquad
        \end{SBOpGroup}
        \fi
    \end{song}

    \begin{song}{Demos gracias al Se�or}{C}
    {} %copyright \SBPubDom
    {}
    {} %pasaje
    {}

        \begin{SBOpGroup}
            // Demos gracias al Se�or, demos gracias,
            demos gracias por su amor //
        \end{SBOpGroup}

        \begin{SBOpGroup}
            Por la ma�ana, las aves cantan
            sus alabanza a Dios el Creador,
            tambi�n nosotros a �l cantemos
            y alabemos a Cristo el Redentor
        \end{SBOpGroup}

        \ifChordBk
        \begin{SBOpGroup}
            Acordes:
            \upchord{\CPiano\C}{\qquad Do} Mayor \qquad\qquad
        \end{SBOpGroup}
        \fi
    \end{song}

    \begin{song}{Me dice que me ama}{G}
    {} %copyright \SBPubDom
    {Jes�s Adri�n Romero}
    {} %pasaje
    {\NotCCLIed} %\NotCCLIed

%        \SBRef{El aire de tu casa}{2005}%fuente \#

        \SBIntro[N]{\Ch{G}{~} \Ch{C}{~} \Ch{Am}{~} \Ch{D}{~}}

        \begin{SBOpGroup}
            Me dice que me ama cuando escucho llover,
            G C
            me dice que ama con un atardecer,
            G C Em D
            lo dice sin palabras, con las olas del mar,
            Em C D
            lo dice en la ma�ana con mi respirar.
        \end{SBOpGroup}

        \begin{SBChorus}
            G C G C G
            //Me dice que me ama y que conmigo quiere estar,
            C Am7 D
            me dice que me busca cuando salgo yo a pasear,
            Am7 C Em
            que ha hecho lo que existe para llamar mi atenci�n,
            Am D G C D
            que quiere conquistarme y alegrar mi coraz�n.
        \end{SBChorus}

        \begin{SBOpGroup}
            G C
            Me dice que me ama cuando veo la cruz,
            G C
            sus manos extendidas as� tan grande es su amor,
            G C Em D
            lo dicen las heridas de sus manos y pies,
            Em C D
            me dice que me ama una y otra vez.
        \end{SBOpGroup}
        \ifChordBk
        \begin{SBOpGroup}
            Acordes:
            \upchord{\GPiano\G}{\qquad Sol} Mayor \qquad\qquad
        \end{SBOpGroup}

        \fi
    \end{song}

    \begin{song}{Par\'abola}{G}
    {} %copyright \SBPubDom
    { Marcos Vidal }
    {Lucas 10:30} %pasaje
    {\href{https://youtu.be/cxkfqxpuICM}{Escuchar}} %\NotCCLIed

%  \renewcommand{\RevDate}{February~11,~1993}
%  \SBRef{No puedo parar de alabarte}{2006}%fuente \#

        \begin{SBOpGroup}
            \Ch{G# maj7}{////}\Ch{Gm}{in}tro////\Ch{G7}{}
        \end{SBOpGroup}

        \begin{SBVerse}

            regre\Ch{Ddis}{sa}ba a casa un poco mas tem\Ch{G7}{pra}no de lo nor\Ch{Cm}{mal}

            cuando \Ch{Ddis}{vi\'o} que sobre �l ve\Ch{G7}{ni}an \Ch{Cm}{tres}

            y na\Ch{G# maj7}{va}ja en mano le \Ch{Gm}{a}tacaron sin contempla\Ch{Fm}{ci\'on}

            le de\Ch{Bb}{ja}ron incons\Ch{Bb7/B}{cien}te bajo el \Ch{Cm}{sol}
        \end{SBVerse}
        \begin{SBOpGroup}
            \Ch{G# maj7}{}\Ch{Gm}{}\Ch{G7}{}
        \end{SBOpGroup}
        \begin{SBVerse}
            y ca  \Ch{Ddis}{mi}no de la i\Ch{G7}{gle}sia iba el pas\Ch{Cm}{tor} poco despu�s

            la reu\Ch{Ddis}{nion} ya estaba a \Ch{G7}{pun}to de empe\Ch{Cm}{zar}

            iba \Ch{G#maj7}{tar}de y discu\Ch{Gm}{tien}do en el ca\Ch{Fm}{mi}no con su mujer

            inten\Ch{Bb}{tan}do no per\Ch{Bb7/B}{der} su autori\Ch{Cm}{dad}


            \Ch{G#maj7}{ay} si el maestro nos vol\Ch{Bbmaj7}{vie}ra a contar

            alguna his\Ch{Gm}{to}ria que nos hiciera \Ch{Cm}{re}capacitar

            piensa\Ch{Gm}{lo} piensa\Ch{Cm}{lo}

        \end{SBVerse}
        \begin{SBOpGroup}
            \Ch{G# maj7}{}\Ch{Gm}{}\Ch{G7}{}
        \end{SBOpGroup}

        \begin{SBVerse}

            tres mi\Ch{Ddis}{nu}tos mas y el \Ch{G7}{l�}der de ala\Ch{Cm}{ban}za aparecio

            los te\Ch{Ddis}{cla}dos siete \Ch{G7}{ca}bles y un a\Ch{Cm}{tril}

            y aunque \Ch{G#maj7}{si} le pare\Ch{Gm}{ci\'o} ver algo \Ch{Fm}{ro}jo en el arcel

            prefi\Ch{Bb}{ri\'o} pasar de \Ch{Bb7/B}{lar}go y de per\Ch{Cm}{fil}
        \end{SBVerse}
        \begin{SBVerse}

            un gi\Ch{D#dis}{ta}no despei\Ch{G#7}{na}do que pa\Ch{Cm#}{sa}ba por ahi

            no sa\Ch{D#dis}{b�a} ni \Ch{G#7}{leer} ni escri\Ch{Cm#}{bir}

            pero al \Ch{Amaj7}{ver} el pano\Ch{Gm#7}{ra}ma le do\Ch{Fm#7}{li\'o} en el coraz\'on

            y acer\Ch{B}{c�n}dose hasta el \Ch{B7/C}{hom}bre le ayud\Ch{Cm#}{\'o}

            \Ch{G#maj7}{ay} si el maestro nos vol\Ch{Bbmaj7}{vie}ra a contar

            alguna his\Ch{Gm}{to}ria que nos hiciera \Ch{Cm}{re}capacitar

            piensa\Ch{Gm}{lo} piensa\Ch{Cm}{lo}

        \end{SBVerse}


        \ifChordBk
        \begin{SBOpGroup}
            Acordes:

            \upchord{\keyboard[Do][Fo][Go][Bo]\Gs}{Sol} Mayor S\'eptima
        \end{SBOpGroup}
        \fi
    \end{song}

    \begin{song}{Grita, canta, danza}{Cm}
    {\SBPubDom} %copyright \SBPubDom
    {}
    {} %pasaje
    {} %\NotCCLIed

%  \renewcommand{\RevDate}{February~11,~1993}
%  \SBRef{No puedo parar de alabarte}{2006}%fuente \#

        \begin{SBChorus}
            \Ch{Cm}{Gri}ta, canta, danza alegremente en su pre\Ch{Ab}{sen}cia

            Gira, salta dando vueltas para \Ch{Bb}{Cris}to

            �l vive y \Ch{G7}{vi}ve para siempre es el \Ch{Cm}{Rey}\Ch{G7}{}
        \end{SBChorus}

        \begin{SBOpGroup}
            Te alaba\Ch{Cm}{r�}, te exaltar� y te agra\Ch{Bb}{de}cer�

            Tu \Ch{Eb}{gran}de amor Jes�s

            Cam\Ch{Ab}{bias}te mi lamento en ala\Ch{Cm}{ban}za

            Sa\Ch{Bb}{nas}te mi \Ch{G7}{he}rido cora\Ch{Cm}{z�n} \Ch{G7}{}
        \end{SBOpGroup}

        \ifChordBk
        \begin{SBOpGroup}
            Acordes:
            \upchord{\CmPiano\Cm}{\qquad Do} Menor \qquad\qquad
            \upchord{\AflatPiano\Aflat}{La bemol} Mayor \hfill \break
            \upchord{\BflatPiano\Bflat}{\qquad Si bemol} Mayor \hfill \break
        \end{SBOpGroup}
        \fi
    \end{song}

    \begin{song}{As� como David danzaba}{Am}
    {\SBPubDom} %copyright \SBPubDom
    {}
    {} %pasaje
    {} %\NotCCLIed

%  \renewcommand{\RevDate}{February~11,~1993}
%  \SBRef{No puedo parar de alabarte}{2006}%fuente \#

        \begin{SBOpGroup}
            \Ch{Am}{Cuan}do el Se�or hiciere volver la cautivi\Ch{G}{dad}

            seremos \Ch{Dm}{co}mo los que sue\Ch{E}{�an}
        \end{SBOpGroup}

        \begin{SBOpGroup}

            \Ch{Am}{Mi} boca llenar� de risa, \Ch{G}{mis} labios de alabanza,

            \Ch{Dm}{En}tonces dir�n las naciones:

            \Ch{E}{Gran}des cosas ha hecho el Se�or
        \end{SBOpGroup}

        \begin{SBOpGroup}
            Me goza\Ch{Am}{r�}, me gozar�, me gozar�,

            me gozar� en Jeho\Ch{G}{v�}.  [�G�zate!]

            Pues ha lle\Ch{Dm}{va}do todo dolor, me ha hecho \Ch{E}{li}bre
        \end{SBOpGroup}

        \begin{SBChorus}
            \Ch{Am}{A}s� como David cantaba, \Ch{G}{a}s� como David danzaba,

            \Ch{Dm}{a}s� como David flu�a en su pre\Ch{E}{sen}cia
        \end{SBChorus}
        \ifChordBk
        \begin{SBOpGroup}
            Acordes:
            \upchord{\AmPiano\Am}{\qquad La} Menor \qquad\qquad
            \upchord{\GPiano\G}{\qquad Sol} Mayor \hfill \break
            \upchord{\DmPiano\Dm}{\qquad Re} Menor \qquad\qquad
        \end{SBOpGroup}
        \fi
    \end{song}

    %\begin{document}

\ifguitarra

\lhead{\LHeadFont Acodes~para~Guitarra}
\chead{\CHeadFont ({\rm\thepage})}
\rhead{\RHeadFont\RelDate}
{\parindent 8pt
        {\myTitleFont --- Acodes para Guitarra ---}}\par
\vskip 20pt
\textbf{Acodes Mayores}

%\small{El s\'imbolo \# significa sostenido y {\flat}~significa~bemol}
\small
\upchord{\A}{La Mayor} \upchord{\B}{Si Mayor} \upchord{\C}{Do Mayor} \upchord{\D}{Re Mayor} \upchord{\E}{Mi Mayor} \upchord{\F}{Fa Mayor} \upchord{\G}{Sol Mayor}

\upchord{\As}{A\#/$B\flat$ Mayor} \upchord{\Cs}{C\#/$D\flat$ Mayor} \upchord{\Ds}{D\#/$E\flat$ Mayor}  \upchord{\Fs}{F\#/$G\flat$ Mayor} \upchord{\Gs}{G\#/$A\flat$ Mayor} \upchord{\As}{A\#/$B\flat$ Mayor}
\normalsize

\textbf{Acodes Menores}

\small
\upchord{\Am}{La} Menor \upchord{\Bm}{Si} Menor \upchord{\Cm}{Do} Menor \upchord{\Dm}{Re} Menor \upchord{\Em}{Mi} Menor \upchord{\Fm}{Fa} Menor \upchord{\Gm}{Sol} Menor

\upchord{\Asm}{\small{A\#/B\flat Menor}} \upchord{\Csm}{\small{C\#/D\flat Menor}} \upchord{\Dsm}{\small{D\#/E\flat Menor}}  \upchord{\Fsm}{\small{F\#/G\flat Menor}} \upchord{\Gsm}{\small{G\#/A\flat Menor}} \upchord{\Asm}{\small{A\#/B\flat Menor}}
\normalsize

\vskip 20pt
\textbf{Acodes Mayores S\'eptima}

\upchord{\Aseven}{La} Mayor s\'eptima
\upchord{\Bflatseven}{Si} bemol Mayor s\'eptima
\upchord{\Bseven}{Si} Mayor s\'eptima
\upchord{\Cseven}{\small{Do Mayor s\'eptima}}
\upchord{\Csseven}{\small{Do sostenidoMayor s\'eptima}}
\upchord{\Dseven}{\small{Re Mayor s\'eptima}}
\upchord{\Eflatseven}{\small{Mi bemol Mayor s\'eptima}}
\upchord{\Eseven}{\small{Mi Mayor s\'eptima}}
\upchord{\Fseven}{\small{Fa Mayor s\'eptima}}
\upchord{\Gseven}{\small{Sol Mayor s\'eptima}}
\vskip 20pt

\textbf{Acodes Menores S\'eptima}

\small
\upchord{\Amseven}{La} Menor s\'eptima
\upchord{\Bmseven}{Si} Menor s\'eptima
\upchord{\Cmseven}{Do} Menor s\'eptima
\upchord{\Csmseven}{Do} Menor s\'eptima
\upchord{\Dmseven}{Re} Menor s\'eptima
\upchord{\Dsmseven}{Re} Sostenido Menor s\'eptima
\upchord{\Emseven}{Mi} Menor s\'eptima
\upchord{\Emseventr}{Mi} Menor s\'eptima
\upchord{\Fmseven}{Fa} Menor s\'eptima
\upchord{\Fsmseven}{Fa sostenido} Menor s\'eptima
\upchord{\Gmseven}{Sol} Menor s\'eptima
\upchord{\Gsmseven}{Sol} Sostenido Menor s\'eptima
\upchord{\Bflatmseven}{Si bemol} Menor s\'eptima
\normalsize

\vskip 20pt
\textbf{Acodes Mayores Suspendido cuarta}
\vskip 25pt

\small
\upchord{\Asus}{La} Suspendida cuarta
\upchord{\Bsus}{Si} Suspendida cuarta
\upchord{\Csus}{Do} Suspendida cuarta
\upchord{\Dsus}{Re} Suspendida cuarta
\upchord{\Esus}{Re} Suspendida cuarta
\upchord{\Fsus}{Fa} Suspendida cuarta
\upchord{\Gsus}{Sol} Suspendida cuarta

\upchord{\Fssus}{Fa} sostenido Suspendida cuarta
\upchord{\Gssus}{Sol sostenido} Suspendida cuarta
\normalsize

\vskip 20pt
\textbf{Acodes Mayor Aumentada}
\vskip 25pt

\small
\upchord{\CMaj}{Do} Maj
\upchord{\DMaj}{Re} Maj
\upchord{\GMaj}{Sol} Maj
\normalsize

\vskip 20pt
\textbf{Acodes Mayor S\'eptima Aumentada}
\vskip 25pt

\small
\upchord{\AsevenMaj}{La} Maj S\'eptima Aumentada
\upchord{\Fmajseven}{Fa} Maj S\'eptima Aumentada
\normalsize

\vskip 20pt
\textbf{Acodes Aumentada 2}
\vskip 25pt

\small
\upchord{\Atwo}{La} Aumentada 2
\upchord{\Ctwo}{Do} Aumentada 2
\normalsize

\vskip 20pt
\textbf{Acodes Novena}
\vskip 25pt

\small
\upchord{\Cnine}{Do} Novena
\upchord{\Gnine}{Sol} Novena
\normalsize

\vskip 20pt
\textbf{Acodes Disminuidos}
\vskip 25pt

\small
\upchord{\Gsdim}{Sol} sostenido disminuido
\normalsize


\vskip 20pt
\textbf{Acodes Con Bajo cambiado}

\small
\upchord{\AAs}{La Mayor bajo Bb}
\upchord{\ACs}{La Mayor bajo C\#}
\upchord{\AEg}{La Mayor bajo E}
\upchord{\AmF}{La Menor bajo F}
\vskip 20pt
\upchord{\CE}{Do Mayor bajo E}
\upchord{\CG}{Do Mayor bajo G}
\upchord{\DflatF}{Re bemol bajo F}
\upchord{\DA}{Re Mayor bajo A}
\vskip 20pt
\upchord{\DE}{Re Mayor bajo E}
\upchord{\DFs}{Re Mayor bajo F\#}
\upchord{\EGs}{Mi Mayor Bajo G\#}
\vskip 20pt
\upchord{\GB}{Sol Mayor Bajo B}
\upchord{\GD}{Sol Mayor Bajo D}
\upchord{\GE}{Sol Mayor Bajo E}
\upchord{\AflatC}{La bemol Bajo C}
\vskip 20pt
\upchord{\AflatEflat}{La bemol Bajo Eb}
\upchord{\DsusFs}{Re} Suspendida cuarta bajo F\#
\normalsize

\vskip 20pt
\textbf{Acodes semidisminuidos}

\small
\upchord{\Bmsevenbfive}{Si} Menor s\'eptima semidisminuido
\normalsize

\vskip 20pt
\textbf{Acodes 13 suspendida cuarta}

\small
\upchord{\Csusthirteen}{Do} 13 suspendida cuarta
\normalsize

\clearpage
\fi

\ifpiano
\lhead{\LHeadFont Acodes~para~Piano}
{\parindent 8pt
        {\myTitleFont --- Acordes para Piano ---}}\par
\vskip 20pt
\textbf{Acodes Mayores}
\vskip 25pt

%\small{El s\'imbolo \# significa sostenido y {\flat}~significa~bemol}
\small
\upchord{\APiano}{\qquad La Mayor} \qquad\qquad \upchord{\BPiano}{Si Mayor} \qquad\qquad \upchord{\CPiano}{\qquad Do Mayor} \qquad\qquad \upchord{\DPiano}{\qquad Re Mayor} \hfill \break
\vskip 25pt
\upchord{\EPiano}{\qquad Mi Mayor} \qquad\qquad  \upchord{\FPiano}{\qquad Fa Mayor} \qquad\qquad \upchord{\GPiano}{\qquad Sol Mayor}
\vskip 25pt
\upchord{\AsPiano}{A\#/$B\flat$ Mayor} \qquad\qquad \upchord{\CsPiano}{C\#/$D\flat$ Mayor} \qquad\qquad \upchord{\DsPiano}{D\#/$E\flat$ Mayor} \qquad\qquad \upchord{\FsPiano}{F\#/$G\flat$ Mayor} \hfill \break
\vskip 25pt
\upchord{\GsPiano}{G\#/$A\flat$ Mayor} \qquad\qquad \upchord{\AsPiano}{A\#/$B\flat$ Mayor}
\normalsize

\textbf{Acodes Menores}
\vskip 25pt

\small
\upchord{\AmPiano}{\qquad La} Menor \qquad\qquad \upchord{\BmPiano}{\qquad Si} Menor \qquad\qquad \upchord{\CmPiano}{\qquad Do} Menor \qquad\qquad \upchord{\DmPiano}{\qquad Re} Menor \hfill \break
\vskip 25pt
\upchord{\EmPiano}{\qquad Mi} Menor \qquad\qquad \upchord{\FmPiano}{\qquad Fa} Menor \qquad\qquad \upchord{\GmPiano}{\qquad Sol} Menor
\vskip 25pt
\upchord{\AsmPiano}{\small{A\#/B\flat Menor}}  \qquad\qquad  \upchord{\CsmPiano}{C\#/D\flat Menor}  \qquad\qquad  \upchord{\DsmPiano}{D\#/E\flat Menor} \qquad\qquad \upchord{\FsmPiano}{F\#/G\flat Menor} \hfill \break
\vskip 25pt
\upchord{\GsmPiano}{G\#/A\flat Menor}  \qquad\qquad  \upchord{\AsmPiano}{A\#/B\flat Menor}
\normalsize

\clearpage
%\vskip 20pt
\textbf{Acodes Mayores S\'eptima}
\vskip 25pt

\small
\upchord{\AsevenPiano}{La Mayor s\'eptima} \qquad\qquad \upchord{\BsevenPiano}{Si Mayor s\'eptima} \qquad\qquad \upchord{\CsevenPiano}{Do Mayor s\'eptima} \qquad\qquad
\vskip 25pt
\upchord{\DsevenPiano}{Re Mayor s\'eptima} \qquad\qquad \upchord{\EsevenPiano}{Mi Mayor s\'eptima} \qquad\qquad \upchord{\FsevenPiano}{Fa Mayor s\'eptima}
\vskip 25pt
\upchord{\GsevenPiano}{Sol Mayor s\'eptima}  \qquad\qquad \upchord{\BflatsevenPiano}{Si} bemol Mayor s\'eptima
\vskip 25pt
\upchord{\EflatsevenPiano}{Mi bemol Mayor s\'eptima} \qquad\qquad
\normalsize
\vskip 20pt

\textbf{Acodes Menores S\'eptima}
\vskip 25pt

\small
\upchord{\AmsevenPiano}{La} Menor s\'eptima
\upchord{\BmsevenPiano}{Si} Menor s\'eptima
\upchord{\CmsevenPiano}{Do} Menor s\'eptima
\vskip 25pt
\upchord{\DmsevenPiano}{Re} Menor s\'eptima
\upchord{\EmsevenPiano}{Mi} Menor s\'eptima
\upchord{\FmsevenPiano}{Fa} Menor s\'eptima
\vskip 25pt
\upchord{\GmsevenPiano}{Sol} Menor s\'eptima
\upchord{\BflatmsevenPiano}{Si bemol} Menor s\'eptima
\upchord{\FsmsevenPiano}{Fa sostenido} Menor s\'eptima
\vskip 25pt
\upchord{\CssevenPiano}{Do sostenido} Mayor s\'eptima
\upchord{\DsmsevenPiano}{Re} Sostenido Menor s\'eptima
\upchord{\GsmsevenPiano}{Sol} Sostenido Menor s\'eptima
\normalsize

\vskip 20pt

\textbf{Acodes Mayores Suspendido cuarta}
\vskip 25pt

\small
\upchord{\AsusPiano}{La} Suspendida cuarta
\upchord{\BsusPiano}{Si} Suspendida cuarta
\upchord{\CsusPiano}{Do} Suspendida cuarta
\vskip 25pt
\upchord{\DsusPiano}{Re} Suspendida cuarta
\upchord{\EsusPiano}{Re} Suspendida cuarta
\upchord{\FsusPiano}{Fa} Suspendida cuarta
\vskip 25pt
\upchord{\GsusPiano}{Sol} Suspendida cuarta
\upchord{\GssusPiano}{Sol sostenido} Suspendida cuarta
\normalsize

\vskip 20pt
\textbf{Acodes Mayor Aumentada}
\vskip 25pt

\small
\upchord{\CMajPiano}{Do} Maj  \qquad\qquad
\upchord{\DMajPiano}{Re} Maj
\upchord{\GMajPiano}{Sol} Maj
\normalsize

\vskip 20pt
\textbf{Acodes Mayor S\'eptima Aumentada}
\vskip 25pt

\small
\upchord{\AsevenMajPiano}{La} Maj S\'eptima Aumentada
\upchord{\FmajsevenPiano}{Fa} Maj S\'eptima Aumentada
\normalsize

\vskip 20pt
\textbf{Acodes Aumentada 2}
\vskip 25pt

\small
\upchord{\AtwoPiano}{La} Aumentada 2
\upchord{\CtwoPiano}{Do} Aumentada 2
\normalsize


\vskip 20pt
\textbf{Acodes Disminuidos}
\vskip 25pt

\small
\upchord{\GsdimPiano}{Sol} sostenido disminuido
\normalsize


\vskip 20pt
\textbf{Acodes Con Bajo cambiado}
\vskip 25pt

\small
\upchord{\ACsPiano}{La Mayor bajo C\#}
\upchord{\AEPiano}{La Mayor bajo E}
\vskip 20pt
\upchord{\AmFPiano}{La Menor bajo F}
\upchord{\CEPiano}{Do Mayor bajo E}
\vskip 20pt
\upchord{\CGPiano}{Do Mayor bajo G}
\upchord{\DAPiano}{Re Mayor bajo A}
\upchord{\DFsPiano}{Re Mayor bajo F\#}
\upchord{\GDPiano}{Sol Mayor Bajo D}
\vskip 20pt
\upchord{\GBPiano}{Sol Mayor Bajo B}
\vskip 25pt
\upchord{\DsusFsPiano}{Re} Suspendida cuarta bajo F\#
\normalsize

\vskip 20pt
\textbf{Acodes medio disminuido s\'eptima}
\vskip 25pt

\small
\upchord{\BmsevenbfivePiano}{Si} medio disminuido s\'eptima
\normalsize

\vskip 20pt
\textbf{Acodes 13 suspendida cuarta}
\vskip 25pt

\small
\upchord{\CsusthirteenPiano}{Do} 13 suspendida cuarta
\normalsize

\clearpage
\fi
%\end{document}
%\bye
    %%%%%% rcsid = @(#)$Id:$
%%%%%%
%%
%%      ================================
%%      Sample Key Index (sampleAdx.tex)
%%      ================================
%%
%%      Version 4.5, 30 April, 2010
%%
%%      Copyright 1992--2010 Christopher Rath <christopher@rath.ca>
%%
%%	This package is free software; you can redistribute it and/or
%%	modify it under the terms of version 2.1 of the GNU Lesser 
%%	General Public License as published by the Free Software
%%	Foundation.
%%
%%	This package is distributed in the hope that it will be
%%	useful, but WITHOUT ANY WARRANTY; without even the implied
%%	warranty of MERCHANTABILITY or FITNESS FOR A PARTICULAR
%%	PURPOSE.  See the GNU Lesser General Public License for more
%%	details.
%%
%%      This file is provided as a template for Song Artist
%%      Index generation.
%%
%%%%%%
%%%%%%

%%%%%%%%%%%%%%%%%%%%%%%%%%%%%%%%%%%%%%%%%%%%%%%%%%%%%%%%%%
%%%%%%%%%%%%%%%%%%%%%%%%%%%%%%%%%%%%%%%%%%%%%%%%%%%%%%%%%%
%%                                                      %%
%%           P R E A M B L E   B E G I N S              %%
%%                                                      %%
%%%%%%%%%%%%%%%%%%%%%%%%%%%%%%%%%%%%%%%%%%%%%%%%%%%%%%%%%%
%%%%%%%%%%%%%%%%%%%%%%%%%%%%%%%%%%%%%%%%%%%%%%%%%%%%%%%%%%

%\documentclass[12pt,twocolumn]{book}
%\usepackage{latexsym,fancyhdr}
%\usepackage[wordbk]{songbook}

%%%
% Revision Date and Release Date definitions.
%
%       \RelDate - The last time this songbook was released.
%       \RevDate - The last time this file was revised in any way.
%%%
%\newcommand{\RelDate}{30~May'96}
%\newcommand{\RevDate}{\RelDate}

%%%
% Redefine fonts from SongBook style that I don't like, and define
% any extra fonts I require.
%%%
\font\myTinySF=cmss8    at  8pt
\font\myHugeSF=cmssbx10 at 25pt
\renewcommand{\CpyRtInfoFont}{\tiny\myTinySF}
%\newcommand{\myTitleFont}{\Huge\myHugeSF}
%\newcommand{\mySubTitleFont}{\large\sf}

%%%
% Define fonts to use in the headers and footers of the songbook.
%%%
%\newcommand{\LHeadFont}{\normalsize}            % = cmr12  at 12pt
%\newcommand{\CHeadFont}{\normalsize\rm}         % = cmr12  at 12pt
%\newcommand{\RHeadFont}{\normalsize}            % = cmr12  at 12pt
%\newcommand{\LFootFont}{\scriptsize}            % = cmr8   at  8pt
%\newcommand{\CFootFont}{\tiny\myTinySF}         % = cmss8  at  8pt
%\newcommand{\RFootFont}{\scriptsize}            % = cmr8   at  8pt

%%%
% Turn on and define fancy page heading/footing definition.
%%%
\pagestyle{fancy}

%\addtolength{\headwidth}{\marginparsep}
%\addtolength{\headwidth}{\marginparwidth}
%\renewcommand{\footrulewidth}{0.4pt}
\lhead{\LHeadFont \'Indice~de~Autores}
       \chead{\CHeadFont ({\rm\thepage})}
       \rhead{\RHeadFont\RelDate}
%
%\lfoot{\LFootFont Property of a Church}
%       \cfoot{\CFootFont Last Revised:  \RevDate}
%       \rfoot{\RFootFont Material used by permission.}


%%%
% Index entries command definition.
%%%
\renewcommand{\item}{\par\hangindent=40pt}
\renewcommand{\subitem}{\par\hangindent=40pt \hspace*{20pt}}
\renewcommand{\subsubitem}{\par\hangindent=40pt \hspace*{30pt}}


%%%%%%%%%%%%%%%%%%%%%%%%%%%%%%%%%%%%%%%%%%%%%%%%%%%%%%%%%%
%%%%%%%%%%%%%%%%%%%%%%%%%%%%%%%%%%%%%%%%%%%%%%%%%%%%%%%%%%
%%                                                      %%
%%           D O C U M E N T   B E G I N S              %%
%%                                                      %%
%%%%%%%%%%%%%%%%%%%%%%%%%%%%%%%%%%%%%%%%%%%%%%%%%%%%%%%%%%
%%%%%%%%%%%%%%%%%%%%%%%%%%%%%%%%%%%%%%%%%%%%%%%%%%%%%%%%%%
%\begin{document}

%%%
% Index begins.
%%%
\pdfbookmark[0]{\'Indice~de~autores}{autores}
{\parindent 8pt
  {\myTitleFont --- INDICE DE AUTORES ---}}\par
\vskip 20pt

\input{Estribillero.adx}

%\end{document}
%\bye
%
%%%
% Document ends.
%%%

% Local Variables:
%   LaTeX-item-indent:     -1
%   LaTeX-indent-level:     2
%   TeX-brace-indent-level: 2
%   TeX-auto-untabify:      nil
%   TeX-style-local:        style/
% End:

    %%%%%% rcsid = @(#)$Id: sampleKdx.tex,v 1.16 2010-04-12 18:04:30 rathc Exp $
%%%%%%
%%
%%      ================================
%%      Sample Key Index (sampleKdx.tex)
%%      ================================
%%
%%      Version 4.5, 30 April, 2010
%%
%%      Copyright 1992--2010 Christopher Rath <christopher@rath.ca>
%%
%%	This package is free software; you can redistribute it and/or
%%	modify it under the terms of version 2.1 of the GNU Lesser 
%%	General Public License as published by the Free Software
%%	Foundation.
%%
%%	This package is distributed in the hope that it will be
%%	useful, but WITHOUT ANY WARRANTY; without even the implied
%%	warranty of MERCHANTABILITY or FITNESS FOR A PARTICULAR
%%	PURPOSE.  See the GNU Lesser General Public License for more
%%	details.
%%
%%      This file is provided as a template for Song Key
%%      Index generation.
%%
%%%%%%
%%%%%%

%%%%%%%%%%%%%%%%%%%%%%%%%%%%%%%%%%%%%%%%%%%%%%%%%%%%%%%%%%
%%%%%%%%%%%%%%%%%%%%%%%%%%%%%%%%%%%%%%%%%%%%%%%%%%%%%%%%%%
%%                                                      %%
%%           P R E A M B L E   B E G I N S              %%
%%                                                      %%
%%%%%%%%%%%%%%%%%%%%%%%%%%%%%%%%%%%%%%%%%%%%%%%%%%%%%%%%%%
%%%%%%%%%%%%%%%%%%%%%%%%%%%%%%%%%%%%%%%%%%%%%%%%%%%%%%%%%%

%\documentclass[12pt,twocolumn]{book}
%\usepackage{latexsym,fancyhdr}
%\usepackage[wordbk]{songbook}

%%%
% Revision Date and Release Date definitions.
%
%       \RelDate - The last time this songbook was released.
%       \RevDate - The last time this file was revised in any way.
%%%
%\newcommand{\RelDate}{30~May'96}
%\newcommand{\RevDate}{\RelDate}

%%%
% Redefine fonts from SongBook style that I don't like, and define
% any extra fonts I require.
%%%
\font\myTinySF=cmss8    at  8pt
\font\myHugeSF=cmssbx10 at 25pt
\renewcommand{\CpyRtInfoFont}{\tiny\myTinySF}
%\newcommand{\myTitleFont}{\Huge\myHugeSF}
%\newcommand{\mySubTitleFont}{\large\sf}

%%%
% Define fonts to use in the headers and footers of the songbook.
%%%
%\newcommand{\LHeadFont}{\normalsize}            % = cmr12  at 12pt
%\newcommand{\CHeadFont}{\normalsize\rm}         % = cmr12  at 12pt
%\newcommand{\RHeadFont}{\normalsize}            % = cmr12  at 12pt
%\newcommand{\LFootFont}{\scriptsize}            % = cmr8   at  8pt
%\newcommand{\CFootFont}{\tiny\myTinySF}         % = cmss8  at  8pt
%\newcommand{\RFootFont}{\scriptsize}            % = cmr8   at  8pt

%%%
% Turn on and define fancy page heading/footing definition.
%%%
\pagestyle{fancy}
\pdfbookmark[0]{\'Indice~Tonal}{tonal}
%\addtolength{\headwidth}{\marginparsep}
%\addtolength{\headwidth}{\marginparwidth}
%\renewcommand{\footrulewidth}{0.4pt}
\lhead{\LHeadFont \'Indice~Tonal}
       \chead{\CHeadFont ({\rm\thepage})}
       \rhead{\RHeadFont\RelDate}

%\lfoot{\LFootFont Property of a Church}
%       \cfoot{\CFootFont Last Revised:  \RevDate}
%       \rfoot{\RFootFont Material used by permission.}


%%%
% Index entries command definition.
%%%
\renewcommand{\item}{\par\hangindent=40pt}
\renewcommand{\subitem}{\par\hangindent=40pt \hspace*{20pt}}
\renewcommand{\subsubitem}{\par\hangindent=40pt \hspace*{30pt}}


%%%%%%%%%%%%%%%%%%%%%%%%%%%%%%%%%%%%%%%%%%%%%%%%%%%%%%%%%%
%%%%%%%%%%%%%%%%%%%%%%%%%%%%%%%%%%%%%%%%%%%%%%%%%%%%%%%%%%
%%                                                      %%
%%           D O C U M E N T   B E G I N S              %%
%%                                                      %%
%%%%%%%%%%%%%%%%%%%%%%%%%%%%%%%%%%%%%%%%%%%%%%%%%%%%%%%%%%
%%%%%%%%%%%%%%%%%%%%%%%%%%%%%%%%%%%%%%%%%%%%%%%%%%%%%%%%%%
%\begin{document}

%%%
% Index begins.
%%%
{\parindent 8pt
  {\myTitleFont --- INDICE TONAL ---}}\par
\vskip 20pt

\input{Estribillero.kdx}
%
%\end{document}
%\bye
%
%%%
% Document ends.
%%%

% Local Variables:
%   LaTeX-item-indent:     -1
%   LaTeX-indent-level:     2
%   TeX-brace-indent-level: 2
%   TeX-auto-untabify:      nil
%   TeX-style-local:        style/
% End:

    %%%%%% rcsid = @(#)$Id: sampleTdx.tex,v 1.18 2010-04-12 18:04:31 rathc Exp $
%%%%%%
%%
%%      ===============================================
%%      Sample Title & First Line Index (sampleTdx.tex)
%%      ===============================================
%%
%%      Version 4.5, 30 April, 2010
%%
%%      Copyright 1992--2010 Christopher Rath <christopher@rath.ca>
%%
%%      This package is free software; you can redistribute it and/or
%%      modify it under the terms of version 2.1 of the GNU Lesser 
%%	General Public License as published by the Free Software 
%%	Foundation.
%%
%%      This package is distributed in the hope that it will be
%%      useful, but WITHOUT ANY WARRANTY; without even the implied
%%      warranty of MERCHANTABILITY or FITNESS FOR A PARTICULAR
%%      PURPOSE.  See the GNU Lesser General Public License for more
%%      details.
%%
%%      This file is provided as a template for Title and First Line
%%      Index generation.
%%
%%%%%%
%%%%%%

%%%%%%%%%%%%%%%%%%%%%%%%%%%%%%%%%%%%%%%%%%%%%%%%%%%%%%%%%%
%%%%%%%%%%%%%%%%%%%%%%%%%%%%%%%%%%%%%%%%%%%%%%%%%%%%%%%%%%
%%                                                      %%
%%           P R E A M B L E   B E G I N S              %%
%%                                                      %%
%%%%%%%%%%%%%%%%%%%%%%%%%%%%%%%%%%%%%%%%%%%%%%%%%%%%%%%%%%
%%%%%%%%%%%%%%%%%%%%%%%%%%%%%%%%%%%%%%%%%%%%%%%%%%%%%%%%%%

%\documentclass[12pt,twocolumn,spanish]{book}
%\usepackage{latexsym,fancyhdr}
%\usepackage[wordbk]{songbook}


%%%
% Revision Date and Release Date definitions.
%
%       \RelDate - The last time this songbook was released.
%       \RevDate - The last time this file was revised in any way.
%%%
%\newcommand{\RelDate}{30 May'96}
%\newcommand{\RevDate}{\today}

%%%
% Redefine fonts from SongBook style that I don't like, and define
% any extra fonts I require.
%%%
\font\myTinySF=cmss8    at  8pt
\font\myHugeSF=cmssbx10 at 25pt
\renewcommand{\CpyRtInfoFont}{\tiny\myTinySF}
%\newcommand{\myTitleFont}{\Huge\myHugeSF}
%\newcommand{\mySubTitleFont}{\large\sf}

%%%
% Define fonts to use in the headers and footers of the songbook.
%%%
%\newcommand{\LHeadFont}{\normalsize}            % = cmr12  at 12pt
%\newcommand{\CHeadFont}{\normalsize\rm}         % = cmr12  at 12pt
%\newcommand{\RHeadFont}{\normalsize}            % = cmr12  at 12pt
%\newcommand{\LFootFont}{\scriptsize}            % = cmr8   at  8pt
%\newcommand{\CFootFont}{\tiny\myTinySF}         % = cmss8  at  8pt
%\newcommand{\RFootFont}{\scriptsize}            % = cmr8   at  8pt

%%%
% Turn on and define fancy page heading/footing definition.
%%%
\pagestyle{fancy}

\addtolength{\headwidth}{\marginparsep}
\addtolength{\headwidth}{\marginparwidth}
\renewcommand{\footrulewidth}{0.4pt}
\lhead{\LHeadFont A Church Songbook}
       \chead{\CHeadFont \I'ndice~por~T\'itulo({\rm\thepage})}
       \rhead{\RHeadFont\RelDate}

\lfoot{\LFootFont Property of a Church}
       \cfoot{\CFootFont Last Revised:  \RevDate}
       \rfoot{\RFootFont Material used by permission.}

%%%
% Index entries command definition.
%%%
\renewcommand{\item}{\par\hangindent=40pt}
\renewcommand{\subitem}{\par\hangindent=40pt \hspace*{20pt}}
\renewcommand{\subsubitem}{\par\hangindent=40pt \hspace*{30pt}}


%%%%%%%%%%%%%%%%%%%%%%%%%%%%%%%%%%%%%%%%%%%%%%%%%%%%%%%%%%
%%%%%%%%%%%%%%%%%%%%%%%%%%%%%%%%%%%%%%%%%%%%%%%%%%%%%%%%%%
%%                                                      %%
%%           D O C U M E N T   B E G I N S              %%
%%                                                      %%
%%%%%%%%%%%%%%%%%%%%%%%%%%%%%%%%%%%%%%%%%%%%%%%%%%%%%%%%%%
%%%%%%%%%%%%%%%%%%%%%%%%%%%%%%%%%%%%%%%%%%%%%%%%%%%%%%%%%%
%\begin{document}

%%%
% Begin the Index.
%%%
{\parindent 8pt
  {\myTitleFont --- Title Index ---}}\par
\vskip 5pt
{\parindent 20pt
  {\mySubTitleFont --- with first lines in italic ---}}
\vskip 20pt

\input{Estribillero.tdx}

%\end{document}
%\bye
%
%%%
% Document ends.
%%%

\end{document}
\bye
%
%%%
% Document ends.
%%%


%\end{document}
%\bye
%
%%%
% Document ends.
%%%

%%%%%% rcsid = @(#)$Id: sampleKdx.tex,v 1.16 2010-04-12 18:04:30 rathc Exp $
%%%%%%
%%
%%      ================================
%%      Sample Key Index (sampleKdx.tex)
%%      ================================
%%
%%      Version 4.5, 30 April, 2010
%%
%%      Copyright 1992--2010 Christopher Rath <christopher@rath.ca>
%%
%%	This package is free software; you can redistribute it and/or
%%	modify it under the terms of version 2.1 of the GNU Lesser 
%%	General Public License as published by the Free Software
%%	Foundation.
%%
%%	This package is distributed in the hope that it will be
%%	useful, but WITHOUT ANY WARRANTY; without even the implied
%%	warranty of MERCHANTABILITY or FITNESS FOR A PARTICULAR
%%	PURPOSE.  See the GNU Lesser General Public License for more
%%	details.
%%
%%      This file is provided as a template for Song Key
%%      Index generation.
%%
%%%%%%
%%%%%%

%%%%%%%%%%%%%%%%%%%%%%%%%%%%%%%%%%%%%%%%%%%%%%%%%%%%%%%%%%
%%%%%%%%%%%%%%%%%%%%%%%%%%%%%%%%%%%%%%%%%%%%%%%%%%%%%%%%%%
%%                                                      %%
%%           P R E A M B L E   B E G I N S              %%
%%                                                      %%
%%%%%%%%%%%%%%%%%%%%%%%%%%%%%%%%%%%%%%%%%%%%%%%%%%%%%%%%%%
%%%%%%%%%%%%%%%%%%%%%%%%%%%%%%%%%%%%%%%%%%%%%%%%%%%%%%%%%%

%\documentclass[12pt,twocolumn]{book}
%\usepackage{latexsym,fancyhdr}
%\usepackage[wordbk]{songbook}

%%%
% Revision Date and Release Date definitions.
%
%       \RelDate - The last time this songbook was released.
%       \RevDate - The last time this file was revised in any way.
%%%
%\newcommand{\RelDate}{30~May'96}
%\newcommand{\RevDate}{\RelDate}

%%%
% Redefine fonts from SongBook style that I don't like, and define
% any extra fonts I require.
%%%
\font\myTinySF=cmss8    at  8pt
\font\myHugeSF=cmssbx10 at 25pt
\renewcommand{\CpyRtInfoFont}{\tiny\myTinySF}
%\newcommand{\myTitleFont}{\Huge\myHugeSF}
%\newcommand{\mySubTitleFont}{\large\sf}

%%%
% Define fonts to use in the headers and footers of the songbook.
%%%
%\newcommand{\LHeadFont}{\normalsize}            % = cmr12  at 12pt
%\newcommand{\CHeadFont}{\normalsize\rm}         % = cmr12  at 12pt
%\newcommand{\RHeadFont}{\normalsize}            % = cmr12  at 12pt
%\newcommand{\LFootFont}{\scriptsize}            % = cmr8   at  8pt
%\newcommand{\CFootFont}{\tiny\myTinySF}         % = cmss8  at  8pt
%\newcommand{\RFootFont}{\scriptsize}            % = cmr8   at  8pt

%%%
% Turn on and define fancy page heading/footing definition.
%%%
\pagestyle{fancy}
\pdfbookmark[0]{\'Indice~Tonal}{tonal}
%\addtolength{\headwidth}{\marginparsep}
%\addtolength{\headwidth}{\marginparwidth}
%\renewcommand{\footrulewidth}{0.4pt}
\lhead{\LHeadFont \'Indice~Tonal}
       \chead{\CHeadFont ({\rm\thepage})}
       \rhead{\RHeadFont\RelDate}

%\lfoot{\LFootFont Property of a Church}
%       \cfoot{\CFootFont Last Revised:  \RevDate}
%       \rfoot{\RFootFont Material used by permission.}


%%%
% Index entries command definition.
%%%
\renewcommand{\item}{\par\hangindent=40pt}
\renewcommand{\subitem}{\par\hangindent=40pt \hspace*{20pt}}
\renewcommand{\subsubitem}{\par\hangindent=40pt \hspace*{30pt}}


%%%%%%%%%%%%%%%%%%%%%%%%%%%%%%%%%%%%%%%%%%%%%%%%%%%%%%%%%%
%%%%%%%%%%%%%%%%%%%%%%%%%%%%%%%%%%%%%%%%%%%%%%%%%%%%%%%%%%
%%                                                      %%
%%           D O C U M E N T   B E G I N S              %%
%%                                                      %%
%%%%%%%%%%%%%%%%%%%%%%%%%%%%%%%%%%%%%%%%%%%%%%%%%%%%%%%%%%
%%%%%%%%%%%%%%%%%%%%%%%%%%%%%%%%%%%%%%%%%%%%%%%%%%%%%%%%%%
%\begin{document}

%%%
% Index begins.
%%%
{\parindent 8pt
  {\myTitleFont --- INDICE TONAL ---}}\par
\vskip 20pt

%%%%%% rcsid = @(#)$Id: sample-sb.tex,v 1.23 2010-04-12 18:04:11 rathc Exp $
%%%%%%
%%
%%      ===============================
%%      Sample Songbook (sample-sb.tex)
%%      ===============================
%%
%%      Version 4.5, 30 April, 2010
%%
%%      Copyright 1992--2010 Christopher Rath <christopher@rath.ca>
%%
%%      This package is free software; you can redistribute it and/or
%%      modify it under the terms of version 2.1 of the GNU Lesser
%%	General Public License as published by the Free Software 
%%	Foundation.
%%
%%      This package is distributed in the hope that it will be
%%      useful, but WITHOUT ANY WARRANTY; without even the implied
%%      warranty of MERCHANTABILITY or FITNESS FOR A PARTICULAR
%%      PURPOSE.  See the GNU Lesser General Public License for more
%%      details.
%%
%%      This file contains a subset of the songbook we distribute
%%      at our church.  To the best of my knowledge, all of the lyrics
%%      contained herein are freely distributable.  This file has been
%%      provided as a sample of what can be produced by the chordbk,
%%      wordbk, and overhead LaTeX styles.
%%
%%      NEEDED:  The fancyhdr LaTeX style is required to properly
%%              format this file.  If you don't have that then comment
%%              out the commands in the preamble which deal with the
%%              fancyhdr style.
%%
%%%%%%
%%%%%%
%%
%%      1. Chord notation.  Within this songbook the following
%%         conventions have been adopted:
%%
%%              "Minor" is entered as "m";
%%                      e.g. Cm7 for C minor 7th.
%%              "Major" is entered as "M";
%%                      e.g. CM7 for C major 7th.
%%
%%%%%%
%%%%%%
%%      ============
%%      Bibliography
%%      ============
%%
%%    
%%
%%%%%%
%%%%%%

%%%%%%%%%%%%%%%%%%%%%%%%%%%%%%%%%%%%%%%%%%%%%%%%%%%%%%%%%%
%%%%%%%%%%%%%%%%%%%%%%%%%%%%%%%%%%%%%%%%%%%%%%%%%%%%%%%%%%
%%                                                      %%
%%           P R E A M B L E   B E G I N S              %%
%%                                                      %%
%%%%%%%%%%%%%%%%%%%%%%%%%%%%%%%%%%%%%%%%%%%%%%%%%%%%%%%%%%
%%%%%%%%%%%%%%%%%%%%%%%%%%%%%%%%%%%%%%%%%%%%%%%%%%%%%%%%%%

\documentclass[12pt, spanish, titlepage]{book}
\usepackage[T1]{fontenc}
\usepackage[latin9]{inputenc}
\usepackage{babel}
\usepackage{mypiano}
\usepackage{gchords}
\usepackage{latexsym,fancyhdr}
\usepackage{imakeidx}
\usepackage[unicode=true,pdfusetitle, bookmarks=true,bookmarksnumbered=false,bookmarksopen=false,
    breaklinks=false,pdfborder={0 0 1},backref=false,colorlinks=true]{hyperref}
\usepackage[chordbk]{songbook} %% Words & Chords edition.
%%\usepackage[compactallsongs,chordbk]{songbook}    %% Words & Chords edition.
%%\usepackage[wordbk]{songbook}                 %% Words Only edition.
%%\usepackage[overhead]{songbook}               %% Overhead Transparency edition.


% genera acordes de guitarra
\newif\ifguitarra

%genera acordes de piano
\newif\ifpiano

\guitarratrue
\pianotrue



\renewcommand{\SBChorusTag}{Coro:}
\renewcommand{\SBBridgeTag}{Puente:}
\newcommand{\myTitleFont}{\Huge\myHugeSF}
\newcommand{\mySubTitleFont}{\large\sf}
%%%
% Revision Date and Release Date definitions.
%
%       \RelDate - The last time this songbook was released.  Set this
%                  date each time a new release/update of the songbook
%                  is generated.
%       \RevDate - The last time a particular song was revised in any
%                  way.  This command will be renewed inside every
%                  song.
%%%
\newcommand{\RelDate}{26~Marzo,~2014}
\newcommand{\RevDate}{\today}

%%%
% C.C.L.I. license number definition; for copyright licensing info.
% One of these macros will be manually inserted into the {CpyRt}
% parameter of the {song} environment.
%
%       \CCLInumber - The actual copyright license number.  Don't
%               insert this command in the {CpyRt} parameter, use one
%               of the others.
%       \CCLIed - Indicates a song falls under our CCLI license.
%       \NotCCLIed - Indicates a song doesn't fall under our CCLI
%               license.  Public Domain songs fall into this category.
%       \PGranted - We have received specific permission from the
%               copyright holder to use this song.
%       \PPending - We are in the process of obtaining permission to
%               use this song.
%%%
\newcommand{\CCLInumber}{Your CCLI Number}
\newcommand{\CCLIed}{{\CpyRtInfoFont (CCLI \CCLInumber)}}
\newcommand{\NotCCLIed}{\relax}
\newcommand{\PGranted}{\relax}
\newcommand{\PPending}{{\CpyRtInfoFont (Permission Pending)}}

% comandos para pintar acordes de guitarra
\newcommand{\A}{\chord{t}{x,n,p2,p2,p2,n}{A}}
\newcommand{\Aseven}{\chord{t}{x,n,p2,n,p2,n}{A7}}
\newcommand{\AsevenMaj}{\chord{t}{x,n,p2,p1,p2,n}{A7+}}
\newcommand{\Am}{\chord{t}{x,n,p2,p2,p1,n}{Am}}
\newcommand{\Amseven}{\chord{t}{x,n,p2,n,p1,n}{Am7}}
\newcommand{\ACs}{\chord{t}{x,p3,p2,p2,p2,n}{A/C\#}}

\newcommand{\As}{\chord{t}{p1,p1,p3,p3,p3,p1}{A\#}}
\newcommand{\Bflat}{\chord{t}{p1,p1,p3,p3,p3,p1}{Bb}}

\newcommand{\B}{\chord{t}{x,bf1p2,f2p4,f3p4,f4p4,f1p2}{B}}
\newcommand{\Bseven}{\chord{t}{x,f1p2,p4,f1p2,p4,f1p2,}{B7}}
\newcommand{\BsevenBasDs}{\chord{t}{x,x,p1,p2,n,p2}{B7/D\#}}
\newcommand{\Bm}{\chord{t}{x,p2,p4,p4,p3,p2}{Bm}}
\newcommand{\Bmseven}{\chord{t}{x,p2,p4,p2,p3,p2}{Bm7}}
\newcommand{\BmseveN}{\chord{t}{x,p2,p4,p3,p3,p2}{Bm7+}}
\newcommand{\BmsevenA}{\chord{t}{x,n,p4,p4,p3,n}{Bm/A}}

\newcommand{\C}{\chord{t}{x,p3,n,p2,p1,n}{C}}
\newcommand{\Cseven}{\chord{t}{x,p3,p3,p2,p1,n}{C7}}
\newcommand{\Cm}{\chord{t}{x,p3,p1,n,p1,p3}{Cm}}
%\newcommand{\Cmseven}{\chord{t}{x,p3,p1,n,p1,p3}{Cm}}
\newcommand{\CE}{\chord{t}{o,p3,n,p2,p1,n}{C/E}}

\newcommand{\Cs}{\chord{4}{n,n,p2,p2,p2,n}{C\#}}
\newcommand{\Csm}{\chord{t}{p4,p4,p6,p6,p5,p4}{C\#m}}
\newcommand{\CssevenLight}{\chord{t}{x,p4,p3,p4,p2,x}{C\#7}}

\newcommand{\D}{\chord{t}{x,x,n,p2,p3,p2}{D}}
\newcommand{\Dseven}{\chord{t}{x,x,n,p2,p1,p2}{D7}}
\newcommand{\DseveN}{\chord{t}{x,x,n,p2,p2,p2}{D7+}}
\newcommand{\Dm}{\chord{t}{x,x,n,p2,p3,p1}{Dm}}
\newcommand{\Dmseven}{\chord{t}{n,n,n,p2,p1,p1}{Dm7}}
\newcommand{\DmsevenG}{\chord{t}{p3,n,n,p2,p1,p1}{Dm7/G}}
\newcommand{\Dsix}{\chord{t}{x,x,n,p2,n,p2}{D6}}
\newcommand{\DmBasB}{\chord{t}{x,p2,p3,p2,p3,x}{Dm/B}}
\newcommand{\DA}{\chord{t}{x,o,n,p2,p3,p2}{D/A}}
\newcommand{\DFs}{\chord{t}{p2,n,n,p2,p3,p2}{D/F\#}}

\newcommand{\Ds}{\chord{t}{n,n,p1,p3,p4,p3}{D\#}}
\newcommand{\Eflat}{\chord{t}{n,n,p1,p3,p4,p3}{E$\flat$}}

\newcommand{\E}{\chord{t}{n,p2,p2,p1,n,n}{E}}
\newcommand{\Eseven}{\chord{t}{n,p2,p2,p1,p3,n}{E7}}
\newcommand{\EseveN}{\chord{t}{n,p2,p2,p4,p3,p4}{E7}}
\newcommand{\Em}{\chord{t}{n,p2,p2,n,n,n}{Em}}
\newcommand{\EsevenFour}{\chord{t}{n,p2,p2,p4,p3,p5}{E7,11}}
\newcommand{\EseveNNine}{\chord{t}{n,f1p2,f1p2,p4,p3,f1p2,}{E79}}

\newcommand{\F}{\chord{t}{p1,p3,p3,p2,p1,p1}{F}}

\newcommand{\Fs}{\chord{t}{p2,p4,p4,p3,p2,p2}{F\#}}
\newcommand{\Fsm}{\chord{t}{f1p2,p4,p4,f1p2,f1p2,f1p2,}{F\#m}}
\newcommand{\FsminLight}{\chord{t}{x,x,f3p4,f1p2,f1p2,f1p2,}{F\#m}}
\newcommand{\FsminBasSeveN}{\chord{t}{x,x,f3p3,f1p2,f1p2,f1p2,}{F\#m/E\#}}
\newcommand{\FsminBasSeven}{\chord{t}{x,x,f2p2,f1p2,f1p2,f1p2,}{F\#m/E}}
\newcommand{\FsminSeven}{\chord{t}{f1p2,p4,p4,f1p2,p5,f1p2,}{F\#7m}}

\newcommand{\G}{\chord{t}{p3,p2,n,n,n,p2}{G}}
\newcommand{\Gseven}{\chord{t}{p3,p2,n,n,n,p1}{G7}}
\newcommand{\GB}{\chord{t}{n,p2,n,n,p3,n}{G/B}}
\newcommand{\GD}{\chord{t}{x,p2,p2,n,p3,p3}{G/D}} %verificar porque parece m�s ien G/E o G/A o Em7algo
\newcommand{\Gnine}{\chord{t}{p3,p2,n,p0,p2,p2}{G9}}
\newcommand{\Gm}{\chord{t}{f1p3,p5,p5,f1p3,f1p3,f1p3,}{Gm}}
\newcommand{\Gmseven}{\chord{t}{f1p3,p5,p3,f1p3,f1p3,f1p3,}{Gm7}}

\newcommand{\Gs}{\chord{3}{x,x,p4,p3,p2,p2,}{G\#}}
\newcommand{\Aflat}{\chord{3}{x,x,p4,p3,p2,p2,}{A$\flat$}}
\newcommand{\Gsmseven}{\chord{t}{f2p4,x,f4p4,f4p4,f4p4,f4p4,}{G\#7}}
\newcommand{\Gssus}{\chord{t}{p4,p6,p6,p6,p4,p4}{Gsus4}}

% comandos para pintar acordes de piano
\newcommand{\APiano}{\keyboardf[Ao][Cso][Eo]}
\newcommand{\AsevenPiano}{\keyboardf[Ao][Cso][Eo][Go]}
\newcommand{\AmPiano}{\keyboardf[Ao][Co][Eo]}
\newcommand{\AmsevenPiano}{\keyboardf[Ao][Co][Eo][Go]}
\newcommand{\ACsPiano}{\keyboard[Cso][Eo][Ao]}

\newcommand{\AsPiano}{\keyboard[Do][Fo][Aso]}
\newcommand{\BflatPiano}{\keyboard[Do][Fo][Aso]}

\newcommand{\BPiano}{\keyboard[Dso][Fso][Bo]}
\newcommand{\BsevenPiano}{\keyboard[Dso][Fso][Bo][Ao]}
\newcommand{\BmPiano}{\keyboard[Do][Fso][Bo]}
\newcommand{\BmsevenPiano}{\keyboard[Do][Fso][Bo][Ao]}

\newcommand{\CPiano}{\keyboard[Co][Eo][Go]}
\newcommand{\CsevenPiano}{\keyboard[Co][Eo][Go][Aso]}
\newcommand{\CmPiano}{\keyboard[Co][Dso][Go]}

\newcommand{\CsPiano}{\keyboard[Cso][Fo][Gso]}
\newcommand{\CsmPiano}{\keyboard[Cso][Eo][Gso]}

\newcommand{\DPiano}{\keyboard[Do][Fso][Ao]}
\newcommand{\DsevenPiano}{\keyboard[Do][Fso][Ao][Co]}
\newcommand{\DmPiano}{\keyboard[Do][Fo][Ao]}
\newcommand{\DmsevenPiano}{\keyboard[Do][Fo][Ao][Co]}
\newcommand{\DAPiano}{\keyboardf[Ao][Do][Fso]}
\newcommand{\DFsPiano}{\keyboardf[Fso][Ao][Do]}

\newcommand{\DsPiano}{\keyboard[Dso][Go][Aso]}
\newcommand{\EflatPiano}{\keyboard[Dso][Go][Aso]}

\newcommand{\EPiano}{\keyboard[Eo][Gso][Bo]}
\newcommand{\EsevenPiano}{\keyboard[Eo][Gso][Bo][Do]}
\newcommand{\EmPiano}{\keyboard[Eo][Go][Bo]}
\newcommand{\EmsevenPiano}{\keyboard[Eo][Go][Bo][Do]}

\newcommand{\FPiano}{\keyboard[Co][Fo][Ao]}

\newcommand{\FsPiano}{\keyboard[Cso][Fso][Aso]}
\newcommand{\FsmPiano}{\keyboard[Cso][Fso][Ao]}
\newcommand{\FsmsevenPiano}{\keyboard[Cso][Fso][Ao][Eo]}

\newcommand{\GPiano}{\keyboard[Do][Go][Bo]}
\newcommand{\GsevenPiano}{\keyboard[Do][Fo][Go][Bo]}
\newcommand{\GBPiano}{\keyboardtwooctaves[Bo][Dt][Gt]}
\newcommand{\GDPiano}{\keyboard[Do][Go][Bo]}
\newcommand{\GmPiano}{\keyboard[Do][Go][Aso]}
\newcommand{\GmsevenPiano}{\keyboard[Do][Go][Aso][Fo]}

\newcommand{\GsPiano}{\keyboard[Dso][Gso][Co]}
\newcommand{\AflatPiano}{\keyboard[Dso][Gso][Co]}

%%%
% Title page information.
%%%
\title{Cuaderno de Himnos Tradicionales y Contempor�neos}
\author{Ruslan L\'opez}
\date{\'Ultima Revisi\'on:  \RevDate}

%%%
% Redefine fonts from SongBook style that I don't like.
%%%
\font\myTinySF=cmss8 at 8pt
\renewcommand{\CpyRtInfoFont}{\tiny\myTinySF}

%%%
% Define fonts to use in the headers and footers of the songbook.
%%%
\newcommand{\LHeadFont}{\normalsize}            % = cmr12  at 12pt
\newcommand{\CHeadFont}{\normalsize\rm}         % = cmr12  at 12pt
\newcommand{\RHeadFont}{\normalsize}            % = cmr12  at 12pt
\newcommand{\LFootFont}{\scriptsize}            % = cmr8   at  8pt
\newcommand{\CFootFont}{\tiny\myTinySF}         % = cmss8  at  8pt
\newcommand{\RFootFont}{\scriptsize}            % = cmr8   at  8pt

%%%
% Turn on and define fancy page heading/footing definition.
%%%
\pagestyle{fancy}

\ifChordBk
% It's a words & chords songbook...
\addtolength{\headwidth}{\marginparsep}
\addtolength{\headwidth}{\marginparwidth}
\renewcommand{\headrulewidth}{0.4pt}
\renewcommand{\footrulewidth}{0.4pt}
\fancyhead[LE,RO]{\LHeadFont\emph{\leftmark\/}\SBContinueMark}
\fancyhead[CE,CO]{\CHeadFont\thepage}
\fancyhead[RE,LO]{\RHeadFont\RelDate}
\else\ifOverhead
% It's an overhead...
\renewcommand{\footrulewidth}{0pt}
\renewcommand{\headrulewidth}{0pt}
\fancyhead[LE,RO]{}
\fancyhead[CE,CO]{}
\fancyhead[RE,LO]{}
\else\ifWordBk
% It's a words only songbook...
\addtolength{\headwidth}{\marginparsep}
\addtolength{\headwidth}{\marginparwidth}
\renewcommand{\headrulewidth}{0.4pt}
\renewcommand{\footrulewidth}{0.4pt}
\fancyhead[LE,RO]{\LHeadFont Estribillero}
\fancyhead[CE,CO]{\CHeadFont\thepage}
\fancyhead[RE,LO]{\RHeadFont\RelDate}
\fi\fi\fi

\fancyfoot[LE,RO]{\LFootFont Transcripciones}
\ifSongEject
\fancyfoot[CE,CO]{\CFootFont \RevDate}
\else
\fancyfoot[CE,CO]{\CFootFont}
\fi
\fancyfoot[RE,LO]{\RFootFont Todo el material son transcripciones personales.}

%%%
% Turn on/off index-file generation.  Uncomment the \makeindex line to
% turn index generation on;  comment it out to turn index generation
% off.
%%%
\makeTitleIndex         %% Title and First Line Index.
\makeTitleContents      %% Table of Contents.
\makeKeyIndex           %% Index of song by key.
\makeArtistIndex        %% Index of song by artist.
\makeindex

%%%%%%%%%%%%%%%%%%%%%%%%%%%%%%%%%%%%%%%%%%%%%%%%%%%%%%%%%%
%%%%%%%%%%%%%%%%%%%%%%%%%%%%%%%%%%%%%%%%%%%%%%%%%%%%%%%%%%
%%                                                      %%
%%           D O C U M E N T   B E G I N S              %%
%%                                                      %%
%%%%%%%%%%%%%%%%%%%%%%%%%%%%%%%%%%%%%%%%%%%%%%%%%%%%%%%%%%
%%%%%%%%%%%%%%%%%%%%%%%%%%%%%%%%%%%%%%%%%%%%%%%%%%%%%%%%%%
\begin{document}

%%%
% Uncomment "\maketitle" statement to make a title page.
%%%
    \maketitle
    %\mainmatter
    \ifWordBk
    \twocolumn
    \fi
%%%
% Songbook begins.
%%%
    \begin{song}{Abba Padre}{D}
    {} %copyright \SBPubDom
    {Marco Barrientos}
    {Romanos 8:15} %pasaje
    {\href{http://open.spotify.com/track/0yj0zBaa7Ckn6ZMQPmCmfF}{Escuchar}} %\NotCCLIed

%        \SBRef{No puedo parar de alabarte}{2006}%fuente \#
        \FLineIdx{Una Vez m�s}

        \begin{SBOptional}
            \Ch{D}{Una} vez m�s

            me a\Ch{Bm}{cer}co a T�

            con \Ch{Em}{li}bertad

            en adora\Ch{A}{ci\'o}n
        \end{SBOptional}

        \begin{SBOptional}
            T\'u e\Ch{D}{res} mi Dios

            tu \Ch{Bm}{hi}jo soy

            mi \Ch{Em}{co}muni\'on contigo

            es una \Ch{A}{dul}ce bendici\'on
        \end{SBOptional}

        \begin{SBChorus}
            // �\Ch{D}{A}bba Pa\Ch{G}{dre}! �\Ch{D}{A}bba Pa\Ch{G}{dre}!

            Es\Ch{D}{tar}\Ch{A/C#}{} conti\Ch{Bm}{go}

            \Ch{D/A}{es} una \Ch{Bm}{dul}ce bendi\Ch{A}{ci\'on}

            �\Ch{D}{A}bba Pa\Ch{G}{dre}! te \Ch{F#m}{a}mo Se\Ch{Bm}{�or}

            \Ch{G}{quie}ro estar en \Ch{D}{co}muni\'on

            \Ch{Em}{quie}ro es\Ch{A}{tar} con\Ch{D}{ti}go. //
        \end{SBChorus}
        \ifChordBk
        \begin{SBOpGroup}
            Acordes: \break

%\keyboardtwooctaves[Do][Fso][Ao]
            \upchord{\DPiano\D}{\qquad Re} Mayor \qquad\qquad
            \upchord{\BmPiano\Bm}{\qquad Si} Menor \hfill \break
            \upchord{\EmPiano\Em}{\qquad Mi} Menor\qquad\qquad
            \upchord{\APiano\A}{\qquad La} Mayor \hfill \break
            \upchord{\GPiano\G}{\qquad Sol} Mayor \qquad\qquad
            \upchord{\FsmPiano\Fsm}{\qquad Fa\#} Menor \hfill \break
            \upchord{\ACsPiano\ACs}{\qquad La} Mayor con bajo C\# \qquad\qquad
            \upchord{\DAPiano\DA}{\qquad Re} Mayor con Bajo La \qquad\qquad
        \end{SBOpGroup}
        \fi

    \end{song}

    \begin{song}{A Cristo solo a Cristo}{G}
    {Proyecto AA} %copyright \SBPubDom
    {Marcos Witt}
    {Hechos 4:12} %pasaje
    {\href{https://open.spotify.com/track/31LsSTJ8P7HZWu6KVpY9Fz}{Escuchar}} %\NotCCLIed

%  \renewcommand{\RevDate}{February~11,~1993}
%  \SBRef{No puedo parar de alabarte}{2006}%fuente \#

        \begin{SBOptional}
            \Ch{G}{A} Cristo \Ch{G7}{so}lo a \Ch{C}{C}risto\Ch{Bm}{~}, \Ch{Am}{yo} exalta\Ch{D}{r\'e}

            \Ch{G}{A} Cristo \Ch{G7}{so}lo a \Ch{C}{C}risto\Ch{Bm}{~}, \Ch{Am}{yo} adora\Ch{D}{r\'e}
        \end{SBOptional}

        \begin{SBOptional}
            \Ch{Am}{Por}que \'El me ha dado vida e\Ch{D}{ter}na

            \Ch{Am}{Por}que \'El me ha dado el po\Ch{D}{der}

            \Ch{Am}{Por}que \'El me ha dado la vic\Ch{D}{to}ria

            \Ch{D}{�l} \Ch{Em}{es} \Ch{D/F\#}{mi} \Ch{G}{Rey}.

            a \Ch{C}{Cris}to he \Ch{Bm}{pro}cla\Ch{Am}{ma}\Ch{D}{do} \Ch{G}{Rey}
        \end{SBOptional}
        \ifChordBk
        \begin{SBOpGroup}
            Acordes:
            \upchord{\GPiano\G}{\qquad Sol} Mayor \qquad\qquad
            \upchord{\GsevenPiano\Gseven}{\qquad Sol} Mayor S\'eptima \hfill \break
            \upchord{\CPiano\C}{\qquad Do} Mayor \qquad\qquad
            \upchord{\BmPiano\Bm}{\qquad Si} Menor \hfill \break
            \upchord{\AmPiano\Am}{\qquad La} Menor \qquad\qquad
            \upchord{\DPiano\D}{\qquad Re} Mayor \hfill \break
            \upchord{\EmPiano\Em}{\qquad Mi} Menor \qquad\qquad
            \upchord{\DFsPiano\DFs}{\qquad Re} Mayor con bajo F\# \hfill \break
        \end{SBOpGroup}
        \fi
    \end{song}

    \begin{song}{Doxolog\'ia}{G}
    {} %copyright \SBPubDom
    {Thomas Ken, Genevan Psalter, 1551, atr. a Louis Bourgeois}
    {Juan 17:22} %pasaje
    {\href{https://open.spotify.com/intl-es/track/24I9oe6qBsY3JHvL6g4yTT}{Escuchar}} %\NotCCLIed

        \FLineIdx{A Dios el Padre Celestial}

        \begin{SBVerse}
            \Ch{G}{A} Dios \Ch{D}{el} \Ch{Em}{Pa}\Ch{Bm}{dre} \Ch{Em}{Ce}\Ch{D}{les}\Ch{G}{tial},

            Al Hijo \Ch{D}{nues}\Ch{Em}{tro} \Ch{C}{Re}\Ch{G}{den}\Ch{D}{tor}.

            \Ch{Em}{Y} al \Ch{D}{E}\Ch{G}{ter}\Ch{D}{nal} \Ch{G}{Con}\Ch{C}{so}\Ch{Am7}{la}\Ch{G}{dor},

            uni\Ch{Em}{dos} \Ch{D}{to}\Ch{Am}{dos} \Ch{G/B}{a}\Ch{D}{la}\Ch{G}{bad}.

            \Ch{C}{A}\Ch{G}{m\'en}.
        \end{SBVerse}
        \ifChordBk
        \begin{SBOpGroup}
            Acordes:
            \upchord{\GPiano\G}{\qquad Sol} Mayor \qquad\qquad
            \upchord{\DPiano\D}{Re} Mayor \hfill \break
            \upchord{\EmPiano\Em}{\qquad Mi} Menor \qquad\qquad
            \upchord{\BmPiano\Bm}{\qquad Si} Menor \hfill \break
            \upchord{\CPiano\C}{\qquad Do} Mayor \qquad\qquad
            \upchord{\AmsevenPiano\Amseven}{\qquad La} Menor S\'eptima \hfill \break
            \upchord{\AmPiano\Am}{\qquad La} Menor \qquad\qquad
            \upchord{\GBPiano\GB}{\qquad Sol} con bajo B \hfill \break
        \end{SBOpGroup}
        \fi
    \end{song}


    \begin{song}{Admirable}{Dm7}
    {} %copyright \SBPubDom
    {Danilo Montero}
    {Apocalipsis 1:18} %pasaje
    {\href{https://open.spotify.com/intl-es/track/3EigVRcP0VQ5MhAUkCfZX8}{Escuchar}} %\NotCCLIed

%  \renewcommand{\RevDate}{February~11,~1993}
%  \SBRef{No puedo parar de alabarte}{2006}%fuente \#
        \FLineIdx{Con poder y autoridad}
        \begin{SBOpGroup}
            \Ch{Dm7}{Con} poder y au\Ch{Bb}{to}ridad \Ch{F}{nues}tro Dios ven\Ch{C}{ci\'o} a la
            muerte

            \Ch{Dm7}{So}bre el trono \Ch{Bb}{ce}lestial \Ch{Am7}{siem}pre reina\Ch{Dm7}{r\'a}.

            \Ch{Bb}{Sen}tado en \Ch{C}{ma}jestad \Ch{Bb}{su}yo es el reino por los
            \Ch{F}{Si}\Ch{C}{glos} \Ch{Bb}{y} por la \Ch{C}{e}ternidad \Ch{Bb}{su} luz de \Ch{Gm7}{glo}ria brilla\Ch{Am7}{r\'a}.
        \end{SBOpGroup}

        \begin{SBChorus}
            Admi\Ch{C}{ra}\Ch{Dm7}{ble}, conse\Ch{C}{je}\Ch{Dm7}{ro} \Ch{Bb}{mi} Dios \Ch{C}{con}sola\Ch{Dm7}{dor},\Ch{Am7}{}

            Eres \Ch{C}{dig}\Ch{Dm7}{no} de a\Ch{C}{la}ban\Ch{Dm7}{za}, \Ch{Bb}{Pr�n}ci\Ch{C}{pe} de \Ch{Dm7}{paz}.
        \end{SBChorus}
        \ifChordBk
        \begin{SBOpGroup}
            Acordes:
            \upchord{\DmsevenPiano\Dmseven}{Re} menor s\'eptima \qquad\qquad
            \upchord{\BflatPiano\Bflat}{\qquad Si bemol} Mayor \hfill \break
            \upchord{\FPiano\F}{\qquad Fa} Mayor \qquad\qquad
            \upchord{\CPiano\C}{\qquad Do} Mayor \hfill \break
            \upchord{\AmsevenPiano\Amseven}{\qquad La} Menor S\'eptima \qquad\qquad
            \upchord{\GmsevenPiano\Gmseven}{\qquad Sol} Menor S\'eptima \hfill \break
        \end{SBOpGroup}
        \fi
    \end{song}

    \begin{song}{A Nuestro Padre Dios}{F}
    {} %copyright \SBPubDom
    {An\'onimo en Thesaurus Musicus 1744}
    {Juan 3:16} %pasaje
    {} %\NotCCLIed

%  \renewcommand{\RevDate}{February~11,~1993}
%  \SBRef{No puedo parar de alabarte}{2006}%fuente \#

        \begin{SBVerse}
            \Ch{F}{A} \Ch{Dm}{nues}\Ch{Gm}{tro} \Ch{C}{Pa}\Ch{Dm7}{dre} \Ch{C}{Dios}

            \Ch{F}{Al}\Ch{Dm}{ce}\Ch{Gm}{mos} \Ch{F}{nues}\Ch{C7}{tra} \Ch{Dm}{voz}

            �\Ch{Gm}{Glo}\Ch{F}{ria} \Ch{C}{a} \Ch{F}{\'El}!

            Tal \Ch{Am}{fu�} su a\Ch{F}{mor} que di\'o

            \Ch{C7}{Al} hijo que \Ch{F}{mu}\Ch{C}{ri\'o,}

            \Ch{F}{En} \Ch{Bb}{quien} con\Ch{F}{f�o} yo;

            �\Ch{Bb}{Glo}\Ch{F}{ria} \Ch{C7}{a} \Ch{F}{\'El}!
        \end{SBVerse}

        \begin{SBVerse}
            \Ch{F}{A} \Ch{Dm}{nues}\Ch{Gm}{tro} \Ch{C}{Sal}\Ch{Dm}{va}\Ch{C}{dor}

            \Ch{F}{De}\Ch{Dm}{mos} \Ch{Gm}{con} \Ch{F}{fe} \Ch{C7}{lo}\Ch{Dm}{or}

            �\Ch{Gm}{Glo}ria \Ch{C}{a} \Ch{F}{\'El}!

            Su \Ch{Am}{san}gre \Ch{F}{de}rram\'o

            \Ch{C7}{Con} ella me \Ch{F}{la}\Ch{C}{v\'o,}

            \Ch{F}{Y} el \Ch{Bb}{cie}lo \Ch{F}{me} abri\'o

            �\Ch{Bb}{Glo}\Ch{F}{ria} \Ch{C7}{a} \Ch{F}{\'El}!
        \end{SBVerse}

        \begin{SBVerse}
            \Ch{F}{Es}\Ch{Dm}{p�}\Ch{Gm}{ri}\Ch{C}{tu} \Ch{Dm7}{de} \Ch{C}{Dios},

            \Ch{F}{E}\Ch{Dm}{le}\Ch{Gm}{vo} a Ti \Ch{C7}{mi} \Ch{Dm}{voz};

            �\Ch{Gm}{Glo}\Ch{F}{ria} \Ch{C}{a} \Ch{F}{Ti}!

            Con \Ch{Am}{ce}les\Ch{F}{tial} fulgor

            \Ch{C7}{Me} muestras el \Ch{F}{a}\Ch{C}{mor}

            \Ch{F}{De} \Ch{Bb}{Cris}to \Ch{F}{mi} Se�or

            �\Ch{Bb}{Glo}\Ch{F}{ria} \Ch{C7}{a} \Ch{F}{Ti}!
        \end{SBVerse}

        \begin{SBVerse}
            \Ch{F}{Con} \Ch{Dm}{go}\Ch{Gm}{zo} y \Ch{Dm}{a}\Ch{C}{mor},

            \Ch{F}{Can}\Ch{Dm}{te}\Ch{Gm}{mos} con \Ch{C7}{fer}\Ch{Dm}{vor}

            \Ch{Gm}{Al} \Ch{F}{Tri}\Ch{C}{no} \Ch{F}{Dios}.

            En \Ch{Am}{la} e\Ch{F}{ter}nidad

            \Ch{C7}{Mo}ra la Tri\Ch{F}{ni}\Ch{C}{dad};

            �\Ch{F}{Por} \Ch{Bb}{siem}pre \Ch{F}{a}labad

            \Ch{Bb}{Al} \Ch{F}{Tri}\Ch{C7}{no} \Ch{F}{Dios}!
        \end{SBVerse}

        \ifChordBk
        \begin{SBOpGroup}
            Acordes:
            \upchord{\FPiano\F}{\qquad Fa} Mayor \qquad\qquad
            \upchord{\DmPiano\Dm}{\qquad Re} menor  \hfill \break
            \upchord{\GmPiano\Gm}{\qquad Sol} Menor \qquad\qquad
            \upchord{\CPiano\C}{\qquad Do} Mayor \hfill \break
            \upchord{\DmsevenPiano\Dmseven}{\qquad Re} menor s\'eptima \qquad\qquad
            \upchord{\CsevenPiano\Cseven}{\qquad Do} Mayor s\'eptima \hfill \break
            \upchord{\AmPiano\Am}{\qquad La} Menor \qquad\qquad
            \upchord{\BflatPiano\Bflat}{\qquad Si bemol} Mayor \hfill \break
        \end{SBOpGroup}
        \fi
    \end{song}


    \begin{song}{Adonai}{Em}
    {} %copyright \SBPubDom
    {Marcos Witt}
    {} %pasaje
    {} %\NotCCLIed

%  \renewcommand{\RevDate}{February~11,~1993}
%  \SBRef{No puedo parar de alabarte}{2006}%fuente \#

        \begin{SBChorus}
            \Ch{Em}{Oh} Ado\Ch{B7}{nai}. Oh Ado\Ch{Em}{nai}.
            \Ch{C}{Dios} \Ch{D}{del} Uni\Ch{Em}{ver}so, Se\Ch{B7}{�or} de la Crea\Ch{Em}{ci�n}.
        \end{SBChorus}

        \begin{SBVerse}
            Los \Ch{D}{cie}los cuentan tu \Ch{G}{glo}ria,
            tus \Ch{D}{hi}jos hoy te a\Ch{G}{do}ran,
            por \Ch{B7}{to}das \Ch{Em}{tus} mara\Ch{Am}{vi}llas, Ado\Ch{B7}{nai}.
        \end{SBVerse}

        \ifChordBk
        \begin{SBOpGroup}
            Acordes:
            \upchord{\EmPiano\Em}{\qquad Mi} Menor \qquad\qquad
            \upchord{\BsevenPiano\Bseven}{\qquad Si} S\'eptima \hfill \break
            \upchord{\CPiano\C}{\qquad Do} Mayor \qquad\qquad
            \upchord{\DPiano\D}{Re} Mayor \hfill \break
            \upchord{\GPiano\G}{\qquad Sol} Mayor \qquad\qquad
            \upchord{\AmPiano\Am}{\qquad La} Menor \hfill \break
        \end{SBOpGroup}
        \fi
    \end{song}

    \begin{song}{Ahora Que Estoy Contigo}{A}
    {\SBPubDom} %copyright \SBPubDom
    {}
    {} %pasaje
    {} %\NotCCLIed

        \begin{SBVerse}
            \Ch{A}{A}hora que estoy con\Ch{D}{ti}go en tus \Ch{A}{bra}zos de amor,

            \Ch{F\#m}{pue}do escuchar de \Ch{D}{T�} y de m� un \Ch{E7}{la}tido

            \Ch{A}{el} tuyo dice ``siempre te \Ch{D}{a}mar�'', el m�o ``\Ch{A}{te} adorar�'',

            que \Ch{F\#m}{dul}ce comu\Ch{D}{ni�n} estar con\Ch{E7}{ti}go.

            \Ch{D}{A}hora que estoy conti\Ch{A}{go}, ante tus \Ch{G}{pies}, oh Padre, yo
            me rin\Ch{D}{do},

            y en \Ch{Dm7}{mi} necesidad nada te \Ch{Em}{pe}dir�, solo te a\Ch{G}{do}rar�
        \end{SBVerse}

        \ifChordBk
        \begin{SBOpGroup}
            Acordes:
            \upchord{\APiano\A}{\qquad La} Mayor \qquad\qquad
            \upchord{\DPiano\D}{\qquad Re} Mayor \hfill \break
            \upchord{\FsmPiano\Fsm}{\qquad Fa\#} Menor \qquad\qquad
            \upchord{\EsevenPiano\Eseven}{\qquad Mi} S\'eptima \hfill \break
            \upchord{\DmsevenPiano\Dmseven}{Re} menor s\'eptima \qquad\qquad
            \upchord{\EmPiano\Em}{\qquad Mi} Menor \hfill \break
            \upchord{\GPiano\G}{\qquad Sol} Mayor \qquad\qquad
        \end{SBOpGroup}
        \fi
    \end{song}

    \begin{song}{Aleluya}{G}
    {} %copyright \SBPubDom
    {Jes�s Adri�n Romero}
    {} %pasaje
    {} %\NotCCLIed

        \SBIntro[N]{\Ch{G}{~} \Ch{Dm7/G}{~} \Ch{C}{~}}

        \FLineIdx{}

        \begin{SBOpGroup}
            En el cielo y en la tierra te alabamos OH Se\Ch{G}{�or}.

            Eres \Ch{G}{dig}no de alabanza y de suprema adoraci�n.

            Te proclamamos Se�or. Te proclamamos Se\Ch{G}{�or}.
        \end{SBOpGroup}

        \begin{SBChorus}
            Ale\Ch{G}{lu}ya, Alelu\Ch{Em7}{ya}, Ale\Ch{C2}{lu}ya \Ch{Am7}{al} Se\Ch{Dsus4}{�or}.
            Ale\Ch{G}{lu}ya, Alelu\Ch{G}{ya}, Aleluya al Se\Ch{G}{�or}.
        \end{SBChorus}

        \begin{SBOpGroup}
            \Ch{G}{En} la cruz por m� te diste para darme liber\Ch{G}{tad}.

            De la tumba resurgiste y en tu trono ahora es\Ch{G}{t�s}.

            OH Jes�s te proclamamos Se�or.te proclamamos Se�or
        \end{SBOpGroup}

        \ifChordBk
        \begin{SBOpGroup}
            Acordes:
            \upchord{\GPiano\G}{\qquad Sol} Mayor \qquad\qquad
            \upchord{\DPiano\D}{Re} Mayor \hfill \break
            \upchord{\EmsevenPiano\Emseven}{\qquad Mi} Menor S\'eptima \qquad\qquad
        \end{SBOpGroup}
        \fi
    \end{song}


    \begin{song}{Al que me ci�e de poder}{E}
    {} %copyright \SBPubDom
    {Jes�s Adri�n Romero}
    {} %pasaje
    {} %\NotCCLIed

        \begin{SBChorus}
            Al que me ci�e de po\Ch{E}{der}

            a aqu�l que mi victoria \Ch{C\#m}{es}

            s�lo a �l alaba\Ch{A}{r�}, \Ch{B7}{~}s�lo a �l exalta\Ch{E}{r�}
        \end{SBChorus}

        \begin{SBOpGroup}
            De t� ser� mi alabanza
            en la congregaci�n
            cantar� y alabar�
            tu nombre Se�or
        \end{SBOpGroup}

        \ifChordBk
        \begin{SBOpGroup}
            Acordes:
            \upchord{\EPiano\E}{\qquad Mi} Mayor \qquad\qquad
            \upchord{\CsmPiano\Csm}{\qquad Do} Sostenido Menor \hfill \break
            \upchord{\APiano\A}{\qquad La} Mayor \qquad\qquad
            \upchord{\BsevenPiano\Bseven}{\qquad Si} S\'eptima \hfill \break
        \end{SBOpGroup}
        \fi
    \end{song}


    \begin{song}{Al Trono Majestuoso}{Eb}
    {} %copyright \SBPubDom
    {Aurelia \& Samuel sebastian Wesley}
    {} %pasaje
    {} %\NotCCLIed

        \begin{SBChorus}
            \Ch{Eb}{Al} trono majestuoso
            \Ch{Eb}{Del} Dios omnipoten\Ch{Eb}{te},
            Hu\Ch{Eb}{mil}des vuestra frente,
            naciones inclinad
            �l es el ser supre\Ch{Eb}{mo},
            Se�or de cuanto existe,
            y nada al fin \Ch{Eb}{res}iste
            Al grande Jeho\Ch{Eb}{v�}
        \end{SBChorus}

        \begin{SBOpGroup}
            Del polvo de la tierra
            formonos complacida
            su mano, y dionos vida
            su aliento creador.
            Y al vernos despu�s ciegos,
            en la maldad sumidos,
            Cual padre a hijos queridos
            Salud nos provey�.
        \end{SBOpGroup}

        \begin{SBOpGroup}
            La gratitud sincera
            nos dictar� canciones
            y en coro dulces sones
            al cielo subir�n
            con los celestes himnos
            arm�nica alianza
            formando, su alabanza
            doquier resonar�.
        \end{SBOpGroup}

        \begin{SBOpGroup}
            Se�or, a tu palabra
            los mundos obedecen,
            y del mortal perecen
            la ciencia y altivez.
            Tu amor y verdad solos
            en nada habr�n menguado,
            despu�s que hayan cesado
            los siglos de correr.
        \end{SBOpGroup}

        \ifChordBk
        \begin{SBOpGroup}
            Acordes:
            \upchord{\EflatPiano\Eflat}{\qquad Mi} bemol Mayor \qquad\qquad
        \end{SBOpGroup}
        \fi
    \end{song}

    \begin{song}{Alza tus ojos}{Am}
    {} %copyright \SBPubDom
    {Marcos Barrientos}
    {} %pasaje
    {} %\NotCCLIed


        \begin{SBOpGroup}
            \Ch{Am}{Al}za tus ojos y \Ch{F}{mi}ra, \Ch{C}{la} cosecha est� \Ch{E}{lis}ta

            el tiempo ha llegado la mies est� madura
            esfu�rzate y s� valiente lev�ntate y predica
            a todas las naciones que Cristo es la Vida
        \end{SBOpGroup}

        \begin{SBChorus}
            Y ser� llena la tierra de su gloria
            se cubrir� como las aguas cubren la mar
        \end{SBChorus}

        \begin{SBOpGroup}
            No, no hay otro nombre,
            dado a los hombres,
            Jesucristo es el Se�or
        \end{SBOpGroup}

        \ifChordBk
        \begin{SBOpGroup}
            Acordes:
            \upchord{\AmPiano\Am}{\qquad La} Menor \qquad\qquad
            \upchord{\FPiano\F}{\qquad Fa} Mayor \hfill \break
            \upchord{\CPiano\C}{\qquad Do} Mayor \qquad\qquad
            \upchord{\EPiano\E}{\qquad Mi} Mayor \hfill \break
            \upchord{\GPiano\G}{\qquad Sol} Mayor \qquad\qquad
            \upchord{\DmPiano\Dm}{\qquad Re} Menor \hfill \break
        \end{SBOpGroup}
        \fi
    \end{song}

    \begin{song}{Aqu� estoy}{A}
    {} %copyright \SBPubDom
    {Jes�s Adri�n Romero}
    {} %pasaje
    {} %\NotCCLIed


        \begin{SBOpGroup}
            A\Ch{A}{qu�} estoy, te ofrezco todo lo que soy
            Aqu� estoy, un sacrificio quiero ser
            Toma mi ser, mi vida entrego a T�
        \end{SBOpGroup}

        \begin{SBOpGroup}

            Porque T� eres mi Dios,
            Eres digno de adoraci�n
            Una ofrenda de amor ser�
            Para T�

        \end{SBOpGroup}

        \ifChordBk
        \begin{SBOpGroup}
            Acordes:
            \upchord{\APiano\A}{\qquad La} Mayor \qquad\qquad
        \end{SBOpGroup}
        \fi
    \end{song}

    \begin{song}{A sus pies}{Am}
    {} %copyright \SBPubDom
    {Jes�s Adri�n Romero}
    {} %pasaje
    {} %\NotCCLIed


        \begin{SBOpGroup}
            Cuando el mundo te inunda de fatalidad
            y te agobia la vida con su mucho af�n
            y se llena tu alma de preocupaci�n
            y se seca la fuente de tu coraz�n
        \end{SBOpGroup}

        \begin{SBChorus}
            Puedes sentarte a sus pies
            y de sus manos beber
            la plenitud que tu alma necesita
            puedes sentarte a sus pies
            y cada d�a tener
            una nueva canci�n y nueva vida
            a sus pies hay paz,
            gracia y bendici�n
            a sus pies tendr�s
            luz y direcci�n.
            la plenitud en �l
            nunca se agotar�
            puedes descansar en su presencia
        \end{SBChorus}

        \begin{SBOpGroup}
            Cuando quieras huir por que no puedes m�s
            por que s�lo te sientes entre los dem�s
            y no hay m�s en tus ojos brillo y emoci�n
            y se cierra tu boca por que no hay canci�n
        \end{SBOpGroup}



        \ifChordBk
        \begin{SBOpGroup}
            Acordes:
            \upchord{\AmPiano\Am}{\qquad La} Menor \qquad\qquad
        \end{SBOpGroup}
        \fi
    \end{song}

    \begin{song}{Bendice Nuestra Ofrenda}{G}
    {} %copyright \SBPubDom
    {D. Tinoco, G Franc}
    {} %pasaje
    {} %\NotCCLIed

        \begin{SBOpGroup}
            \Ch{G}{Ben}di\Ch{D}{ce} \Ch{Em}{Nues}\Ch{Bm}{tra} o\Ch{Em}{fren}\Ch{D}{da} oh \Ch{G}{Dios},

            Que pre\Ch{D}{sen}ta\Ch{Em}{mos} \Ch{C}{con} \Ch{G}{a}\Ch{D}{mor}.

            \Ch{Em}{Es} \Ch{D}{prue}\Ch{G}{ba} \Ch{D}{fiel} \Ch{G}{de} \Ch{C}{gra}\Ch{Am7}{ti}\Ch{G}{tud},

            Por tu \Ch{Em}{bon}\Ch{D}{dad} \Ch{Am}{en} ple\Ch{D}{ni}\Ch{G}{tud}.

            \Ch{C}{A}\Ch{G}{m�n}
        \end{SBOpGroup}

        \ifChordBk
        \begin{SBOpGroup}
            Acordes:
            \upchord{\GPiano\G}{\qquad Sol} Mayor \qquad\qquad
            \upchord{\DPiano\D}{\qquad Re} Mayor \hfill \break
            \upchord{\EmPiano\Em}{\qquad Mi} Menor \qquad\qquad
            \upchord{\BmPiano\Bm}{\qquad Si} Menor \hfill \break
            \upchord{\CPiano\C}{\qquad Do} Mayor \qquad\qquad
            \upchord{\AmsevenPiano\Amseven}{\qquad La} Menor S\'eptima \hfill \break
            \upchord{\AmPiano\Am}{\qquad La} Menor \qquad\qquad
            \upchord{\GBPiano\GB}{\qquad Sol} Mayor con bajo B \hfill \break
        \end{SBOpGroup}
        \fi
    \end{song}

    \begin{song}{Bueno es alabar}{G}
    {} %copyright \SBPubDom
    {Danilo Montero}
    {} %pasaje
    {\href{https://open.spotify.com/intl-es/track/6S90J1ENA9sgzfBDEHNyJ7}{Escuchar}}

        \begin{SBOpGroup}
            \Ch{G}{Bue}no es ala\Ch{C}{bar} �oh Se\Ch{D}{�or}!, tu \Ch{C}{nom}\Ch{D}{bre}

            \Ch{G}{dar}te honra, \Ch{C}{glo}ria y ho\Ch{D}{nor} por \Ch{C}{siem}\Ch{D}{pre},

            \Ch{G}{Bue}no es ala\Ch{C}{bar}te Je\Ch{D}{s�s}

            y go\Ch{Am7}{zar}me en tu po\Ch{D}{der}
        \end{SBOpGroup}

        \begin{SBChorus}
            Porque \Ch{G}{gran}de \Ch{C}{e}res \Ch{D}{T�}

            \Ch{G}{gran}des \Ch{C}{son} tus o\Ch{D}{bras},

            porque \Ch{G}{gran}de \Ch{C}{e}res \Ch{D}{T�}

            grande es tu a\Ch{Em}{mor}, grande es tu \Ch{D}{glo}ria
        \end{SBChorus}
        \ifChordBk
        \begin{SBOpGroup}
            Acordes:
            \upchord{\GPiano\G}{\qquad Sol} Mayor \qquad\qquad
            \upchord{\CPiano\C}{\qquad Do} Mayor  \hfill \break
            \upchord{\DPiano\D}{\qquad Re} Mayor \qquad\qquad
            \upchord{\AmsevenPiano\Amseven}{\qquad La} Menor S\'eptima \hfill \break
            \upchord{\EmPiano\Em}{\qquad Mi} Menor \qquad\qquad
        \end{SBOpGroup}
        \fi
    \end{song}

    \begin{song}{Cada d�a con Cristo}{D}
    {} %copyright \SBPubDom
    {}
    {Isa�as 26:3} %pasaje
    {\href{https://open.spotify.com/intl-es/track/0EtXTYDHwK81dnsloMPNha}{Escuchar}} %\NotCCLIed

        \begin{SBOpGroup}
            \Ch{D}{Ca}da d�a con Cristo me llena de perfecta \Ch{Em}{paz}

            \Ch{A7}{ca}da d�a con \Ch{Em}{Cris}to le amo \Ch{A7}{m�s} y m�s

            \Ch{D}{\'El} me salva y guarda y \Ch{D7}{s�} que pronto volve\Ch{G}{r�}

            y vi\Ch{Gm}{vir} con \Ch{D}{Cris}to m�s \Ch{A}{dul}ce cada d�a se\Ch{D}{r�}
        \end{SBOpGroup}

        \ifChordBk
        \begin{SBOpGroup}
            Acordes:
            \upchord{\DPiano\D}{\qquad Re} Mayor \qquad\qquad
            \upchord{\EmPiano\Em}{\qquad Mi} Menor \hfill \break
            \upchord{\AsevenPiano\Aseven}{\qquad La} Mayor S\'eptima \qquad\qquad
            \upchord{\DsevenPiano\Dseven}{\qquad Re} Mayor S\'eptima \hfill \break
            \upchord{\GPiano\G}{\qquad Sol} Mayor \qquad\qquad
            \upchord{\GmPiano\Gm}{\qquad Sol} Menor \hfill \break
            \upchord{\APiano\A}{\qquad La} Mayor \qquad\qquad
        \end{SBOpGroup}
        \fi
    \end{song}

    \begin{song}{Cada Ma�ana}{D}
    {} %copyright \SBPubDom
    {Jes�s Adri�n Romero}
    {} %pasaje
    {} %\NotCCLIed

        \begin{SBOpGroup}
            \Ch{D}{Ca}da ma�ana al despertar,
            y por la noche al descansar,
            agradezco tus bondades a mi vida
            por todo lo que me permites disfrutar
        \end{SBOpGroup}

        \begin{SBChorus}
            /// Ale-lu-u-ya ///
            // Agradecido estoy por tu bondad.//
            Agradecido estoy por tu bondad.
        \end{SBChorus}

        \ifChordBk
        \begin{SBOpGroup}
            Acordes:
            \upchord{\FsPiano\Fs}{\qquad Fa} Sostenido Mayor \qquad\qquad
            \upchord{\APiano\A}{\qquad La} Mayor \hfill \break
            \upchord{\DPiano\D}{\qquad Re} Mayor \qquad\qquad
            \upchord{\GPiano\G}{\qquad Sol} Mayor \hfill \break
            \upchord{\BmsevenPiano\Bmseven}{\qquad Si} Menor S\'eptima \qquad\qquad
            \upchord{\EmPiano\Em}{\qquad Mi} Menor \hfill \break
            \upchord{\DsevenPiano\Dseven}{\qquad Re} Mayor S\'eptima \qquad\qquad
        \end{SBOpGroup}
        \fi
    \end{song}

    \begin{song}{Cantar� de tu Amor}{F}
    {} %copyright \SBPubDom
    {Danilo Montero}
    {} %pasaje
    {\href{https://open.spotify.com/intl-es/track/7kzqc0u8SNQAU5Wca0WNKo}{Escuchar}} %\NotCCLIed

%        \SBRef{El aire de tu casa}{2005}%fuente \#

        \begin{SBOpGroup}
            \Ch{F}{Por} mucho tiempo bus\Ch{C/E}{qu�}

            \Ch{F}{u}na raz�n de vi\Ch{C/E}{vir}

            \Ch{F}{en} medio de \Ch{G}{mil} pre\Ch{Am7}{gun}tas

            \Ch{F}{tu} amor me respon\Ch{G}{di�}
        \end{SBOpGroup}


        \begin{SBOpGroup}
            \Ch{F}{A}hora veo la \Ch{C/E}{luz}

            \Ch{F}{y} ya no tengo te\Ch{C/E}{mor}

            \Ch{F}{tu} reino \Ch{G}{vi}no a mi \Ch{Am7}{vi}da

            \Ch{F}{y} ahora vivo para \Ch{G}{T�}
        \end{SBOpGroup}

        \begin{SBChorus}
            Canta\Ch{C}{r�} de tu a\Ch{G/B}{mor} rendi\Ch{Am7}{r�} mi cora\Ch{C}{z�n} ante \Ch{F}{t�}

            tu se\Ch{C}{r�s} mi pa\Ch{G/B}{si�n} y mis \Ch{Am7}{pa}sos se guia\Ch{C}{r�n} por tu \Ch{F}{voz}

            mi Je\Ch{Dm}{s�s} y mi \Ch{G}{Rey} de tu \Ch{F}{gran} amor canta\Ch{G}{r�}
        \end{SBChorus}

        \ifChordBk
        \begin{SBOpGroup}
            Acordes:
            \upchord{\FPiano\F}{\qquad Fa} Mayor \qquad\qquad
            \upchord{\CEPiano\CE}{\qquad Do} Mayor Bajo E  \hfill \break
            \upchord{\GPiano\G}{\qquad Sol} Mayor \qquad\qquad
            \upchord{\AmsevenPiano\Amseven}{\qquad La} Menor S\'eptima \hfill \break
            \upchord{\GBPiano\GB}{\qquad Sol} Mayor Bajo B \qquad\qquad\qquad\qquad\qquad
            \upchord{\DmPiano\Dm}{\qquad Re} Menor \hfill \break
            \upchord{\CPiano\C}{\qquad Do} Mayor \qquad\qquad
        \end{SBOpGroup}

        \fi
    \end{song}

    \begin{song}{Cara A Cara}{D}
    {Vidal Music} %copyright \SBPubDom
    {Marcos Vidal}
    {} %pasaje
    {} %\NotCCLIed


        \begin{SBChorus}
            Solamente una palabra,
            solamente una oraci�n,
            cuando llegue a Tu presencia, oh Se�or,
            no me importa en que lugar
            de la mesa me hagas sentar,
            o el color de mi corona, si la llego a ganar

            S�lo d�jame mirarte, cara a cara,
            y perderme como un ni�o en Tu mirada,
            y que pase mucho tiempo,
            y que nadie diga nada,
            por que estoy viendo al Maestro,
            cara a cara
        \end{SBChorus}

        \begin{SBOpGroup}
            Solamente una palabra,
            si es que a�n me queda voz,
            y si logro articularla en Tu presencia,
            no te quiero hacer preguntas, s�lo una petici�n,
            y si puede ser a solas, mucho mejor
        \end{SBOpGroup}

        \ifChordBk
        \begin{SBOpGroup}
            Acordes:
            \upchord{\DPiano\D}{\qquad Re} Mayor \qquad\qquad
        \end{SBOpGroup}
        \fi
    \end{song}


    \begin{song}{Cerca de T�}{G}
    {Vidal Music} %copyright \SBPubDom
    {Marcos Vidal}
    {} %pasaje
    {} %\NotCCLIed

        \begin{SBOpGroup}
            Si decidiera negar mi fe
            y no con ar nunca m�s en �l
            no tengo a donde ir, no tengo a donde ir
        \end{SBOpGroup}


        \begin{SBOpGroup}
            Si despreciare en mi coraz�n
            la santa gracia que me salv�
            no tengo a donde ir, no tengo a donde ir
        \end{SBOpGroup}

        \begin{SBOpGroup}
            Convencido estoy que sin tu amor se acabar�an mis
            fuerzas
            y sin T� mi coraz�n sediento se muere, se seca
        \end{SBOpGroup}

        \begin{SBOpGroup}
            Cerca de T�, yo quiero estar
            de tu presencia no me quiero alejar
        \end{SBOpGroup}

        \ifChordBk
        \begin{SBOpGroup}
            Acordes:
            \upchord{\DPiano\D}{\qquad Re} Mayor \qquad\qquad
        \end{SBOpGroup}
        \fi
    \end{song}

    \begin{song}{�C�mo Podr� Estar Triste?}{C}
    {} %copyright \SBPubDom
    {}
    {} %pasaje
    {} %\NotCCLIed

        \begin{SBOpGroup}
            C�mo podr� estar triste,
            c�mo entre sombras ir,
            c�mo sentirme solo
            y en el dolor vivir,
            siCristo es mi consuelo,
            mi amigo siempre el,
            // si aun las aves tienen
            Seguro asilo en �l //
        \end{SBOpGroup}

        \begin{SBChorus}
            Feliz cantando alegre
            yo vivo siempre aqu�;
            si El cuida de las aves
            cuidar� tambi�n de m�.
        \end{SBChorus}

        \begin{SBOpGroup}
            Nunca te desalientes,
            oigo al Se�or decir,
            y en su Palabra ado
            hago al dolor huir.
            A Cristo paso a paso
            yo sigo sin cesar,
            //y todas sus bondades
            me da sin limitar
        \end{SBOpGroup}

        \begin{SBOpGroup}
            Siempre que soy tentado
            o que en la sombra estoy,
            m�s cerca de �l camino
            y protegido voy.
            Si en mi la fe desmaya
            y caigo en la ansiedad
            //tan s�lo El me levanta,
            me da seguridad//
        \end{SBOpGroup}

        \ifChordBk
        \begin{SBOpGroup}
            Acordes:
            \upchord{\DPiano\D}{\qquad Re} Mayor \qquad\qquad
        \end{SBOpGroup}
        \fi
    \end{song}

    \begin{song}{Con C�nticos Se�or}{C}
    {} %copyright \SBPubDom
    {}
    {} %pasaje
    {} %\NotCCLIed

        \begin{SBOpGroup}

            Con c�nticos Se�or mi coraz�n y voz
            te adoran con fervor oh trino Santo Dios
            Tu mano paternal marc� mi senda aqu�
            mis pasos, cada cual, velados son por T�
            Innumerables son tus bienes y sin par
            que por tu compasi�n recibo sin cesar
            T� eres oh Se�or mi sumo todo bien
            mil lenguas tu amor cantando siempre est�n

        \end{SBOpGroup}

        \begin{SBChorus}

            En tu Mansi�n yo te ver�
            de t� perd�n fel�z tendr�


            En tu Mansi�n yo te ver�
            y galard�n fel�z tendr�

        \end{SBChorus}

        \ifChordBk
        \begin{SBOpGroup}
            Acordes:
            \upchord{\DPiano\D}{\qquad Re} Mayor \qquad\qquad
        \end{SBOpGroup}
        \fi
    \end{song}

    \begin{song}{Con Manos Vac�as}{E}
    {} %copyright \SBPubDom
    {Jes�s Adri�n Romero}
    {} %pasaje
    {} %\NotCCLIed

        \begin{SBOpGroup}
            \Ch{E}{Con} manos vac�as \Ch{A}{ven}go a T�

            \Ch{B}{no} tengo nada que dar\Ch{C\#m}{te}

            \Ch{F\#m}{no} hay nada de valor en \Ch{C\#m}{m�}

            \Ch{A}{no} puedo im\Ch{F\#m}{pre}sio\Ch{B}{nar}te
        \end{SBOpGroup}

        \begin{SBOpGroup}
            \Ch{E}{Te} puedo entregar mi \Ch{A}{co}raz�n,

            \Ch{B}{pe}ro est� quebran\Ch{C\#m}{ta}do

            \Ch{F\#m}{re}c�belo mi buen Pas\Ch{C\#m}{tor},

            \Ch{A}{Tu} puedes \Ch{F\#m}{res}tau\Ch{B}{rar}lo
        \end{SBOpGroup}


        \begin{SBChorus}
            \Ch{A}{Pon}go mi \Ch{B}{vi}da a tu servicio Se\Ch{C\#m}{�or}

            \Ch{A}{no} ser� \Ch{B}{mu}cho, pero la entrego \Ch{C\#m}{hoy}

            \Ch{A}{y} si mis \Ch{B}{ma}nos hoy vac�as es\Ch{C\#m}{t�n},

            \Ch{F\#m7}{pue}des lle\Ch{C\#m/G\#}{nar}las con tu \Ch{C\#m}{gran} po\Ch{B}{der} y a\Ch{A}{mor}.

            \Ch{B}{U}sa mis manos Se\Ch{C\#m}{�or}
        \end{SBChorus}

        \ifChordBk
        \begin{SBOpGroup}
            Acordes:
            \upchord{\EPiano\E}{\qquad Mi} Mayor \qquad\qquad
            \upchord{\APiano\A}{\qquad La} Mayor \hfill \break
            \upchord{\BPiano\B}{\qquad Si} Mayor \qquad\qquad
            \upchord{\CsmPiano\Csm}{\qquad Do sostenido} Menor \hfill \break
            \upchord{\FsmPiano\Fsm}{\qquad Fa sostenido} Menor \qquad\qquad
            \upchord{\GssusPiano\Gssus}{\qquad Sol sostenido} suspendida cuarta \hfill \break
            \upchord{\CsmPiano\Csm}{\qquad Do sostenido} Menor \qquad\qquad
            \upchord{\FsmsevenPiano\Fsmseven}{\qquad Fa sostenido} S\'eptima Menor \qquad\qquad
        \end{SBOpGroup}
        \fi

        \begin{SBExtraKeys}{
            \STitle{Con Manos Vac�as}{F}

            \begin{SBOpGroup}
                \Ch{F}{Con} manos vac�as \Ch{Bb}{ven}go a T� \Ch{C}{no} tengo nada que darte

                no hay nada de valor en m� \Ch{Bb}{no} puedo impresio\Ch{C}{nar}te
            \end{SBOpGroup}

            \begin{SBOpGroup}
                \Ch{F}{Te} puedo entregar mi \Ch{Bb}{co}raz�n,

                \Ch{C}{pe}ro est� quebrantado

                rec�belo mi buen Pastor,

                \Ch{Bb}{Tu} puedes restau\Ch{C}{rar}lo
            \end{SBOpGroup}

            \begin{SBChorus}
                \Ch{Bb}{Pon}go mi \Ch{C}{vi}da a tu servicio Se�or

                \Ch{Bb}{no} ser� \Ch{C}{mu}cho, pero la entrego hoy

                \Ch{Bb}{y} si mis \Ch{C}{ma}nos hoy vac�as est�n,

                puedes llenarlas con tu gran po\Ch{C}{der} y a\Ch{Bb}{mor}.

                \Ch{C}{U}sa mis manos Se�or
            \end{SBChorus}

            \ifChordBk
            \begin{SBOpGroup}
                Acordes:
                \upchord{\FPiano\F}{\qquad Fa} Mayor \qquad\qquad
                \upchord{\BflatPiano\Bflat}{\qquad Si bemol} Mayor \hfill \break
                \upchord{\CPiano\C}{\qquad Do} Mayor \qquad\qquad
            \end{SBOpGroup}
            \fi
        }\end{SBExtraKeys}
    \end{song}

    \begin{song}{Con mi Dios}{Em}
    {} %copyright \SBPubDom
    {}
    {} %pasaje
    {} %\NotCCLIed

        \begin{SBOpGroup}
            Con mi Dios yo saltar� los muros
            Con mi Dios Ej�rcitos derribar
        \end{SBOpGroup}

        \begin{SBOpGroup}
            �l adiestra mis manos para la batalla,
            puedo tomar con mis manos el arco de bronce
        \end{SBOpGroup}

        \begin{SBOpGroup}
            �l es escudo, la roca m�a,
            �l es la fuerza de mi salvaci�n
            Mi alto refugio, mi fortaleza
            �l es mi libertador
        \end{SBOpGroup}


        \ifChordBk
        \begin{SBOpGroup}
            Acordes:
            \upchord{\EmPiano\Em}{\qquad Mi} Menor \qquad\qquad
        \end{SBOpGroup}
        \fi
    \end{song}

    \begin{song}{Con Mis Labios}{D}
    {} %copyright \SBPubDom
    {}
    {} %pasaje
    {} %\NotCCLIed

        \begin{SBOpGroup}
            Con mis labios y mi vida
            te alabo Se�or, te alabo Se�or,
            con mis labios y mi vida te alabo bendito Se�or,
        \end{SBOpGroup}

        \begin{SBChorus}
            Porque T� has sido precioso para m�, precioso
            para m�,precioso para m�,
            Porque T� has sido precioso para m�, te alabo
            bendito Se�or
        \end{SBChorus}

        \ifChordBk
        \begin{SBOpGroup}
            Acordes:
            \upchord{\DPiano\D}{\qquad Re} Mayor \qquad\qquad
        \end{SBOpGroup}
        \fi
    \end{song}

    \begin{song}{Con Mis Manos Levantadas}{G}
    {} %copyright \SBPubDom
    {}
    {} %pasaje
    {} %\NotCCLIed

        \begin{SBOpGroup}
            Con mis manos levantadas hacia el cielo me
            Presento ante Ti hoy Se�or para recibir de Ti
            La fuerza y el poder para vivir junto a Ti.
        \end{SBOpGroup}

        \begin{SBChorus}
            Llenas hoy mi coraz�n con Tu presencia
            Llenas de alegr�a y paz todo mi ser
            De cualquier necesidad T� me responder�s
            Porque me amas, me amas
        \end{SBChorus}

        \ifChordBk
        \begin{SBOpGroup}
            Acordes:
            \upchord{\GPiano\G}{\qquad Sol} Mayor \qquad\qquad
        \end{SBOpGroup}
        \fi
    \end{song}

    \begin{song}{Cristo me Ayuda por �l a Vivir}{F}
    {} %copyright \SBPubDom
    {}
    {} %pasaje
    {} %\NotCCLIed

        \begin{SBOpGroup}
            Cristo me ayuda por �l a vivir
            Cristo me ayuda por �l a morir;
            Hasta que llegue su gloria a ver,
            Cada momento le entrego mi ser
        \end{SBOpGroup}

        \begin{SBChorus}
            Cada momento la vida me da,
            Cada momento conmigo �l est�;
            Hasta que llegue su gloria a ver,
            Cada momento le entrego ni s�r.
        \end{SBChorus}

        \begin{SBOpGroup}
            Siento pesares, muy cerca �l est�,
            Siento dolores, a livio me da;
            Tengo aflicciones, me muestra su amor;
            Cada momento me cuidas Se�or.
        \end{SBOpGroup}

        \begin{SBOpGroup}
            Tengo amarguras, o tengo temor
            Tengo tristezas, me inspira valor;
            Tengo conflictos o penas aqu�
            Cada momento te acuerdas de m�
        \end{SBOpGroup}

        \begin{SBOpGroup}
            Tengo aquezas, o d�bil estoy,
            Cristo de dice Tu amparo yo soy ;
            Cada momento en tinieblas o en luz,
            Siempre conmigo est� mi Jes�s.
        \end{SBOpGroup}

        \ifChordBk
        \begin{SBOpGroup}
            Acordes:
            \upchord{\FPiano\F}{\qquad Fa} Mayor \qquad\qquad
        \end{SBOpGroup}
        \fi
    \end{song}

    \begin{song}{Cristo t� me has amado}{G}
    {} %copyright \SBPubDom
    {}
    {} %pasaje
    {Ingrid Rosario} %\NotCCLIed

        \begin{SBOpGroup}
            Cristo t� me has amado
            Cristo nunca de t� me apartar�
            Del pecado T� me has rescatado
            Mis pies pusiste en la roca y yo s� que
        \end{SBOpGroup}

        \begin{SBChorus}
            Te amo, por siempre
            Aunque el mundo alrededor pueda cambiar
            Tu eres mi Salvador
            Yo te adorar� por la eternidad.
        \end{SBChorus}

        \ifChordBk
        \begin{SBOpGroup}
            Acordes:
            \upchord{\GPiano\G}{\qquad Sol} Mayor \qquad\qquad
        \end{SBOpGroup}
        \fi
    \end{song}

    \begin{song}{Cu�n bello es el Se�or}{D}
    {} %copyright \SBPubDom
    {}
    {} %pasaje
    {Ingrid Rosario} %\NotCCLIed

        \begin{SBOpGroup}
            //Cu�n bello es el Se�or
            Cu�n hermoso es el Se�or
            //Cu�n bello es el Se�or
            Hoy le quiero adorar//
        \end{SBOpGroup}

        \begin{SBChorus}
            La belleza de mi Se�or.
            Nunca se agotar�
            La hermosura de mi Se�or
            Siempre resplandecer�
        \end{SBChorus}

        \ifChordBk
        \begin{SBOpGroup}
            Acordes:
            \upchord{\DPiano\D}{\qquad Re} Mayor \qquad\qquad
        \end{SBOpGroup}
        \fi
    \end{song}

    \begin{song}{Dad a Dios Inmortal alabanza}{D}
    {} %copyright \SBPubDom
    {}
    {} %pasaje
    {} %\NotCCLIed

        \begin{SBOpGroup}
            Dad a \Ch{D}{Dios} inmortal ala\Ch{G/D}{ban}\Ch{D}{za};
            Su merced, su verdad \Ch{G}{nos} i\Ch{A}{nun}da;
            \Ch{D}{Es} su gracia en prodigios fe\Ch{G/D}{cun}\Ch{D}{da},
            Sus mer\Ch{C\#}{ce}des, humildes can\Ch{F\#m}{tad}
            �\Ch{D}{Al} Se\Ch{A7}{�or} de se�ores dad \Ch{D}{glo}ria,
            Rey de reyes, poder \Ch{A}{sin} se\Ch{D}{gun}do!
            Mori\Ch{A}{r�n} los se�ores del \Ch{D}{mun}do,
            mas su reino no a\Ch{A7}{ca}ba Ja\Ch{D}{m�s}.
        \end{SBOpGroup}

        \begin{SBOpGroup}
            Las naciones vi� en vicios sumidas
            Y sinti� compasi�n en su seno;
            De prodigios de gracia est� lleno;
            Sus mercedes, humildes cantad
            A su pueblo llev� por la mano
            A la tierra por �l prometida
            Por los siglos sin n le da vida
            y el pecado y la muerte caer�n.
        \end{SBOpGroup}

        \begin{SBOpGroup}
            A su Hijo envi� por salvarnos
            Del pecado y la muerte inherente:
            De prodigios de gracia es torrente,
            Sus mercedes, humildes cantad
            Por el mundo su mano nos lleva
            Y al celeste descanso nos gu�a;
            Su bondad vivir� eterno d�a,
            Cuando el mundo no exista ya m�s
        \end{SBOpGroup}

        \ifChordBk
        \begin{SBOpGroup}
            Acordes:
            \upchord{\DPiano\D}{\qquad Re} Mayor \qquad\qquad
            \upchord{\GDPiano\GD}{\qquad Sol} Mayor Bajo D \hfill \break
            \upchord{\APiano\A}{\qquad La} Mayor \qquad\qquad
            \upchord{\CsPiano\Cs}{\qquad Do sostenido} Mayor \hfill \break
            \upchord{\FsmPiano\Fsm}{\qquad Fa sostenido} Menor \qquad\qquad
            \upchord{\AsevenPiano\Aseven}{\qquad La} Mayor S\'eptima \hfill \break
        \end{SBOpGroup}
        \fi
    \end{song}

    \begin{song}{Damos Honor a Ti}{E}
    {} %copyright \SBPubDom
    {}
    {} %pasaje
    {} %\NotCCLIed

        \begin{SBOpGroup}
            Damos honor a Ti, damos honor a Ti.
            Creador de vida eres T�.
            Damos honor a Ti, damos honor a Ti.
            Porque no hay otro Dios como T�.
            Rey de reyes, Admirable.
            Eres el Principio y Fin.
            Soberano y sublime.
            Eres nuestro Salvador.

        \end{SBOpGroup}

        \ifChordBk
        \begin{SBOpGroup}
            Acordes:
            \upchord{\EPiano\E}{\qquad Mi} Mayor \qquad\qquad
        \end{SBOpGroup}
        \fi
    \end{song}

    \begin{song}{DeGloriaEnGloria}{D}
    {} %copyright \SBPubDom
    {}
    {} %pasaje
    {Marcos Witt} %\NotCCLIed

        \begin{SBOpGroup}
            De \Ch{D}{glo}ria en \Ch{A/C\#}{glo}ria te \Ch{Bm}{veo} \Ch{Am7}{~}
            cuanto m�s te conozco
            Quiero saber m�s de T�
        \end{SBOpGroup}

        \begin{SBOpGroup}
            Mi \Ch{D}{Dios} cuan buen alfarero
            Quebr�ntame transf�rmame mold�ame a tu imagen,
            Se�or
        \end{SBOpGroup}

        \begin{SBOpGroup}
            Quiero ser mas como Tu
            ver la vida como Tu
            Saturarme de tu esp�ritu
        \end{SBOpGroup}

        \SBEnd{y reflejar al mundo tu a\Ch{D}{mor}}

        \ifChordBk
        \begin{SBOpGroup}
            Acordes:
            \upchord{\DPiano\D}{\qquad Re} Mayor \qquad\qquad
            \upchord{\ACsPiano\ACs}{\qquad La} Mayor con bajo C\# \hfill \break
            \upchord{\BmPiano\Bm}{\qquad Si} Menor \qquad\qquad
            \upchord{\AmsevenPiano\Amseven}{\qquad La} Menor S\'eptima \hfill \break
        \end{SBOpGroup}
        \fi
    \end{song}

    \begin{song}{Delante de tu Trono}{G}
    {} %copyright \SBPubDom
    {}
    {} %pasaje
    {Marco Barrientos}

        \begin{SBOpGroup}
            // De\Ch{G}{lan}te de tu trono
            Se\Ch{Em}{�or} yo quiero estar
            \Ch{C}{pa}ra contemplar
            tu hermo\Ch{D}{su}ra y santidad //
        \end{SBOpGroup}

        \begin{SBOpGroup}
            Y de\Ch{G}{ci}\Ch{D}{rte}: Te \Ch{Em}{a}mo
            Y decir\Ch{G}{te}: Te a\Ch{Am}{do}\Ch{D}{ro}
            Y de\Ch{G}{cir}\Ch{D}{te}: Te \Ch{Em}{a}mo
            Y que eres todo \Ch{D}{pa}ra \Ch{G}{m�}.
        \end{SBOpGroup}

        \ifChordBk
        \begin{SBOpGroup}
            Acordes:
            \upchord{\GPiano\G}{\qquad Sol} Mayor \qquad\qquad
            \upchord{\EmPiano\Em}{\qquad Mi} Menor \hfill \break
            \upchord{\CPiano\C}{\qquad Do} Mayor \qquad\qquad
            \upchord{\DPiano\D}{\qquad Re} Mayor \hfill \break
            \upchord{\AmPiano\Am}{\qquad La} Menor \qquad\qquad
        \end{SBOpGroup}
        \fi
    \end{song}

    \begin{song}{Demos gracias al Se�or}{C}
    {} %copyright \SBPubDom
    {}
    {} %pasaje
    {}

        \begin{SBOpGroup}
            // Demos gracias al Se�or, demos gracias,
            demos gracias por su amor //
        \end{SBOpGroup}

        \begin{SBOpGroup}
            Por la ma�ana, las aves cantan
            sus alabanza a Dios el Creador,
            tambi�n nosotros a �l cantemos
            y alabemos a Cristo el Redentor
        \end{SBOpGroup}

        \ifChordBk
        \begin{SBOpGroup}
            Acordes:
            \upchord{\CPiano\C}{\qquad Do} Mayor \qquad\qquad
        \end{SBOpGroup}
        \fi
    \end{song}

    \begin{song}{Me dice que me ama}{G}
    {} %copyright \SBPubDom
    {Jes�s Adri�n Romero}
    {} %pasaje
    {\NotCCLIed} %\NotCCLIed

%        \SBRef{El aire de tu casa}{2005}%fuente \#

        \SBIntro[N]{\Ch{G}{~} \Ch{C}{~} \Ch{Am}{~} \Ch{D}{~}}

        \begin{SBOpGroup}
            Me dice que me ama cuando escucho llover,
            G C
            me dice que ama con un atardecer,
            G C Em D
            lo dice sin palabras, con las olas del mar,
            Em C D
            lo dice en la ma�ana con mi respirar.
        \end{SBOpGroup}

        \begin{SBChorus}
            G C G C G
            //Me dice que me ama y que conmigo quiere estar,
            C Am7 D
            me dice que me busca cuando salgo yo a pasear,
            Am7 C Em
            que ha hecho lo que existe para llamar mi atenci�n,
            Am D G C D
            que quiere conquistarme y alegrar mi coraz�n.
        \end{SBChorus}

        \begin{SBOpGroup}
            G C
            Me dice que me ama cuando veo la cruz,
            G C
            sus manos extendidas as� tan grande es su amor,
            G C Em D
            lo dicen las heridas de sus manos y pies,
            Em C D
            me dice que me ama una y otra vez.
        \end{SBOpGroup}
        \ifChordBk
        \begin{SBOpGroup}
            Acordes:
            \upchord{\GPiano\G}{\qquad Sol} Mayor \qquad\qquad
        \end{SBOpGroup}

        \fi
    \end{song}

    \begin{song}{Par\'abola}{G}
    {} %copyright \SBPubDom
    { Marcos Vidal }
    {Lucas 10:30} %pasaje
    {\href{https://youtu.be/cxkfqxpuICM}{Escuchar}} %\NotCCLIed

%  \renewcommand{\RevDate}{February~11,~1993}
%  \SBRef{No puedo parar de alabarte}{2006}%fuente \#

        \begin{SBOpGroup}
            \Ch{G# maj7}{////}\Ch{Gm}{in}tro////\Ch{G7}{}
        \end{SBOpGroup}

        \begin{SBVerse}

            regre\Ch{Ddis}{sa}ba a casa un poco mas tem\Ch{G7}{pra}no de lo nor\Ch{Cm}{mal}

            cuando \Ch{Ddis}{vi\'o} que sobre �l ve\Ch{G7}{ni}an \Ch{Cm}{tres}

            y na\Ch{G# maj7}{va}ja en mano le \Ch{Gm}{a}tacaron sin contempla\Ch{Fm}{ci\'on}

            le de\Ch{Bb}{ja}ron incons\Ch{Bb7/B}{cien}te bajo el \Ch{Cm}{sol}
        \end{SBVerse}
        \begin{SBOpGroup}
            \Ch{G# maj7}{}\Ch{Gm}{}\Ch{G7}{}
        \end{SBOpGroup}
        \begin{SBVerse}
            y ca  \Ch{Ddis}{mi}no de la i\Ch{G7}{gle}sia iba el pas\Ch{Cm}{tor} poco despu�s

            la reu\Ch{Ddis}{nion} ya estaba a \Ch{G7}{pun}to de empe\Ch{Cm}{zar}

            iba \Ch{G#maj7}{tar}de y discu\Ch{Gm}{tien}do en el ca\Ch{Fm}{mi}no con su mujer

            inten\Ch{Bb}{tan}do no per\Ch{Bb7/B}{der} su autori\Ch{Cm}{dad}


            \Ch{G#maj7}{ay} si el maestro nos vol\Ch{Bbmaj7}{vie}ra a contar

            alguna his\Ch{Gm}{to}ria que nos hiciera \Ch{Cm}{re}capacitar

            piensa\Ch{Gm}{lo} piensa\Ch{Cm}{lo}

        \end{SBVerse}
        \begin{SBOpGroup}
            \Ch{G# maj7}{}\Ch{Gm}{}\Ch{G7}{}
        \end{SBOpGroup}

        \begin{SBVerse}

            tres mi\Ch{Ddis}{nu}tos mas y el \Ch{G7}{l�}der de ala\Ch{Cm}{ban}za aparecio

            los te\Ch{Ddis}{cla}dos siete \Ch{G7}{ca}bles y un a\Ch{Cm}{tril}

            y aunque \Ch{G#maj7}{si} le pare\Ch{Gm}{ci\'o} ver algo \Ch{Fm}{ro}jo en el arcel

            prefi\Ch{Bb}{ri\'o} pasar de \Ch{Bb7/B}{lar}go y de per\Ch{Cm}{fil}
        \end{SBVerse}
        \begin{SBVerse}

            un gi\Ch{D#dis}{ta}no despei\Ch{G#7}{na}do que pa\Ch{Cm#}{sa}ba por ahi

            no sa\Ch{D#dis}{b�a} ni \Ch{G#7}{leer} ni escri\Ch{Cm#}{bir}

            pero al \Ch{Amaj7}{ver} el pano\Ch{Gm#7}{ra}ma le do\Ch{Fm#7}{li\'o} en el coraz\'on

            y acer\Ch{B}{c�n}dose hasta el \Ch{B7/C}{hom}bre le ayud\Ch{Cm#}{\'o}

            \Ch{G#maj7}{ay} si el maestro nos vol\Ch{Bbmaj7}{vie}ra a contar

            alguna his\Ch{Gm}{to}ria que nos hiciera \Ch{Cm}{re}capacitar

            piensa\Ch{Gm}{lo} piensa\Ch{Cm}{lo}

        \end{SBVerse}


        \ifChordBk
        \begin{SBOpGroup}
            Acordes:

            \upchord{\keyboard[Do][Fo][Go][Bo]\Gs}{Sol} Mayor S\'eptima
        \end{SBOpGroup}
        \fi
    \end{song}

    \begin{song}{Grita, canta, danza}{Cm}
    {\SBPubDom} %copyright \SBPubDom
    {}
    {} %pasaje
    {} %\NotCCLIed

%  \renewcommand{\RevDate}{February~11,~1993}
%  \SBRef{No puedo parar de alabarte}{2006}%fuente \#

        \begin{SBChorus}
            \Ch{Cm}{Gri}ta, canta, danza alegremente en su pre\Ch{Ab}{sen}cia

            Gira, salta dando vueltas para \Ch{Bb}{Cris}to

            �l vive y \Ch{G7}{vi}ve para siempre es el \Ch{Cm}{Rey}\Ch{G7}{}
        \end{SBChorus}

        \begin{SBOpGroup}
            Te alaba\Ch{Cm}{r�}, te exaltar� y te agra\Ch{Bb}{de}cer�

            Tu \Ch{Eb}{gran}de amor Jes�s

            Cam\Ch{Ab}{bias}te mi lamento en ala\Ch{Cm}{ban}za

            Sa\Ch{Bb}{nas}te mi \Ch{G7}{he}rido cora\Ch{Cm}{z�n} \Ch{G7}{}
        \end{SBOpGroup}

        \ifChordBk
        \begin{SBOpGroup}
            Acordes:
            \upchord{\CmPiano\Cm}{\qquad Do} Menor \qquad\qquad
            \upchord{\AflatPiano\Aflat}{La bemol} Mayor \hfill \break
            \upchord{\BflatPiano\Bflat}{\qquad Si bemol} Mayor \hfill \break
        \end{SBOpGroup}
        \fi
    \end{song}

    \begin{song}{As� como David danzaba}{Am}
    {\SBPubDom} %copyright \SBPubDom
    {}
    {} %pasaje
    {} %\NotCCLIed

%  \renewcommand{\RevDate}{February~11,~1993}
%  \SBRef{No puedo parar de alabarte}{2006}%fuente \#

        \begin{SBOpGroup}
            \Ch{Am}{Cuan}do el Se�or hiciere volver la cautivi\Ch{G}{dad}

            seremos \Ch{Dm}{co}mo los que sue\Ch{E}{�an}
        \end{SBOpGroup}

        \begin{SBOpGroup}

            \Ch{Am}{Mi} boca llenar� de risa, \Ch{G}{mis} labios de alabanza,

            \Ch{Dm}{En}tonces dir�n las naciones:

            \Ch{E}{Gran}des cosas ha hecho el Se�or
        \end{SBOpGroup}

        \begin{SBOpGroup}
            Me goza\Ch{Am}{r�}, me gozar�, me gozar�,

            me gozar� en Jeho\Ch{G}{v�}.  [�G�zate!]

            Pues ha lle\Ch{Dm}{va}do todo dolor, me ha hecho \Ch{E}{li}bre
        \end{SBOpGroup}

        \begin{SBChorus}
            \Ch{Am}{A}s� como David cantaba, \Ch{G}{a}s� como David danzaba,

            \Ch{Dm}{a}s� como David flu�a en su pre\Ch{E}{sen}cia
        \end{SBChorus}
        \ifChordBk
        \begin{SBOpGroup}
            Acordes:
            \upchord{\AmPiano\Am}{\qquad La} Menor \qquad\qquad
            \upchord{\GPiano\G}{\qquad Sol} Mayor \hfill \break
            \upchord{\DmPiano\Dm}{\qquad Re} Menor \qquad\qquad
        \end{SBOpGroup}
        \fi
    \end{song}

    %\begin{document}

\ifguitarra

\lhead{\LHeadFont Acodes~para~Guitarra}
\chead{\CHeadFont ({\rm\thepage})}
\rhead{\RHeadFont\RelDate}
{\parindent 8pt
        {\myTitleFont --- Acodes para Guitarra ---}}\par
\vskip 20pt
\textbf{Acodes Mayores}

%\small{El s\'imbolo \# significa sostenido y {\flat}~significa~bemol}
\small
\upchord{\A}{La Mayor} \upchord{\B}{Si Mayor} \upchord{\C}{Do Mayor} \upchord{\D}{Re Mayor} \upchord{\E}{Mi Mayor} \upchord{\F}{Fa Mayor} \upchord{\G}{Sol Mayor}

\upchord{\As}{A\#/$B\flat$ Mayor} \upchord{\Cs}{C\#/$D\flat$ Mayor} \upchord{\Ds}{D\#/$E\flat$ Mayor}  \upchord{\Fs}{F\#/$G\flat$ Mayor} \upchord{\Gs}{G\#/$A\flat$ Mayor} \upchord{\As}{A\#/$B\flat$ Mayor}
\normalsize

\textbf{Acodes Menores}

\small
\upchord{\Am}{La} Menor \upchord{\Bm}{Si} Menor \upchord{\Cm}{Do} Menor \upchord{\Dm}{Re} Menor \upchord{\Em}{Mi} Menor \upchord{\Fm}{Fa} Menor \upchord{\Gm}{Sol} Menor

\upchord{\Asm}{\small{A\#/B\flat Menor}} \upchord{\Csm}{\small{C\#/D\flat Menor}} \upchord{\Dsm}{\small{D\#/E\flat Menor}}  \upchord{\Fsm}{\small{F\#/G\flat Menor}} \upchord{\Gsm}{\small{G\#/A\flat Menor}} \upchord{\Asm}{\small{A\#/B\flat Menor}}
\normalsize

\vskip 20pt
\textbf{Acodes Mayores S\'eptima}

\upchord{\Aseven}{La} Mayor s\'eptima
\upchord{\Bflatseven}{Si} bemol Mayor s\'eptima
\upchord{\Bseven}{Si} Mayor s\'eptima
\upchord{\Cseven}{\small{Do Mayor s\'eptima}}
\upchord{\Csseven}{\small{Do sostenidoMayor s\'eptima}}
\upchord{\Dseven}{\small{Re Mayor s\'eptima}}
\upchord{\Eflatseven}{\small{Mi bemol Mayor s\'eptima}}
\upchord{\Eseven}{\small{Mi Mayor s\'eptima}}
\upchord{\Fseven}{\small{Fa Mayor s\'eptima}}
\upchord{\Gseven}{\small{Sol Mayor s\'eptima}}
\vskip 20pt

\textbf{Acodes Menores S\'eptima}

\small
\upchord{\Amseven}{La} Menor s\'eptima
\upchord{\Bmseven}{Si} Menor s\'eptima
\upchord{\Cmseven}{Do} Menor s\'eptima
\upchord{\Csmseven}{Do} Menor s\'eptima
\upchord{\Dmseven}{Re} Menor s\'eptima
\upchord{\Dsmseven}{Re} Sostenido Menor s\'eptima
\upchord{\Emseven}{Mi} Menor s\'eptima
\upchord{\Emseventr}{Mi} Menor s\'eptima
\upchord{\Fmseven}{Fa} Menor s\'eptima
\upchord{\Fsmseven}{Fa sostenido} Menor s\'eptima
\upchord{\Gmseven}{Sol} Menor s\'eptima
\upchord{\Gsmseven}{Sol} Sostenido Menor s\'eptima
\upchord{\Bflatmseven}{Si bemol} Menor s\'eptima
\normalsize

\vskip 20pt
\textbf{Acodes Mayores Suspendido cuarta}
\vskip 25pt

\small
\upchord{\Asus}{La} Suspendida cuarta
\upchord{\Bsus}{Si} Suspendida cuarta
\upchord{\Csus}{Do} Suspendida cuarta
\upchord{\Dsus}{Re} Suspendida cuarta
\upchord{\Esus}{Re} Suspendida cuarta
\upchord{\Fsus}{Fa} Suspendida cuarta
\upchord{\Gsus}{Sol} Suspendida cuarta

\upchord{\Fssus}{Fa} sostenido Suspendida cuarta
\upchord{\Gssus}{Sol sostenido} Suspendida cuarta
\normalsize

\vskip 20pt
\textbf{Acodes Mayor Aumentada}
\vskip 25pt

\small
\upchord{\CMaj}{Do} Maj
\upchord{\DMaj}{Re} Maj
\upchord{\GMaj}{Sol} Maj
\normalsize

\vskip 20pt
\textbf{Acodes Mayor S\'eptima Aumentada}
\vskip 25pt

\small
\upchord{\AsevenMaj}{La} Maj S\'eptima Aumentada
\upchord{\Fmajseven}{Fa} Maj S\'eptima Aumentada
\normalsize

\vskip 20pt
\textbf{Acodes Aumentada 2}
\vskip 25pt

\small
\upchord{\Atwo}{La} Aumentada 2
\upchord{\Ctwo}{Do} Aumentada 2
\normalsize

\vskip 20pt
\textbf{Acodes Novena}
\vskip 25pt

\small
\upchord{\Cnine}{Do} Novena
\upchord{\Gnine}{Sol} Novena
\normalsize

\vskip 20pt
\textbf{Acodes Disminuidos}
\vskip 25pt

\small
\upchord{\Gsdim}{Sol} sostenido disminuido
\normalsize


\vskip 20pt
\textbf{Acodes Con Bajo cambiado}

\small
\upchord{\AAs}{La Mayor bajo Bb}
\upchord{\ACs}{La Mayor bajo C\#}
\upchord{\AEg}{La Mayor bajo E}
\upchord{\AmF}{La Menor bajo F}
\vskip 20pt
\upchord{\CE}{Do Mayor bajo E}
\upchord{\CG}{Do Mayor bajo G}
\upchord{\DflatF}{Re bemol bajo F}
\upchord{\DA}{Re Mayor bajo A}
\vskip 20pt
\upchord{\DE}{Re Mayor bajo E}
\upchord{\DFs}{Re Mayor bajo F\#}
\upchord{\EGs}{Mi Mayor Bajo G\#}
\vskip 20pt
\upchord{\GB}{Sol Mayor Bajo B}
\upchord{\GD}{Sol Mayor Bajo D}
\upchord{\GE}{Sol Mayor Bajo E}
\upchord{\AflatC}{La bemol Bajo C}
\vskip 20pt
\upchord{\AflatEflat}{La bemol Bajo Eb}
\upchord{\DsusFs}{Re} Suspendida cuarta bajo F\#
\normalsize

\vskip 20pt
\textbf{Acodes semidisminuidos}

\small
\upchord{\Bmsevenbfive}{Si} Menor s\'eptima semidisminuido
\normalsize

\vskip 20pt
\textbf{Acodes 13 suspendida cuarta}

\small
\upchord{\Csusthirteen}{Do} 13 suspendida cuarta
\normalsize

\clearpage
\fi

\ifpiano
\lhead{\LHeadFont Acodes~para~Piano}
{\parindent 8pt
        {\myTitleFont --- Acordes para Piano ---}}\par
\vskip 20pt
\textbf{Acodes Mayores}
\vskip 25pt

%\small{El s\'imbolo \# significa sostenido y {\flat}~significa~bemol}
\small
\upchord{\APiano}{\qquad La Mayor} \qquad\qquad \upchord{\BPiano}{Si Mayor} \qquad\qquad \upchord{\CPiano}{\qquad Do Mayor} \qquad\qquad \upchord{\DPiano}{\qquad Re Mayor} \hfill \break
\vskip 25pt
\upchord{\EPiano}{\qquad Mi Mayor} \qquad\qquad  \upchord{\FPiano}{\qquad Fa Mayor} \qquad\qquad \upchord{\GPiano}{\qquad Sol Mayor}
\vskip 25pt
\upchord{\AsPiano}{A\#/$B\flat$ Mayor} \qquad\qquad \upchord{\CsPiano}{C\#/$D\flat$ Mayor} \qquad\qquad \upchord{\DsPiano}{D\#/$E\flat$ Mayor} \qquad\qquad \upchord{\FsPiano}{F\#/$G\flat$ Mayor} \hfill \break
\vskip 25pt
\upchord{\GsPiano}{G\#/$A\flat$ Mayor} \qquad\qquad \upchord{\AsPiano}{A\#/$B\flat$ Mayor}
\normalsize

\textbf{Acodes Menores}
\vskip 25pt

\small
\upchord{\AmPiano}{\qquad La} Menor \qquad\qquad \upchord{\BmPiano}{\qquad Si} Menor \qquad\qquad \upchord{\CmPiano}{\qquad Do} Menor \qquad\qquad \upchord{\DmPiano}{\qquad Re} Menor \hfill \break
\vskip 25pt
\upchord{\EmPiano}{\qquad Mi} Menor \qquad\qquad \upchord{\FmPiano}{\qquad Fa} Menor \qquad\qquad \upchord{\GmPiano}{\qquad Sol} Menor
\vskip 25pt
\upchord{\AsmPiano}{\small{A\#/B\flat Menor}}  \qquad\qquad  \upchord{\CsmPiano}{C\#/D\flat Menor}  \qquad\qquad  \upchord{\DsmPiano}{D\#/E\flat Menor} \qquad\qquad \upchord{\FsmPiano}{F\#/G\flat Menor} \hfill \break
\vskip 25pt
\upchord{\GsmPiano}{G\#/A\flat Menor}  \qquad\qquad  \upchord{\AsmPiano}{A\#/B\flat Menor}
\normalsize

\clearpage
%\vskip 20pt
\textbf{Acodes Mayores S\'eptima}
\vskip 25pt

\small
\upchord{\AsevenPiano}{La Mayor s\'eptima} \qquad\qquad \upchord{\BsevenPiano}{Si Mayor s\'eptima} \qquad\qquad \upchord{\CsevenPiano}{Do Mayor s\'eptima} \qquad\qquad
\vskip 25pt
\upchord{\DsevenPiano}{Re Mayor s\'eptima} \qquad\qquad \upchord{\EsevenPiano}{Mi Mayor s\'eptima} \qquad\qquad \upchord{\FsevenPiano}{Fa Mayor s\'eptima}
\vskip 25pt
\upchord{\GsevenPiano}{Sol Mayor s\'eptima}  \qquad\qquad \upchord{\BflatsevenPiano}{Si} bemol Mayor s\'eptima
\vskip 25pt
\upchord{\EflatsevenPiano}{Mi bemol Mayor s\'eptima} \qquad\qquad
\normalsize
\vskip 20pt

\textbf{Acodes Menores S\'eptima}
\vskip 25pt

\small
\upchord{\AmsevenPiano}{La} Menor s\'eptima
\upchord{\BmsevenPiano}{Si} Menor s\'eptima
\upchord{\CmsevenPiano}{Do} Menor s\'eptima
\vskip 25pt
\upchord{\DmsevenPiano}{Re} Menor s\'eptima
\upchord{\EmsevenPiano}{Mi} Menor s\'eptima
\upchord{\FmsevenPiano}{Fa} Menor s\'eptima
\vskip 25pt
\upchord{\GmsevenPiano}{Sol} Menor s\'eptima
\upchord{\BflatmsevenPiano}{Si bemol} Menor s\'eptima
\upchord{\FsmsevenPiano}{Fa sostenido} Menor s\'eptima
\vskip 25pt
\upchord{\CssevenPiano}{Do sostenido} Mayor s\'eptima
\upchord{\DsmsevenPiano}{Re} Sostenido Menor s\'eptima
\upchord{\GsmsevenPiano}{Sol} Sostenido Menor s\'eptima
\normalsize

\vskip 20pt

\textbf{Acodes Mayores Suspendido cuarta}
\vskip 25pt

\small
\upchord{\AsusPiano}{La} Suspendida cuarta
\upchord{\BsusPiano}{Si} Suspendida cuarta
\upchord{\CsusPiano}{Do} Suspendida cuarta
\vskip 25pt
\upchord{\DsusPiano}{Re} Suspendida cuarta
\upchord{\EsusPiano}{Re} Suspendida cuarta
\upchord{\FsusPiano}{Fa} Suspendida cuarta
\vskip 25pt
\upchord{\GsusPiano}{Sol} Suspendida cuarta
\upchord{\GssusPiano}{Sol sostenido} Suspendida cuarta
\normalsize

\vskip 20pt
\textbf{Acodes Mayor Aumentada}
\vskip 25pt

\small
\upchord{\CMajPiano}{Do} Maj  \qquad\qquad
\upchord{\DMajPiano}{Re} Maj
\upchord{\GMajPiano}{Sol} Maj
\normalsize

\vskip 20pt
\textbf{Acodes Mayor S\'eptima Aumentada}
\vskip 25pt

\small
\upchord{\AsevenMajPiano}{La} Maj S\'eptima Aumentada
\upchord{\FmajsevenPiano}{Fa} Maj S\'eptima Aumentada
\normalsize

\vskip 20pt
\textbf{Acodes Aumentada 2}
\vskip 25pt

\small
\upchord{\AtwoPiano}{La} Aumentada 2
\upchord{\CtwoPiano}{Do} Aumentada 2
\normalsize


\vskip 20pt
\textbf{Acodes Disminuidos}
\vskip 25pt

\small
\upchord{\GsdimPiano}{Sol} sostenido disminuido
\normalsize


\vskip 20pt
\textbf{Acodes Con Bajo cambiado}
\vskip 25pt

\small
\upchord{\ACsPiano}{La Mayor bajo C\#}
\upchord{\AEPiano}{La Mayor bajo E}
\vskip 20pt
\upchord{\AmFPiano}{La Menor bajo F}
\upchord{\CEPiano}{Do Mayor bajo E}
\vskip 20pt
\upchord{\CGPiano}{Do Mayor bajo G}
\upchord{\DAPiano}{Re Mayor bajo A}
\upchord{\DFsPiano}{Re Mayor bajo F\#}
\upchord{\GDPiano}{Sol Mayor Bajo D}
\vskip 20pt
\upchord{\GBPiano}{Sol Mayor Bajo B}
\vskip 25pt
\upchord{\DsusFsPiano}{Re} Suspendida cuarta bajo F\#
\normalsize

\vskip 20pt
\textbf{Acodes medio disminuido s\'eptima}
\vskip 25pt

\small
\upchord{\BmsevenbfivePiano}{Si} medio disminuido s\'eptima
\normalsize

\vskip 20pt
\textbf{Acodes 13 suspendida cuarta}
\vskip 25pt

\small
\upchord{\CsusthirteenPiano}{Do} 13 suspendida cuarta
\normalsize

\clearpage
\fi
%\end{document}
%\bye
    %%%%%% rcsid = @(#)$Id:$
%%%%%%
%%
%%      ================================
%%      Sample Key Index (sampleAdx.tex)
%%      ================================
%%
%%      Version 4.5, 30 April, 2010
%%
%%      Copyright 1992--2010 Christopher Rath <christopher@rath.ca>
%%
%%	This package is free software; you can redistribute it and/or
%%	modify it under the terms of version 2.1 of the GNU Lesser 
%%	General Public License as published by the Free Software
%%	Foundation.
%%
%%	This package is distributed in the hope that it will be
%%	useful, but WITHOUT ANY WARRANTY; without even the implied
%%	warranty of MERCHANTABILITY or FITNESS FOR A PARTICULAR
%%	PURPOSE.  See the GNU Lesser General Public License for more
%%	details.
%%
%%      This file is provided as a template for Song Artist
%%      Index generation.
%%
%%%%%%
%%%%%%

%%%%%%%%%%%%%%%%%%%%%%%%%%%%%%%%%%%%%%%%%%%%%%%%%%%%%%%%%%
%%%%%%%%%%%%%%%%%%%%%%%%%%%%%%%%%%%%%%%%%%%%%%%%%%%%%%%%%%
%%                                                      %%
%%           P R E A M B L E   B E G I N S              %%
%%                                                      %%
%%%%%%%%%%%%%%%%%%%%%%%%%%%%%%%%%%%%%%%%%%%%%%%%%%%%%%%%%%
%%%%%%%%%%%%%%%%%%%%%%%%%%%%%%%%%%%%%%%%%%%%%%%%%%%%%%%%%%

%\documentclass[12pt,twocolumn]{book}
%\usepackage{latexsym,fancyhdr}
%\usepackage[wordbk]{songbook}

%%%
% Revision Date and Release Date definitions.
%
%       \RelDate - The last time this songbook was released.
%       \RevDate - The last time this file was revised in any way.
%%%
%\newcommand{\RelDate}{30~May'96}
%\newcommand{\RevDate}{\RelDate}

%%%
% Redefine fonts from SongBook style that I don't like, and define
% any extra fonts I require.
%%%
\font\myTinySF=cmss8    at  8pt
\font\myHugeSF=cmssbx10 at 25pt
\renewcommand{\CpyRtInfoFont}{\tiny\myTinySF}
%\newcommand{\myTitleFont}{\Huge\myHugeSF}
%\newcommand{\mySubTitleFont}{\large\sf}

%%%
% Define fonts to use in the headers and footers of the songbook.
%%%
%\newcommand{\LHeadFont}{\normalsize}            % = cmr12  at 12pt
%\newcommand{\CHeadFont}{\normalsize\rm}         % = cmr12  at 12pt
%\newcommand{\RHeadFont}{\normalsize}            % = cmr12  at 12pt
%\newcommand{\LFootFont}{\scriptsize}            % = cmr8   at  8pt
%\newcommand{\CFootFont}{\tiny\myTinySF}         % = cmss8  at  8pt
%\newcommand{\RFootFont}{\scriptsize}            % = cmr8   at  8pt

%%%
% Turn on and define fancy page heading/footing definition.
%%%
\pagestyle{fancy}

%\addtolength{\headwidth}{\marginparsep}
%\addtolength{\headwidth}{\marginparwidth}
%\renewcommand{\footrulewidth}{0.4pt}
\lhead{\LHeadFont \'Indice~de~Autores}
       \chead{\CHeadFont ({\rm\thepage})}
       \rhead{\RHeadFont\RelDate}
%
%\lfoot{\LFootFont Property of a Church}
%       \cfoot{\CFootFont Last Revised:  \RevDate}
%       \rfoot{\RFootFont Material used by permission.}


%%%
% Index entries command definition.
%%%
\renewcommand{\item}{\par\hangindent=40pt}
\renewcommand{\subitem}{\par\hangindent=40pt \hspace*{20pt}}
\renewcommand{\subsubitem}{\par\hangindent=40pt \hspace*{30pt}}


%%%%%%%%%%%%%%%%%%%%%%%%%%%%%%%%%%%%%%%%%%%%%%%%%%%%%%%%%%
%%%%%%%%%%%%%%%%%%%%%%%%%%%%%%%%%%%%%%%%%%%%%%%%%%%%%%%%%%
%%                                                      %%
%%           D O C U M E N T   B E G I N S              %%
%%                                                      %%
%%%%%%%%%%%%%%%%%%%%%%%%%%%%%%%%%%%%%%%%%%%%%%%%%%%%%%%%%%
%%%%%%%%%%%%%%%%%%%%%%%%%%%%%%%%%%%%%%%%%%%%%%%%%%%%%%%%%%
%\begin{document}

%%%
% Index begins.
%%%
\pdfbookmark[0]{\'Indice~de~autores}{autores}
{\parindent 8pt
  {\myTitleFont --- INDICE DE AUTORES ---}}\par
\vskip 20pt

\input{Estribillero.adx}

%\end{document}
%\bye
%
%%%
% Document ends.
%%%

% Local Variables:
%   LaTeX-item-indent:     -1
%   LaTeX-indent-level:     2
%   TeX-brace-indent-level: 2
%   TeX-auto-untabify:      nil
%   TeX-style-local:        style/
% End:

    %%%%%% rcsid = @(#)$Id: sampleKdx.tex,v 1.16 2010-04-12 18:04:30 rathc Exp $
%%%%%%
%%
%%      ================================
%%      Sample Key Index (sampleKdx.tex)
%%      ================================
%%
%%      Version 4.5, 30 April, 2010
%%
%%      Copyright 1992--2010 Christopher Rath <christopher@rath.ca>
%%
%%	This package is free software; you can redistribute it and/or
%%	modify it under the terms of version 2.1 of the GNU Lesser 
%%	General Public License as published by the Free Software
%%	Foundation.
%%
%%	This package is distributed in the hope that it will be
%%	useful, but WITHOUT ANY WARRANTY; without even the implied
%%	warranty of MERCHANTABILITY or FITNESS FOR A PARTICULAR
%%	PURPOSE.  See the GNU Lesser General Public License for more
%%	details.
%%
%%      This file is provided as a template for Song Key
%%      Index generation.
%%
%%%%%%
%%%%%%

%%%%%%%%%%%%%%%%%%%%%%%%%%%%%%%%%%%%%%%%%%%%%%%%%%%%%%%%%%
%%%%%%%%%%%%%%%%%%%%%%%%%%%%%%%%%%%%%%%%%%%%%%%%%%%%%%%%%%
%%                                                      %%
%%           P R E A M B L E   B E G I N S              %%
%%                                                      %%
%%%%%%%%%%%%%%%%%%%%%%%%%%%%%%%%%%%%%%%%%%%%%%%%%%%%%%%%%%
%%%%%%%%%%%%%%%%%%%%%%%%%%%%%%%%%%%%%%%%%%%%%%%%%%%%%%%%%%

%\documentclass[12pt,twocolumn]{book}
%\usepackage{latexsym,fancyhdr}
%\usepackage[wordbk]{songbook}

%%%
% Revision Date and Release Date definitions.
%
%       \RelDate - The last time this songbook was released.
%       \RevDate - The last time this file was revised in any way.
%%%
%\newcommand{\RelDate}{30~May'96}
%\newcommand{\RevDate}{\RelDate}

%%%
% Redefine fonts from SongBook style that I don't like, and define
% any extra fonts I require.
%%%
\font\myTinySF=cmss8    at  8pt
\font\myHugeSF=cmssbx10 at 25pt
\renewcommand{\CpyRtInfoFont}{\tiny\myTinySF}
%\newcommand{\myTitleFont}{\Huge\myHugeSF}
%\newcommand{\mySubTitleFont}{\large\sf}

%%%
% Define fonts to use in the headers and footers of the songbook.
%%%
%\newcommand{\LHeadFont}{\normalsize}            % = cmr12  at 12pt
%\newcommand{\CHeadFont}{\normalsize\rm}         % = cmr12  at 12pt
%\newcommand{\RHeadFont}{\normalsize}            % = cmr12  at 12pt
%\newcommand{\LFootFont}{\scriptsize}            % = cmr8   at  8pt
%\newcommand{\CFootFont}{\tiny\myTinySF}         % = cmss8  at  8pt
%\newcommand{\RFootFont}{\scriptsize}            % = cmr8   at  8pt

%%%
% Turn on and define fancy page heading/footing definition.
%%%
\pagestyle{fancy}
\pdfbookmark[0]{\'Indice~Tonal}{tonal}
%\addtolength{\headwidth}{\marginparsep}
%\addtolength{\headwidth}{\marginparwidth}
%\renewcommand{\footrulewidth}{0.4pt}
\lhead{\LHeadFont \'Indice~Tonal}
       \chead{\CHeadFont ({\rm\thepage})}
       \rhead{\RHeadFont\RelDate}

%\lfoot{\LFootFont Property of a Church}
%       \cfoot{\CFootFont Last Revised:  \RevDate}
%       \rfoot{\RFootFont Material used by permission.}


%%%
% Index entries command definition.
%%%
\renewcommand{\item}{\par\hangindent=40pt}
\renewcommand{\subitem}{\par\hangindent=40pt \hspace*{20pt}}
\renewcommand{\subsubitem}{\par\hangindent=40pt \hspace*{30pt}}


%%%%%%%%%%%%%%%%%%%%%%%%%%%%%%%%%%%%%%%%%%%%%%%%%%%%%%%%%%
%%%%%%%%%%%%%%%%%%%%%%%%%%%%%%%%%%%%%%%%%%%%%%%%%%%%%%%%%%
%%                                                      %%
%%           D O C U M E N T   B E G I N S              %%
%%                                                      %%
%%%%%%%%%%%%%%%%%%%%%%%%%%%%%%%%%%%%%%%%%%%%%%%%%%%%%%%%%%
%%%%%%%%%%%%%%%%%%%%%%%%%%%%%%%%%%%%%%%%%%%%%%%%%%%%%%%%%%
%\begin{document}

%%%
% Index begins.
%%%
{\parindent 8pt
  {\myTitleFont --- INDICE TONAL ---}}\par
\vskip 20pt

\input{Estribillero.kdx}
%
%\end{document}
%\bye
%
%%%
% Document ends.
%%%

% Local Variables:
%   LaTeX-item-indent:     -1
%   LaTeX-indent-level:     2
%   TeX-brace-indent-level: 2
%   TeX-auto-untabify:      nil
%   TeX-style-local:        style/
% End:

    %%%%%% rcsid = @(#)$Id: sampleTdx.tex,v 1.18 2010-04-12 18:04:31 rathc Exp $
%%%%%%
%%
%%      ===============================================
%%      Sample Title & First Line Index (sampleTdx.tex)
%%      ===============================================
%%
%%      Version 4.5, 30 April, 2010
%%
%%      Copyright 1992--2010 Christopher Rath <christopher@rath.ca>
%%
%%      This package is free software; you can redistribute it and/or
%%      modify it under the terms of version 2.1 of the GNU Lesser 
%%	General Public License as published by the Free Software 
%%	Foundation.
%%
%%      This package is distributed in the hope that it will be
%%      useful, but WITHOUT ANY WARRANTY; without even the implied
%%      warranty of MERCHANTABILITY or FITNESS FOR A PARTICULAR
%%      PURPOSE.  See the GNU Lesser General Public License for more
%%      details.
%%
%%      This file is provided as a template for Title and First Line
%%      Index generation.
%%
%%%%%%
%%%%%%

%%%%%%%%%%%%%%%%%%%%%%%%%%%%%%%%%%%%%%%%%%%%%%%%%%%%%%%%%%
%%%%%%%%%%%%%%%%%%%%%%%%%%%%%%%%%%%%%%%%%%%%%%%%%%%%%%%%%%
%%                                                      %%
%%           P R E A M B L E   B E G I N S              %%
%%                                                      %%
%%%%%%%%%%%%%%%%%%%%%%%%%%%%%%%%%%%%%%%%%%%%%%%%%%%%%%%%%%
%%%%%%%%%%%%%%%%%%%%%%%%%%%%%%%%%%%%%%%%%%%%%%%%%%%%%%%%%%

%\documentclass[12pt,twocolumn,spanish]{book}
%\usepackage{latexsym,fancyhdr}
%\usepackage[wordbk]{songbook}


%%%
% Revision Date and Release Date definitions.
%
%       \RelDate - The last time this songbook was released.
%       \RevDate - The last time this file was revised in any way.
%%%
%\newcommand{\RelDate}{30 May'96}
%\newcommand{\RevDate}{\today}

%%%
% Redefine fonts from SongBook style that I don't like, and define
% any extra fonts I require.
%%%
\font\myTinySF=cmss8    at  8pt
\font\myHugeSF=cmssbx10 at 25pt
\renewcommand{\CpyRtInfoFont}{\tiny\myTinySF}
%\newcommand{\myTitleFont}{\Huge\myHugeSF}
%\newcommand{\mySubTitleFont}{\large\sf}

%%%
% Define fonts to use in the headers and footers of the songbook.
%%%
%\newcommand{\LHeadFont}{\normalsize}            % = cmr12  at 12pt
%\newcommand{\CHeadFont}{\normalsize\rm}         % = cmr12  at 12pt
%\newcommand{\RHeadFont}{\normalsize}            % = cmr12  at 12pt
%\newcommand{\LFootFont}{\scriptsize}            % = cmr8   at  8pt
%\newcommand{\CFootFont}{\tiny\myTinySF}         % = cmss8  at  8pt
%\newcommand{\RFootFont}{\scriptsize}            % = cmr8   at  8pt

%%%
% Turn on and define fancy page heading/footing definition.
%%%
\pagestyle{fancy}

\addtolength{\headwidth}{\marginparsep}
\addtolength{\headwidth}{\marginparwidth}
\renewcommand{\footrulewidth}{0.4pt}
\lhead{\LHeadFont A Church Songbook}
       \chead{\CHeadFont \I'ndice~por~T\'itulo({\rm\thepage})}
       \rhead{\RHeadFont\RelDate}

\lfoot{\LFootFont Property of a Church}
       \cfoot{\CFootFont Last Revised:  \RevDate}
       \rfoot{\RFootFont Material used by permission.}

%%%
% Index entries command definition.
%%%
\renewcommand{\item}{\par\hangindent=40pt}
\renewcommand{\subitem}{\par\hangindent=40pt \hspace*{20pt}}
\renewcommand{\subsubitem}{\par\hangindent=40pt \hspace*{30pt}}


%%%%%%%%%%%%%%%%%%%%%%%%%%%%%%%%%%%%%%%%%%%%%%%%%%%%%%%%%%
%%%%%%%%%%%%%%%%%%%%%%%%%%%%%%%%%%%%%%%%%%%%%%%%%%%%%%%%%%
%%                                                      %%
%%           D O C U M E N T   B E G I N S              %%
%%                                                      %%
%%%%%%%%%%%%%%%%%%%%%%%%%%%%%%%%%%%%%%%%%%%%%%%%%%%%%%%%%%
%%%%%%%%%%%%%%%%%%%%%%%%%%%%%%%%%%%%%%%%%%%%%%%%%%%%%%%%%%
%\begin{document}

%%%
% Begin the Index.
%%%
{\parindent 8pt
  {\myTitleFont --- Title Index ---}}\par
\vskip 5pt
{\parindent 20pt
  {\mySubTitleFont --- with first lines in italic ---}}
\vskip 20pt

\input{Estribillero.tdx}

%\end{document}
%\bye
%
%%%
% Document ends.
%%%

\end{document}
\bye
%
%%%
% Document ends.
%%%

%
%\end{document}
%\bye
%
%%%
% Document ends.
%%%

% Local Variables:
%   LaTeX-item-indent:     -1
%   LaTeX-indent-level:     2
%   TeX-brace-indent-level: 2
%   TeX-auto-untabify:      nil
%   TeX-style-local:        style/
% End:

%%%%%% rcsid = @(#)$Id: sampleKdx.tex,v 1.16 2010-04-12 18:04:30 rathc Exp $
%%%%%%
%%
%%      ================================
%%      Sample Key Index (sampleKdx.tex)
%%      ================================
%%
%%      Version 4.5, 30 April, 2010
%%
%%      Copyright 1992--2010 Christopher Rath <christopher@rath.ca>
%%
%%	This package is free software; you can redistribute it and/or
%%	modify it under the terms of version 2.1 of the GNU Lesser 
%%	General Public License as published by the Free Software
%%	Foundation.
%%
%%	This package is distributed in the hope that it will be
%%	useful, but WITHOUT ANY WARRANTY; without even the implied
%%	warranty of MERCHANTABILITY or FITNESS FOR A PARTICULAR
%%	PURPOSE.  See the GNU Lesser General Public License for more
%%	details.
%%
%%      This file is provided as a template for Song Key
%%      Index generation.
%%
%%%%%%
%%%%%%

%%%%%%%%%%%%%%%%%%%%%%%%%%%%%%%%%%%%%%%%%%%%%%%%%%%%%%%%%%
%%%%%%%%%%%%%%%%%%%%%%%%%%%%%%%%%%%%%%%%%%%%%%%%%%%%%%%%%%
%%                                                      %%
%%           P R E A M B L E   B E G I N S              %%
%%                                                      %%
%%%%%%%%%%%%%%%%%%%%%%%%%%%%%%%%%%%%%%%%%%%%%%%%%%%%%%%%%%
%%%%%%%%%%%%%%%%%%%%%%%%%%%%%%%%%%%%%%%%%%%%%%%%%%%%%%%%%%

%\documentclass[12pt,twocolumn]{book}
%\usepackage{latexsym,fancyhdr}
%\usepackage[wordbk]{songbook}

%%%
% Revision Date and Release Date definitions.
%
%       \RelDate - The last time this songbook was released.
%       \RevDate - The last time this file was revised in any way.
%%%
%\newcommand{\RelDate}{30~May'96}
%\newcommand{\RevDate}{\RelDate}

%%%
% Redefine fonts from SongBook style that I don't like, and define
% any extra fonts I require.
%%%
\font\myTinySF=cmss8    at  8pt
\font\myHugeSF=cmssbx10 at 25pt
\renewcommand{\CpyRtInfoFont}{\tiny\myTinySF}
%\newcommand{\myTitleFont}{\Huge\myHugeSF}
%\newcommand{\mySubTitleFont}{\large\sf}

%%%
% Define fonts to use in the headers and footers of the songbook.
%%%
%\newcommand{\LHeadFont}{\normalsize}            % = cmr12  at 12pt
%\newcommand{\CHeadFont}{\normalsize\rm}         % = cmr12  at 12pt
%\newcommand{\RHeadFont}{\normalsize}            % = cmr12  at 12pt
%\newcommand{\LFootFont}{\scriptsize}            % = cmr8   at  8pt
%\newcommand{\CFootFont}{\tiny\myTinySF}         % = cmss8  at  8pt
%\newcommand{\RFootFont}{\scriptsize}            % = cmr8   at  8pt

%%%
% Turn on and define fancy page heading/footing definition.
%%%
\pagestyle{fancy}
\pdfbookmark[0]{\'Indice~Tonal}{tonal}
%\addtolength{\headwidth}{\marginparsep}
%\addtolength{\headwidth}{\marginparwidth}
%\renewcommand{\footrulewidth}{0.4pt}
\lhead{\LHeadFont \'Indice~Tonal}
       \chead{\CHeadFont ({\rm\thepage})}
       \rhead{\RHeadFont\RelDate}

%\lfoot{\LFootFont Property of a Church}
%       \cfoot{\CFootFont Last Revised:  \RevDate}
%       \rfoot{\RFootFont Material used by permission.}


%%%
% Index entries command definition.
%%%
\renewcommand{\item}{\par\hangindent=40pt}
\renewcommand{\subitem}{\par\hangindent=40pt \hspace*{20pt}}
\renewcommand{\subsubitem}{\par\hangindent=40pt \hspace*{30pt}}


%%%%%%%%%%%%%%%%%%%%%%%%%%%%%%%%%%%%%%%%%%%%%%%%%%%%%%%%%%
%%%%%%%%%%%%%%%%%%%%%%%%%%%%%%%%%%%%%%%%%%%%%%%%%%%%%%%%%%
%%                                                      %%
%%           D O C U M E N T   B E G I N S              %%
%%                                                      %%
%%%%%%%%%%%%%%%%%%%%%%%%%%%%%%%%%%%%%%%%%%%%%%%%%%%%%%%%%%
%%%%%%%%%%%%%%%%%%%%%%%%%%%%%%%%%%%%%%%%%%%%%%%%%%%%%%%%%%
%\begin{document}

%%%
% Index begins.
%%%
{\parindent 8pt
  {\myTitleFont --- INDICE TONAL ---}}\par
\vskip 20pt

%%%%%% rcsid = @(#)$Id: sample-sb.tex,v 1.23 2010-04-12 18:04:11 rathc Exp $
%%%%%%
%%
%%      ===============================
%%      Sample Songbook (sample-sb.tex)
%%      ===============================
%%
%%      Version 4.5, 30 April, 2010
%%
%%      Copyright 1992--2010 Christopher Rath <christopher@rath.ca>
%%
%%      This package is free software; you can redistribute it and/or
%%      modify it under the terms of version 2.1 of the GNU Lesser
%%	General Public License as published by the Free Software 
%%	Foundation.
%%
%%      This package is distributed in the hope that it will be
%%      useful, but WITHOUT ANY WARRANTY; without even the implied
%%      warranty of MERCHANTABILITY or FITNESS FOR A PARTICULAR
%%      PURPOSE.  See the GNU Lesser General Public License for more
%%      details.
%%
%%      This file contains a subset of the songbook we distribute
%%      at our church.  To the best of my knowledge, all of the lyrics
%%      contained herein are freely distributable.  This file has been
%%      provided as a sample of what can be produced by the chordbk,
%%      wordbk, and overhead LaTeX styles.
%%
%%      NEEDED:  The fancyhdr LaTeX style is required to properly
%%              format this file.  If you don't have that then comment
%%              out the commands in the preamble which deal with the
%%              fancyhdr style.
%%
%%%%%%
%%%%%%
%%
%%      1. Chord notation.  Within this songbook the following
%%         conventions have been adopted:
%%
%%              "Minor" is entered as "m";
%%                      e.g. Cm7 for C minor 7th.
%%              "Major" is entered as "M";
%%                      e.g. CM7 for C major 7th.
%%
%%%%%%
%%%%%%
%%      ============
%%      Bibliography
%%      ============
%%
%%    
%%
%%%%%%
%%%%%%

%%%%%%%%%%%%%%%%%%%%%%%%%%%%%%%%%%%%%%%%%%%%%%%%%%%%%%%%%%
%%%%%%%%%%%%%%%%%%%%%%%%%%%%%%%%%%%%%%%%%%%%%%%%%%%%%%%%%%
%%                                                      %%
%%           P R E A M B L E   B E G I N S              %%
%%                                                      %%
%%%%%%%%%%%%%%%%%%%%%%%%%%%%%%%%%%%%%%%%%%%%%%%%%%%%%%%%%%
%%%%%%%%%%%%%%%%%%%%%%%%%%%%%%%%%%%%%%%%%%%%%%%%%%%%%%%%%%

\documentclass[12pt, spanish, titlepage]{book}
\usepackage[T1]{fontenc}
\usepackage[latin9]{inputenc}
\usepackage{babel}
\usepackage{mypiano}
\usepackage{gchords}
\usepackage{latexsym,fancyhdr}
\usepackage{imakeidx}
\usepackage[unicode=true,pdfusetitle, bookmarks=true,bookmarksnumbered=false,bookmarksopen=false,
    breaklinks=false,pdfborder={0 0 1},backref=false,colorlinks=true]{hyperref}
\usepackage[chordbk]{songbook} %% Words & Chords edition.
%%\usepackage[compactallsongs,chordbk]{songbook}    %% Words & Chords edition.
%%\usepackage[wordbk]{songbook}                 %% Words Only edition.
%%\usepackage[overhead]{songbook}               %% Overhead Transparency edition.


% genera acordes de guitarra
\newif\ifguitarra

%genera acordes de piano
\newif\ifpiano

\guitarratrue
\pianotrue



\renewcommand{\SBChorusTag}{Coro:}
\renewcommand{\SBBridgeTag}{Puente:}
\newcommand{\myTitleFont}{\Huge\myHugeSF}
\newcommand{\mySubTitleFont}{\large\sf}
%%%
% Revision Date and Release Date definitions.
%
%       \RelDate - The last time this songbook was released.  Set this
%                  date each time a new release/update of the songbook
%                  is generated.
%       \RevDate - The last time a particular song was revised in any
%                  way.  This command will be renewed inside every
%                  song.
%%%
\newcommand{\RelDate}{26~Marzo,~2014}
\newcommand{\RevDate}{\today}

%%%
% C.C.L.I. license number definition; for copyright licensing info.
% One of these macros will be manually inserted into the {CpyRt}
% parameter of the {song} environment.
%
%       \CCLInumber - The actual copyright license number.  Don't
%               insert this command in the {CpyRt} parameter, use one
%               of the others.
%       \CCLIed - Indicates a song falls under our CCLI license.
%       \NotCCLIed - Indicates a song doesn't fall under our CCLI
%               license.  Public Domain songs fall into this category.
%       \PGranted - We have received specific permission from the
%               copyright holder to use this song.
%       \PPending - We are in the process of obtaining permission to
%               use this song.
%%%
\newcommand{\CCLInumber}{Your CCLI Number}
\newcommand{\CCLIed}{{\CpyRtInfoFont (CCLI \CCLInumber)}}
\newcommand{\NotCCLIed}{\relax}
\newcommand{\PGranted}{\relax}
\newcommand{\PPending}{{\CpyRtInfoFont (Permission Pending)}}

% comandos para pintar acordes de guitarra
\newcommand{\A}{\chord{t}{x,n,p2,p2,p2,n}{A}}
\newcommand{\Aseven}{\chord{t}{x,n,p2,n,p2,n}{A7}}
\newcommand{\AsevenMaj}{\chord{t}{x,n,p2,p1,p2,n}{A7+}}
\newcommand{\Am}{\chord{t}{x,n,p2,p2,p1,n}{Am}}
\newcommand{\Amseven}{\chord{t}{x,n,p2,n,p1,n}{Am7}}
\newcommand{\ACs}{\chord{t}{x,p3,p2,p2,p2,n}{A/C\#}}

\newcommand{\As}{\chord{t}{p1,p1,p3,p3,p3,p1}{A\#}}
\newcommand{\Bflat}{\chord{t}{p1,p1,p3,p3,p3,p1}{Bb}}

\newcommand{\B}{\chord{t}{x,bf1p2,f2p4,f3p4,f4p4,f1p2}{B}}
\newcommand{\Bseven}{\chord{t}{x,f1p2,p4,f1p2,p4,f1p2,}{B7}}
\newcommand{\BsevenBasDs}{\chord{t}{x,x,p1,p2,n,p2}{B7/D\#}}
\newcommand{\Bm}{\chord{t}{x,p2,p4,p4,p3,p2}{Bm}}
\newcommand{\Bmseven}{\chord{t}{x,p2,p4,p2,p3,p2}{Bm7}}
\newcommand{\BmseveN}{\chord{t}{x,p2,p4,p3,p3,p2}{Bm7+}}
\newcommand{\BmsevenA}{\chord{t}{x,n,p4,p4,p3,n}{Bm/A}}

\newcommand{\C}{\chord{t}{x,p3,n,p2,p1,n}{C}}
\newcommand{\Cseven}{\chord{t}{x,p3,p3,p2,p1,n}{C7}}
\newcommand{\Cm}{\chord{t}{x,p3,p1,n,p1,p3}{Cm}}
%\newcommand{\Cmseven}{\chord{t}{x,p3,p1,n,p1,p3}{Cm}}
\newcommand{\CE}{\chord{t}{o,p3,n,p2,p1,n}{C/E}}

\newcommand{\Cs}{\chord{4}{n,n,p2,p2,p2,n}{C\#}}
\newcommand{\Csm}{\chord{t}{p4,p4,p6,p6,p5,p4}{C\#m}}
\newcommand{\CssevenLight}{\chord{t}{x,p4,p3,p4,p2,x}{C\#7}}

\newcommand{\D}{\chord{t}{x,x,n,p2,p3,p2}{D}}
\newcommand{\Dseven}{\chord{t}{x,x,n,p2,p1,p2}{D7}}
\newcommand{\DseveN}{\chord{t}{x,x,n,p2,p2,p2}{D7+}}
\newcommand{\Dm}{\chord{t}{x,x,n,p2,p3,p1}{Dm}}
\newcommand{\Dmseven}{\chord{t}{n,n,n,p2,p1,p1}{Dm7}}
\newcommand{\DmsevenG}{\chord{t}{p3,n,n,p2,p1,p1}{Dm7/G}}
\newcommand{\Dsix}{\chord{t}{x,x,n,p2,n,p2}{D6}}
\newcommand{\DmBasB}{\chord{t}{x,p2,p3,p2,p3,x}{Dm/B}}
\newcommand{\DA}{\chord{t}{x,o,n,p2,p3,p2}{D/A}}
\newcommand{\DFs}{\chord{t}{p2,n,n,p2,p3,p2}{D/F\#}}

\newcommand{\Ds}{\chord{t}{n,n,p1,p3,p4,p3}{D\#}}
\newcommand{\Eflat}{\chord{t}{n,n,p1,p3,p4,p3}{E$\flat$}}

\newcommand{\E}{\chord{t}{n,p2,p2,p1,n,n}{E}}
\newcommand{\Eseven}{\chord{t}{n,p2,p2,p1,p3,n}{E7}}
\newcommand{\EseveN}{\chord{t}{n,p2,p2,p4,p3,p4}{E7}}
\newcommand{\Em}{\chord{t}{n,p2,p2,n,n,n}{Em}}
\newcommand{\EsevenFour}{\chord{t}{n,p2,p2,p4,p3,p5}{E7,11}}
\newcommand{\EseveNNine}{\chord{t}{n,f1p2,f1p2,p4,p3,f1p2,}{E79}}

\newcommand{\F}{\chord{t}{p1,p3,p3,p2,p1,p1}{F}}

\newcommand{\Fs}{\chord{t}{p2,p4,p4,p3,p2,p2}{F\#}}
\newcommand{\Fsm}{\chord{t}{f1p2,p4,p4,f1p2,f1p2,f1p2,}{F\#m}}
\newcommand{\FsminLight}{\chord{t}{x,x,f3p4,f1p2,f1p2,f1p2,}{F\#m}}
\newcommand{\FsminBasSeveN}{\chord{t}{x,x,f3p3,f1p2,f1p2,f1p2,}{F\#m/E\#}}
\newcommand{\FsminBasSeven}{\chord{t}{x,x,f2p2,f1p2,f1p2,f1p2,}{F\#m/E}}
\newcommand{\FsminSeven}{\chord{t}{f1p2,p4,p4,f1p2,p5,f1p2,}{F\#7m}}

\newcommand{\G}{\chord{t}{p3,p2,n,n,n,p2}{G}}
\newcommand{\Gseven}{\chord{t}{p3,p2,n,n,n,p1}{G7}}
\newcommand{\GB}{\chord{t}{n,p2,n,n,p3,n}{G/B}}
\newcommand{\GD}{\chord{t}{x,p2,p2,n,p3,p3}{G/D}} %verificar porque parece m�s ien G/E o G/A o Em7algo
\newcommand{\Gnine}{\chord{t}{p3,p2,n,p0,p2,p2}{G9}}
\newcommand{\Gm}{\chord{t}{f1p3,p5,p5,f1p3,f1p3,f1p3,}{Gm}}
\newcommand{\Gmseven}{\chord{t}{f1p3,p5,p3,f1p3,f1p3,f1p3,}{Gm7}}

\newcommand{\Gs}{\chord{3}{x,x,p4,p3,p2,p2,}{G\#}}
\newcommand{\Aflat}{\chord{3}{x,x,p4,p3,p2,p2,}{A$\flat$}}
\newcommand{\Gsmseven}{\chord{t}{f2p4,x,f4p4,f4p4,f4p4,f4p4,}{G\#7}}
\newcommand{\Gssus}{\chord{t}{p4,p6,p6,p6,p4,p4}{Gsus4}}

% comandos para pintar acordes de piano
\newcommand{\APiano}{\keyboardf[Ao][Cso][Eo]}
\newcommand{\AsevenPiano}{\keyboardf[Ao][Cso][Eo][Go]}
\newcommand{\AmPiano}{\keyboardf[Ao][Co][Eo]}
\newcommand{\AmsevenPiano}{\keyboardf[Ao][Co][Eo][Go]}
\newcommand{\ACsPiano}{\keyboard[Cso][Eo][Ao]}

\newcommand{\AsPiano}{\keyboard[Do][Fo][Aso]}
\newcommand{\BflatPiano}{\keyboard[Do][Fo][Aso]}

\newcommand{\BPiano}{\keyboard[Dso][Fso][Bo]}
\newcommand{\BsevenPiano}{\keyboard[Dso][Fso][Bo][Ao]}
\newcommand{\BmPiano}{\keyboard[Do][Fso][Bo]}
\newcommand{\BmsevenPiano}{\keyboard[Do][Fso][Bo][Ao]}

\newcommand{\CPiano}{\keyboard[Co][Eo][Go]}
\newcommand{\CsevenPiano}{\keyboard[Co][Eo][Go][Aso]}
\newcommand{\CmPiano}{\keyboard[Co][Dso][Go]}

\newcommand{\CsPiano}{\keyboard[Cso][Fo][Gso]}
\newcommand{\CsmPiano}{\keyboard[Cso][Eo][Gso]}

\newcommand{\DPiano}{\keyboard[Do][Fso][Ao]}
\newcommand{\DsevenPiano}{\keyboard[Do][Fso][Ao][Co]}
\newcommand{\DmPiano}{\keyboard[Do][Fo][Ao]}
\newcommand{\DmsevenPiano}{\keyboard[Do][Fo][Ao][Co]}
\newcommand{\DAPiano}{\keyboardf[Ao][Do][Fso]}
\newcommand{\DFsPiano}{\keyboardf[Fso][Ao][Do]}

\newcommand{\DsPiano}{\keyboard[Dso][Go][Aso]}
\newcommand{\EflatPiano}{\keyboard[Dso][Go][Aso]}

\newcommand{\EPiano}{\keyboard[Eo][Gso][Bo]}
\newcommand{\EsevenPiano}{\keyboard[Eo][Gso][Bo][Do]}
\newcommand{\EmPiano}{\keyboard[Eo][Go][Bo]}
\newcommand{\EmsevenPiano}{\keyboard[Eo][Go][Bo][Do]}

\newcommand{\FPiano}{\keyboard[Co][Fo][Ao]}

\newcommand{\FsPiano}{\keyboard[Cso][Fso][Aso]}
\newcommand{\FsmPiano}{\keyboard[Cso][Fso][Ao]}
\newcommand{\FsmsevenPiano}{\keyboard[Cso][Fso][Ao][Eo]}

\newcommand{\GPiano}{\keyboard[Do][Go][Bo]}
\newcommand{\GsevenPiano}{\keyboard[Do][Fo][Go][Bo]}
\newcommand{\GBPiano}{\keyboardtwooctaves[Bo][Dt][Gt]}
\newcommand{\GDPiano}{\keyboard[Do][Go][Bo]}
\newcommand{\GmPiano}{\keyboard[Do][Go][Aso]}
\newcommand{\GmsevenPiano}{\keyboard[Do][Go][Aso][Fo]}

\newcommand{\GsPiano}{\keyboard[Dso][Gso][Co]}
\newcommand{\AflatPiano}{\keyboard[Dso][Gso][Co]}

%%%
% Title page information.
%%%
\title{Cuaderno de Himnos Tradicionales y Contempor�neos}
\author{Ruslan L\'opez}
\date{\'Ultima Revisi\'on:  \RevDate}

%%%
% Redefine fonts from SongBook style that I don't like.
%%%
\font\myTinySF=cmss8 at 8pt
\renewcommand{\CpyRtInfoFont}{\tiny\myTinySF}

%%%
% Define fonts to use in the headers and footers of the songbook.
%%%
\newcommand{\LHeadFont}{\normalsize}            % = cmr12  at 12pt
\newcommand{\CHeadFont}{\normalsize\rm}         % = cmr12  at 12pt
\newcommand{\RHeadFont}{\normalsize}            % = cmr12  at 12pt
\newcommand{\LFootFont}{\scriptsize}            % = cmr8   at  8pt
\newcommand{\CFootFont}{\tiny\myTinySF}         % = cmss8  at  8pt
\newcommand{\RFootFont}{\scriptsize}            % = cmr8   at  8pt

%%%
% Turn on and define fancy page heading/footing definition.
%%%
\pagestyle{fancy}

\ifChordBk
% It's a words & chords songbook...
\addtolength{\headwidth}{\marginparsep}
\addtolength{\headwidth}{\marginparwidth}
\renewcommand{\headrulewidth}{0.4pt}
\renewcommand{\footrulewidth}{0.4pt}
\fancyhead[LE,RO]{\LHeadFont\emph{\leftmark\/}\SBContinueMark}
\fancyhead[CE,CO]{\CHeadFont\thepage}
\fancyhead[RE,LO]{\RHeadFont\RelDate}
\else\ifOverhead
% It's an overhead...
\renewcommand{\footrulewidth}{0pt}
\renewcommand{\headrulewidth}{0pt}
\fancyhead[LE,RO]{}
\fancyhead[CE,CO]{}
\fancyhead[RE,LO]{}
\else\ifWordBk
% It's a words only songbook...
\addtolength{\headwidth}{\marginparsep}
\addtolength{\headwidth}{\marginparwidth}
\renewcommand{\headrulewidth}{0.4pt}
\renewcommand{\footrulewidth}{0.4pt}
\fancyhead[LE,RO]{\LHeadFont Estribillero}
\fancyhead[CE,CO]{\CHeadFont\thepage}
\fancyhead[RE,LO]{\RHeadFont\RelDate}
\fi\fi\fi

\fancyfoot[LE,RO]{\LFootFont Transcripciones}
\ifSongEject
\fancyfoot[CE,CO]{\CFootFont \RevDate}
\else
\fancyfoot[CE,CO]{\CFootFont}
\fi
\fancyfoot[RE,LO]{\RFootFont Todo el material son transcripciones personales.}

%%%
% Turn on/off index-file generation.  Uncomment the \makeindex line to
% turn index generation on;  comment it out to turn index generation
% off.
%%%
\makeTitleIndex         %% Title and First Line Index.
\makeTitleContents      %% Table of Contents.
\makeKeyIndex           %% Index of song by key.
\makeArtistIndex        %% Index of song by artist.
\makeindex

%%%%%%%%%%%%%%%%%%%%%%%%%%%%%%%%%%%%%%%%%%%%%%%%%%%%%%%%%%
%%%%%%%%%%%%%%%%%%%%%%%%%%%%%%%%%%%%%%%%%%%%%%%%%%%%%%%%%%
%%                                                      %%
%%           D O C U M E N T   B E G I N S              %%
%%                                                      %%
%%%%%%%%%%%%%%%%%%%%%%%%%%%%%%%%%%%%%%%%%%%%%%%%%%%%%%%%%%
%%%%%%%%%%%%%%%%%%%%%%%%%%%%%%%%%%%%%%%%%%%%%%%%%%%%%%%%%%
\begin{document}

%%%
% Uncomment "\maketitle" statement to make a title page.
%%%
    \maketitle
    %\mainmatter
    \ifWordBk
    \twocolumn
    \fi
%%%
% Songbook begins.
%%%
    \begin{song}{Abba Padre}{D}
    {} %copyright \SBPubDom
    {Marco Barrientos}
    {Romanos 8:15} %pasaje
    {\href{http://open.spotify.com/track/0yj0zBaa7Ckn6ZMQPmCmfF}{Escuchar}} %\NotCCLIed

%        \SBRef{No puedo parar de alabarte}{2006}%fuente \#
        \FLineIdx{Una Vez m�s}

        \begin{SBOptional}
            \Ch{D}{Una} vez m�s

            me a\Ch{Bm}{cer}co a T�

            con \Ch{Em}{li}bertad

            en adora\Ch{A}{ci\'o}n
        \end{SBOptional}

        \begin{SBOptional}
            T\'u e\Ch{D}{res} mi Dios

            tu \Ch{Bm}{hi}jo soy

            mi \Ch{Em}{co}muni\'on contigo

            es una \Ch{A}{dul}ce bendici\'on
        \end{SBOptional}

        \begin{SBChorus}
            // �\Ch{D}{A}bba Pa\Ch{G}{dre}! �\Ch{D}{A}bba Pa\Ch{G}{dre}!

            Es\Ch{D}{tar}\Ch{A/C#}{} conti\Ch{Bm}{go}

            \Ch{D/A}{es} una \Ch{Bm}{dul}ce bendi\Ch{A}{ci\'on}

            �\Ch{D}{A}bba Pa\Ch{G}{dre}! te \Ch{F#m}{a}mo Se\Ch{Bm}{�or}

            \Ch{G}{quie}ro estar en \Ch{D}{co}muni\'on

            \Ch{Em}{quie}ro es\Ch{A}{tar} con\Ch{D}{ti}go. //
        \end{SBChorus}
        \ifChordBk
        \begin{SBOpGroup}
            Acordes: \break

%\keyboardtwooctaves[Do][Fso][Ao]
            \upchord{\DPiano\D}{\qquad Re} Mayor \qquad\qquad
            \upchord{\BmPiano\Bm}{\qquad Si} Menor \hfill \break
            \upchord{\EmPiano\Em}{\qquad Mi} Menor\qquad\qquad
            \upchord{\APiano\A}{\qquad La} Mayor \hfill \break
            \upchord{\GPiano\G}{\qquad Sol} Mayor \qquad\qquad
            \upchord{\FsmPiano\Fsm}{\qquad Fa\#} Menor \hfill \break
            \upchord{\ACsPiano\ACs}{\qquad La} Mayor con bajo C\# \qquad\qquad
            \upchord{\DAPiano\DA}{\qquad Re} Mayor con Bajo La \qquad\qquad
        \end{SBOpGroup}
        \fi

    \end{song}

    \begin{song}{A Cristo solo a Cristo}{G}
    {Proyecto AA} %copyright \SBPubDom
    {Marcos Witt}
    {Hechos 4:12} %pasaje
    {\href{https://open.spotify.com/track/31LsSTJ8P7HZWu6KVpY9Fz}{Escuchar}} %\NotCCLIed

%  \renewcommand{\RevDate}{February~11,~1993}
%  \SBRef{No puedo parar de alabarte}{2006}%fuente \#

        \begin{SBOptional}
            \Ch{G}{A} Cristo \Ch{G7}{so}lo a \Ch{C}{C}risto\Ch{Bm}{~}, \Ch{Am}{yo} exalta\Ch{D}{r\'e}

            \Ch{G}{A} Cristo \Ch{G7}{so}lo a \Ch{C}{C}risto\Ch{Bm}{~}, \Ch{Am}{yo} adora\Ch{D}{r\'e}
        \end{SBOptional}

        \begin{SBOptional}
            \Ch{Am}{Por}que \'El me ha dado vida e\Ch{D}{ter}na

            \Ch{Am}{Por}que \'El me ha dado el po\Ch{D}{der}

            \Ch{Am}{Por}que \'El me ha dado la vic\Ch{D}{to}ria

            \Ch{D}{�l} \Ch{Em}{es} \Ch{D/F\#}{mi} \Ch{G}{Rey}.

            a \Ch{C}{Cris}to he \Ch{Bm}{pro}cla\Ch{Am}{ma}\Ch{D}{do} \Ch{G}{Rey}
        \end{SBOptional}
        \ifChordBk
        \begin{SBOpGroup}
            Acordes:
            \upchord{\GPiano\G}{\qquad Sol} Mayor \qquad\qquad
            \upchord{\GsevenPiano\Gseven}{\qquad Sol} Mayor S\'eptima \hfill \break
            \upchord{\CPiano\C}{\qquad Do} Mayor \qquad\qquad
            \upchord{\BmPiano\Bm}{\qquad Si} Menor \hfill \break
            \upchord{\AmPiano\Am}{\qquad La} Menor \qquad\qquad
            \upchord{\DPiano\D}{\qquad Re} Mayor \hfill \break
            \upchord{\EmPiano\Em}{\qquad Mi} Menor \qquad\qquad
            \upchord{\DFsPiano\DFs}{\qquad Re} Mayor con bajo F\# \hfill \break
        \end{SBOpGroup}
        \fi
    \end{song}

    \begin{song}{Doxolog\'ia}{G}
    {} %copyright \SBPubDom
    {Thomas Ken, Genevan Psalter, 1551, atr. a Louis Bourgeois}
    {Juan 17:22} %pasaje
    {\href{https://open.spotify.com/intl-es/track/24I9oe6qBsY3JHvL6g4yTT}{Escuchar}} %\NotCCLIed

        \FLineIdx{A Dios el Padre Celestial}

        \begin{SBVerse}
            \Ch{G}{A} Dios \Ch{D}{el} \Ch{Em}{Pa}\Ch{Bm}{dre} \Ch{Em}{Ce}\Ch{D}{les}\Ch{G}{tial},

            Al Hijo \Ch{D}{nues}\Ch{Em}{tro} \Ch{C}{Re}\Ch{G}{den}\Ch{D}{tor}.

            \Ch{Em}{Y} al \Ch{D}{E}\Ch{G}{ter}\Ch{D}{nal} \Ch{G}{Con}\Ch{C}{so}\Ch{Am7}{la}\Ch{G}{dor},

            uni\Ch{Em}{dos} \Ch{D}{to}\Ch{Am}{dos} \Ch{G/B}{a}\Ch{D}{la}\Ch{G}{bad}.

            \Ch{C}{A}\Ch{G}{m\'en}.
        \end{SBVerse}
        \ifChordBk
        \begin{SBOpGroup}
            Acordes:
            \upchord{\GPiano\G}{\qquad Sol} Mayor \qquad\qquad
            \upchord{\DPiano\D}{Re} Mayor \hfill \break
            \upchord{\EmPiano\Em}{\qquad Mi} Menor \qquad\qquad
            \upchord{\BmPiano\Bm}{\qquad Si} Menor \hfill \break
            \upchord{\CPiano\C}{\qquad Do} Mayor \qquad\qquad
            \upchord{\AmsevenPiano\Amseven}{\qquad La} Menor S\'eptima \hfill \break
            \upchord{\AmPiano\Am}{\qquad La} Menor \qquad\qquad
            \upchord{\GBPiano\GB}{\qquad Sol} con bajo B \hfill \break
        \end{SBOpGroup}
        \fi
    \end{song}


    \begin{song}{Admirable}{Dm7}
    {} %copyright \SBPubDom
    {Danilo Montero}
    {Apocalipsis 1:18} %pasaje
    {\href{https://open.spotify.com/intl-es/track/3EigVRcP0VQ5MhAUkCfZX8}{Escuchar}} %\NotCCLIed

%  \renewcommand{\RevDate}{February~11,~1993}
%  \SBRef{No puedo parar de alabarte}{2006}%fuente \#
        \FLineIdx{Con poder y autoridad}
        \begin{SBOpGroup}
            \Ch{Dm7}{Con} poder y au\Ch{Bb}{to}ridad \Ch{F}{nues}tro Dios ven\Ch{C}{ci\'o} a la
            muerte

            \Ch{Dm7}{So}bre el trono \Ch{Bb}{ce}lestial \Ch{Am7}{siem}pre reina\Ch{Dm7}{r\'a}.

            \Ch{Bb}{Sen}tado en \Ch{C}{ma}jestad \Ch{Bb}{su}yo es el reino por los
            \Ch{F}{Si}\Ch{C}{glos} \Ch{Bb}{y} por la \Ch{C}{e}ternidad \Ch{Bb}{su} luz de \Ch{Gm7}{glo}ria brilla\Ch{Am7}{r\'a}.
        \end{SBOpGroup}

        \begin{SBChorus}
            Admi\Ch{C}{ra}\Ch{Dm7}{ble}, conse\Ch{C}{je}\Ch{Dm7}{ro} \Ch{Bb}{mi} Dios \Ch{C}{con}sola\Ch{Dm7}{dor},\Ch{Am7}{}

            Eres \Ch{C}{dig}\Ch{Dm7}{no} de a\Ch{C}{la}ban\Ch{Dm7}{za}, \Ch{Bb}{Pr�n}ci\Ch{C}{pe} de \Ch{Dm7}{paz}.
        \end{SBChorus}
        \ifChordBk
        \begin{SBOpGroup}
            Acordes:
            \upchord{\DmsevenPiano\Dmseven}{Re} menor s\'eptima \qquad\qquad
            \upchord{\BflatPiano\Bflat}{\qquad Si bemol} Mayor \hfill \break
            \upchord{\FPiano\F}{\qquad Fa} Mayor \qquad\qquad
            \upchord{\CPiano\C}{\qquad Do} Mayor \hfill \break
            \upchord{\AmsevenPiano\Amseven}{\qquad La} Menor S\'eptima \qquad\qquad
            \upchord{\GmsevenPiano\Gmseven}{\qquad Sol} Menor S\'eptima \hfill \break
        \end{SBOpGroup}
        \fi
    \end{song}

    \begin{song}{A Nuestro Padre Dios}{F}
    {} %copyright \SBPubDom
    {An\'onimo en Thesaurus Musicus 1744}
    {Juan 3:16} %pasaje
    {} %\NotCCLIed

%  \renewcommand{\RevDate}{February~11,~1993}
%  \SBRef{No puedo parar de alabarte}{2006}%fuente \#

        \begin{SBVerse}
            \Ch{F}{A} \Ch{Dm}{nues}\Ch{Gm}{tro} \Ch{C}{Pa}\Ch{Dm7}{dre} \Ch{C}{Dios}

            \Ch{F}{Al}\Ch{Dm}{ce}\Ch{Gm}{mos} \Ch{F}{nues}\Ch{C7}{tra} \Ch{Dm}{voz}

            �\Ch{Gm}{Glo}\Ch{F}{ria} \Ch{C}{a} \Ch{F}{\'El}!

            Tal \Ch{Am}{fu�} su a\Ch{F}{mor} que di\'o

            \Ch{C7}{Al} hijo que \Ch{F}{mu}\Ch{C}{ri\'o,}

            \Ch{F}{En} \Ch{Bb}{quien} con\Ch{F}{f�o} yo;

            �\Ch{Bb}{Glo}\Ch{F}{ria} \Ch{C7}{a} \Ch{F}{\'El}!
        \end{SBVerse}

        \begin{SBVerse}
            \Ch{F}{A} \Ch{Dm}{nues}\Ch{Gm}{tro} \Ch{C}{Sal}\Ch{Dm}{va}\Ch{C}{dor}

            \Ch{F}{De}\Ch{Dm}{mos} \Ch{Gm}{con} \Ch{F}{fe} \Ch{C7}{lo}\Ch{Dm}{or}

            �\Ch{Gm}{Glo}ria \Ch{C}{a} \Ch{F}{\'El}!

            Su \Ch{Am}{san}gre \Ch{F}{de}rram\'o

            \Ch{C7}{Con} ella me \Ch{F}{la}\Ch{C}{v\'o,}

            \Ch{F}{Y} el \Ch{Bb}{cie}lo \Ch{F}{me} abri\'o

            �\Ch{Bb}{Glo}\Ch{F}{ria} \Ch{C7}{a} \Ch{F}{\'El}!
        \end{SBVerse}

        \begin{SBVerse}
            \Ch{F}{Es}\Ch{Dm}{p�}\Ch{Gm}{ri}\Ch{C}{tu} \Ch{Dm7}{de} \Ch{C}{Dios},

            \Ch{F}{E}\Ch{Dm}{le}\Ch{Gm}{vo} a Ti \Ch{C7}{mi} \Ch{Dm}{voz};

            �\Ch{Gm}{Glo}\Ch{F}{ria} \Ch{C}{a} \Ch{F}{Ti}!

            Con \Ch{Am}{ce}les\Ch{F}{tial} fulgor

            \Ch{C7}{Me} muestras el \Ch{F}{a}\Ch{C}{mor}

            \Ch{F}{De} \Ch{Bb}{Cris}to \Ch{F}{mi} Se�or

            �\Ch{Bb}{Glo}\Ch{F}{ria} \Ch{C7}{a} \Ch{F}{Ti}!
        \end{SBVerse}

        \begin{SBVerse}
            \Ch{F}{Con} \Ch{Dm}{go}\Ch{Gm}{zo} y \Ch{Dm}{a}\Ch{C}{mor},

            \Ch{F}{Can}\Ch{Dm}{te}\Ch{Gm}{mos} con \Ch{C7}{fer}\Ch{Dm}{vor}

            \Ch{Gm}{Al} \Ch{F}{Tri}\Ch{C}{no} \Ch{F}{Dios}.

            En \Ch{Am}{la} e\Ch{F}{ter}nidad

            \Ch{C7}{Mo}ra la Tri\Ch{F}{ni}\Ch{C}{dad};

            �\Ch{F}{Por} \Ch{Bb}{siem}pre \Ch{F}{a}labad

            \Ch{Bb}{Al} \Ch{F}{Tri}\Ch{C7}{no} \Ch{F}{Dios}!
        \end{SBVerse}

        \ifChordBk
        \begin{SBOpGroup}
            Acordes:
            \upchord{\FPiano\F}{\qquad Fa} Mayor \qquad\qquad
            \upchord{\DmPiano\Dm}{\qquad Re} menor  \hfill \break
            \upchord{\GmPiano\Gm}{\qquad Sol} Menor \qquad\qquad
            \upchord{\CPiano\C}{\qquad Do} Mayor \hfill \break
            \upchord{\DmsevenPiano\Dmseven}{\qquad Re} menor s\'eptima \qquad\qquad
            \upchord{\CsevenPiano\Cseven}{\qquad Do} Mayor s\'eptima \hfill \break
            \upchord{\AmPiano\Am}{\qquad La} Menor \qquad\qquad
            \upchord{\BflatPiano\Bflat}{\qquad Si bemol} Mayor \hfill \break
        \end{SBOpGroup}
        \fi
    \end{song}


    \begin{song}{Adonai}{Em}
    {} %copyright \SBPubDom
    {Marcos Witt}
    {} %pasaje
    {} %\NotCCLIed

%  \renewcommand{\RevDate}{February~11,~1993}
%  \SBRef{No puedo parar de alabarte}{2006}%fuente \#

        \begin{SBChorus}
            \Ch{Em}{Oh} Ado\Ch{B7}{nai}. Oh Ado\Ch{Em}{nai}.
            \Ch{C}{Dios} \Ch{D}{del} Uni\Ch{Em}{ver}so, Se\Ch{B7}{�or} de la Crea\Ch{Em}{ci�n}.
        \end{SBChorus}

        \begin{SBVerse}
            Los \Ch{D}{cie}los cuentan tu \Ch{G}{glo}ria,
            tus \Ch{D}{hi}jos hoy te a\Ch{G}{do}ran,
            por \Ch{B7}{to}das \Ch{Em}{tus} mara\Ch{Am}{vi}llas, Ado\Ch{B7}{nai}.
        \end{SBVerse}

        \ifChordBk
        \begin{SBOpGroup}
            Acordes:
            \upchord{\EmPiano\Em}{\qquad Mi} Menor \qquad\qquad
            \upchord{\BsevenPiano\Bseven}{\qquad Si} S\'eptima \hfill \break
            \upchord{\CPiano\C}{\qquad Do} Mayor \qquad\qquad
            \upchord{\DPiano\D}{Re} Mayor \hfill \break
            \upchord{\GPiano\G}{\qquad Sol} Mayor \qquad\qquad
            \upchord{\AmPiano\Am}{\qquad La} Menor \hfill \break
        \end{SBOpGroup}
        \fi
    \end{song}

    \begin{song}{Ahora Que Estoy Contigo}{A}
    {\SBPubDom} %copyright \SBPubDom
    {}
    {} %pasaje
    {} %\NotCCLIed

        \begin{SBVerse}
            \Ch{A}{A}hora que estoy con\Ch{D}{ti}go en tus \Ch{A}{bra}zos de amor,

            \Ch{F\#m}{pue}do escuchar de \Ch{D}{T�} y de m� un \Ch{E7}{la}tido

            \Ch{A}{el} tuyo dice ``siempre te \Ch{D}{a}mar�'', el m�o ``\Ch{A}{te} adorar�'',

            que \Ch{F\#m}{dul}ce comu\Ch{D}{ni�n} estar con\Ch{E7}{ti}go.

            \Ch{D}{A}hora que estoy conti\Ch{A}{go}, ante tus \Ch{G}{pies}, oh Padre, yo
            me rin\Ch{D}{do},

            y en \Ch{Dm7}{mi} necesidad nada te \Ch{Em}{pe}dir�, solo te a\Ch{G}{do}rar�
        \end{SBVerse}

        \ifChordBk
        \begin{SBOpGroup}
            Acordes:
            \upchord{\APiano\A}{\qquad La} Mayor \qquad\qquad
            \upchord{\DPiano\D}{\qquad Re} Mayor \hfill \break
            \upchord{\FsmPiano\Fsm}{\qquad Fa\#} Menor \qquad\qquad
            \upchord{\EsevenPiano\Eseven}{\qquad Mi} S\'eptima \hfill \break
            \upchord{\DmsevenPiano\Dmseven}{Re} menor s\'eptima \qquad\qquad
            \upchord{\EmPiano\Em}{\qquad Mi} Menor \hfill \break
            \upchord{\GPiano\G}{\qquad Sol} Mayor \qquad\qquad
        \end{SBOpGroup}
        \fi
    \end{song}

    \begin{song}{Aleluya}{G}
    {} %copyright \SBPubDom
    {Jes�s Adri�n Romero}
    {} %pasaje
    {} %\NotCCLIed

        \SBIntro[N]{\Ch{G}{~} \Ch{Dm7/G}{~} \Ch{C}{~}}

        \FLineIdx{}

        \begin{SBOpGroup}
            En el cielo y en la tierra te alabamos OH Se\Ch{G}{�or}.

            Eres \Ch{G}{dig}no de alabanza y de suprema adoraci�n.

            Te proclamamos Se�or. Te proclamamos Se\Ch{G}{�or}.
        \end{SBOpGroup}

        \begin{SBChorus}
            Ale\Ch{G}{lu}ya, Alelu\Ch{Em7}{ya}, Ale\Ch{C2}{lu}ya \Ch{Am7}{al} Se\Ch{Dsus4}{�or}.
            Ale\Ch{G}{lu}ya, Alelu\Ch{G}{ya}, Aleluya al Se\Ch{G}{�or}.
        \end{SBChorus}

        \begin{SBOpGroup}
            \Ch{G}{En} la cruz por m� te diste para darme liber\Ch{G}{tad}.

            De la tumba resurgiste y en tu trono ahora es\Ch{G}{t�s}.

            OH Jes�s te proclamamos Se�or.te proclamamos Se�or
        \end{SBOpGroup}

        \ifChordBk
        \begin{SBOpGroup}
            Acordes:
            \upchord{\GPiano\G}{\qquad Sol} Mayor \qquad\qquad
            \upchord{\DPiano\D}{Re} Mayor \hfill \break
            \upchord{\EmsevenPiano\Emseven}{\qquad Mi} Menor S\'eptima \qquad\qquad
        \end{SBOpGroup}
        \fi
    \end{song}


    \begin{song}{Al que me ci�e de poder}{E}
    {} %copyright \SBPubDom
    {Jes�s Adri�n Romero}
    {} %pasaje
    {} %\NotCCLIed

        \begin{SBChorus}
            Al que me ci�e de po\Ch{E}{der}

            a aqu�l que mi victoria \Ch{C\#m}{es}

            s�lo a �l alaba\Ch{A}{r�}, \Ch{B7}{~}s�lo a �l exalta\Ch{E}{r�}
        \end{SBChorus}

        \begin{SBOpGroup}
            De t� ser� mi alabanza
            en la congregaci�n
            cantar� y alabar�
            tu nombre Se�or
        \end{SBOpGroup}

        \ifChordBk
        \begin{SBOpGroup}
            Acordes:
            \upchord{\EPiano\E}{\qquad Mi} Mayor \qquad\qquad
            \upchord{\CsmPiano\Csm}{\qquad Do} Sostenido Menor \hfill \break
            \upchord{\APiano\A}{\qquad La} Mayor \qquad\qquad
            \upchord{\BsevenPiano\Bseven}{\qquad Si} S\'eptima \hfill \break
        \end{SBOpGroup}
        \fi
    \end{song}


    \begin{song}{Al Trono Majestuoso}{Eb}
    {} %copyright \SBPubDom
    {Aurelia \& Samuel sebastian Wesley}
    {} %pasaje
    {} %\NotCCLIed

        \begin{SBChorus}
            \Ch{Eb}{Al} trono majestuoso
            \Ch{Eb}{Del} Dios omnipoten\Ch{Eb}{te},
            Hu\Ch{Eb}{mil}des vuestra frente,
            naciones inclinad
            �l es el ser supre\Ch{Eb}{mo},
            Se�or de cuanto existe,
            y nada al fin \Ch{Eb}{res}iste
            Al grande Jeho\Ch{Eb}{v�}
        \end{SBChorus}

        \begin{SBOpGroup}
            Del polvo de la tierra
            formonos complacida
            su mano, y dionos vida
            su aliento creador.
            Y al vernos despu�s ciegos,
            en la maldad sumidos,
            Cual padre a hijos queridos
            Salud nos provey�.
        \end{SBOpGroup}

        \begin{SBOpGroup}
            La gratitud sincera
            nos dictar� canciones
            y en coro dulces sones
            al cielo subir�n
            con los celestes himnos
            arm�nica alianza
            formando, su alabanza
            doquier resonar�.
        \end{SBOpGroup}

        \begin{SBOpGroup}
            Se�or, a tu palabra
            los mundos obedecen,
            y del mortal perecen
            la ciencia y altivez.
            Tu amor y verdad solos
            en nada habr�n menguado,
            despu�s que hayan cesado
            los siglos de correr.
        \end{SBOpGroup}

        \ifChordBk
        \begin{SBOpGroup}
            Acordes:
            \upchord{\EflatPiano\Eflat}{\qquad Mi} bemol Mayor \qquad\qquad
        \end{SBOpGroup}
        \fi
    \end{song}

    \begin{song}{Alza tus ojos}{Am}
    {} %copyright \SBPubDom
    {Marcos Barrientos}
    {} %pasaje
    {} %\NotCCLIed


        \begin{SBOpGroup}
            \Ch{Am}{Al}za tus ojos y \Ch{F}{mi}ra, \Ch{C}{la} cosecha est� \Ch{E}{lis}ta

            el tiempo ha llegado la mies est� madura
            esfu�rzate y s� valiente lev�ntate y predica
            a todas las naciones que Cristo es la Vida
        \end{SBOpGroup}

        \begin{SBChorus}
            Y ser� llena la tierra de su gloria
            se cubrir� como las aguas cubren la mar
        \end{SBChorus}

        \begin{SBOpGroup}
            No, no hay otro nombre,
            dado a los hombres,
            Jesucristo es el Se�or
        \end{SBOpGroup}

        \ifChordBk
        \begin{SBOpGroup}
            Acordes:
            \upchord{\AmPiano\Am}{\qquad La} Menor \qquad\qquad
            \upchord{\FPiano\F}{\qquad Fa} Mayor \hfill \break
            \upchord{\CPiano\C}{\qquad Do} Mayor \qquad\qquad
            \upchord{\EPiano\E}{\qquad Mi} Mayor \hfill \break
            \upchord{\GPiano\G}{\qquad Sol} Mayor \qquad\qquad
            \upchord{\DmPiano\Dm}{\qquad Re} Menor \hfill \break
        \end{SBOpGroup}
        \fi
    \end{song}

    \begin{song}{Aqu� estoy}{A}
    {} %copyright \SBPubDom
    {Jes�s Adri�n Romero}
    {} %pasaje
    {} %\NotCCLIed


        \begin{SBOpGroup}
            A\Ch{A}{qu�} estoy, te ofrezco todo lo que soy
            Aqu� estoy, un sacrificio quiero ser
            Toma mi ser, mi vida entrego a T�
        \end{SBOpGroup}

        \begin{SBOpGroup}

            Porque T� eres mi Dios,
            Eres digno de adoraci�n
            Una ofrenda de amor ser�
            Para T�

        \end{SBOpGroup}

        \ifChordBk
        \begin{SBOpGroup}
            Acordes:
            \upchord{\APiano\A}{\qquad La} Mayor \qquad\qquad
        \end{SBOpGroup}
        \fi
    \end{song}

    \begin{song}{A sus pies}{Am}
    {} %copyright \SBPubDom
    {Jes�s Adri�n Romero}
    {} %pasaje
    {} %\NotCCLIed


        \begin{SBOpGroup}
            Cuando el mundo te inunda de fatalidad
            y te agobia la vida con su mucho af�n
            y se llena tu alma de preocupaci�n
            y se seca la fuente de tu coraz�n
        \end{SBOpGroup}

        \begin{SBChorus}
            Puedes sentarte a sus pies
            y de sus manos beber
            la plenitud que tu alma necesita
            puedes sentarte a sus pies
            y cada d�a tener
            una nueva canci�n y nueva vida
            a sus pies hay paz,
            gracia y bendici�n
            a sus pies tendr�s
            luz y direcci�n.
            la plenitud en �l
            nunca se agotar�
            puedes descansar en su presencia
        \end{SBChorus}

        \begin{SBOpGroup}
            Cuando quieras huir por que no puedes m�s
            por que s�lo te sientes entre los dem�s
            y no hay m�s en tus ojos brillo y emoci�n
            y se cierra tu boca por que no hay canci�n
        \end{SBOpGroup}



        \ifChordBk
        \begin{SBOpGroup}
            Acordes:
            \upchord{\AmPiano\Am}{\qquad La} Menor \qquad\qquad
        \end{SBOpGroup}
        \fi
    \end{song}

    \begin{song}{Bendice Nuestra Ofrenda}{G}
    {} %copyright \SBPubDom
    {D. Tinoco, G Franc}
    {} %pasaje
    {} %\NotCCLIed

        \begin{SBOpGroup}
            \Ch{G}{Ben}di\Ch{D}{ce} \Ch{Em}{Nues}\Ch{Bm}{tra} o\Ch{Em}{fren}\Ch{D}{da} oh \Ch{G}{Dios},

            Que pre\Ch{D}{sen}ta\Ch{Em}{mos} \Ch{C}{con} \Ch{G}{a}\Ch{D}{mor}.

            \Ch{Em}{Es} \Ch{D}{prue}\Ch{G}{ba} \Ch{D}{fiel} \Ch{G}{de} \Ch{C}{gra}\Ch{Am7}{ti}\Ch{G}{tud},

            Por tu \Ch{Em}{bon}\Ch{D}{dad} \Ch{Am}{en} ple\Ch{D}{ni}\Ch{G}{tud}.

            \Ch{C}{A}\Ch{G}{m�n}
        \end{SBOpGroup}

        \ifChordBk
        \begin{SBOpGroup}
            Acordes:
            \upchord{\GPiano\G}{\qquad Sol} Mayor \qquad\qquad
            \upchord{\DPiano\D}{\qquad Re} Mayor \hfill \break
            \upchord{\EmPiano\Em}{\qquad Mi} Menor \qquad\qquad
            \upchord{\BmPiano\Bm}{\qquad Si} Menor \hfill \break
            \upchord{\CPiano\C}{\qquad Do} Mayor \qquad\qquad
            \upchord{\AmsevenPiano\Amseven}{\qquad La} Menor S\'eptima \hfill \break
            \upchord{\AmPiano\Am}{\qquad La} Menor \qquad\qquad
            \upchord{\GBPiano\GB}{\qquad Sol} Mayor con bajo B \hfill \break
        \end{SBOpGroup}
        \fi
    \end{song}

    \begin{song}{Bueno es alabar}{G}
    {} %copyright \SBPubDom
    {Danilo Montero}
    {} %pasaje
    {\href{https://open.spotify.com/intl-es/track/6S90J1ENA9sgzfBDEHNyJ7}{Escuchar}}

        \begin{SBOpGroup}
            \Ch{G}{Bue}no es ala\Ch{C}{bar} �oh Se\Ch{D}{�or}!, tu \Ch{C}{nom}\Ch{D}{bre}

            \Ch{G}{dar}te honra, \Ch{C}{glo}ria y ho\Ch{D}{nor} por \Ch{C}{siem}\Ch{D}{pre},

            \Ch{G}{Bue}no es ala\Ch{C}{bar}te Je\Ch{D}{s�s}

            y go\Ch{Am7}{zar}me en tu po\Ch{D}{der}
        \end{SBOpGroup}

        \begin{SBChorus}
            Porque \Ch{G}{gran}de \Ch{C}{e}res \Ch{D}{T�}

            \Ch{G}{gran}des \Ch{C}{son} tus o\Ch{D}{bras},

            porque \Ch{G}{gran}de \Ch{C}{e}res \Ch{D}{T�}

            grande es tu a\Ch{Em}{mor}, grande es tu \Ch{D}{glo}ria
        \end{SBChorus}
        \ifChordBk
        \begin{SBOpGroup}
            Acordes:
            \upchord{\GPiano\G}{\qquad Sol} Mayor \qquad\qquad
            \upchord{\CPiano\C}{\qquad Do} Mayor  \hfill \break
            \upchord{\DPiano\D}{\qquad Re} Mayor \qquad\qquad
            \upchord{\AmsevenPiano\Amseven}{\qquad La} Menor S\'eptima \hfill \break
            \upchord{\EmPiano\Em}{\qquad Mi} Menor \qquad\qquad
        \end{SBOpGroup}
        \fi
    \end{song}

    \begin{song}{Cada d�a con Cristo}{D}
    {} %copyright \SBPubDom
    {}
    {Isa�as 26:3} %pasaje
    {\href{https://open.spotify.com/intl-es/track/0EtXTYDHwK81dnsloMPNha}{Escuchar}} %\NotCCLIed

        \begin{SBOpGroup}
            \Ch{D}{Ca}da d�a con Cristo me llena de perfecta \Ch{Em}{paz}

            \Ch{A7}{ca}da d�a con \Ch{Em}{Cris}to le amo \Ch{A7}{m�s} y m�s

            \Ch{D}{\'El} me salva y guarda y \Ch{D7}{s�} que pronto volve\Ch{G}{r�}

            y vi\Ch{Gm}{vir} con \Ch{D}{Cris}to m�s \Ch{A}{dul}ce cada d�a se\Ch{D}{r�}
        \end{SBOpGroup}

        \ifChordBk
        \begin{SBOpGroup}
            Acordes:
            \upchord{\DPiano\D}{\qquad Re} Mayor \qquad\qquad
            \upchord{\EmPiano\Em}{\qquad Mi} Menor \hfill \break
            \upchord{\AsevenPiano\Aseven}{\qquad La} Mayor S\'eptima \qquad\qquad
            \upchord{\DsevenPiano\Dseven}{\qquad Re} Mayor S\'eptima \hfill \break
            \upchord{\GPiano\G}{\qquad Sol} Mayor \qquad\qquad
            \upchord{\GmPiano\Gm}{\qquad Sol} Menor \hfill \break
            \upchord{\APiano\A}{\qquad La} Mayor \qquad\qquad
        \end{SBOpGroup}
        \fi
    \end{song}

    \begin{song}{Cada Ma�ana}{D}
    {} %copyright \SBPubDom
    {Jes�s Adri�n Romero}
    {} %pasaje
    {} %\NotCCLIed

        \begin{SBOpGroup}
            \Ch{D}{Ca}da ma�ana al despertar,
            y por la noche al descansar,
            agradezco tus bondades a mi vida
            por todo lo que me permites disfrutar
        \end{SBOpGroup}

        \begin{SBChorus}
            /// Ale-lu-u-ya ///
            // Agradecido estoy por tu bondad.//
            Agradecido estoy por tu bondad.
        \end{SBChorus}

        \ifChordBk
        \begin{SBOpGroup}
            Acordes:
            \upchord{\FsPiano\Fs}{\qquad Fa} Sostenido Mayor \qquad\qquad
            \upchord{\APiano\A}{\qquad La} Mayor \hfill \break
            \upchord{\DPiano\D}{\qquad Re} Mayor \qquad\qquad
            \upchord{\GPiano\G}{\qquad Sol} Mayor \hfill \break
            \upchord{\BmsevenPiano\Bmseven}{\qquad Si} Menor S\'eptima \qquad\qquad
            \upchord{\EmPiano\Em}{\qquad Mi} Menor \hfill \break
            \upchord{\DsevenPiano\Dseven}{\qquad Re} Mayor S\'eptima \qquad\qquad
        \end{SBOpGroup}
        \fi
    \end{song}

    \begin{song}{Cantar� de tu Amor}{F}
    {} %copyright \SBPubDom
    {Danilo Montero}
    {} %pasaje
    {\href{https://open.spotify.com/intl-es/track/7kzqc0u8SNQAU5Wca0WNKo}{Escuchar}} %\NotCCLIed

%        \SBRef{El aire de tu casa}{2005}%fuente \#

        \begin{SBOpGroup}
            \Ch{F}{Por} mucho tiempo bus\Ch{C/E}{qu�}

            \Ch{F}{u}na raz�n de vi\Ch{C/E}{vir}

            \Ch{F}{en} medio de \Ch{G}{mil} pre\Ch{Am7}{gun}tas

            \Ch{F}{tu} amor me respon\Ch{G}{di�}
        \end{SBOpGroup}


        \begin{SBOpGroup}
            \Ch{F}{A}hora veo la \Ch{C/E}{luz}

            \Ch{F}{y} ya no tengo te\Ch{C/E}{mor}

            \Ch{F}{tu} reino \Ch{G}{vi}no a mi \Ch{Am7}{vi}da

            \Ch{F}{y} ahora vivo para \Ch{G}{T�}
        \end{SBOpGroup}

        \begin{SBChorus}
            Canta\Ch{C}{r�} de tu a\Ch{G/B}{mor} rendi\Ch{Am7}{r�} mi cora\Ch{C}{z�n} ante \Ch{F}{t�}

            tu se\Ch{C}{r�s} mi pa\Ch{G/B}{si�n} y mis \Ch{Am7}{pa}sos se guia\Ch{C}{r�n} por tu \Ch{F}{voz}

            mi Je\Ch{Dm}{s�s} y mi \Ch{G}{Rey} de tu \Ch{F}{gran} amor canta\Ch{G}{r�}
        \end{SBChorus}

        \ifChordBk
        \begin{SBOpGroup}
            Acordes:
            \upchord{\FPiano\F}{\qquad Fa} Mayor \qquad\qquad
            \upchord{\CEPiano\CE}{\qquad Do} Mayor Bajo E  \hfill \break
            \upchord{\GPiano\G}{\qquad Sol} Mayor \qquad\qquad
            \upchord{\AmsevenPiano\Amseven}{\qquad La} Menor S\'eptima \hfill \break
            \upchord{\GBPiano\GB}{\qquad Sol} Mayor Bajo B \qquad\qquad\qquad\qquad\qquad
            \upchord{\DmPiano\Dm}{\qquad Re} Menor \hfill \break
            \upchord{\CPiano\C}{\qquad Do} Mayor \qquad\qquad
        \end{SBOpGroup}

        \fi
    \end{song}

    \begin{song}{Cara A Cara}{D}
    {Vidal Music} %copyright \SBPubDom
    {Marcos Vidal}
    {} %pasaje
    {} %\NotCCLIed


        \begin{SBChorus}
            Solamente una palabra,
            solamente una oraci�n,
            cuando llegue a Tu presencia, oh Se�or,
            no me importa en que lugar
            de la mesa me hagas sentar,
            o el color de mi corona, si la llego a ganar

            S�lo d�jame mirarte, cara a cara,
            y perderme como un ni�o en Tu mirada,
            y que pase mucho tiempo,
            y que nadie diga nada,
            por que estoy viendo al Maestro,
            cara a cara
        \end{SBChorus}

        \begin{SBOpGroup}
            Solamente una palabra,
            si es que a�n me queda voz,
            y si logro articularla en Tu presencia,
            no te quiero hacer preguntas, s�lo una petici�n,
            y si puede ser a solas, mucho mejor
        \end{SBOpGroup}

        \ifChordBk
        \begin{SBOpGroup}
            Acordes:
            \upchord{\DPiano\D}{\qquad Re} Mayor \qquad\qquad
        \end{SBOpGroup}
        \fi
    \end{song}


    \begin{song}{Cerca de T�}{G}
    {Vidal Music} %copyright \SBPubDom
    {Marcos Vidal}
    {} %pasaje
    {} %\NotCCLIed

        \begin{SBOpGroup}
            Si decidiera negar mi fe
            y no con ar nunca m�s en �l
            no tengo a donde ir, no tengo a donde ir
        \end{SBOpGroup}


        \begin{SBOpGroup}
            Si despreciare en mi coraz�n
            la santa gracia que me salv�
            no tengo a donde ir, no tengo a donde ir
        \end{SBOpGroup}

        \begin{SBOpGroup}
            Convencido estoy que sin tu amor se acabar�an mis
            fuerzas
            y sin T� mi coraz�n sediento se muere, se seca
        \end{SBOpGroup}

        \begin{SBOpGroup}
            Cerca de T�, yo quiero estar
            de tu presencia no me quiero alejar
        \end{SBOpGroup}

        \ifChordBk
        \begin{SBOpGroup}
            Acordes:
            \upchord{\DPiano\D}{\qquad Re} Mayor \qquad\qquad
        \end{SBOpGroup}
        \fi
    \end{song}

    \begin{song}{�C�mo Podr� Estar Triste?}{C}
    {} %copyright \SBPubDom
    {}
    {} %pasaje
    {} %\NotCCLIed

        \begin{SBOpGroup}
            C�mo podr� estar triste,
            c�mo entre sombras ir,
            c�mo sentirme solo
            y en el dolor vivir,
            siCristo es mi consuelo,
            mi amigo siempre el,
            // si aun las aves tienen
            Seguro asilo en �l //
        \end{SBOpGroup}

        \begin{SBChorus}
            Feliz cantando alegre
            yo vivo siempre aqu�;
            si El cuida de las aves
            cuidar� tambi�n de m�.
        \end{SBChorus}

        \begin{SBOpGroup}
            Nunca te desalientes,
            oigo al Se�or decir,
            y en su Palabra ado
            hago al dolor huir.
            A Cristo paso a paso
            yo sigo sin cesar,
            //y todas sus bondades
            me da sin limitar
        \end{SBOpGroup}

        \begin{SBOpGroup}
            Siempre que soy tentado
            o que en la sombra estoy,
            m�s cerca de �l camino
            y protegido voy.
            Si en mi la fe desmaya
            y caigo en la ansiedad
            //tan s�lo El me levanta,
            me da seguridad//
        \end{SBOpGroup}

        \ifChordBk
        \begin{SBOpGroup}
            Acordes:
            \upchord{\DPiano\D}{\qquad Re} Mayor \qquad\qquad
        \end{SBOpGroup}
        \fi
    \end{song}

    \begin{song}{Con C�nticos Se�or}{C}
    {} %copyright \SBPubDom
    {}
    {} %pasaje
    {} %\NotCCLIed

        \begin{SBOpGroup}

            Con c�nticos Se�or mi coraz�n y voz
            te adoran con fervor oh trino Santo Dios
            Tu mano paternal marc� mi senda aqu�
            mis pasos, cada cual, velados son por T�
            Innumerables son tus bienes y sin par
            que por tu compasi�n recibo sin cesar
            T� eres oh Se�or mi sumo todo bien
            mil lenguas tu amor cantando siempre est�n

        \end{SBOpGroup}

        \begin{SBChorus}

            En tu Mansi�n yo te ver�
            de t� perd�n fel�z tendr�


            En tu Mansi�n yo te ver�
            y galard�n fel�z tendr�

        \end{SBChorus}

        \ifChordBk
        \begin{SBOpGroup}
            Acordes:
            \upchord{\DPiano\D}{\qquad Re} Mayor \qquad\qquad
        \end{SBOpGroup}
        \fi
    \end{song}

    \begin{song}{Con Manos Vac�as}{E}
    {} %copyright \SBPubDom
    {Jes�s Adri�n Romero}
    {} %pasaje
    {} %\NotCCLIed

        \begin{SBOpGroup}
            \Ch{E}{Con} manos vac�as \Ch{A}{ven}go a T�

            \Ch{B}{no} tengo nada que dar\Ch{C\#m}{te}

            \Ch{F\#m}{no} hay nada de valor en \Ch{C\#m}{m�}

            \Ch{A}{no} puedo im\Ch{F\#m}{pre}sio\Ch{B}{nar}te
        \end{SBOpGroup}

        \begin{SBOpGroup}
            \Ch{E}{Te} puedo entregar mi \Ch{A}{co}raz�n,

            \Ch{B}{pe}ro est� quebran\Ch{C\#m}{ta}do

            \Ch{F\#m}{re}c�belo mi buen Pas\Ch{C\#m}{tor},

            \Ch{A}{Tu} puedes \Ch{F\#m}{res}tau\Ch{B}{rar}lo
        \end{SBOpGroup}


        \begin{SBChorus}
            \Ch{A}{Pon}go mi \Ch{B}{vi}da a tu servicio Se\Ch{C\#m}{�or}

            \Ch{A}{no} ser� \Ch{B}{mu}cho, pero la entrego \Ch{C\#m}{hoy}

            \Ch{A}{y} si mis \Ch{B}{ma}nos hoy vac�as es\Ch{C\#m}{t�n},

            \Ch{F\#m7}{pue}des lle\Ch{C\#m/G\#}{nar}las con tu \Ch{C\#m}{gran} po\Ch{B}{der} y a\Ch{A}{mor}.

            \Ch{B}{U}sa mis manos Se\Ch{C\#m}{�or}
        \end{SBChorus}

        \ifChordBk
        \begin{SBOpGroup}
            Acordes:
            \upchord{\EPiano\E}{\qquad Mi} Mayor \qquad\qquad
            \upchord{\APiano\A}{\qquad La} Mayor \hfill \break
            \upchord{\BPiano\B}{\qquad Si} Mayor \qquad\qquad
            \upchord{\CsmPiano\Csm}{\qquad Do sostenido} Menor \hfill \break
            \upchord{\FsmPiano\Fsm}{\qquad Fa sostenido} Menor \qquad\qquad
            \upchord{\GssusPiano\Gssus}{\qquad Sol sostenido} suspendida cuarta \hfill \break
            \upchord{\CsmPiano\Csm}{\qquad Do sostenido} Menor \qquad\qquad
            \upchord{\FsmsevenPiano\Fsmseven}{\qquad Fa sostenido} S\'eptima Menor \qquad\qquad
        \end{SBOpGroup}
        \fi

        \begin{SBExtraKeys}{
            \STitle{Con Manos Vac�as}{F}

            \begin{SBOpGroup}
                \Ch{F}{Con} manos vac�as \Ch{Bb}{ven}go a T� \Ch{C}{no} tengo nada que darte

                no hay nada de valor en m� \Ch{Bb}{no} puedo impresio\Ch{C}{nar}te
            \end{SBOpGroup}

            \begin{SBOpGroup}
                \Ch{F}{Te} puedo entregar mi \Ch{Bb}{co}raz�n,

                \Ch{C}{pe}ro est� quebrantado

                rec�belo mi buen Pastor,

                \Ch{Bb}{Tu} puedes restau\Ch{C}{rar}lo
            \end{SBOpGroup}

            \begin{SBChorus}
                \Ch{Bb}{Pon}go mi \Ch{C}{vi}da a tu servicio Se�or

                \Ch{Bb}{no} ser� \Ch{C}{mu}cho, pero la entrego hoy

                \Ch{Bb}{y} si mis \Ch{C}{ma}nos hoy vac�as est�n,

                puedes llenarlas con tu gran po\Ch{C}{der} y a\Ch{Bb}{mor}.

                \Ch{C}{U}sa mis manos Se�or
            \end{SBChorus}

            \ifChordBk
            \begin{SBOpGroup}
                Acordes:
                \upchord{\FPiano\F}{\qquad Fa} Mayor \qquad\qquad
                \upchord{\BflatPiano\Bflat}{\qquad Si bemol} Mayor \hfill \break
                \upchord{\CPiano\C}{\qquad Do} Mayor \qquad\qquad
            \end{SBOpGroup}
            \fi
        }\end{SBExtraKeys}
    \end{song}

    \begin{song}{Con mi Dios}{Em}
    {} %copyright \SBPubDom
    {}
    {} %pasaje
    {} %\NotCCLIed

        \begin{SBOpGroup}
            Con mi Dios yo saltar� los muros
            Con mi Dios Ej�rcitos derribar
        \end{SBOpGroup}

        \begin{SBOpGroup}
            �l adiestra mis manos para la batalla,
            puedo tomar con mis manos el arco de bronce
        \end{SBOpGroup}

        \begin{SBOpGroup}
            �l es escudo, la roca m�a,
            �l es la fuerza de mi salvaci�n
            Mi alto refugio, mi fortaleza
            �l es mi libertador
        \end{SBOpGroup}


        \ifChordBk
        \begin{SBOpGroup}
            Acordes:
            \upchord{\EmPiano\Em}{\qquad Mi} Menor \qquad\qquad
        \end{SBOpGroup}
        \fi
    \end{song}

    \begin{song}{Con Mis Labios}{D}
    {} %copyright \SBPubDom
    {}
    {} %pasaje
    {} %\NotCCLIed

        \begin{SBOpGroup}
            Con mis labios y mi vida
            te alabo Se�or, te alabo Se�or,
            con mis labios y mi vida te alabo bendito Se�or,
        \end{SBOpGroup}

        \begin{SBChorus}
            Porque T� has sido precioso para m�, precioso
            para m�,precioso para m�,
            Porque T� has sido precioso para m�, te alabo
            bendito Se�or
        \end{SBChorus}

        \ifChordBk
        \begin{SBOpGroup}
            Acordes:
            \upchord{\DPiano\D}{\qquad Re} Mayor \qquad\qquad
        \end{SBOpGroup}
        \fi
    \end{song}

    \begin{song}{Con Mis Manos Levantadas}{G}
    {} %copyright \SBPubDom
    {}
    {} %pasaje
    {} %\NotCCLIed

        \begin{SBOpGroup}
            Con mis manos levantadas hacia el cielo me
            Presento ante Ti hoy Se�or para recibir de Ti
            La fuerza y el poder para vivir junto a Ti.
        \end{SBOpGroup}

        \begin{SBChorus}
            Llenas hoy mi coraz�n con Tu presencia
            Llenas de alegr�a y paz todo mi ser
            De cualquier necesidad T� me responder�s
            Porque me amas, me amas
        \end{SBChorus}

        \ifChordBk
        \begin{SBOpGroup}
            Acordes:
            \upchord{\GPiano\G}{\qquad Sol} Mayor \qquad\qquad
        \end{SBOpGroup}
        \fi
    \end{song}

    \begin{song}{Cristo me Ayuda por �l a Vivir}{F}
    {} %copyright \SBPubDom
    {}
    {} %pasaje
    {} %\NotCCLIed

        \begin{SBOpGroup}
            Cristo me ayuda por �l a vivir
            Cristo me ayuda por �l a morir;
            Hasta que llegue su gloria a ver,
            Cada momento le entrego mi ser
        \end{SBOpGroup}

        \begin{SBChorus}
            Cada momento la vida me da,
            Cada momento conmigo �l est�;
            Hasta que llegue su gloria a ver,
            Cada momento le entrego ni s�r.
        \end{SBChorus}

        \begin{SBOpGroup}
            Siento pesares, muy cerca �l est�,
            Siento dolores, a livio me da;
            Tengo aflicciones, me muestra su amor;
            Cada momento me cuidas Se�or.
        \end{SBOpGroup}

        \begin{SBOpGroup}
            Tengo amarguras, o tengo temor
            Tengo tristezas, me inspira valor;
            Tengo conflictos o penas aqu�
            Cada momento te acuerdas de m�
        \end{SBOpGroup}

        \begin{SBOpGroup}
            Tengo aquezas, o d�bil estoy,
            Cristo de dice Tu amparo yo soy ;
            Cada momento en tinieblas o en luz,
            Siempre conmigo est� mi Jes�s.
        \end{SBOpGroup}

        \ifChordBk
        \begin{SBOpGroup}
            Acordes:
            \upchord{\FPiano\F}{\qquad Fa} Mayor \qquad\qquad
        \end{SBOpGroup}
        \fi
    \end{song}

    \begin{song}{Cristo t� me has amado}{G}
    {} %copyright \SBPubDom
    {}
    {} %pasaje
    {Ingrid Rosario} %\NotCCLIed

        \begin{SBOpGroup}
            Cristo t� me has amado
            Cristo nunca de t� me apartar�
            Del pecado T� me has rescatado
            Mis pies pusiste en la roca y yo s� que
        \end{SBOpGroup}

        \begin{SBChorus}
            Te amo, por siempre
            Aunque el mundo alrededor pueda cambiar
            Tu eres mi Salvador
            Yo te adorar� por la eternidad.
        \end{SBChorus}

        \ifChordBk
        \begin{SBOpGroup}
            Acordes:
            \upchord{\GPiano\G}{\qquad Sol} Mayor \qquad\qquad
        \end{SBOpGroup}
        \fi
    \end{song}

    \begin{song}{Cu�n bello es el Se�or}{D}
    {} %copyright \SBPubDom
    {}
    {} %pasaje
    {Ingrid Rosario} %\NotCCLIed

        \begin{SBOpGroup}
            //Cu�n bello es el Se�or
            Cu�n hermoso es el Se�or
            //Cu�n bello es el Se�or
            Hoy le quiero adorar//
        \end{SBOpGroup}

        \begin{SBChorus}
            La belleza de mi Se�or.
            Nunca se agotar�
            La hermosura de mi Se�or
            Siempre resplandecer�
        \end{SBChorus}

        \ifChordBk
        \begin{SBOpGroup}
            Acordes:
            \upchord{\DPiano\D}{\qquad Re} Mayor \qquad\qquad
        \end{SBOpGroup}
        \fi
    \end{song}

    \begin{song}{Dad a Dios Inmortal alabanza}{D}
    {} %copyright \SBPubDom
    {}
    {} %pasaje
    {} %\NotCCLIed

        \begin{SBOpGroup}
            Dad a \Ch{D}{Dios} inmortal ala\Ch{G/D}{ban}\Ch{D}{za};
            Su merced, su verdad \Ch{G}{nos} i\Ch{A}{nun}da;
            \Ch{D}{Es} su gracia en prodigios fe\Ch{G/D}{cun}\Ch{D}{da},
            Sus mer\Ch{C\#}{ce}des, humildes can\Ch{F\#m}{tad}
            �\Ch{D}{Al} Se\Ch{A7}{�or} de se�ores dad \Ch{D}{glo}ria,
            Rey de reyes, poder \Ch{A}{sin} se\Ch{D}{gun}do!
            Mori\Ch{A}{r�n} los se�ores del \Ch{D}{mun}do,
            mas su reino no a\Ch{A7}{ca}ba Ja\Ch{D}{m�s}.
        \end{SBOpGroup}

        \begin{SBOpGroup}
            Las naciones vi� en vicios sumidas
            Y sinti� compasi�n en su seno;
            De prodigios de gracia est� lleno;
            Sus mercedes, humildes cantad
            A su pueblo llev� por la mano
            A la tierra por �l prometida
            Por los siglos sin n le da vida
            y el pecado y la muerte caer�n.
        \end{SBOpGroup}

        \begin{SBOpGroup}
            A su Hijo envi� por salvarnos
            Del pecado y la muerte inherente:
            De prodigios de gracia es torrente,
            Sus mercedes, humildes cantad
            Por el mundo su mano nos lleva
            Y al celeste descanso nos gu�a;
            Su bondad vivir� eterno d�a,
            Cuando el mundo no exista ya m�s
        \end{SBOpGroup}

        \ifChordBk
        \begin{SBOpGroup}
            Acordes:
            \upchord{\DPiano\D}{\qquad Re} Mayor \qquad\qquad
            \upchord{\GDPiano\GD}{\qquad Sol} Mayor Bajo D \hfill \break
            \upchord{\APiano\A}{\qquad La} Mayor \qquad\qquad
            \upchord{\CsPiano\Cs}{\qquad Do sostenido} Mayor \hfill \break
            \upchord{\FsmPiano\Fsm}{\qquad Fa sostenido} Menor \qquad\qquad
            \upchord{\AsevenPiano\Aseven}{\qquad La} Mayor S\'eptima \hfill \break
        \end{SBOpGroup}
        \fi
    \end{song}

    \begin{song}{Damos Honor a Ti}{E}
    {} %copyright \SBPubDom
    {}
    {} %pasaje
    {} %\NotCCLIed

        \begin{SBOpGroup}
            Damos honor a Ti, damos honor a Ti.
            Creador de vida eres T�.
            Damos honor a Ti, damos honor a Ti.
            Porque no hay otro Dios como T�.
            Rey de reyes, Admirable.
            Eres el Principio y Fin.
            Soberano y sublime.
            Eres nuestro Salvador.

        \end{SBOpGroup}

        \ifChordBk
        \begin{SBOpGroup}
            Acordes:
            \upchord{\EPiano\E}{\qquad Mi} Mayor \qquad\qquad
        \end{SBOpGroup}
        \fi
    \end{song}

    \begin{song}{DeGloriaEnGloria}{D}
    {} %copyright \SBPubDom
    {}
    {} %pasaje
    {Marcos Witt} %\NotCCLIed

        \begin{SBOpGroup}
            De \Ch{D}{glo}ria en \Ch{A/C\#}{glo}ria te \Ch{Bm}{veo} \Ch{Am7}{~}
            cuanto m�s te conozco
            Quiero saber m�s de T�
        \end{SBOpGroup}

        \begin{SBOpGroup}
            Mi \Ch{D}{Dios} cuan buen alfarero
            Quebr�ntame transf�rmame mold�ame a tu imagen,
            Se�or
        \end{SBOpGroup}

        \begin{SBOpGroup}
            Quiero ser mas como Tu
            ver la vida como Tu
            Saturarme de tu esp�ritu
        \end{SBOpGroup}

        \SBEnd{y reflejar al mundo tu a\Ch{D}{mor}}

        \ifChordBk
        \begin{SBOpGroup}
            Acordes:
            \upchord{\DPiano\D}{\qquad Re} Mayor \qquad\qquad
            \upchord{\ACsPiano\ACs}{\qquad La} Mayor con bajo C\# \hfill \break
            \upchord{\BmPiano\Bm}{\qquad Si} Menor \qquad\qquad
            \upchord{\AmsevenPiano\Amseven}{\qquad La} Menor S\'eptima \hfill \break
        \end{SBOpGroup}
        \fi
    \end{song}

    \begin{song}{Delante de tu Trono}{G}
    {} %copyright \SBPubDom
    {}
    {} %pasaje
    {Marco Barrientos}

        \begin{SBOpGroup}
            // De\Ch{G}{lan}te de tu trono
            Se\Ch{Em}{�or} yo quiero estar
            \Ch{C}{pa}ra contemplar
            tu hermo\Ch{D}{su}ra y santidad //
        \end{SBOpGroup}

        \begin{SBOpGroup}
            Y de\Ch{G}{ci}\Ch{D}{rte}: Te \Ch{Em}{a}mo
            Y decir\Ch{G}{te}: Te a\Ch{Am}{do}\Ch{D}{ro}
            Y de\Ch{G}{cir}\Ch{D}{te}: Te \Ch{Em}{a}mo
            Y que eres todo \Ch{D}{pa}ra \Ch{G}{m�}.
        \end{SBOpGroup}

        \ifChordBk
        \begin{SBOpGroup}
            Acordes:
            \upchord{\GPiano\G}{\qquad Sol} Mayor \qquad\qquad
            \upchord{\EmPiano\Em}{\qquad Mi} Menor \hfill \break
            \upchord{\CPiano\C}{\qquad Do} Mayor \qquad\qquad
            \upchord{\DPiano\D}{\qquad Re} Mayor \hfill \break
            \upchord{\AmPiano\Am}{\qquad La} Menor \qquad\qquad
        \end{SBOpGroup}
        \fi
    \end{song}

    \begin{song}{Demos gracias al Se�or}{C}
    {} %copyright \SBPubDom
    {}
    {} %pasaje
    {}

        \begin{SBOpGroup}
            // Demos gracias al Se�or, demos gracias,
            demos gracias por su amor //
        \end{SBOpGroup}

        \begin{SBOpGroup}
            Por la ma�ana, las aves cantan
            sus alabanza a Dios el Creador,
            tambi�n nosotros a �l cantemos
            y alabemos a Cristo el Redentor
        \end{SBOpGroup}

        \ifChordBk
        \begin{SBOpGroup}
            Acordes:
            \upchord{\CPiano\C}{\qquad Do} Mayor \qquad\qquad
        \end{SBOpGroup}
        \fi
    \end{song}

    \begin{song}{Me dice que me ama}{G}
    {} %copyright \SBPubDom
    {Jes�s Adri�n Romero}
    {} %pasaje
    {\NotCCLIed} %\NotCCLIed

%        \SBRef{El aire de tu casa}{2005}%fuente \#

        \SBIntro[N]{\Ch{G}{~} \Ch{C}{~} \Ch{Am}{~} \Ch{D}{~}}

        \begin{SBOpGroup}
            Me dice que me ama cuando escucho llover,
            G C
            me dice que ama con un atardecer,
            G C Em D
            lo dice sin palabras, con las olas del mar,
            Em C D
            lo dice en la ma�ana con mi respirar.
        \end{SBOpGroup}

        \begin{SBChorus}
            G C G C G
            //Me dice que me ama y que conmigo quiere estar,
            C Am7 D
            me dice que me busca cuando salgo yo a pasear,
            Am7 C Em
            que ha hecho lo que existe para llamar mi atenci�n,
            Am D G C D
            que quiere conquistarme y alegrar mi coraz�n.
        \end{SBChorus}

        \begin{SBOpGroup}
            G C
            Me dice que me ama cuando veo la cruz,
            G C
            sus manos extendidas as� tan grande es su amor,
            G C Em D
            lo dicen las heridas de sus manos y pies,
            Em C D
            me dice que me ama una y otra vez.
        \end{SBOpGroup}
        \ifChordBk
        \begin{SBOpGroup}
            Acordes:
            \upchord{\GPiano\G}{\qquad Sol} Mayor \qquad\qquad
        \end{SBOpGroup}

        \fi
    \end{song}

    \begin{song}{Par\'abola}{G}
    {} %copyright \SBPubDom
    { Marcos Vidal }
    {Lucas 10:30} %pasaje
    {\href{https://youtu.be/cxkfqxpuICM}{Escuchar}} %\NotCCLIed

%  \renewcommand{\RevDate}{February~11,~1993}
%  \SBRef{No puedo parar de alabarte}{2006}%fuente \#

        \begin{SBOpGroup}
            \Ch{G# maj7}{////}\Ch{Gm}{in}tro////\Ch{G7}{}
        \end{SBOpGroup}

        \begin{SBVerse}

            regre\Ch{Ddis}{sa}ba a casa un poco mas tem\Ch{G7}{pra}no de lo nor\Ch{Cm}{mal}

            cuando \Ch{Ddis}{vi\'o} que sobre �l ve\Ch{G7}{ni}an \Ch{Cm}{tres}

            y na\Ch{G# maj7}{va}ja en mano le \Ch{Gm}{a}tacaron sin contempla\Ch{Fm}{ci\'on}

            le de\Ch{Bb}{ja}ron incons\Ch{Bb7/B}{cien}te bajo el \Ch{Cm}{sol}
        \end{SBVerse}
        \begin{SBOpGroup}
            \Ch{G# maj7}{}\Ch{Gm}{}\Ch{G7}{}
        \end{SBOpGroup}
        \begin{SBVerse}
            y ca  \Ch{Ddis}{mi}no de la i\Ch{G7}{gle}sia iba el pas\Ch{Cm}{tor} poco despu�s

            la reu\Ch{Ddis}{nion} ya estaba a \Ch{G7}{pun}to de empe\Ch{Cm}{zar}

            iba \Ch{G#maj7}{tar}de y discu\Ch{Gm}{tien}do en el ca\Ch{Fm}{mi}no con su mujer

            inten\Ch{Bb}{tan}do no per\Ch{Bb7/B}{der} su autori\Ch{Cm}{dad}


            \Ch{G#maj7}{ay} si el maestro nos vol\Ch{Bbmaj7}{vie}ra a contar

            alguna his\Ch{Gm}{to}ria que nos hiciera \Ch{Cm}{re}capacitar

            piensa\Ch{Gm}{lo} piensa\Ch{Cm}{lo}

        \end{SBVerse}
        \begin{SBOpGroup}
            \Ch{G# maj7}{}\Ch{Gm}{}\Ch{G7}{}
        \end{SBOpGroup}

        \begin{SBVerse}

            tres mi\Ch{Ddis}{nu}tos mas y el \Ch{G7}{l�}der de ala\Ch{Cm}{ban}za aparecio

            los te\Ch{Ddis}{cla}dos siete \Ch{G7}{ca}bles y un a\Ch{Cm}{tril}

            y aunque \Ch{G#maj7}{si} le pare\Ch{Gm}{ci\'o} ver algo \Ch{Fm}{ro}jo en el arcel

            prefi\Ch{Bb}{ri\'o} pasar de \Ch{Bb7/B}{lar}go y de per\Ch{Cm}{fil}
        \end{SBVerse}
        \begin{SBVerse}

            un gi\Ch{D#dis}{ta}no despei\Ch{G#7}{na}do que pa\Ch{Cm#}{sa}ba por ahi

            no sa\Ch{D#dis}{b�a} ni \Ch{G#7}{leer} ni escri\Ch{Cm#}{bir}

            pero al \Ch{Amaj7}{ver} el pano\Ch{Gm#7}{ra}ma le do\Ch{Fm#7}{li\'o} en el coraz\'on

            y acer\Ch{B}{c�n}dose hasta el \Ch{B7/C}{hom}bre le ayud\Ch{Cm#}{\'o}

            \Ch{G#maj7}{ay} si el maestro nos vol\Ch{Bbmaj7}{vie}ra a contar

            alguna his\Ch{Gm}{to}ria que nos hiciera \Ch{Cm}{re}capacitar

            piensa\Ch{Gm}{lo} piensa\Ch{Cm}{lo}

        \end{SBVerse}


        \ifChordBk
        \begin{SBOpGroup}
            Acordes:

            \upchord{\keyboard[Do][Fo][Go][Bo]\Gs}{Sol} Mayor S\'eptima
        \end{SBOpGroup}
        \fi
    \end{song}

    \begin{song}{Grita, canta, danza}{Cm}
    {\SBPubDom} %copyright \SBPubDom
    {}
    {} %pasaje
    {} %\NotCCLIed

%  \renewcommand{\RevDate}{February~11,~1993}
%  \SBRef{No puedo parar de alabarte}{2006}%fuente \#

        \begin{SBChorus}
            \Ch{Cm}{Gri}ta, canta, danza alegremente en su pre\Ch{Ab}{sen}cia

            Gira, salta dando vueltas para \Ch{Bb}{Cris}to

            �l vive y \Ch{G7}{vi}ve para siempre es el \Ch{Cm}{Rey}\Ch{G7}{}
        \end{SBChorus}

        \begin{SBOpGroup}
            Te alaba\Ch{Cm}{r�}, te exaltar� y te agra\Ch{Bb}{de}cer�

            Tu \Ch{Eb}{gran}de amor Jes�s

            Cam\Ch{Ab}{bias}te mi lamento en ala\Ch{Cm}{ban}za

            Sa\Ch{Bb}{nas}te mi \Ch{G7}{he}rido cora\Ch{Cm}{z�n} \Ch{G7}{}
        \end{SBOpGroup}

        \ifChordBk
        \begin{SBOpGroup}
            Acordes:
            \upchord{\CmPiano\Cm}{\qquad Do} Menor \qquad\qquad
            \upchord{\AflatPiano\Aflat}{La bemol} Mayor \hfill \break
            \upchord{\BflatPiano\Bflat}{\qquad Si bemol} Mayor \hfill \break
        \end{SBOpGroup}
        \fi
    \end{song}

    \begin{song}{As� como David danzaba}{Am}
    {\SBPubDom} %copyright \SBPubDom
    {}
    {} %pasaje
    {} %\NotCCLIed

%  \renewcommand{\RevDate}{February~11,~1993}
%  \SBRef{No puedo parar de alabarte}{2006}%fuente \#

        \begin{SBOpGroup}
            \Ch{Am}{Cuan}do el Se�or hiciere volver la cautivi\Ch{G}{dad}

            seremos \Ch{Dm}{co}mo los que sue\Ch{E}{�an}
        \end{SBOpGroup}

        \begin{SBOpGroup}

            \Ch{Am}{Mi} boca llenar� de risa, \Ch{G}{mis} labios de alabanza,

            \Ch{Dm}{En}tonces dir�n las naciones:

            \Ch{E}{Gran}des cosas ha hecho el Se�or
        \end{SBOpGroup}

        \begin{SBOpGroup}
            Me goza\Ch{Am}{r�}, me gozar�, me gozar�,

            me gozar� en Jeho\Ch{G}{v�}.  [�G�zate!]

            Pues ha lle\Ch{Dm}{va}do todo dolor, me ha hecho \Ch{E}{li}bre
        \end{SBOpGroup}

        \begin{SBChorus}
            \Ch{Am}{A}s� como David cantaba, \Ch{G}{a}s� como David danzaba,

            \Ch{Dm}{a}s� como David flu�a en su pre\Ch{E}{sen}cia
        \end{SBChorus}
        \ifChordBk
        \begin{SBOpGroup}
            Acordes:
            \upchord{\AmPiano\Am}{\qquad La} Menor \qquad\qquad
            \upchord{\GPiano\G}{\qquad Sol} Mayor \hfill \break
            \upchord{\DmPiano\Dm}{\qquad Re} Menor \qquad\qquad
        \end{SBOpGroup}
        \fi
    \end{song}

    %\begin{document}

\ifguitarra

\lhead{\LHeadFont Acodes~para~Guitarra}
\chead{\CHeadFont ({\rm\thepage})}
\rhead{\RHeadFont\RelDate}
{\parindent 8pt
        {\myTitleFont --- Acodes para Guitarra ---}}\par
\vskip 20pt
\textbf{Acodes Mayores}

%\small{El s\'imbolo \# significa sostenido y {\flat}~significa~bemol}
\small
\upchord{\A}{La Mayor} \upchord{\B}{Si Mayor} \upchord{\C}{Do Mayor} \upchord{\D}{Re Mayor} \upchord{\E}{Mi Mayor} \upchord{\F}{Fa Mayor} \upchord{\G}{Sol Mayor}

\upchord{\As}{A\#/$B\flat$ Mayor} \upchord{\Cs}{C\#/$D\flat$ Mayor} \upchord{\Ds}{D\#/$E\flat$ Mayor}  \upchord{\Fs}{F\#/$G\flat$ Mayor} \upchord{\Gs}{G\#/$A\flat$ Mayor} \upchord{\As}{A\#/$B\flat$ Mayor}
\normalsize

\textbf{Acodes Menores}

\small
\upchord{\Am}{La} Menor \upchord{\Bm}{Si} Menor \upchord{\Cm}{Do} Menor \upchord{\Dm}{Re} Menor \upchord{\Em}{Mi} Menor \upchord{\Fm}{Fa} Menor \upchord{\Gm}{Sol} Menor

\upchord{\Asm}{\small{A\#/B\flat Menor}} \upchord{\Csm}{\small{C\#/D\flat Menor}} \upchord{\Dsm}{\small{D\#/E\flat Menor}}  \upchord{\Fsm}{\small{F\#/G\flat Menor}} \upchord{\Gsm}{\small{G\#/A\flat Menor}} \upchord{\Asm}{\small{A\#/B\flat Menor}}
\normalsize

\vskip 20pt
\textbf{Acodes Mayores S\'eptima}

\upchord{\Aseven}{La} Mayor s\'eptima
\upchord{\Bflatseven}{Si} bemol Mayor s\'eptima
\upchord{\Bseven}{Si} Mayor s\'eptima
\upchord{\Cseven}{\small{Do Mayor s\'eptima}}
\upchord{\Csseven}{\small{Do sostenidoMayor s\'eptima}}
\upchord{\Dseven}{\small{Re Mayor s\'eptima}}
\upchord{\Eflatseven}{\small{Mi bemol Mayor s\'eptima}}
\upchord{\Eseven}{\small{Mi Mayor s\'eptima}}
\upchord{\Fseven}{\small{Fa Mayor s\'eptima}}
\upchord{\Gseven}{\small{Sol Mayor s\'eptima}}
\vskip 20pt

\textbf{Acodes Menores S\'eptima}

\small
\upchord{\Amseven}{La} Menor s\'eptima
\upchord{\Bmseven}{Si} Menor s\'eptima
\upchord{\Cmseven}{Do} Menor s\'eptima
\upchord{\Csmseven}{Do} Menor s\'eptima
\upchord{\Dmseven}{Re} Menor s\'eptima
\upchord{\Dsmseven}{Re} Sostenido Menor s\'eptima
\upchord{\Emseven}{Mi} Menor s\'eptima
\upchord{\Emseventr}{Mi} Menor s\'eptima
\upchord{\Fmseven}{Fa} Menor s\'eptima
\upchord{\Fsmseven}{Fa sostenido} Menor s\'eptima
\upchord{\Gmseven}{Sol} Menor s\'eptima
\upchord{\Gsmseven}{Sol} Sostenido Menor s\'eptima
\upchord{\Bflatmseven}{Si bemol} Menor s\'eptima
\normalsize

\vskip 20pt
\textbf{Acodes Mayores Suspendido cuarta}
\vskip 25pt

\small
\upchord{\Asus}{La} Suspendida cuarta
\upchord{\Bsus}{Si} Suspendida cuarta
\upchord{\Csus}{Do} Suspendida cuarta
\upchord{\Dsus}{Re} Suspendida cuarta
\upchord{\Esus}{Re} Suspendida cuarta
\upchord{\Fsus}{Fa} Suspendida cuarta
\upchord{\Gsus}{Sol} Suspendida cuarta

\upchord{\Fssus}{Fa} sostenido Suspendida cuarta
\upchord{\Gssus}{Sol sostenido} Suspendida cuarta
\normalsize

\vskip 20pt
\textbf{Acodes Mayor Aumentada}
\vskip 25pt

\small
\upchord{\CMaj}{Do} Maj
\upchord{\DMaj}{Re} Maj
\upchord{\GMaj}{Sol} Maj
\normalsize

\vskip 20pt
\textbf{Acodes Mayor S\'eptima Aumentada}
\vskip 25pt

\small
\upchord{\AsevenMaj}{La} Maj S\'eptima Aumentada
\upchord{\Fmajseven}{Fa} Maj S\'eptima Aumentada
\normalsize

\vskip 20pt
\textbf{Acodes Aumentada 2}
\vskip 25pt

\small
\upchord{\Atwo}{La} Aumentada 2
\upchord{\Ctwo}{Do} Aumentada 2
\normalsize

\vskip 20pt
\textbf{Acodes Novena}
\vskip 25pt

\small
\upchord{\Cnine}{Do} Novena
\upchord{\Gnine}{Sol} Novena
\normalsize

\vskip 20pt
\textbf{Acodes Disminuidos}
\vskip 25pt

\small
\upchord{\Gsdim}{Sol} sostenido disminuido
\normalsize


\vskip 20pt
\textbf{Acodes Con Bajo cambiado}

\small
\upchord{\AAs}{La Mayor bajo Bb}
\upchord{\ACs}{La Mayor bajo C\#}
\upchord{\AEg}{La Mayor bajo E}
\upchord{\AmF}{La Menor bajo F}
\vskip 20pt
\upchord{\CE}{Do Mayor bajo E}
\upchord{\CG}{Do Mayor bajo G}
\upchord{\DflatF}{Re bemol bajo F}
\upchord{\DA}{Re Mayor bajo A}
\vskip 20pt
\upchord{\DE}{Re Mayor bajo E}
\upchord{\DFs}{Re Mayor bajo F\#}
\upchord{\EGs}{Mi Mayor Bajo G\#}
\vskip 20pt
\upchord{\GB}{Sol Mayor Bajo B}
\upchord{\GD}{Sol Mayor Bajo D}
\upchord{\GE}{Sol Mayor Bajo E}
\upchord{\AflatC}{La bemol Bajo C}
\vskip 20pt
\upchord{\AflatEflat}{La bemol Bajo Eb}
\upchord{\DsusFs}{Re} Suspendida cuarta bajo F\#
\normalsize

\vskip 20pt
\textbf{Acodes semidisminuidos}

\small
\upchord{\Bmsevenbfive}{Si} Menor s\'eptima semidisminuido
\normalsize

\vskip 20pt
\textbf{Acodes 13 suspendida cuarta}

\small
\upchord{\Csusthirteen}{Do} 13 suspendida cuarta
\normalsize

\clearpage
\fi

\ifpiano
\lhead{\LHeadFont Acodes~para~Piano}
{\parindent 8pt
        {\myTitleFont --- Acordes para Piano ---}}\par
\vskip 20pt
\textbf{Acodes Mayores}
\vskip 25pt

%\small{El s\'imbolo \# significa sostenido y {\flat}~significa~bemol}
\small
\upchord{\APiano}{\qquad La Mayor} \qquad\qquad \upchord{\BPiano}{Si Mayor} \qquad\qquad \upchord{\CPiano}{\qquad Do Mayor} \qquad\qquad \upchord{\DPiano}{\qquad Re Mayor} \hfill \break
\vskip 25pt
\upchord{\EPiano}{\qquad Mi Mayor} \qquad\qquad  \upchord{\FPiano}{\qquad Fa Mayor} \qquad\qquad \upchord{\GPiano}{\qquad Sol Mayor}
\vskip 25pt
\upchord{\AsPiano}{A\#/$B\flat$ Mayor} \qquad\qquad \upchord{\CsPiano}{C\#/$D\flat$ Mayor} \qquad\qquad \upchord{\DsPiano}{D\#/$E\flat$ Mayor} \qquad\qquad \upchord{\FsPiano}{F\#/$G\flat$ Mayor} \hfill \break
\vskip 25pt
\upchord{\GsPiano}{G\#/$A\flat$ Mayor} \qquad\qquad \upchord{\AsPiano}{A\#/$B\flat$ Mayor}
\normalsize

\textbf{Acodes Menores}
\vskip 25pt

\small
\upchord{\AmPiano}{\qquad La} Menor \qquad\qquad \upchord{\BmPiano}{\qquad Si} Menor \qquad\qquad \upchord{\CmPiano}{\qquad Do} Menor \qquad\qquad \upchord{\DmPiano}{\qquad Re} Menor \hfill \break
\vskip 25pt
\upchord{\EmPiano}{\qquad Mi} Menor \qquad\qquad \upchord{\FmPiano}{\qquad Fa} Menor \qquad\qquad \upchord{\GmPiano}{\qquad Sol} Menor
\vskip 25pt
\upchord{\AsmPiano}{\small{A\#/B\flat Menor}}  \qquad\qquad  \upchord{\CsmPiano}{C\#/D\flat Menor}  \qquad\qquad  \upchord{\DsmPiano}{D\#/E\flat Menor} \qquad\qquad \upchord{\FsmPiano}{F\#/G\flat Menor} \hfill \break
\vskip 25pt
\upchord{\GsmPiano}{G\#/A\flat Menor}  \qquad\qquad  \upchord{\AsmPiano}{A\#/B\flat Menor}
\normalsize

\clearpage
%\vskip 20pt
\textbf{Acodes Mayores S\'eptima}
\vskip 25pt

\small
\upchord{\AsevenPiano}{La Mayor s\'eptima} \qquad\qquad \upchord{\BsevenPiano}{Si Mayor s\'eptima} \qquad\qquad \upchord{\CsevenPiano}{Do Mayor s\'eptima} \qquad\qquad
\vskip 25pt
\upchord{\DsevenPiano}{Re Mayor s\'eptima} \qquad\qquad \upchord{\EsevenPiano}{Mi Mayor s\'eptima} \qquad\qquad \upchord{\FsevenPiano}{Fa Mayor s\'eptima}
\vskip 25pt
\upchord{\GsevenPiano}{Sol Mayor s\'eptima}  \qquad\qquad \upchord{\BflatsevenPiano}{Si} bemol Mayor s\'eptima
\vskip 25pt
\upchord{\EflatsevenPiano}{Mi bemol Mayor s\'eptima} \qquad\qquad
\normalsize
\vskip 20pt

\textbf{Acodes Menores S\'eptima}
\vskip 25pt

\small
\upchord{\AmsevenPiano}{La} Menor s\'eptima
\upchord{\BmsevenPiano}{Si} Menor s\'eptima
\upchord{\CmsevenPiano}{Do} Menor s\'eptima
\vskip 25pt
\upchord{\DmsevenPiano}{Re} Menor s\'eptima
\upchord{\EmsevenPiano}{Mi} Menor s\'eptima
\upchord{\FmsevenPiano}{Fa} Menor s\'eptima
\vskip 25pt
\upchord{\GmsevenPiano}{Sol} Menor s\'eptima
\upchord{\BflatmsevenPiano}{Si bemol} Menor s\'eptima
\upchord{\FsmsevenPiano}{Fa sostenido} Menor s\'eptima
\vskip 25pt
\upchord{\CssevenPiano}{Do sostenido} Mayor s\'eptima
\upchord{\DsmsevenPiano}{Re} Sostenido Menor s\'eptima
\upchord{\GsmsevenPiano}{Sol} Sostenido Menor s\'eptima
\normalsize

\vskip 20pt

\textbf{Acodes Mayores Suspendido cuarta}
\vskip 25pt

\small
\upchord{\AsusPiano}{La} Suspendida cuarta
\upchord{\BsusPiano}{Si} Suspendida cuarta
\upchord{\CsusPiano}{Do} Suspendida cuarta
\vskip 25pt
\upchord{\DsusPiano}{Re} Suspendida cuarta
\upchord{\EsusPiano}{Re} Suspendida cuarta
\upchord{\FsusPiano}{Fa} Suspendida cuarta
\vskip 25pt
\upchord{\GsusPiano}{Sol} Suspendida cuarta
\upchord{\GssusPiano}{Sol sostenido} Suspendida cuarta
\normalsize

\vskip 20pt
\textbf{Acodes Mayor Aumentada}
\vskip 25pt

\small
\upchord{\CMajPiano}{Do} Maj  \qquad\qquad
\upchord{\DMajPiano}{Re} Maj
\upchord{\GMajPiano}{Sol} Maj
\normalsize

\vskip 20pt
\textbf{Acodes Mayor S\'eptima Aumentada}
\vskip 25pt

\small
\upchord{\AsevenMajPiano}{La} Maj S\'eptima Aumentada
\upchord{\FmajsevenPiano}{Fa} Maj S\'eptima Aumentada
\normalsize

\vskip 20pt
\textbf{Acodes Aumentada 2}
\vskip 25pt

\small
\upchord{\AtwoPiano}{La} Aumentada 2
\upchord{\CtwoPiano}{Do} Aumentada 2
\normalsize


\vskip 20pt
\textbf{Acodes Disminuidos}
\vskip 25pt

\small
\upchord{\GsdimPiano}{Sol} sostenido disminuido
\normalsize


\vskip 20pt
\textbf{Acodes Con Bajo cambiado}
\vskip 25pt

\small
\upchord{\ACsPiano}{La Mayor bajo C\#}
\upchord{\AEPiano}{La Mayor bajo E}
\vskip 20pt
\upchord{\AmFPiano}{La Menor bajo F}
\upchord{\CEPiano}{Do Mayor bajo E}
\vskip 20pt
\upchord{\CGPiano}{Do Mayor bajo G}
\upchord{\DAPiano}{Re Mayor bajo A}
\upchord{\DFsPiano}{Re Mayor bajo F\#}
\upchord{\GDPiano}{Sol Mayor Bajo D}
\vskip 20pt
\upchord{\GBPiano}{Sol Mayor Bajo B}
\vskip 25pt
\upchord{\DsusFsPiano}{Re} Suspendida cuarta bajo F\#
\normalsize

\vskip 20pt
\textbf{Acodes medio disminuido s\'eptima}
\vskip 25pt

\small
\upchord{\BmsevenbfivePiano}{Si} medio disminuido s\'eptima
\normalsize

\vskip 20pt
\textbf{Acodes 13 suspendida cuarta}
\vskip 25pt

\small
\upchord{\CsusthirteenPiano}{Do} 13 suspendida cuarta
\normalsize

\clearpage
\fi
%\end{document}
%\bye
    %%%%%% rcsid = @(#)$Id:$
%%%%%%
%%
%%      ================================
%%      Sample Key Index (sampleAdx.tex)
%%      ================================
%%
%%      Version 4.5, 30 April, 2010
%%
%%      Copyright 1992--2010 Christopher Rath <christopher@rath.ca>
%%
%%	This package is free software; you can redistribute it and/or
%%	modify it under the terms of version 2.1 of the GNU Lesser 
%%	General Public License as published by the Free Software
%%	Foundation.
%%
%%	This package is distributed in the hope that it will be
%%	useful, but WITHOUT ANY WARRANTY; without even the implied
%%	warranty of MERCHANTABILITY or FITNESS FOR A PARTICULAR
%%	PURPOSE.  See the GNU Lesser General Public License for more
%%	details.
%%
%%      This file is provided as a template for Song Artist
%%      Index generation.
%%
%%%%%%
%%%%%%

%%%%%%%%%%%%%%%%%%%%%%%%%%%%%%%%%%%%%%%%%%%%%%%%%%%%%%%%%%
%%%%%%%%%%%%%%%%%%%%%%%%%%%%%%%%%%%%%%%%%%%%%%%%%%%%%%%%%%
%%                                                      %%
%%           P R E A M B L E   B E G I N S              %%
%%                                                      %%
%%%%%%%%%%%%%%%%%%%%%%%%%%%%%%%%%%%%%%%%%%%%%%%%%%%%%%%%%%
%%%%%%%%%%%%%%%%%%%%%%%%%%%%%%%%%%%%%%%%%%%%%%%%%%%%%%%%%%

%\documentclass[12pt,twocolumn]{book}
%\usepackage{latexsym,fancyhdr}
%\usepackage[wordbk]{songbook}

%%%
% Revision Date and Release Date definitions.
%
%       \RelDate - The last time this songbook was released.
%       \RevDate - The last time this file was revised in any way.
%%%
%\newcommand{\RelDate}{30~May'96}
%\newcommand{\RevDate}{\RelDate}

%%%
% Redefine fonts from SongBook style that I don't like, and define
% any extra fonts I require.
%%%
\font\myTinySF=cmss8    at  8pt
\font\myHugeSF=cmssbx10 at 25pt
\renewcommand{\CpyRtInfoFont}{\tiny\myTinySF}
%\newcommand{\myTitleFont}{\Huge\myHugeSF}
%\newcommand{\mySubTitleFont}{\large\sf}

%%%
% Define fonts to use in the headers and footers of the songbook.
%%%
%\newcommand{\LHeadFont}{\normalsize}            % = cmr12  at 12pt
%\newcommand{\CHeadFont}{\normalsize\rm}         % = cmr12  at 12pt
%\newcommand{\RHeadFont}{\normalsize}            % = cmr12  at 12pt
%\newcommand{\LFootFont}{\scriptsize}            % = cmr8   at  8pt
%\newcommand{\CFootFont}{\tiny\myTinySF}         % = cmss8  at  8pt
%\newcommand{\RFootFont}{\scriptsize}            % = cmr8   at  8pt

%%%
% Turn on and define fancy page heading/footing definition.
%%%
\pagestyle{fancy}

%\addtolength{\headwidth}{\marginparsep}
%\addtolength{\headwidth}{\marginparwidth}
%\renewcommand{\footrulewidth}{0.4pt}
\lhead{\LHeadFont \'Indice~de~Autores}
       \chead{\CHeadFont ({\rm\thepage})}
       \rhead{\RHeadFont\RelDate}
%
%\lfoot{\LFootFont Property of a Church}
%       \cfoot{\CFootFont Last Revised:  \RevDate}
%       \rfoot{\RFootFont Material used by permission.}


%%%
% Index entries command definition.
%%%
\renewcommand{\item}{\par\hangindent=40pt}
\renewcommand{\subitem}{\par\hangindent=40pt \hspace*{20pt}}
\renewcommand{\subsubitem}{\par\hangindent=40pt \hspace*{30pt}}


%%%%%%%%%%%%%%%%%%%%%%%%%%%%%%%%%%%%%%%%%%%%%%%%%%%%%%%%%%
%%%%%%%%%%%%%%%%%%%%%%%%%%%%%%%%%%%%%%%%%%%%%%%%%%%%%%%%%%
%%                                                      %%
%%           D O C U M E N T   B E G I N S              %%
%%                                                      %%
%%%%%%%%%%%%%%%%%%%%%%%%%%%%%%%%%%%%%%%%%%%%%%%%%%%%%%%%%%
%%%%%%%%%%%%%%%%%%%%%%%%%%%%%%%%%%%%%%%%%%%%%%%%%%%%%%%%%%
%\begin{document}

%%%
% Index begins.
%%%
\pdfbookmark[0]{\'Indice~de~autores}{autores}
{\parindent 8pt
  {\myTitleFont --- INDICE DE AUTORES ---}}\par
\vskip 20pt

\input{Estribillero.adx}

%\end{document}
%\bye
%
%%%
% Document ends.
%%%

% Local Variables:
%   LaTeX-item-indent:     -1
%   LaTeX-indent-level:     2
%   TeX-brace-indent-level: 2
%   TeX-auto-untabify:      nil
%   TeX-style-local:        style/
% End:

    %%%%%% rcsid = @(#)$Id: sampleKdx.tex,v 1.16 2010-04-12 18:04:30 rathc Exp $
%%%%%%
%%
%%      ================================
%%      Sample Key Index (sampleKdx.tex)
%%      ================================
%%
%%      Version 4.5, 30 April, 2010
%%
%%      Copyright 1992--2010 Christopher Rath <christopher@rath.ca>
%%
%%	This package is free software; you can redistribute it and/or
%%	modify it under the terms of version 2.1 of the GNU Lesser 
%%	General Public License as published by the Free Software
%%	Foundation.
%%
%%	This package is distributed in the hope that it will be
%%	useful, but WITHOUT ANY WARRANTY; without even the implied
%%	warranty of MERCHANTABILITY or FITNESS FOR A PARTICULAR
%%	PURPOSE.  See the GNU Lesser General Public License for more
%%	details.
%%
%%      This file is provided as a template for Song Key
%%      Index generation.
%%
%%%%%%
%%%%%%

%%%%%%%%%%%%%%%%%%%%%%%%%%%%%%%%%%%%%%%%%%%%%%%%%%%%%%%%%%
%%%%%%%%%%%%%%%%%%%%%%%%%%%%%%%%%%%%%%%%%%%%%%%%%%%%%%%%%%
%%                                                      %%
%%           P R E A M B L E   B E G I N S              %%
%%                                                      %%
%%%%%%%%%%%%%%%%%%%%%%%%%%%%%%%%%%%%%%%%%%%%%%%%%%%%%%%%%%
%%%%%%%%%%%%%%%%%%%%%%%%%%%%%%%%%%%%%%%%%%%%%%%%%%%%%%%%%%

%\documentclass[12pt,twocolumn]{book}
%\usepackage{latexsym,fancyhdr}
%\usepackage[wordbk]{songbook}

%%%
% Revision Date and Release Date definitions.
%
%       \RelDate - The last time this songbook was released.
%       \RevDate - The last time this file was revised in any way.
%%%
%\newcommand{\RelDate}{30~May'96}
%\newcommand{\RevDate}{\RelDate}

%%%
% Redefine fonts from SongBook style that I don't like, and define
% any extra fonts I require.
%%%
\font\myTinySF=cmss8    at  8pt
\font\myHugeSF=cmssbx10 at 25pt
\renewcommand{\CpyRtInfoFont}{\tiny\myTinySF}
%\newcommand{\myTitleFont}{\Huge\myHugeSF}
%\newcommand{\mySubTitleFont}{\large\sf}

%%%
% Define fonts to use in the headers and footers of the songbook.
%%%
%\newcommand{\LHeadFont}{\normalsize}            % = cmr12  at 12pt
%\newcommand{\CHeadFont}{\normalsize\rm}         % = cmr12  at 12pt
%\newcommand{\RHeadFont}{\normalsize}            % = cmr12  at 12pt
%\newcommand{\LFootFont}{\scriptsize}            % = cmr8   at  8pt
%\newcommand{\CFootFont}{\tiny\myTinySF}         % = cmss8  at  8pt
%\newcommand{\RFootFont}{\scriptsize}            % = cmr8   at  8pt

%%%
% Turn on and define fancy page heading/footing definition.
%%%
\pagestyle{fancy}
\pdfbookmark[0]{\'Indice~Tonal}{tonal}
%\addtolength{\headwidth}{\marginparsep}
%\addtolength{\headwidth}{\marginparwidth}
%\renewcommand{\footrulewidth}{0.4pt}
\lhead{\LHeadFont \'Indice~Tonal}
       \chead{\CHeadFont ({\rm\thepage})}
       \rhead{\RHeadFont\RelDate}

%\lfoot{\LFootFont Property of a Church}
%       \cfoot{\CFootFont Last Revised:  \RevDate}
%       \rfoot{\RFootFont Material used by permission.}


%%%
% Index entries command definition.
%%%
\renewcommand{\item}{\par\hangindent=40pt}
\renewcommand{\subitem}{\par\hangindent=40pt \hspace*{20pt}}
\renewcommand{\subsubitem}{\par\hangindent=40pt \hspace*{30pt}}


%%%%%%%%%%%%%%%%%%%%%%%%%%%%%%%%%%%%%%%%%%%%%%%%%%%%%%%%%%
%%%%%%%%%%%%%%%%%%%%%%%%%%%%%%%%%%%%%%%%%%%%%%%%%%%%%%%%%%
%%                                                      %%
%%           D O C U M E N T   B E G I N S              %%
%%                                                      %%
%%%%%%%%%%%%%%%%%%%%%%%%%%%%%%%%%%%%%%%%%%%%%%%%%%%%%%%%%%
%%%%%%%%%%%%%%%%%%%%%%%%%%%%%%%%%%%%%%%%%%%%%%%%%%%%%%%%%%
%\begin{document}

%%%
% Index begins.
%%%
{\parindent 8pt
  {\myTitleFont --- INDICE TONAL ---}}\par
\vskip 20pt

\input{Estribillero.kdx}
%
%\end{document}
%\bye
%
%%%
% Document ends.
%%%

% Local Variables:
%   LaTeX-item-indent:     -1
%   LaTeX-indent-level:     2
%   TeX-brace-indent-level: 2
%   TeX-auto-untabify:      nil
%   TeX-style-local:        style/
% End:

    %%%%%% rcsid = @(#)$Id: sampleTdx.tex,v 1.18 2010-04-12 18:04:31 rathc Exp $
%%%%%%
%%
%%      ===============================================
%%      Sample Title & First Line Index (sampleTdx.tex)
%%      ===============================================
%%
%%      Version 4.5, 30 April, 2010
%%
%%      Copyright 1992--2010 Christopher Rath <christopher@rath.ca>
%%
%%      This package is free software; you can redistribute it and/or
%%      modify it under the terms of version 2.1 of the GNU Lesser 
%%	General Public License as published by the Free Software 
%%	Foundation.
%%
%%      This package is distributed in the hope that it will be
%%      useful, but WITHOUT ANY WARRANTY; without even the implied
%%      warranty of MERCHANTABILITY or FITNESS FOR A PARTICULAR
%%      PURPOSE.  See the GNU Lesser General Public License for more
%%      details.
%%
%%      This file is provided as a template for Title and First Line
%%      Index generation.
%%
%%%%%%
%%%%%%

%%%%%%%%%%%%%%%%%%%%%%%%%%%%%%%%%%%%%%%%%%%%%%%%%%%%%%%%%%
%%%%%%%%%%%%%%%%%%%%%%%%%%%%%%%%%%%%%%%%%%%%%%%%%%%%%%%%%%
%%                                                      %%
%%           P R E A M B L E   B E G I N S              %%
%%                                                      %%
%%%%%%%%%%%%%%%%%%%%%%%%%%%%%%%%%%%%%%%%%%%%%%%%%%%%%%%%%%
%%%%%%%%%%%%%%%%%%%%%%%%%%%%%%%%%%%%%%%%%%%%%%%%%%%%%%%%%%

%\documentclass[12pt,twocolumn,spanish]{book}
%\usepackage{latexsym,fancyhdr}
%\usepackage[wordbk]{songbook}


%%%
% Revision Date and Release Date definitions.
%
%       \RelDate - The last time this songbook was released.
%       \RevDate - The last time this file was revised in any way.
%%%
%\newcommand{\RelDate}{30 May'96}
%\newcommand{\RevDate}{\today}

%%%
% Redefine fonts from SongBook style that I don't like, and define
% any extra fonts I require.
%%%
\font\myTinySF=cmss8    at  8pt
\font\myHugeSF=cmssbx10 at 25pt
\renewcommand{\CpyRtInfoFont}{\tiny\myTinySF}
%\newcommand{\myTitleFont}{\Huge\myHugeSF}
%\newcommand{\mySubTitleFont}{\large\sf}

%%%
% Define fonts to use in the headers and footers of the songbook.
%%%
%\newcommand{\LHeadFont}{\normalsize}            % = cmr12  at 12pt
%\newcommand{\CHeadFont}{\normalsize\rm}         % = cmr12  at 12pt
%\newcommand{\RHeadFont}{\normalsize}            % = cmr12  at 12pt
%\newcommand{\LFootFont}{\scriptsize}            % = cmr8   at  8pt
%\newcommand{\CFootFont}{\tiny\myTinySF}         % = cmss8  at  8pt
%\newcommand{\RFootFont}{\scriptsize}            % = cmr8   at  8pt

%%%
% Turn on and define fancy page heading/footing definition.
%%%
\pagestyle{fancy}

\addtolength{\headwidth}{\marginparsep}
\addtolength{\headwidth}{\marginparwidth}
\renewcommand{\footrulewidth}{0.4pt}
\lhead{\LHeadFont A Church Songbook}
       \chead{\CHeadFont \I'ndice~por~T\'itulo({\rm\thepage})}
       \rhead{\RHeadFont\RelDate}

\lfoot{\LFootFont Property of a Church}
       \cfoot{\CFootFont Last Revised:  \RevDate}
       \rfoot{\RFootFont Material used by permission.}

%%%
% Index entries command definition.
%%%
\renewcommand{\item}{\par\hangindent=40pt}
\renewcommand{\subitem}{\par\hangindent=40pt \hspace*{20pt}}
\renewcommand{\subsubitem}{\par\hangindent=40pt \hspace*{30pt}}


%%%%%%%%%%%%%%%%%%%%%%%%%%%%%%%%%%%%%%%%%%%%%%%%%%%%%%%%%%
%%%%%%%%%%%%%%%%%%%%%%%%%%%%%%%%%%%%%%%%%%%%%%%%%%%%%%%%%%
%%                                                      %%
%%           D O C U M E N T   B E G I N S              %%
%%                                                      %%
%%%%%%%%%%%%%%%%%%%%%%%%%%%%%%%%%%%%%%%%%%%%%%%%%%%%%%%%%%
%%%%%%%%%%%%%%%%%%%%%%%%%%%%%%%%%%%%%%%%%%%%%%%%%%%%%%%%%%
%\begin{document}

%%%
% Begin the Index.
%%%
{\parindent 8pt
  {\myTitleFont --- Title Index ---}}\par
\vskip 5pt
{\parindent 20pt
  {\mySubTitleFont --- with first lines in italic ---}}
\vskip 20pt

\input{Estribillero.tdx}

%\end{document}
%\bye
%
%%%
% Document ends.
%%%

\end{document}
\bye
%
%%%
% Document ends.
%%%

%
%\end{document}
%\bye
%
%%%
% Document ends.
%%%

% Local Variables:
%   LaTeX-item-indent:     -1
%   LaTeX-indent-level:     2
%   TeX-brace-indent-level: 2
%   TeX-auto-untabify:      nil
%   TeX-style-local:        style/
% End:

%\begin{document}

\ifguitarra

\lhead{\LHeadFont Acodes~para~Guitarra}
\chead{\CHeadFont ({\rm\thepage})}
\rhead{\RHeadFont\RelDate}
{\parindent 8pt
        {\myTitleFont --- Acodes para Guitarra ---}}\par
\vskip 20pt
\textbf{Acodes Mayores}

%\small{El s\'imbolo \# significa sostenido y {\flat}~significa~bemol}
\small
\upchord{\A}{La Mayor} \upchord{\B}{Si Mayor} \upchord{\C}{Do Mayor} \upchord{\D}{Re Mayor} \upchord{\E}{Mi Mayor} \upchord{\F}{Fa Mayor} \upchord{\G}{Sol Mayor}

\upchord{\As}{A\#/$B\flat$ Mayor} \upchord{\Cs}{C\#/$D\flat$ Mayor} \upchord{\Ds}{D\#/$E\flat$ Mayor}  \upchord{\Fs}{F\#/$G\flat$ Mayor} \upchord{\Gs}{G\#/$A\flat$ Mayor} \upchord{\As}{A\#/$B\flat$ Mayor}
\normalsize

\textbf{Acodes Menores}

\small
\upchord{\Am}{La} Menor \upchord{\Bm}{Si} Menor \upchord{\Cm}{Do} Menor \upchord{\Dm}{Re} Menor \upchord{\Em}{Mi} Menor \upchord{\Fm}{Fa} Menor \upchord{\Gm}{Sol} Menor

\upchord{\Asm}{\small{A\#/B\flat Menor}} \upchord{\Csm}{\small{C\#/D\flat Menor}} \upchord{\Dsm}{\small{D\#/E\flat Menor}}  \upchord{\Fsm}{\small{F\#/G\flat Menor}} \upchord{\Gsm}{\small{G\#/A\flat Menor}} \upchord{\Asm}{\small{A\#/B\flat Menor}}
\normalsize

\vskip 20pt
\textbf{Acodes Mayores S\'eptima}

\upchord{\Aseven}{La} Mayor s\'eptima
\upchord{\Bflatseven}{Si} bemol Mayor s\'eptima
\upchord{\Bseven}{Si} Mayor s\'eptima
\upchord{\Cseven}{\small{Do Mayor s\'eptima}}
\upchord{\Csseven}{\small{Do sostenidoMayor s\'eptima}}
\upchord{\Dseven}{\small{Re Mayor s\'eptima}}
\upchord{\Eflatseven}{\small{Mi bemol Mayor s\'eptima}}
\upchord{\Eseven}{\small{Mi Mayor s\'eptima}}
\upchord{\Fseven}{\small{Fa Mayor s\'eptima}}
\upchord{\Gseven}{\small{Sol Mayor s\'eptima}}
\vskip 20pt

\textbf{Acodes Menores S\'eptima}

\small
\upchord{\Amseven}{La} Menor s\'eptima
\upchord{\Bmseven}{Si} Menor s\'eptima
\upchord{\Cmseven}{Do} Menor s\'eptima
\upchord{\Csmseven}{Do} Menor s\'eptima
\upchord{\Dmseven}{Re} Menor s\'eptima
\upchord{\Dsmseven}{Re} Sostenido Menor s\'eptima
\upchord{\Emseven}{Mi} Menor s\'eptima
\upchord{\Emseventr}{Mi} Menor s\'eptima
\upchord{\Fmseven}{Fa} Menor s\'eptima
\upchord{\Fsmseven}{Fa sostenido} Menor s\'eptima
\upchord{\Gmseven}{Sol} Menor s\'eptima
\upchord{\Gsmseven}{Sol} Sostenido Menor s\'eptima
\upchord{\Bflatmseven}{Si bemol} Menor s\'eptima
\normalsize

\vskip 20pt
\textbf{Acodes Mayores Suspendido cuarta}
\vskip 25pt

\small
\upchord{\Asus}{La} Suspendida cuarta
\upchord{\Bsus}{Si} Suspendida cuarta
\upchord{\Csus}{Do} Suspendida cuarta
\upchord{\Dsus}{Re} Suspendida cuarta
\upchord{\Esus}{Re} Suspendida cuarta
\upchord{\Fsus}{Fa} Suspendida cuarta
\upchord{\Gsus}{Sol} Suspendida cuarta

\upchord{\Fssus}{Fa} sostenido Suspendida cuarta
\upchord{\Gssus}{Sol sostenido} Suspendida cuarta
\normalsize

\vskip 20pt
\textbf{Acodes Mayor Aumentada}
\vskip 25pt

\small
\upchord{\CMaj}{Do} Maj
\upchord{\DMaj}{Re} Maj
\upchord{\GMaj}{Sol} Maj
\normalsize

\vskip 20pt
\textbf{Acodes Mayor S\'eptima Aumentada}
\vskip 25pt

\small
\upchord{\AsevenMaj}{La} Maj S\'eptima Aumentada
\upchord{\Fmajseven}{Fa} Maj S\'eptima Aumentada
\normalsize

\vskip 20pt
\textbf{Acodes Aumentada 2}
\vskip 25pt

\small
\upchord{\Atwo}{La} Aumentada 2
\upchord{\Ctwo}{Do} Aumentada 2
\normalsize

\vskip 20pt
\textbf{Acodes Novena}
\vskip 25pt

\small
\upchord{\Cnine}{Do} Novena
\upchord{\Gnine}{Sol} Novena
\normalsize

\vskip 20pt
\textbf{Acodes Disminuidos}
\vskip 25pt

\small
\upchord{\Gsdim}{Sol} sostenido disminuido
\normalsize


\vskip 20pt
\textbf{Acodes Con Bajo cambiado}

\small
\upchord{\AAs}{La Mayor bajo Bb}
\upchord{\ACs}{La Mayor bajo C\#}
\upchord{\AEg}{La Mayor bajo E}
\upchord{\AmF}{La Menor bajo F}
\vskip 20pt
\upchord{\CE}{Do Mayor bajo E}
\upchord{\CG}{Do Mayor bajo G}
\upchord{\DflatF}{Re bemol bajo F}
\upchord{\DA}{Re Mayor bajo A}
\vskip 20pt
\upchord{\DE}{Re Mayor bajo E}
\upchord{\DFs}{Re Mayor bajo F\#}
\upchord{\EGs}{Mi Mayor Bajo G\#}
\vskip 20pt
\upchord{\GB}{Sol Mayor Bajo B}
\upchord{\GD}{Sol Mayor Bajo D}
\upchord{\GE}{Sol Mayor Bajo E}
\upchord{\AflatC}{La bemol Bajo C}
\vskip 20pt
\upchord{\AflatEflat}{La bemol Bajo Eb}
\upchord{\DsusFs}{Re} Suspendida cuarta bajo F\#
\normalsize

\vskip 20pt
\textbf{Acodes semidisminuidos}

\small
\upchord{\Bmsevenbfive}{Si} Menor s\'eptima semidisminuido
\normalsize

\vskip 20pt
\textbf{Acodes 13 suspendida cuarta}

\small
\upchord{\Csusthirteen}{Do} 13 suspendida cuarta
\normalsize

\clearpage
\fi

\ifpiano
\lhead{\LHeadFont Acodes~para~Piano}
{\parindent 8pt
        {\myTitleFont --- Acordes para Piano ---}}\par
\vskip 20pt
\textbf{Acodes Mayores}
\vskip 25pt

%\small{El s\'imbolo \# significa sostenido y {\flat}~significa~bemol}
\small
\upchord{\APiano}{\qquad La Mayor} \qquad\qquad \upchord{\BPiano}{Si Mayor} \qquad\qquad \upchord{\CPiano}{\qquad Do Mayor} \qquad\qquad \upchord{\DPiano}{\qquad Re Mayor} \hfill \break
\vskip 25pt
\upchord{\EPiano}{\qquad Mi Mayor} \qquad\qquad  \upchord{\FPiano}{\qquad Fa Mayor} \qquad\qquad \upchord{\GPiano}{\qquad Sol Mayor}
\vskip 25pt
\upchord{\AsPiano}{A\#/$B\flat$ Mayor} \qquad\qquad \upchord{\CsPiano}{C\#/$D\flat$ Mayor} \qquad\qquad \upchord{\DsPiano}{D\#/$E\flat$ Mayor} \qquad\qquad \upchord{\FsPiano}{F\#/$G\flat$ Mayor} \hfill \break
\vskip 25pt
\upchord{\GsPiano}{G\#/$A\flat$ Mayor} \qquad\qquad \upchord{\AsPiano}{A\#/$B\flat$ Mayor}
\normalsize

\textbf{Acodes Menores}
\vskip 25pt

\small
\upchord{\AmPiano}{\qquad La} Menor \qquad\qquad \upchord{\BmPiano}{\qquad Si} Menor \qquad\qquad \upchord{\CmPiano}{\qquad Do} Menor \qquad\qquad \upchord{\DmPiano}{\qquad Re} Menor \hfill \break
\vskip 25pt
\upchord{\EmPiano}{\qquad Mi} Menor \qquad\qquad \upchord{\FmPiano}{\qquad Fa} Menor \qquad\qquad \upchord{\GmPiano}{\qquad Sol} Menor
\vskip 25pt
\upchord{\AsmPiano}{\small{A\#/B\flat Menor}}  \qquad\qquad  \upchord{\CsmPiano}{C\#/D\flat Menor}  \qquad\qquad  \upchord{\DsmPiano}{D\#/E\flat Menor} \qquad\qquad \upchord{\FsmPiano}{F\#/G\flat Menor} \hfill \break
\vskip 25pt
\upchord{\GsmPiano}{G\#/A\flat Menor}  \qquad\qquad  \upchord{\AsmPiano}{A\#/B\flat Menor}
\normalsize

\clearpage
%\vskip 20pt
\textbf{Acodes Mayores S\'eptima}
\vskip 25pt

\small
\upchord{\AsevenPiano}{La Mayor s\'eptima} \qquad\qquad \upchord{\BsevenPiano}{Si Mayor s\'eptima} \qquad\qquad \upchord{\CsevenPiano}{Do Mayor s\'eptima} \qquad\qquad
\vskip 25pt
\upchord{\DsevenPiano}{Re Mayor s\'eptima} \qquad\qquad \upchord{\EsevenPiano}{Mi Mayor s\'eptima} \qquad\qquad \upchord{\FsevenPiano}{Fa Mayor s\'eptima}
\vskip 25pt
\upchord{\GsevenPiano}{Sol Mayor s\'eptima}  \qquad\qquad \upchord{\BflatsevenPiano}{Si} bemol Mayor s\'eptima
\vskip 25pt
\upchord{\EflatsevenPiano}{Mi bemol Mayor s\'eptima} \qquad\qquad
\normalsize
\vskip 20pt

\textbf{Acodes Menores S\'eptima}
\vskip 25pt

\small
\upchord{\AmsevenPiano}{La} Menor s\'eptima
\upchord{\BmsevenPiano}{Si} Menor s\'eptima
\upchord{\CmsevenPiano}{Do} Menor s\'eptima
\vskip 25pt
\upchord{\DmsevenPiano}{Re} Menor s\'eptima
\upchord{\EmsevenPiano}{Mi} Menor s\'eptima
\upchord{\FmsevenPiano}{Fa} Menor s\'eptima
\vskip 25pt
\upchord{\GmsevenPiano}{Sol} Menor s\'eptima
\upchord{\BflatmsevenPiano}{Si bemol} Menor s\'eptima
\upchord{\FsmsevenPiano}{Fa sostenido} Menor s\'eptima
\vskip 25pt
\upchord{\CssevenPiano}{Do sostenido} Mayor s\'eptima
\upchord{\DsmsevenPiano}{Re} Sostenido Menor s\'eptima
\upchord{\GsmsevenPiano}{Sol} Sostenido Menor s\'eptima
\normalsize

\vskip 20pt

\textbf{Acodes Mayores Suspendido cuarta}
\vskip 25pt

\small
\upchord{\AsusPiano}{La} Suspendida cuarta
\upchord{\BsusPiano}{Si} Suspendida cuarta
\upchord{\CsusPiano}{Do} Suspendida cuarta
\vskip 25pt
\upchord{\DsusPiano}{Re} Suspendida cuarta
\upchord{\EsusPiano}{Re} Suspendida cuarta
\upchord{\FsusPiano}{Fa} Suspendida cuarta
\vskip 25pt
\upchord{\GsusPiano}{Sol} Suspendida cuarta
\upchord{\GssusPiano}{Sol sostenido} Suspendida cuarta
\normalsize

\vskip 20pt
\textbf{Acodes Mayor Aumentada}
\vskip 25pt

\small
\upchord{\CMajPiano}{Do} Maj  \qquad\qquad
\upchord{\DMajPiano}{Re} Maj
\upchord{\GMajPiano}{Sol} Maj
\normalsize

\vskip 20pt
\textbf{Acodes Mayor S\'eptima Aumentada}
\vskip 25pt

\small
\upchord{\AsevenMajPiano}{La} Maj S\'eptima Aumentada
\upchord{\FmajsevenPiano}{Fa} Maj S\'eptima Aumentada
\normalsize

\vskip 20pt
\textbf{Acodes Aumentada 2}
\vskip 25pt

\small
\upchord{\AtwoPiano}{La} Aumentada 2
\upchord{\CtwoPiano}{Do} Aumentada 2
\normalsize


\vskip 20pt
\textbf{Acodes Disminuidos}
\vskip 25pt

\small
\upchord{\GsdimPiano}{Sol} sostenido disminuido
\normalsize


\vskip 20pt
\textbf{Acodes Con Bajo cambiado}
\vskip 25pt

\small
\upchord{\ACsPiano}{La Mayor bajo C\#}
\upchord{\AEPiano}{La Mayor bajo E}
\vskip 20pt
\upchord{\AmFPiano}{La Menor bajo F}
\upchord{\CEPiano}{Do Mayor bajo E}
\vskip 20pt
\upchord{\CGPiano}{Do Mayor bajo G}
\upchord{\DAPiano}{Re Mayor bajo A}
\upchord{\DFsPiano}{Re Mayor bajo F\#}
\upchord{\GDPiano}{Sol Mayor Bajo D}
\vskip 20pt
\upchord{\GBPiano}{Sol Mayor Bajo B}
\vskip 25pt
\upchord{\DsusFsPiano}{Re} Suspendida cuarta bajo F\#
\normalsize

\vskip 20pt
\textbf{Acodes medio disminuido s\'eptima}
\vskip 25pt

\small
\upchord{\BmsevenbfivePiano}{Si} medio disminuido s\'eptima
\normalsize

\vskip 20pt
\textbf{Acodes 13 suspendida cuarta}
\vskip 25pt

\small
\upchord{\CsusthirteenPiano}{Do} 13 suspendida cuarta
\normalsize

\clearpage
\fi
%\end{document}
%\bye
\end{document}
\bye
%
%%%
% Document ends.
%%%
