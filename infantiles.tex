%! Author = Ruslan López
%! Date = 19/08/2025

% Preamble
% genera acordes de guitarra
\newif\ifguitarra

%genera acordes de piano
\newif\ifpiano

\guitarratrue
\pianotrue

% comandos para pintar acordes de guitarra
\newcommand{\AGuitar}{\chord{t}{x,o,f1p2,f2p2,f3p2,n}{A}}
\newcommand{\AsusGuitar}{\chord{t}{x,o,f1p2,f2p2,f3p3,n}{Asus4}}
\newcommand{\AsevenGuitar}{\chord{t}{x,o,f1p2,n,f3p2,n}{A7}}
\newcommand{\AMajGuitar}{\chord{t}{x,o,f2p2,f1p1,f3p2,n}{A+7}}
\newcommand{\ABGuitar}{\chord{t}{x,bf1p2,f2p2,f3p2,f4p2,n}{A/B}}
\newcommand{\ACsGuitar}{\chord{t}{x,bf4p3,f2p2,f3p2,f1p2,n}{A/C$\#$}}
\newcommand{\ADGuitar}{\chord{t}{x,x,o,f2p2,f3p2,n}{A/D}}
\newcommand{\AEGuitar}{\chord{t}{o,n,f1p2,f2p2,f3p2,n}{A/E}}
\newcommand{\AGGuitar}{\chord{t}{bf3p3,n,f1p2,n,f2p2,n}{A/G}}
\newcommand{\AAsGuitar}{\chord{t}{x,bf1p1,f2p2,f3p2,f4p2,n}{A/Bb}}
\newcommand{\AsevenCsGuitar}{\chord{t}{x,bf4p3,f1p2,n,f2p2,n}{A7/C\#}}
\newcommand{\AtwoGuitar}{\chord{t}{x,o,f1p2,f3p4,f2p2,n}{A2}}
\newcommand{\AsustwoGuitar}{\chord{t}{x,o,f1p2,f2p2,n,n}{Asus2}}
\newcommand{\AnineGuitar}{\chord{t}{x,o,f1p2,f3p4,f1p2,f2p3}{A9}}

\newcommand{\AmGuitar}{\chord{t}{x,o,f2p2,f3p2,f1p1,n}{Am}}
\newcommand{\AmsevenGuitar}{\chord{t}{x,n,f2p2,n,f1p1,n}{Am7}}
\newcommand{\AmCGuitar}{\chord{t}{x,bf4p3,f3p2,f2p2,f1p1,x}{Am/C}}
\newcommand{\AmCsGuitar}{\chord{t}{x,bf4p4,f3p2,f2p2,f1p1,x}{Am/C$\#$}}
\newcommand{\AmEGuitar}{\chord{t}{o,n,f2p2,f3p2,f1p1,n}{Am/E}}
\newcommand{\AmFGuitar}{\chord{t}{bf1p1,n,f3p2,f4p2,f2p1,x}{Am/F}}
\newcommand{\AmFsGuitar}{\chord{t}{bf2p2,n,f3p2,f4p2,f1p1,x}{Am/F$\#$}}
\newcommand{\AmGGuitar}{\chord{t}{p3,n,p2,p2,p1,x}{Am/G}}
\newcommand{\AmGsGuitar}{\chord{t}{p4,n,p2,p2,p1,x}{Am/G$\#$}}
\newcommand{\AmsevenFGuitar}{\chord{t}{p1,n,p2,n,p1,x}{Am7/F}}

\newcommand{\AsGuitar}{\chord{t}{x,bf1p1,f2p3,f3p3,f4p3,f1p1}{A$\#$}}
\newcommand{\AsMajGuitar}{\chord{t}{p1,p1,p3,p2,p3,p1}{A$\#$+7}}
\newcommand{\AsmGuitar}{\chord{t}{x,bf1p1,f3p3,f4p3,f2p2,f1p1}{A$\#$m}}
\newcommand{\AsthirteenGuitar}{\chord{t}{x,bf1p1,f1p1,f1p1,f4p3,f4p3}{A$\#$13}}
\newcommand{\BflatGuitar}{\chord{t}{x,bf1p1,f2p3,f3p3,f4p3,f1p1}{Bb}}
\newcommand{\BflatsevenGuitar}{\chord{t}{x,bf1p1,f3p3,f1p1,f4p3,f1p1}{Bb7}}
\newcommand{\BflatmGuitar}{\chord{t}{x,bf1p1,f3p3,f4p3,f2p2,f1p1}{Bbm}}
\newcommand{\BflatmsevenGuitar}{\chord{t}{x,bf1p1,f3p3,f1p1,f2p2,f1p1}{Bbm7}}
\newcommand{\BflatMajGuitar}{\chord{t}{p1,p1,p3,p2,p3,p1}{Bb+7}}
\newcommand{\BflatCGuitar}{\chord{t}{x,p3,n,p3,p1,p1}{Bb/C}}
\newcommand{\BflatDGuitar}{\chord{t}{p1,p1,p3,p3,p3,p1}{Bb/D}}
\newcommand{\BflatFGuitar}{\chord{t}{bf1p1,f1p1,f2p3,f3p3,f4p3,f1p1}{Bb/F}}
\newcommand{\BflatGGuitar}{\chord{t}{bf1p3,f2p1,n,n,f3p3,f4p3}{Bb/G}}

\newcommand{\BGuitar}{\chord{t}{x,bf1p2,f2p4,f3p4,f4p4,f1p2}{B}}
\newcommand{\BsusGuitar}{\chord{t}{x,bf1p2,f2p4,f3p4,f4p5,f1p2}{Bsus4}}
\newcommand{\BsevenGuitar}{\chord{t}{x,f1p2,f3p4,f1p2,f4p4,f1p2}{B7}}
\newcommand{\BsevenFsGuitar}{\chord{t}{f2p2,n,f1p1,f3p2,n,f4p2}{B7/F$\#$}}
\newcommand{\BsevenDsGuitar}{\chord{t}{x,x,p1,p2,n,p2}{B7/D$\#$}}
\newcommand{\BDsGuitar}{\chord{2}{x,bf4p4,f1p2,f2p2,n,x}{B/D$\#$}}
\newcommand{\BEflatGuitar}{\chord{2}{x,bf4p4,f1p2,f2p2,n,x}{B/Eb}}
\newcommand{\BmGuitar}{\chord{t}{x,bf1p2,f3p4,f4p4,f2p3,f1p2}{Bm}}
\newcommand{\BmsevenGuitar}{\chord{t}{x,bf1p2,f3p4,f1p2,f2p3,f1p2}{Bm7}}
\newcommand{\BmseveNGuitar}{\chord{t}{x,p2,p4,p3,p3,p2}{Bm7+}}
\newcommand{\BmsevenAGuitar}{\chord{t}{x,n,p4,p4,p3,n}{Bm7/A}}
\newcommand{\BmsevenbfiveGuitar}{\chord{t}{x,bf1p2,f3p3,f2p2,f4p3,n}{Bm7(b5)}} %si semidisminuido
\newcommand{\BmajsevenGGuitar}{\chord{t}{p3,x,p1,p3,n,p2}{B+7/G}}
\newcommand{\BdimDGuitar}{\chord{t}{x,x,o,p4,n,p1}{Bdim/D}}

\newcommand{\CGuitar}{\chord{t}{x,bf3p3,n,f2p2,f1p1,n}{C}}
\newcommand{\CsusGuitar}{\chord{t}{x,bf3p3,f4p3,n,f1p1,f2p1}{Csus4}}
\newcommand{\CsevenGuitar}{\chord{t}{x,bf3p3,f2p2,f4p3,f1p1,n}{C7}}
\newcommand{\CsevenBflatGuitar}{\chord{t}{x,bf1p1,f3p2,n,f2p1,n}{C7/Bb}}
\newcommand{\CmGuitar}{\chord{t}{x,bf3p3,f1p1,n,f2p1,f4p3}{Cm}}
\newcommand{\CmsevenGuitar}{\chord{2}{x,bf1p1,f3p3,f1p1,f2p2,f1p1}{Cm7}}
\newcommand{\CmAflatGuitar}{\chord{3}{x,x,p4,p3,p2,p1}{Cm/Ab}}
\newcommand{\CDGuitar}{\chord{t}{n,n,o,n,f1p1,n}{C/D}}
\newcommand{\CEGuitar}{\chord{t}{o,bf3p3,n,f2p2,f1p1,n}{C/E}}
\newcommand{\CGGuitar}{\chord{t}{bf3p3,n,f2p2,n,f1p1,n}{C/G}}
\newcommand{\CMajGuitar}{\chord{t}{f2p3,bf3p3,f1p2,n,n,n}{C+7}}
\newcommand{\CMajtwoDGuitar}{\chord{t}{x,x,o,p4,p1,n}{C+7+2/D}}
\newcommand{\CsustwoGuitar}{\chord{t}{x,bf2p3,n,n,f1p1,f4p3}{Csus2}}
\newcommand{\CtwoGuitar}{\chord{t}{x,bf2p3,f1p2,n,f3p3,n}{C2}}
\newcommand{\CnineGuitar}{\chord{t}{x,bf2p3,f1p2,f3p3,f4p3,n}{C9}}
\newcommand{\CsusthirteenGuitar}{\chord{t}{x,bf1p3,f1p3,f1p3,f1p3,f4p5}{C13sus4}}

\newcommand{\CsGuitar}{\chord{4}{n,o,f2p2,f3p2,f4p2,n}{C$\#$}}
\newcommand{\CsmGuitar}{\chord{2}{x,bf1p2,f3p4,f4p4,f2p3,f1p2}{C$\#$m}}
\newcommand{\CsmsevenGuitar}{\chord{t}{x,bf1p4,f2p2,f3p1,o,o}{C$\#$m7}}
\newcommand{\CssevenGuitar}{\chord{t}{x,p4,p3,p4,p2,x}{C$\#$7}}

\newcommand{\DflatGuitar}{\chord{4}{n,o,f2p2,f3p2,f4p2,n}{Db}}
\newcommand{\DflatmGuitar}{\chord{2}{x,bf1p2,f3p4,f4p4,f2p3,f1p2}{Dbm}}
\newcommand{\DflatFGuitar}{\chord{t}{x,x,p3,p5,p4,p4}{Db/F}}

\newcommand{\DGuitar}{\chord{t}{x,x,o,f1p2,f3p3,f2p2}{D}}
\newcommand{\DsevenGuitar}{\chord{t}{x,x,o,f2p2,f1p1,f3p2}{D7}}
\newcommand{\DmGuitar}{\chord{t}{x,x,o,f2p2,f3p3,f1p1}{Dm}}
\newcommand{\DmsevenGuitar}{\chord{t}{n,n,o,f2p2,f1p1,f1p1}{Dm7}}
\newcommand{\DMajGuitar}{\chord{t}{x,x,o,f1p2,f1p2,f1p2}{D+7}}
\newcommand{\DsusGuitar}{\chord{t}{x,x,o,f1p2,f3p3,f4p3}{Dsus4}}
\newcommand{\DsusnineGuitar}{\chord{t}{x,x,o,f2p2,f3p3,n}{Dsus9}}
\newcommand{\DmBGuitar}{\chord{t}{x,bf2p2,n,f3p2,f4p3,f1p1}{Dm/B}}
\newcommand{\DmCGuitar}{\chord{t}{x,bf3p3,n,f2p2,f4p3,f1p1}{Dm/C}}
\newcommand{\DmFGuitar}{\chord{t}{bf1p1,n,n,f2p2,f3p3,x}{Dm/F}}
\newcommand{\DtwoGuitar}{\chord{t}{n,bf4p5,f1p2,f1p2,f2p3,f1p2}{D2}}
\newcommand{\DsixGuitar}{\chord{t}{x,x,n,p2,n,p2}{D6}}
\newcommand{\DmBasBGuitar}{\chord{t}{x,p2,p3,p2,p3,x}{Dm/B}}
\newcommand{\DAGuitar}{\chord{t}{x,o,n,f1p2,f3p3,f2p2}{D/A}}
\newcommand{\DCGuitar}{\chord{t}{x,p3,n,p2,p3,p2}{D/C}}
\newcommand{\DEGuitar}{\chord{t}{n,n,bf1p2,f1p2,f2p3,f1p2}{D/E}}
\newcommand{\DFsGuitar}{\chord{t}{bf1p2,n,n,f2p2,f4p3,f3p2}{D/F$\#$}}
\newcommand{\DsevenFsGuitar}{\chord{t}{bf2p2,n,n,f3p2,f1p1,f4p2}{D7/F$\#$}}
\newcommand{\DmsevenGGuitar}{\chord{t}{p3,n,n,p2,p1,p1}{Dm7/G}}
\newcommand{\DsusFsGuitar}{\chord{t}{bf1p2,x,n,f2p2,f3p3,f4p3}{Dsus/F\#}}

\newcommand{\DsGuitar}{\chord{t}{n,n,bf1p1,f2p3,f4p4,f3p3}{D$\#$}}
\newcommand{\DssevenGuitar}{\chord{t}{n,n,bf1p1,f3p3,f1p4,f4p3}{D\#7}}
\newcommand{\DsmGuitar}{\chord{t}{x,x,bf1p1,f3p3,f4p4,f1p2}{D\#m}}
\newcommand{\DsmsevenGuitar}{\chord{t}{x,x,bf1p1,f4p3,f2p2,f3p2}{D\#m7}}
\newcommand{\EflatGuitar}{\chord{t}{n,n,bf1p1,f2p3,f4p4,f3p3}{E$\flat$}}
\newcommand{\EflattwoGuitar}{\chord{3}{x,bf3p4,f1p1,f1p1,f2p2,f1p1}{E$\flat$2}}
\newcommand{\EflatMajGuitar}{\chord{t}{n,n,bf1p1,f2p3,f3p3,f4p3}{E$\flat$+7}}
\newcommand{\EflatFGuitar}{\chord{t}{bf1p1,f1p1,f1p1,f2p3,f4p4,f3p3}{E$\flat$/F}}
\newcommand{\EflatsevenGuitar}{\chord{t}{n,n,bf1p1,f3p3,f1p4,f4p3}{7}}
\newcommand{\EflatmGuitar}{\chord{t}{x,x,bf1p1,f3p3,f4p4,f1p2}{E$\flat$m}}
\newcommand{\EflatmsevenGuitar}{\chord{t}{x,x,bf1p1,f4p3,f2p2,f3p2}{E$\flat$m7}}

\newcommand{\EGuitar}{\chord{t}{o,f2p2,f3p2,f1p1,n,n}{E}}
\newcommand{\EsusGuitar}{\chord{t}{o,f2p2,f3p2,f4p2,n,n}{Esus4}}
\newcommand{\EfiveGuitar}{\chord{t}{o,f1p2,f2p2,f4p4,n,n}{E5}}
\newcommand{\EdimfiveGuitar}{\chord{t}{o,f1p1,f2p2,f3p3,n,n}{E(b5)}}
\newcommand{\EsevenGuitar}{\chord{t}{o,f2p2,f3p2,f1p1,f4p3,n}{E7}}
\newcommand{\EseveNGuitar}{\chord{t}{o,p2,p2,p4,p3,p4}{E7}}
\newcommand{\EmGuitar}{\chord{t}{o,f1p2,f2p2,n,n,n}{Em}}
\newcommand{\EmsevenGuitar}{\chord{t}{o,f1p2,f2p2,n,f4p3,n}{Em7}}
\newcommand{\EmseventrGuitar}{\chord{t}{o,f1p2,f2p2,n,f3p3,f4p3}{Em7}}
\newcommand{\EmelevenGuitar}{\chord{t}{b,n,n,n,n,n}{Em11}}
\newcommand{\EmBGuitar}{\chord{t}{x,bf1p2,f2p2,n,n,n}{Em/B}}
\newcommand{\EmDGuitar}{\chord{t}{x,x,o,n,n,n}{Em/D}}
\newcommand{\EmGGuitar}{\chord{t}{bf3p3,f1p2,f2p2,n,n,n}{Em/G}}
\newcommand{\EsevenFourGuitar}{\chord{t}{n,p2,p2,p4,p3,p5}{E7,11}}
\newcommand{\EseveNNineGuitar}{\chord{t}{n,f1p2,f1p2,p4,p3,f1p2,}{E79}}
\newcommand{\EBGuitar}{\chord{t}{x,bf2p2,f3p2,f1p1,n,n}{E/B}}
\newcommand{\EDGuitar}{\chord{t}{x,x,o,f1p1,n,n}{E/D}}
\newcommand{\EFsGuitar}{\chord{t}{bf2p2,n,f3p2,f1p1,n,n}{E/F\#}}
\newcommand{\EGGuitar}{\chord{t}{bf3p3,n,f2p2,f1p1,n,n}{E/G}}
\newcommand{\EGsGuitar}{\chord{t}{bf4p4,n,f2p2,f1p1,n,n}{E/G\#}}

\newcommand{\FGuitar}{\chord{t}{bf1p1,f3p3,f4p3,f2p2,f1p1,f1p1}{F}}
\newcommand{\FsusGuitar}{\chord{t}{bf1p1,f3p3,f4p3,f2p3,f1p1,f1p1}{Fsus4}}
\newcommand{\FAGuitar}{\chord{t}{x,o,f3p3,f2p2,f1p1,f1p1}{F/A}}
\newcommand{\FtwoGuitar}{\chord{t}{x,x,bf3p3,f2p2,f1p1,f4p3}{F2}}
\newcommand{\FtwoAGuitar}{\chord{t}{x,o,f3p3,n,f1p1,f1p1}{F2/A}}
\newcommand{\FMajGuitar}{\chord{t}{x,x,bf3p3,f2p2,f1p1,n}{FMaj7}}
\newcommand{\FsevenGuitar}{\chord{t}{bf1p1,f3p3,f1p1,f2p2,f1p1,f1p1}{F7}}
\newcommand{\FnineGuitar}{\chord{t}{bf1p1,n,f2p1,n,f3p1,f4p3}{F9}}
\newcommand{\FmGuitar}{\chord{t}{bf1p1,f3p3,f4p3,f1p1,f1p1,f1p1}{Fm}}
\newcommand{\FmsevenGuitar}{\chord{t}{bf1p1,f3p3,f1p1,f1p1,f1p1,f1p1}{Fm7}}
\newcommand{\FmsixGuitar}{\chord{t}{bf1p1,f2p3,f3p3,f1p1,f4p3,f1p1}{Fm6}}
\newcommand{\FdimGuitar}{\chord{t}{x,x,f3p3,f1p1,n,f2p1}{Fdim}}

\newcommand{\FsGuitar}{\chord{t}{bf1p2,f3p4,f4p4,f2p3,f1p2,f1p2}{F\#}}
\newcommand{\FssevenGuitar}{\chord{t}{bf1p2,f3p4,f1p2,f2p3,f1p2,f1p2}{F\#7}}
\newcommand{\FssusGuitar}{\chord{t}{bf1p2,f1p2,f3p4,f4p4,f1p2,f1p2}{F\#sus}}
\newcommand{\FsmGuitar}{\chord{t}{f1p2,f3p4,f4p4,f1p2,f1p2,f1p2,}{F\#m}}
\newcommand{\FsmLightGuitar}{\chord{t}{x,x,f3p4,f1p2,f1p2,f1p2,}{F\#m}}
\newcommand{\FsmBasSeveNGuitar}{\chord{t}{x,x,f3p3,f1p2,f1p2,f1p2,}{F\#m/E\#}}
\newcommand{\FsmBasSevenGuitar}{\chord{t}{x,x,f2p2,f1p2,f1p2,f1p2,}{F\#m/E}}
\newcommand{\FsmsevenGuitar}{\chord{t}{f1p2,f2p4,f3p4,f1p2,f4p5,f1p2,}{F\#m7}}
\newcommand{\GflatGuitar}{\chord{t}{bf1p2,f3p4,f4p4,f2p3,f1p2,f1p2}{Gb}}
\newcommand{\GflatBflatGuitar}{\chord{t}{x,bp4,p4,p3,f1p2,f1p2}{Gb/Bb}}
\newcommand{\GflatmGuitar}{\chord{t}{f1p2,f3p4,f4p4,f1p2,f1p2,f1p2,}{Gbm}}

\newcommand{\GGuitar}{\chord{t}{bf2p3,f1p2,n,n,n,f3p3}{G}}
\newcommand{\GsevenGuitar}{\chord{t}{bf3p3,f2p2,n,n,n,f1p1}{G7}}
\newcommand{\GMajGuitar}{\chord{t}{bf3p3,f1p2,n,n,n,f2p2}{G+7}}
\newcommand{\GsusGuitar}{\chord{t}{bf3p3,f2p2,n,n,f1p1,f4p3}{Gsus4}}
\newcommand{\GBGuitar}{\chord{t}{n,bf1p2,n,n,f3p3,n}{G/B}}
\newcommand{\GDGuitar}{\chord{t}{x,x,x,o,f3p3,f4p3}{G/D}}
\newcommand{\GEGuitar}{\chord{t}{o,f1p2,f2p2,n,f3p3,f4p3}{G/E}}
\newcommand{\GFGuitar}{\chord{t}{bf1p1,f2p2,n,n,f3p3,f4p3}{G/F}}
\newcommand{\GnineGuitar}{\chord{t}{bf2p3,n,n,n,n,f1p1}{G9}}
\newcommand{\GnineDGuitar}{\chord{t}{n,n,o,n,n,f4p1}{G9/D}}
\newcommand{\GtwoGuitar}{\chord{t}{bf1p3,n,n,n,n,f4p3}{G2}}
\newcommand{\GmGuitar}{\chord{t}{f1p3,f3p5,f4p5,f1p3,f1p3,f1p3,}{Gm}}
\newcommand{\GmsevenGuitar}{\chord{t}{f1p3,f3p5,f1p3,f1p3,f1p3,f1p3,}{Gm7}}
\newcommand{\GmBflatGuitar}{\chord{t}{x,bf1p1,n,n,f3p3,f4p3}{Gm/Bb}}

\newcommand{\GsGuitar}{\chord{3}{x,x,f3p4,f2p3,f1p2,f1p2}{G\#}}
\newcommand{\AflatGuitar}{\chord{3}{x,x,f3p4,f2p3,f1p2,f1p2}{A$\flat$}}
\newcommand{\AflatsevenGuitar}{\chord{4}{bf1p1,f3p3,f1p1,f2p2,f1p1,f1p1}{A$\flat$7}}
\newcommand{\AflatmGuitar}{\chord{t}{p4,p2,p1,p1,n,x}{A$\flat$m}}
\newcommand{\AflatEflatGuitar}{\chord{3}{x,bf3p4,f4p4,f2p3,f1p2,f1p2}{A$\flat$/E$\flat$}}
\newcommand{\AflatFGuitar}{\chord{t}{bf1p1,f3p3,f1p1,f1p1,f1p1,f1p1}{A$\flat$/F}}
\newcommand{\AflatCGuitar}{\chord{t}{x,bf2p3,f1p1,f1p1,f1p1,f4p4}{A$\flat$/C}}
\newcommand{\AflatMajGuitar}{\chord{7}{p4,p3,x,o,p4,p4}{A$\flat$Maj7}}
\newcommand{\GsmGuitar}{\chord{t}{p4,p2,p1,p1,n,x}{G\#m}}
\newcommand{\GsmsevenGuitar}{\chord{t}{f2p4,x,f4p4,f4p4,f4p4,f4p4}{G\#m7}}
\newcommand{\GssusGuitar}{\chord{4}{bf1p1,f2p3,f3p3,f4p3,f1p1,f1p1}{G\#sus4}}
\newcommand{\GsdimGuitar}{\chord{3}{p1,p2,p3,p1,p1,p1}{G\#dim}}

% comandos para pintar acordes de piano
\newcommand{\APiano}{\keyboardf[Ao][Cso][Eo]}
\newcommand{\AsevenPiano}{\keyboardf[Ao][Cso][Eo][Go]}
\newcommand{\AMajPiano}{\keyboardf[Ao][Cso][Eo][Gso]}
\newcommand{\AmPiano}{\keyboardf[Ao][Co][Eo]}
\newcommand{\AmsevenPiano}{\keyboardf[Ao][Co][Eo][Go]}
\newcommand{\AmCPiano}{\keyboard[Co][Eo][Ao]}
\newcommand{\AmCsPiano}{\keyboardtwooctaves[Cso][Ao][Eo][Ct]}
\newcommand{\AmFPiano}{\keyboardf[Fo][Ao][Co][Eo]}
\newcommand{\AmFsPiano}{\keyboardf[Fso][Ao][Co][Eo]}
\newcommand{\AmGPiano}{\keyboardf[Go][Ao][Co][Eo]}
\newcommand{\AmGsPiano}{\keyboardf[Gso][Ao][Co][Eo]}
\newcommand{\AsevenCsPiano}{\keyboard[Cso][Eo][Ao][Go]}
\newcommand{\AmsevenFPiano}{\keyboardf[Fo][Go][Ao][Co][Eo]}
\newcommand{\AAsPiano}{\keyboard[Aso][Cso][Eo][Ao]}
\newcommand{\ABPiano}{\keyboardtwooctaves[Bo][Cst][Et][At]}
\newcommand{\ACsPiano}{\keyboard[Cso][Eo][Ao]}
\newcommand{\ADPiano}{\keyboardtwooctaves[Do][Eo][Ao][Cst]}
\newcommand{\AEPiano}{\keyboardtwooctaves[Eo][Ao][Cst]}
\newcommand{\AGPiano}{\keyboardf[Go][Ao][Cso][Eo]}
\newcommand{\AsusPiano}{\keyboardf[Ao][Do][Eo]}
\newcommand{\AtwoPiano}{\keyboardf[Ao][Bo][Cso][Eo]}
\newcommand{\AsustwoPiano}{\keyboardf[Ao][Bo][Eo]}
\newcommand{\AninePiano}{\keyboardf[Ao][Bo][Cso][Eo][Go]}

\newcommand{\AsPiano}{\keyboard[Do][Fo][Aso]}
\newcommand{\AsMajPiano}{\keyboard[Do][Fo][Ao][Aso]}
\newcommand{\AsmPiano}{\keyboard[Cso][Fo][Aso]}
\newcommand{\AsthirteenPiano}{\keyboard[Aso][Do][Fo][Gso][Co][Dso]}
\newcommand{\BflatPiano}{\keyboard[Do][Fo][Aso]}
\newcommand{\BflatsevenPiano}{\keyboard[Do][Fo][Aso][Gso]}
\newcommand{\BflatmPiano}{\keyboard[Cso][Fo][Aso]}
\newcommand{\BflatmsevenPiano}{\keyboard[Cso][Fo][Aso][Gso]}
\newcommand{\BflatMajPiano}{\keyboard[Do][Fo][Ao][Aso]}
\newcommand{\BflatCPiano}{\keyboard[Co][Do][Fo][Aso]}
\newcommand{\BflatDPiano}{\keyboard[Do][Fo][Aso]}
\newcommand{\BflatFPiano}{\keyboardf[Fo][Aso][Do]}
\newcommand{\BflatGPiano}{\keyboardf[Go][Aso][Do][Fo]}

\newcommand{\BPiano}{\keyboard[Dso][Fso][Bo]}
\newcommand{\BsusPiano}{\keyboard[Eo][Fso][Bo]}
\newcommand{\BDsPiano}{\keyboard[Dso][Fso][Bo]}
\newcommand{\BEflatPiano}{\keyboard[Dso][Fso][Bo]}
\newcommand{\BsevenPiano}{\keyboard[Dso][Fso][Bo][Ao]}
\newcommand{\BsevenFsPiano}{\keyboardf[Fso][Ao][Bo][Dso]}
\newcommand{\BsevenDsPiano}{\keyboard[Dso][Fso][Ao][Bo]}
\newcommand{\BmPiano}{\keyboard[Do][Fso][Bo]}
\newcommand{\BmsevenPiano}{\keyboard[Do][Fso][Bo][Ao]}
\newcommand{\BmsevenbfivePiano}{\keyboard[Bo][Do][Fo][Ao]}
\newcommand{\BmajsevenGPiano}{\keyboardf[Go][Bo][Dso][Fso]}
\newcommand{\BdimDPiano}{\keyboard[Do][Fo][Bo]}

\newcommand{\CPiano}{\keyboard[Co][Eo][Go]}
\newcommand{\CsusPiano}{\keyboard[Co][Fo][Go]}
\newcommand{\CDPiano}{\keyboardtwooctaves[Do][Eo][Go][Ct]}
\newcommand{\CEPiano}{\keyboardtwooctaves[Eo][Go][Ct]}
\newcommand{\CGPiano}{\keyboardf[Go][Co][Eo]}
\newcommand{\CsevenPiano}{\keyboard[Co][Eo][Go][Aso]}
\newcommand{\CsevenBflatPiano}{\keyboardtwooctaves[Aso][Ct][Et][Gt]}
\newcommand{\CmPiano}{\keyboard[Co][Dso][Go]}
\newcommand{\CmsevenPiano}{\keyboard[Co][Dso][Go][Aso]}
\newcommand{\CmAflatPiano}{\keyboardtwooctaves[Gso][Ct][Dst][Gt]}
\newcommand{\CtwoPiano}{\keyboard[Co][Do][Eo][Go]}
\newcommand{\CsustwoPiano}{\keyboard[Co][Do][Go]}
\newcommand{\CninePiano}{\keyboard[Co][Do][Eo][Go][Aso]}
\newcommand{\CMajPiano}{\keyboard[Co][Eo][Go][Bo]}
\newcommand{\CMajtwoDPiano}{\keyboardtwooctaves[Do][Eo][Go][Bo][Ct]}
\newcommand{\CsusthirteenPiano}{\keyboard[Co][Do][Fo][Go][Ao][Aso]}

\newcommand{\CsPiano}{\keyboard[Cso][Fo][Gso]}
\newcommand{\CssevenPiano}{\keyboard[Cso][Fo][Gso][Bo]}
\newcommand{\CsmPiano}{\keyboard[Cso][Eo][Gso]}
\newcommand{\CsmsevenPiano}{\keyboard[Cso][Eo][Gso][Aso]}
\newcommand{\CsFPiano}{\keyboardf[Fo][Gso][Cso]}

\newcommand{\DflatPiano}{\CsPiano}
\newcommand{\DflatmPiano}{\CsmPiano}
\newcommand{\DflatFPiano}{\CsFPiano}

\newcommand{\DPiano}{\keyboard[Do][Fso][Ao]}
\newcommand{\DsevenPiano}{\keyboard[Do][Fso][Ao][Co]}
\newcommand{\DmPiano}{\keyboard[Do][Fo][Ao]}
\newcommand{\DmBPiano}{\keyboard[Bo][Do][Fo][Ao]}
\newcommand{\DmCPiano}{\keyboard[Co][Do][Fo][Ao]}
\newcommand{\DmFPiano}{\keyboardf[Do][Fo][Ao]}
\newcommand{\DmsevenPiano}{\keyboard[Do][Fo][Ao][Co]}
\newcommand{\DmsevenGPiano}{\keyboardf[Go][Do][Fo][Ao][Co]}
\newcommand{\DtwoPiano}{\keyboard[Do][Eo][Fso][Ao]}
\newcommand{\DMajPiano}{\keyboard[Do][Fso][Ao][Cso]}
\newcommand{\DCPiano}{\keyboard[Co][Do][Fso][Ao]}
\newcommand{\DAPiano}{\keyboardf[Ao][Do][Fso]}
\newcommand{\DEPiano}{\keyboardtwooctaves[Eo][Fso][Ao][Dt]}
\newcommand{\DFsPiano}{\keyboardf[Fso][Ao][Do]}
\newcommand{\DsevenFsPiano}{\keyboardf[Fso][Ao][Co][Do]}
\newcommand{\DsusPiano}{\keyboard[Do][Go][Ao]}
\newcommand{\DsusninePiano}{\keyboard[Do][Go][Ao][Eo]}
\newcommand{\DsusFsPiano}{\keyboardf[Fo][Go][Ao][Do]}

\newcommand{\DsPiano}{\keyboard[Dso][Go][Aso]}
\newcommand{\DsmPiano}{\keyboard[Dso][Fso][Aso]}
\newcommand{\DsmsevenPiano}{\keyboard[Dso][Fso][Aso][Cso]}
\newcommand{\EflatPiano}{\keyboard[Dso][Go][Aso]}
\newcommand{\EflattwoPiano}{\keyboard[Dso][Fo][Go][Aso]}
\newcommand{\EflatMajPiano}{\keyboard[Dso][Go][Aso][Do]}
\newcommand{\EflatFPiano}{\keyboardf[Fo][Go][Aso][Dso]}
\newcommand{\EflatsevenPiano}{\keyboard[Dso][Go][Aso][Cso]}
\newcommand{\EflatmPiano}{\DsmPiano}
\newcommand{\EflatmsevenPiano}{\keyboard[Dso][Fso][Aso][Cso]}

\newcommand{\EPiano}{\keyboard[Eo][Gso][Bo]}
\newcommand{\EsusPiano}{\keyboard[Eo][Ao][Bo]}
\newcommand{\EfivePiano}{\keyboard[Eo][Bo]}
\newcommand{\EdimfivePiano}{\keyboard[Eo][Aso][Bo]}
\newcommand{\EsevenPiano}{\keyboard[Eo][Gso][Bo][Do]}
\newcommand{\EmPiano}{\keyboard[Eo][Go][Bo]}
\newcommand{\EmsevenPiano}{\keyboard[Eo][Go][Bo][Do]}
\newcommand{\EmelevenPiano}{\keyboard[Eo][Fso][Go][Ao][Bo][Do]}
\newcommand{\EmBPiano}{\keyboardtwooctaves[Bo][Et][Gt]}
\newcommand{\EmDPiano}{\keyboard[Do][Eo][Go][Bo]}
\newcommand{\EmGPiano}{\keyboardf[Go][Bo][Eo]}
\newcommand{\EBPiano}{\keyboardtwooctaves[Bo][Et][Gst]}
\newcommand{\EDPiano}{\keyboard[Do][Eo][Gso][Bo]}
\newcommand{\EFsPiano}{\keyboardf[Fso][Eo][Gso][Bo]}
\newcommand{\EGPiano}{\keyboardf[Go][Eo][Bo]}
\newcommand{\EGsPiano}{\keyboardf[Gso][Eo][Bo]}

\newcommand{\FPiano}{\keyboardf[Co][Fo][Ao]}
\newcommand{\FsevenPiano}{\keyboard[Co][Fo][Ao][Dso]}
\newcommand{\FninePiano}{\keyboardf[Fo][Go][Ao][Co][Dso]}
\newcommand{\FAPiano}{\keyboardf[Ao][Co][Ft]}
\newcommand{\FtwoPiano}{\keyboardf[Fo][Go][Ao][Co]}
\newcommand{\FtwoAPiano}{\keyboardtwooctaves[Ao][Ct][Ft][Gt]}
\newcommand{\FsusPiano}{\keyboardf[Aso][Co][Fo]}
\newcommand{\FMajPiano}{\keyboardf[Co][Fo][Ao][Eo]}
\newcommand{\FmPiano}{\keyboardf[Fo][Aso][Co]}
\newcommand{\FmsevenPiano}{\keyboard[Co][Fo][Aso][Dso]}
\newcommand{\FmsixPiano}{\keyboardf[Fo][Gso][Co][Do]}
\newcommand{\FdimPiano}{\keyboardf[Fo][Aso][Bo]}

\newcommand{\FsPiano}{\keyboard[Cso][Fso][Aso]}
\newcommand{\FssevenPiano}{\keyboardf[Fso][Aso][Cso][Eo]}
\newcommand{\FssusPiano}{\keyboard[Cso][Fso][Bo]}
\newcommand{\FsmPiano}{\keyboardf[Cso][Fso][Ao]}
\newcommand{\FsmsevenPiano}{\keyboard[Cso][Fso][Ao][Eo]}
\newcommand{\GflatPiano}{\keyboard[Cso][Fso][Aso]}
\newcommand{\GflatmPiano}{\keyboardf[Cso][Fso][Ao]}
\newcommand{\GflatBflatPiano}{\keyboardf[Aso][Cso][Fso]}

\newcommand{\GPiano}{\keyboard[Do][Go][Bo]}
\newcommand{\GsevenPiano}{\keyboard[Do][Fo][Go][Bo]}
\newcommand{\GsusPiano}{\keyboardf[Go][Co][Do]}
\newcommand{\GMajPiano}{\keyboardf[Do][Go][Bo][Fso]}
\newcommand{\GBPiano}{\keyboardtwooctaves[Bo][Dt][Gt]}
\newcommand{\GEPiano}{\keyboardtwooctaves[Eo][Go][Bo][Dt]}
\newcommand{\GFPiano}{\keyboardf[Fo][Go][Bo][Do]}
\newcommand{\GDPiano}{\keyboard[Do][Go][Bo]}
\newcommand{\GtwoPiano}{\keyboardf[Go][Ao][Bo][Do]}
\newcommand{\GninePiano}{\keyboardf[Go][Ao][Bo][Do][Fo]}
\newcommand{\GnineDPiano}{\keyboard[Do][Fo][Go][Ao][Bo]}
\newcommand{\GmPiano}{\keyboard[Do][Go][Aso]}
\newcommand{\GmsevenPiano}{\keyboard[Do][Go][Aso][Fo]}
\newcommand{\GmBflatPiano}{\keyboardf[Aso][Go][Do]}

\newcommand{\GsPiano}{\keyboard[Dso][Gso][Co]}
\newcommand{\GssevenPiano}{\keyboardf[Gso][Co][Dso][Fso]}
\newcommand{\GssusPiano}{\keyboardf[Gso][Cso][Dso]}
\newcommand{\GsmPiano}{\keyboardf[Gso][Bo][Dso]}
\newcommand{\GsmsevenPiano}{\keyboardf[Gso][Bo][Dso][Fso]}
\newcommand{\AflatmPiano}{\keyboardf[Gso][Bo][Dso]}
\newcommand{\GsdimPiano}{\keyboardf[Gso][Bo][Do]}
\newcommand{\AflatPiano}{\keyboard[Dso][Gso][Co]}
\newcommand{\AflatsevenPiano}{\GssevenPiano}
\newcommand{\AflatMajPiano}{\keyboardf[Gso][Dso][Co][Go]}
\newcommand{\AflatCPiano}{\keyboard[Co][Dso][Gso]}
\newcommand{\AflatEflatPiano}{\keyboardtwooctaves[Dso][Gso][Ct]}
\newcommand{\AflatFPiano}{\keyboardf[Fo][Gso][Co][Dso]}

%% impresión de nombres de acordes
\newcommand{\A}{\ifguitarra\AGuitar\fi\ifpiano\APiano\fi}
\newcommand{\Asus}{\ifguitarra\AsusGuitar\fi\ifpiano\AsusPiano\fi}
\newcommand{\Aseven}{\ifguitarra\AsevenGuitar\fi\ifpiano\AsevenPiano\fi}
\newcommand{\AMaj}{\ifguitarra\AMajGuitar\fi\ifpiano\AMajPiano\fi}
\newcommand{\AB}{\ifguitarra\ABGuitar\fi\ifpiano\ABPiano\fi}
\newcommand{\ACs}{\ifguitarra\ACsGuitar\fi\ifpiano\ACsPiano\fi}
\newcommand{\AD}{\ifguitarra\ADGuitar\fi\ifpiano\ADPiano\fi}
\newcommand{\AEg}{\ifguitarra\AEGuitar\fi\ifpiano\AEPiano\fi}
\newcommand{\AG}{\ifguitarra\AGGuitar\fi\ifpiano\AGPiano\fi}
\newcommand{\AAs}{\ifguitarra\AAsGuitar\fi\ifpiano\AAsPiano\fi}
\newcommand{\AsevenCs}{\ifguitarra\AsevenCsGuitar\fi\ifpiano\AsevenCsPiano\fi}
\newcommand{\Atwo}{\ifguitarra\AtwoGuitar\fi\ifpiano\AtwoPiano\fi}
\newcommand{\Asustwo}{\ifguitarra\AsustwoGuitar\fi\ifpiano\AsustwoPiano\fi}
\newcommand{\Anine}{\ifguitarra\AnineGuitar\fi\ifpiano\AninePiano\fi}

\newcommand{\Am}{\ifguitarra\AmGuitar\fi\ifpiano\AmPiano\fi}
\newcommand{\Amseven}{\ifguitarra\AmsevenGuitar\fi\ifpiano\AmsevenPiano\fi}
\newcommand{\AmC}{\ifguitarra\AmCGuitar\fi\ifpiano\AmCPiano\fi}
\newcommand{\AmCs}{\ifguitarra\AmCsGuitar\fi\ifpiano\AmCsPiano\fi}
\newcommand{\AmF}{\ifguitarra\AmFGuitar\fi\ifpiano\AmFPiano\fi}
\newcommand{\AmFs}{\ifguitarra\AmFsGuitar\fi\ifpiano\AmFsPiano\fi}
\newcommand{\AmG}{\ifguitarra\AmGuitar\fi\ifpiano\AmGPiano\fi}
\newcommand{\AmGs}{\ifguitarra\AmGsGuitar\fi\ifpiano\AmGsPiano\fi}
\newcommand{\AmsevenF}{\ifguitarra\AmsevenFGuitar\fi\ifpiano\AmsevenFPiano\fi}

\newcommand{\As}{\ifguitarra\AsGuitar\fi\ifpiano\AsPiano\fi}
\newcommand{\AsMaj}{\ifguitarra\AsMajGuitar\fi\ifpiano\AsMajPiano\fi}
\newcommand{\Asm}{\ifguitarra\AsmGuitar\fi\ifpiano\AsMPiano\fi}
\newcommand{\Asthirteen}{\ifguitarra\AsthirteenGuitar\fi\ifpiano\AsthirteenPiano\fi}
\newcommand{\Bflat}{\ifguitarra\BflatGuitar\fi\ifpiano\BflatPiano\fi}
\newcommand{\Bflatseven}{\ifguitarra\BflatsevenGuitar\fi\ifpiano\BflatsevenPiano\fi}
\newcommand{\Bflatm}{\ifguitarra\BflatmGuitar\fi\ifpiano\BflatmPiano\fi}
\newcommand{\Bflatmseven}{\ifguitarra\BflatmsevenGuitar\fi\ifpiano\BflatmsevenPiano\fi}
\newcommand{\BflatMaj}{\ifguitarra\BflatMajGuitar\fi\ifpiano\BflatMajPiano\fi}
\newcommand{\BflatC}{\ifguitarra\BflatCGuitar\fi\ifpiano\BflatCPiano\fi}
\newcommand{\BflatD}{\ifguitarra\BflatDGuitar\fi\ifpiano\BflatDPiano\fi}
\newcommand{\BflatF}{\ifguitarra\BflatFGuitar\fi\ifpiano\BflatFPiano\fi}
\newcommand{\BflatG}{\ifguitarra\BflatGuitar\fi\ifpiano\BflatGPiano\fi}

\newcommand{\B}{\ifguitarra\BGuitar\fi\ifpiano\BPiano\fi}
\newcommand{\Bsus}{\ifguitarra\BsusGuitar\fi\ifpiano\BsusPiano\fi}
\newcommand{\Bseven}{\ifguitarra\BsevenGuitar\fi\ifpiano\BsevenPiano\fi}
\newcommand{\BsevenFs}{\ifguitarra\BsevenFsGuitar\fi\ifpiano\BsevenFsPiano\fi}
\newcommand{\BsevenDs}{\ifguitarra\BsevenDsGuitar\fi\ifpiano\BsevenDsPiano\fi}
\newcommand{\BsevenBasDs}{\ifguitarra\BsevenBasDsGuitar\fi\ifpiano\BsevenBasDsPiano\fi}
\newcommand{\BDs}{\ifguitarra\BDsGuitar\fi\ifpiano\BDsPiano\fi}
\newcommand{\BEflat}{\ifguitarra\BEflatGuitar\fi\ifpiano\BEflatPiano\fi}
\newcommand{\Bm}{\ifguitarra\BmGuitar\fi\ifpiano\BmPiano\fi}
\newcommand{\Bmseven}{\ifguitarra\BmsevenGuitar\fi\ifpiano\BmsevenPiano\fi}
\newcommand{\BmseveN}{\ifguitarra\BmseveNGuitar\fi\ifpiano\BmseveNPiano\fi}
\newcommand{\BmsevenA}{\ifguitarra\BmsevenAGuitar\fi\ifpiano\BmsevenAPiano\fi}
\newcommand{\Bmsevenbfive}{\ifguitarra\BmsevenbfiveGuitar\fi\ifpiano\BmsevenbfivePiano\fi} %si semidisminuido
\newcommand{\BmajsevenG}{\ifguitarra\BmajsevenGGuitar\fi\ifpiano\BmajsevenGPiano\fi}
\newcommand{\BdimD}{\ifguitarra\BdimDGuitar\fi\ifpiano\BdimDPiano\fi}

\newcommand{\C}{\ifguitarra\CGuitar\fi\ifpiano\CPiano\fi}
\newcommand{\Csus}{\ifguitarra\CsusGuitar\fi\ifpiano\CsusPiano\fi}
\newcommand{\Cseven}{\ifguitarra\CsevenGuitar\fi\ifpiano\CsevenPiano\fi}
\newcommand{\CsevenBflat}{\ifguitarra\CsevenBflatGuitar\fi\ifpiano\CsevenBflatPiano\fi}
\newcommand{\Cm}{\ifguitarra\CmGuitar\fi\ifpiano\CmPiano\fi}
\newcommand{\Cmseven}{\ifguitarra\CmsevenGuitar\fi\ifpiano\CmsevenPiano\fi}
\newcommand{\CmAflat}{\ifguitarra\CmAflatGuitar\fi\ifpiano\CmAflatPiano\fi}
\newcommand{\CD}{\ifguitarra\CDGuitar\fi\ifpiano\CDPiano\fi}
\newcommand{\CE}{\ifguitarra\CEGuitar\fi\ifpiano\CEPiano\fi}
\newcommand{\CG}{\ifguitarra\CGuitar\fi\ifpiano\CGPiano\fi}
\newcommand{\CMaj}{\ifguitarra\CMajGuitar\fi\ifpiano\CMajPiano\fi}
\newcommand{\CMajtwoD}{\ifguitarra\CMajtwoDGuitar\fi\ifpiano\CMajtwoDPiano\fi}
\newcommand{\Csustwo}{\ifguitarra\CsustwoGuitar\fi\ifpiano\CsustwoPiano\fi}
\newcommand{\Ctwo}{\ifguitarra\CtwoGuitar\fi\ifpiano\CtwoPiano\fi}
\newcommand{\Cnine}{\ifguitarra\CnineGuitar\fi\ifpiano\CNinePiano\fi}
\newcommand{\Csusthirteen}{\ifguitarra\CsusthirteenGuitar\fi\ifpiano\CsusthirteenPiano\fi}

\newcommand{\Cs}{\ifguitarra\CsGuitar\fi\ifpiano\CsPiano\fi}
\newcommand{\Csm}{\ifguitarra\CsmGuitar\fi\ifpiano\CsmPiano\fi}
\newcommand{\Csmseven}{\ifguitarra\CsmsevenGuitar\fi\ifpiano\CsmsevenPiano\fi}
\newcommand{\Csseven}{\ifguitarra\CsmsevenGuitar\fi\ifpiano\CsmsevenPiano\fi}
\newcommand{\Dflat}{\ifguitarra\DflatGuitar\fi\ifpiano\DflatPiano\fi}
\newcommand{\Dflatm}{\ifguitarra\DflatmGuitar\fi\ifpiano\DflatmPiano\fi}
\newcommand{\DflatF}{\ifguitarra\DflatFGuitar\fi\ifpiano\DflatFPiano\fi}

\newcommand{\D}{\ifguitarra\DGuitar\fi\ifpiano\DPiano\fi}
\newcommand{\Dseven}{\ifguitarra\DsevenGuitar\fi\ifpiano\DsevenPiano\fi}
\newcommand{\Dm}{\ifguitarra\DmGuitar\fi\ifpiano\DmPiano\fi}
\newcommand{\Dmseven}{\ifguitarra\DmsevenGuitar\fi\ifpiano\DmsevenPiano\fi}
\newcommand{\DMaj}{\ifguitarra\DMajGuitar\fi\ifpiano\DMajPiano\fi}
\newcommand{\Dsus}{\ifguitarra\DsusGuitar\fi\ifpiano\DsusPiano\fi}
\newcommand{\Dsusnine}{\ifguitarra\DsusnineGuitar\fi\ifpiano\DsusninePiano\fi}
\newcommand{\DmB}{\ifguitarra\DmBGuitar\fi\ifpiano\DmBPiano\fi}
\newcommand{\DmC}{\ifguitarra\DmCGuitar\fi\ifpiano\DmCPiano\fi}
\newcommand{\DmF}{\ifguitarra\DmFGuitar\fi\ifpiano\DmFPiano\fi}
\newcommand{\Dtwo}{\ifguitarra\DtwoGuitar\fi\ifpiano\DtwoPiano\fi}
\newcommand{\Dsix}{\ifguitarra\DsixGuitar\fi\ifpiano\DsixPiano\fi}
\newcommand{\DmBasB}{\ifguitarra\DmBasBGuitar\fi\ifpiano\DmBasBPiano\fi}
\newcommand{\DA}{\ifguitarra\DAGuitar\fi\ifpiano\DAPiano\fi}
\newcommand{\DC}{\ifguitarra\DCGuitar\fi\ifpiano\DCPiano\fi}
\newcommand{\DE}{\ifguitarra\DEGuitar\fi\ifpiano\DEPiano\fi}
\newcommand{\DFs}{\ifguitarra\DFsGuitar\fi\ifpiano\DFsPiano\fi}
\newcommand{\DsevenFs}{\ifguitarra\DsevenFsGuitar\fi\ifpiano\DsevenFsPiano\fi}
\newcommand{\DmsevenG}{\ifguitarra\DmsevenGGuitar\fi\ifpiano\DmsevenGPiano\fi}
\newcommand{\DsusFs}{\ifguitarra\DsusFsGuitar\fi\ifpiano\DsusFsPiano\fi}

\newcommand{\Ds}{\ifguitarra\DsGuitar\fi\ifpiano\DsPiano\fi}
\newcommand{\Dsseven}{\ifguitarra\DsevenGuitar\fi\ifpiano\DsevenPiano\fi}
\newcommand{\Dsm}{\ifguitarra\DsmGuitar\fi\ifpiano\DsmPiano\fi}
\newcommand{\Dsmseven}{\ifguitarra\DsmsevenGuitar\fi\ifpiano\DsmsevenPiano\fi}
\newcommand{\Eflat}{\ifguitarra\EflatGuitar\fi\ifpiano\EflatPiano\fi}
\newcommand{\Eflattwo}{\ifguitarra\EflattwoGuitar\fi\ifpiano\EflattwoPiano\fi}
\newcommand{\EflatMaj}{\ifguitarra\EflatMajGuitar\fi\ifpiano\EflatMajPiano\fi}
\newcommand{\EflatF}{\ifguitarra\EflatFGuitar\fi\ifpiano\EflatFPiano\fi}
\newcommand{\Eflatseven}{\ifguitarra\EflatsevenGuitar\fi\ifpiano\EflatsevenPiano\fi}
\newcommand{\Eflatmseven}{\ifguitarra\EflatmsevenGuitar\fi\ifpiano\EflatmsevenPiano\fi}
\newcommand{\Eflatm}{\ifguitarra\EflatmGuitar\fi\ifpiano\EflatmPiano\fi}

\newcommand{\E}{\ifguitarra\EGuitar\fi\ifpiano\EPiano\fi}
\newcommand{\Esus}{\ifguitarra\EsusGuitar\fi\ifpiano\EsusPiano\fi}
\newcommand{\Efive}{\ifguitarra\EfiveGuitar\fi\ifpiano\EfivePiano\fi}
\newcommand{\Edimfive}{\ifguitarra\EdimfiveGuitar\fi\ifpiano\EdimfivePiano\fi}
\newcommand{\Eseven}{\ifguitarra\EsevenGuitar\fi\ifpiano\EsevenPiano\fi}
\newcommand{\EseveN}{\ifguitarra\EseveNineGuitar\fi\ifpiano\EseveNinePiano\fi}
\newcommand{\Em}{\ifguitarra\EmGuitar\fi\ifpiano\EmPiano\fi}
\newcommand{\Emseven}{\ifguitarra\EmsevenGuitar\fi\ifpiano\EmsevenPiano\fi}
\newcommand{\Emseventr}{\ifguitarra\EmseventrGuitar\fi\ifpiano\EmsevenPiano\fi}
\newcommand{\Emeleven}{\ifguitarra\EmelevenGuitar\fi\ifpiano\EmelevenPiano\fi}
\newcommand{\EmB}{\ifguitarra\EmBGuitar\fi\ifpiano\EmBPiano\fi}
\newcommand{\EmD}{\ifguitarra\EmDGuitar\fi\ifpiano\EmDPiano\fi}
\newcommand{\EmG}{\ifguitarra\EmGGuitar\fi\ifpiano\EmGPiano\fi}
\newcommand{\EsevenFour}{\ifguitarra\EsevenFourGuitar\fi\ifpiano\EsevenFourPiano\fi}
\newcommand{\EseveNNine}{\ifguitarra\EsevenNNineGuitar\fi\ifpiano\EsevenNNinePiano\fi}
\newcommand{\EB}{\ifguitarra\EBGuitar\fi\ifpiano\EBPiano\fi}
\newcommand{\ED}{\ifguitarra\EDGuitar\fi\ifpiano\EDPiano\fi}
\newcommand{\EFs}{\ifguitarra\EFsGuitar\fi\ifpiano\EFsPiano\fi}
\newcommand{\EG}{\ifguitarra\EGuitar\fi\ifpiano\EPiano\fi}
\newcommand{\EGs}{\ifguitarra\EGsGuitar\fi\ifpiano\EGsPiano\fi}

\newcommand{\F}{\ifguitarra\FGuitar\fi\ifpiano\FPiano\fi}
\newcommand{\FA}{\ifguitarra\FGuitar\fi\ifpiano\FPiano\fi}
\newcommand{\Ftwo}{\ifguitarra\FtwoGuitar\fi\ifpiano\FtwoPiano\fi}
\newcommand{\FtwoA}{\ifguitarra\FtwoAGuitar\fi\ifpiano\FtwoAPiano\fi}
\newcommand{\FMaj}{\ifguitarra\FMajGuitar\fi\ifpiano\FMajPiano\fi}
\newcommand{\Fseven}{\ifguitarra\FsevenGuitar\fi\ifpiano\FsevenPiano\fi}
\newcommand{\Fnine}{\ifguitarra\FnineGuitar\fi\ifpiano\FninePiano\fi}
\newcommand{\Fm}{\ifguitarra\FMajGuitar\fi\ifpiano\FMajPiano\fi}
\newcommand{\Fmseven}{\ifguitarra\FsevenGuitar\fi\ifpiano\FsevenPiano\fi}
\newcommand{\Fmsix}{\ifguitarra\FmsixGuitar\fi\ifpiano\FmsixPiano\fi}
\newcommand{\Fdim}{\ifguitarra\FdimGuitar\fi\ifpiano\FdimPiano\fi}

\newcommand{\Fs}{\ifguitarra\FsGuitar\fi\ifpiano\FsPiano\fi}
\newcommand{\Fsseven}{\ifguitarra\FssevenGuitar\fi\ifpiano\FssevenPiano\fi}
\newcommand{\Fssus}{\ifguitarra\FssevenGuitar\fi\ifpiano\FssevenPiano\fi}
\newcommand{\Fsm}{\ifguitarra\FsmGuitar\fi\ifpiano\FsmPiano\fi}
\newcommand{\FsmLight}{\ifguitarra\FsmLightGuitar\fi\ifpiano\FsmLightPiano\fi}
\newcommand{\FsmBasSeveN}{\ifguitarra\FsmBasSeveNineGuitar\fi\ifpiano\FsmBasSeveNinePiano\fi}
\newcommand{\FsmBasSeven}{\ifguitarra\FsmBasSevenGuitar\fi\ifpiano\FsmBasSevenPiano\fi}
\newcommand{\Fsmseven}{\ifguitarra\FsmsevenGuitar\fi\ifpiano\FsmsevenPiano\fi}
\newcommand{\Gflat}{\ifguitarra\GflatGuitar\fi\ifpiano\GflatPiano\fi}
\newcommand{\GflatBflat}{\ifguitarra\GflatBflatGuitar\fi\ifpiano\GflatBflatPiano\fi}
\newcommand{\Gflatm}{\ifguitarra\GflatmGuitar\fi\ifpiano\GflatmPiano\fi}

\newcommand{\G}{\ifguitarra\GGuitar\fi\ifpiano\GPiano\fi}
\newcommand{\Gseven}{\ifguitarra\GsevenGuitar\fi\ifpiano\GsevenPiano\fi}
\newcommand{\GMaj}{\ifguitarra\GMajGuitar\fi\ifpiano\GMajPiano\fi}
\newcommand{\Gsus}{\ifguitarra\GsusGuitar\fi\ifpiano\GsusPiano\fi}
\newcommand{\GB}{\ifguitarra\GBGuitar\fi\ifpiano\GPiano\fi}
\newcommand{\GD}{\ifguitarra\GDGuitar\fi\ifpiano\GPiano\fi}
\newcommand{\GE}{\ifguitarra\GEGuitar\fi\ifpiano\GPiano\fi}
\newcommand{\GF}{\ifguitarra\GFGuitar\fi\ifpiano\GPiano\fi}
\newcommand{\GnineD}{\ifguitarra\GnineDGuitar\fi\ifpiano\GnineDPiano\fi}
\newcommand{\Gtwo}{\ifguitarra\GtwoGuitar\fi\ifpiano\GPiano\fi}
\newcommand{\Gm}{\ifguitarra\GmGuitar\fi\ifpiano\GmPiano\fi}
\newcommand{\Gmseven}{\ifguitarra\GsevenGuitar\fi\ifpiano\GsevenPiano\fi}
\newcommand{\GmBflat}{\ifguitarra\GmBflatGuitar\fi\ifpiano\GmBflatPiano\fi}

\newcommand{\Gs}{\ifguitarra\GsGuitar\fi\ifpiano\GsPiano\fi}
\newcommand{\Aflat}{\ifguitarra\AflatGuitar\fi\ifpiano\AflatPiano\fi}
\newcommand{\Aflatseven}{\ifguitarra\AflatsevenGuitar\fi\ifpiano\AflatsevenPiano\fi}
\newcommand{\Aflatm}{\ifguitarra\AflatmGuitar\fi\ifpiano\AflatmPiano\fi}
\newcommand{\AflatEflat}{\ifguitarra\AflatEflatGuitar\fi\ifpiano\AflatEflatPiano\fi}
\newcommand{\AflatF}{\ifguitarra\AflatFGuitar\fi\ifpiano\AflatFPiano\fi}
\newcommand{\AflatC}{\ifguitarra\AflatCGuitar\fi\ifpiano\AflatCPiano\fi}
\newcommand{\AflatMaj}{\ifguitarra\AflatMajGuitar\fi\ifpiano\AflatMajPiano\fi}
\newcommand{\Gsm}{\ifguitarra\GsmGuitar\fi\ifpiano\GsmPiano\fi}
\newcommand{\Gsmseven}{\ifguitarra\GsmsevenGuitar\fi\ifpiano\GsmsevenPiano\fi}
\newcommand{\Gssus}{\ifguitarra\GssusGuitar\fi\ifpiano\GssusPiano\fi}
\newcommand{\Gsdim}{\ifguitarra\GsdimGuitar\fi\ifpiano\GsdimPiano\fi}
\documentclass[12pt, spanish]{book}

% Packages
\usepackage[T1]{fontenc} %pdflatex
%\usepackage{fontspec} %xelatex
\usepackage[utf8]{inputenc}
\usepackage{babel}
\usepackage{mypiano}
\usepackage{gchords}
\usepackage{latexsym,fancyhdr}
\usepackage{imakeidx}
\usepackage[pdfpagelabels,hyperindex,unicode=true,pdfusetitle, bookmarks=true,bookmarksnumbered=false,bookmarksopen=true,
    breaklinks=false,pdfborder={0 0 1},backref=false,colorlinks=true]{hyperref}
\usepackage[chordbk]{songbook} %% Words & Chords edition. Estribillero Musicos
\usepackage{endnotes}
%\usepackage{biblatex}
\usepackage{tikz}
%%\usepackage[compactallsongs,chordbk]{songbook}    %% Words & Chords edition.
%\usepackage[wordbk]{songbook}                 %% Words Only edition. Estribillero publicable
%\usepackage[overhead]{songbook}               %% Overhead Transparency edition. Estribillero Letras

%\newcommand\themarker[1]{\pdfbookmark[1]{#1}{\pdfmdfivesum{#1}}}


% Paso 1: Guardar el entorno original
\let\oldsong\song
\let\endoldsong\endsong


\newcommand{\strconcat}[2]{#1 - #2}

% Paso 2: Redefinir el entorno para insertar el pdfbookmark
%\renewenvironment{song}[7][YF]{%
\RenewDocumentEnvironment{song}{m m m m m m}{%
    \pdfbookmark[1]{\strconcat{#1}{#4}}{\pdfmdfivesum{\strconcat{#1}{#4}}}% usar el primer argumento como título
    \oldsong{#1}{#2}{#3}{#4}{#5}{#6}% llamar al entorno original con los mismos argumentos
    }{%
    \endoldsong
}
\renewcommand*{\UrlFont}{\ttfamily\smaller\relax}

\renewcommand{\SBChorusTag}{Coro:}
\renewcommand{\SBBridgeTag}{Puente:}
\renewcommand{\SBEndTag}{Fin:}
%\newcommand{\myTitleFont}{\Huge\myHugeSF}
\newcommand{\mySubTitleFont}{\large\sf}


%%%
% Turn on/off index-file generation.  Uncomment the \makeindex line to
% turn index generation on;  comment it out to turn index generation
% off.
%%%
\makeTitleIndex         %% Title and First Line Index.
\makeTitleContents      %% Table of Contents.
\makeKeyIndex           %% Index of song by key.
\makeArtistIndex        %% Index of song by artist.
\makeindex

\renewcommand{\notesname}{Notas}

%%%
% Revision Date and Release Date definitions.
%
%       \RelDate - The last time this songbook was released.  Set this
%                  date each time a new release/update of the songbook
%                  is generated.
%       \RevDate - The last time a particular song was revised in any
%                  way.  This command will be renewed inside every
%                  song.
%%%
\newcommand{\RelDate}{\today}
\newcommand{\RevDate}{\today}

%%%
% C.C.L.I. license number definition; for copyright licensing info.
% One of these macros will be manually inserted into the {CpyRt}
% parameter of the {song} environment.
%
%       \CCLInumber - The actual copyright license number.  Don't
%               insert this command in the {CpyRt} parameter, use one
%               of the others.
%       \CCLIed - Indicates a song falls under our CCLI license.
%       \NotCCLIed - Indicates a song doesn't fall under our CCLI
%               license.  Public Domain songs fall into this category.
%       \PGranted - We have received specific permission from the
%               copyright holder to use this song.
%       \PPending - We are in the process of obtaining permission to
%               use this song.
%%%
\newcommand{\CCLInumber}{Your CCLI Number}
\newcommand{\CCLIed}{{\CpyRtInfoFont (CCLI \CCLInumber)}}
\newcommand{\NotCCLIed}{\relax}
\newcommand{\PGranted}{\relax}
\newcommand{\PPending}{{\CpyRtInfoFont (Permission Pending)}}

%%%
% Title page information.
%%%
\title{Cuaderno de Cantos Infantiles}
\author{Ruslan L\'opez}
\date{\'Ultima Revisi\'on:  \RevDate}

%%%
% Redefine fonts from SongBook style that I don't like.
%%%
\font\myTinySF=cmss8 at 8pt
\renewcommand{\CpyRtInfoFont}{\tiny\myTinySF}

%%%
% Define fonts to use in the headers and footers of the songbook.
%%%
\newcommand{\LHeadFont}{\normalsize}            % = cmr12  at 12pt
\newcommand{\CHeadFont}{\normalsize\rm}         % = cmr12  at 12pt
\newcommand{\RHeadFont}{\normalsize}            % = cmr12  at 12pt
\newcommand{\LFootFont}{\scriptsize}            % = cmr8   at  8pt
\newcommand{\CFootFont}{\tiny\myTinySF}         % = cmss8  at  8pt
\newcommand{\RFootFont}{\scriptsize}            % = cmr8   at  8pt

%%%
% Turn on and define fancy page heading/footing definition.
%%%
\pagestyle{fancy}

\ifChordBk
% It's a words & chords songbook...
\headsep=7mm
\oddsidemargin=1in
\evensidemargin=1.2in
\footskip=0.2in
\addtolength{\headwidth}{\marginparsep}
\addtolength{\headwidth}{\marginparwidth}
\renewcommand{\headrulewidth}{0.4pt}
\renewcommand{\footrulewidth}{0.4pt}
\fancyhead[LE,RO]{\LHeadFont\emph{\leftmark\/}\SBContinueMark}
\fancyhead[CE,CO]{\CHeadFont\thepage}
\fancyhead[RE,LO]{\RHeadFont\RelDate}
\else\ifOverhead
% It's an overhead...
\renewcommand{\footrulewidth}{0pt}
\renewcommand{\headrulewidth}{0pt}
\fancyhead[LE,RO]{}
\fancyhead[CE,CO]{}
\fancyhead[RE,LO]{}
\else\ifWordBk
% It's a words only songbook...
\addtolength{\headwidth}{\marginparsep}
\addtolength{\headwidth}{\marginparwidth}
\renewcommand{\headrulewidth}{0.4pt}
\renewcommand{\footrulewidth}{0.4pt}
\fancyhead[LE,RO]{\LHeadFont Estribillero}
\fancyhead[CE,CO]{\CHeadFont\thepage}
\fancyhead[RE,LO]{\RHeadFont\RelDate}
\fi\fi\fi

\fancyfoot[LE,RO]{\LFootFont Transcripciones}
\ifSongEject
\fancyfoot[CE,CO]{\CFootFont \RevDate}
\else
\fancyfoot[CE,CO]{\CFootFont}
\fi
\fancyfoot[RE,LO]{\RFootFont Todo el material son transcripciones personales.}


% Document
\begin{document}
    \hypersetup{pageanchor=false}

%%%
% Custom titlepage for Christian church songbook
%%%
    \begin{titlepage}
        \thispdfpagelabel{Portada}
        \centering

        \vspace*{1cm}

        {\Huge\bfseries\sffamily{$\dagger$}\par}

        \vspace{1.5cm}

        {\Huge\bfseries\sffamily Cuaderno de\par}
        \vspace{0.5cm}
        {\Huge\bfseries\sffamily Cantos Infantiles\par}

        \vspace{2cm}

        \begin{center}
            \Large\itshape ``Cantad alegres a Dios, habitantes de toda la tierra.\\
            Servid a Jehov\'a con alegr\'ia;\\
            Venid ante su presencia con regocijo.''\\
            \vspace{0.3cm}
            --- Salmos 100:1-2
        \end{center}

        \vspace{2cm}

        {\Large\bfseries Compilado por:\par}
        \vspace{0.5cm}
        {\Large Ruslan L\'opez\par}

        \vfill

        {\large \'Ultima Revisi\'on: \RevDate\par}

        \vspace{1cm}

    \end{titlepage}

    \hypersetup{pageanchor=true}
    \pdfbookmark[0]{Piezas musicales}{piezas}
%    \mainmatter
    \ifWordBk
    \twocolumn
    \fi
%%%
% Songbook begins.
%%%

    \begin{song}{Una tortuguita}{C}
    {\SBPubDom} %copyright \SBPubDom
    {Dalberto Gomez Perez}
    {} %pasaje
    {} %\NotCCLIed
        \ifChordBk
        \paragraph{\mbox{}\hfill\protect\href{https://open.spotify.com/intl-es/track/0XgjeE9k0MrDisfz7PPtPf}{Escuchar}\mbox{}\hfill}
        \fi

        \begin{SBOpGroup}
            \Ch{C}{U}na tortuguita saca la cabeza

            es\Ch{G}{ti}ra sus manitas se le quita la pe\Ch{C}{re}za
        \end{SBOpGroup}

        \begin{SBOpGroup}
            \Ch{C}{Di}ce el perezoso, me duele la cabeza

            me \Ch{G}{due}le la cintura tengo ganas de \Ch{C}{dor}mir\Ch{C7}{}
        \end{SBOpGroup}

        \begin{SBChorus}
            \Ch{F}{Es} un buen ejemplo, \Ch{C}{pa}ra los cristianos

            \Ch{G}{que} de mala gana le \Ch{C}{sir}ven al Señor.
        \end{SBChorus}

        \begin{SBOpGroup}
            \Ch{C}{Eso} pasará cuando Cristo venga

            \Ch{G}{los} que estén dormidos aquí se queda\Ch{C}{rán}
        \end{SBOpGroup}

        \ifChordBk
        \begin{SBOpGroup}
            Acordes:
            \upchord{\C}{\qquad Do} Mayor  \hfill
            \upchord{\G}{\qquad Sol} Mayor \hfill\null\break
            \upchord{\Csmseven}{Do Sostenido} Menor S\'eptima \hfill
            \upchord{\F}{\qquad Fa} Mayor \hfill\null\break
        \end{SBOpGroup}
        \fi
    \end{song}

    \begin{song}{Zaqueo}{G}
    {} %copyright \SBPubDom
    {Manuel Bonilla}
    {} %pasaje
    {} %\NotCCLIed
        \ifChordBk
        \paragraph{\mbox{}\hfill\protect\href{https://open.spotify.com/intl-es/track/08vuoQqGMlE0HLEe1GG8ui}{Escuchar}\mbox{}\hfill}
        \fi

%  \SBRef{No puedo parar de alabarte}{2006}%fuente \#

        \begin{SBOpGroup}
Zaqueo era un chaparrito así


Que vivía en Jericó


Y cuando Jesús pasó por allí


A un sicomoro subió
        \end{SBOpGroup}

        \begin{SBChorus}
El Salvador le vio allí


Y le hablo así


Zaqueo, bájate de allí


Porque a tu casa voy a ir


A tu casa voy a ir
        \end{SBChorus}

        \begin{SBOpGroup}
Si tú también chiquitito estás


Y a Cristo quieres ver


Tú puedes hoy venir a Él


No hay nada que temer
        \end{SBOpGroup}

        \begin{SBOpGroup}
Decídete a invitarle hoy


Y dile al Salvador


Cristo, pasa por aquí


Porque contigo quiero ir


Que contigo quiero ir
        \end{SBOpGroup}

        \ifChordBk
        \begin{SBOpGroup}
            Acordes:
            \upchord{\Em}{\qquad Mi} Menor \hfill
            \upchord{\Bseven}{\qquad Si} S\'eptima \hfill\null\break
            \upchord{\C}{\qquad Do} Mayor \hfill
            \upchord{\D}{Re} Mayor \hfill\null\break
            \upchord{\G}{\qquad Sol} Mayor \hfill
            \upchord{\Am}{\qquad La} Menor \hfill\null\break
        \end{SBOpGroup}
        \fi
    \end{song}

    \ifChordBk
        \pdfbookmark[0]{Pads~Para~Yamaha}{pads}
\section*{Pads Para Yamaha E-473}

Estos pads son compatibles con varios teclados de Yamaha, por ejemplo el E-463

Para guardar una configuraci\'on en el registration memory.

Apretamos la tecla bank hasta que aparezca en pantalla el n\'umero de banco en que lo queramos guardar, normalmente hay 8.

Si hay que guardar m\'as de uno, tenemos que tener cuidado porque por defecto guarda el \'ultimo que hemos guardado.
Observe que en registration tiene alg\'un candadito, debemos desactivarla para poder guardar los cambios.

Buscamos la funci\'on VoiceFrz y lo modificamos con la rueda a off para poder guardar los sonidos.
apretar la tecla voice para configurar la voz principal M.Voice.
para cambiar los par\'ametros oprima la tecla function, busque la configuraci\'on a modificar, oprima enter, modifique el valor mediante la rueda que se ubica a la derecha
y oprima la tecla enter para guardar los cambios.
Finalmente oprima al mismo tiempo la tecla bank y una de las 4 teclas numeradas que est\'an a su derecha. Deber\'a aparecer la leyenda Memory OK en el visor.
Acabando de modificar y guardar las voces debe volver a ponerlo en On.

Para cargar la voz basta con oprimir la t�cla del n\'umero a la derecha de la tecla bank para que esta se cargue.

\textbf{Pad adoraci\'on 1}
\vskip 25pt

Touch res: Medium
M. Voice: 002 GrandPno
M. Volume: 115
M. Octave: 0
M.Pan: R30
M. Reverb: 100
M. Chorus: 64
M. Attack: 64
M. Release: 64
M. Cutoff: 64
M. Reso: 64
D. Voice: 661 WarmPad
D. Volume: 100
D. Octave: 0
D. Pan: L30
D. Reverb: 100
D. Chorus: 0
D. Attack: 64
D. Release: 115
D. Cutoff: 64
D. Reso: 64
\vskip 25pt

\textbf{Brass Coritos}
\vskip 25pt

Touch res: Medium
M. Voice: 002 GandPno
M. Volume: 115
M. Octave: 0
M. Pan: R10
M. Reverb: 28
M. Chorus: 0
M. Attack: 64
M. Release: 64
M. Cutoff: 64
M. Reso: 64
D. Voice: 133 OctBrass
D. Volume: 100
D. Octave: 1
D. Pan: L10
D. Reverb: 80
D. Chorus: 0
D. Attack: 64
D. Release: 72
D. Cutoff: 75
D. Reso: 64
\vskip 25pt

\textbf{Acorde\'on coritos}
\vskip 25pt

Touch res: Medium
M. Voice: 370 El.Grand
M. Volume: 90
M. Octave: 0
M. Pan: C
M. Reverb: 22
M. Chorus: 0
M. Attack: 64
M. Release: 64
M. Cutoff: 64
M. Reso: 64
D. Voice: 56 ReedOrgn
D. Volume: 64
D. Octave: 1
D. Pan: C
D. Reverb: 51
D. Chorus: 38
D. Attack: 61
D. Release: 29
D. Cutoff: 64
D. Reso: 64
\vskip 25pt

\textbf{Dyno piano}
\vskip 25pt

Touch res: Medium
M. Voice: 389 DXPhase
M. Volume: 100
M. Octave: 0
M. Pan: C
M. Reverb: 50
M. Chorus: 127
M. Attack: 64
M. Release: 64
M. Cutoff: 64
M. Reso: 64
D. Voice: 388 DXLegend
D. Volume: 90
D. Octave: 0
D. Pan: C
D. Reverb: 40
D. Chorus: 127
D. Attack: 64
D. Release: 64
D. Cutoff: 64
D. Reso: 104
\vskip 25pt

\textbf{Modificaciones simples a instrumentos}

\textbf{Saxof\'on}
\vskip 25pt

Touch res: Medium
M. Voice: 159 S!Soprn
M. Volume: 96
M. Octave: 0
M. Pan: C
M. Reverb: 90
M. Chorus: 0
M. Attack: 64
M. Release: 64
M. Cutoff: 64
M. Reso: 100

\textbf{Synthetizador electr\'onica}
\vskip 25pt

Touch res: Medium
M. Voice: 185 HandsUp!
M. Volume: 100
M. Octave: 0
M. Pan: C
M. Reverb: 100
M. Chorus: 127
M. Attack: 64
M. Release: 105
M. Cutoff: 64
M. Reso: 64

\textbf{Piano bolero sudam}
\vskip 25pt

Touch res: Medium
M. Voice: 241 Bright
M. Volume: 100
M. Octave: -1
M. Pan: C
M. Reverb: 100
M. Chorus: 64
M. Attack: 64
M. Release: 64
M. Cutoff: 64
M. Reso: 64

        \theendnotes
        %\begin{document}
\newcommand{\myTitleFont}{\Huge\myHugeSF}
\ifguitarra
\pdfbookmark[0]{Acordes~para~Guitarra}{acordesguitarra}
\lhead{\LHeadFont Acordes~para~Guitarra}
\chead{\CHeadFont ({\rm\thepage})}
\rhead{\RHeadFont\RelDate}
{\parindent 8pt
        {\myTitleFont --- Acodes para Guitarra ---}}\par
\vskip 20pt
\textbf{Acodes Mayores}

%\small{El s\'imbolo \# significa sostenido y {\flat}~significa~bemol}
\small
\upchord{\AGuitar}{La Mayor} \upchord{\BGuitar}{Si Mayor} \upchord{\CGuitar}{Do Mayor} \upchord{\DGuitar}{Re Mayor} \upchord{\EGuitar}{Mi Mayor} \upchord{\FGuitar}{Fa Mayor} \upchord{\GGuitar}{Sol Mayor}

\upchord{\AsGuitar}{A\#/$B\flat$ Mayor} \upchord{\CsGuitar}{C\#/$D\flat$ Mayor} \upchord{\DsGuitar}{D\#/$E\flat$ Mayor}  \upchord{\FsGuitar}{F\#/$G\flat$ Mayor} \upchord{\GsGuitar}{G\#/$A\flat$ Mayor} \upchord{\AsGuitar}{A\#/$B\flat$ Mayor}
\normalsize

\vskip 20pt
\textbf{Acodes Menores}

Estos acordes tienen las siguientes notaciones:
A-, Amin, Am, Aminor\break
\vskip 20pt

\small
\upchord{\AmGuitar}{La} Menor \upchord{\BmGuitar}{Si} Menor \upchord{\CmGuitar}{Do} Menor \upchord{\DmGuitar}{Re} Menor \upchord{\EmGuitar}{Mi} Menor \upchord{\FmGuitar}{Fa} Menor \upchord{\GmGuitar}{Sol} Menor

\upchord{\AsmGuitar}{\small{A\#/$B\flat$ Menor}} \upchord{\CsmGuitar}{\small{C\#/$D\flat$ Menor}} \upchord{\DsmGuitar}{\small{D\#/$E\flat$ Menor}}  \upchord{\FsmGuitar}{\small{F\#/$G\flat$ Menor}} \upchord{\GsmGuitar}{\small{G\#/$A\flat$ Menor}} \upchord{\AsmGuitar}{\small{A\#/$B\flat$ Menor}}
\normalsize

\vskip 20pt
\textbf{Acodes Mayores S\'eptima}

\upchord{\AsevenGuitar}{La} Mayor s\'eptima
\upchord{\BflatsevenGuitar}{Si} bemol Mayor s\'eptima
\upchord{\BsevenGuitar}{Si} Mayor s\'eptima
\upchord{\CsevenGuitar}{\small{Do Mayor s\'eptima}}
\vskip 20pt
\upchord{\CssevenGuitar}{\small{Do sostenidoMayor s\'eptima}}
\upchord{\DsevenGuitar}{\small{Re Mayor s\'eptima}}
\upchord{\EflatsevenGuitar}{\small{Mi bemol Mayor s\'eptima}}
\upchord{\EsevenGuitar}{\small{Mi Mayor s\'eptima}}
\upchord{\FsevenGuitar}{\small{Fa Mayor s\'eptima}}
\upchord{\FssevenGuitar}{\small{Fa sostenido Mayor s\'eptima}}
\vskip 20pt
\upchord{\GsevenGuitar}{\small{Sol Mayor s\'eptima}}
\upchord{\AflatsevenGuitar}{\qquad La bemol S\'eptima}

\textbf{Acodes Menores S\'eptima}

\small
\upchord{\AmsevenGuitar}{La} Menor s\'eptima
\upchord{\BmsevenGuitar}{Si} Menor s\'eptima
\upchord{\CmsevenGuitar}{Do} Menor s\'eptima
\upchord{\CsmsevenGuitar}{Do} Menor s\'eptima
\upchord{\DmsevenGuitar}{Re} Menor s\'eptima
\vskip 20pt
\upchord{\DsmsevenGuitar}{Re} Sostenido Menor s\'eptima
\upchord{\EmsevenGuitar}{Mi} Menor s\'eptima
\upchord{\EmseventrGuitar}{Mi} Menor s\'eptima
\upchord{\FmsevenGuitar}{Fa} Menor s\'eptima
\vskip 20pt
\upchord{\FsmsevenGuitar}{Fa sostenido} Menor s\'eptima
\upchord{\GmsevenGuitar}{Sol} Menor s\'eptima
\upchord{\GsmsevenGuitar}{Sol} Sostenido Menor s\'eptima
\upchord{\BflatmsevenGuitar}{Si bemol} Menor s\'eptima
\normalsize

\vskip 20pt
\textbf{Acodes Mayores Suspendido cuarta}
\vskip 25pt

\small
\upchord{\AsusGuitar}{La} Suspendida cuarta
\upchord{\BsusGuitar}{Si} Suspendida cuarta
\upchord{\CsusGuitar}{Do} Suspendida cuarta
\upchord{\DsusGuitar}{Re} Suspendida cuarta
\upchord{\EsusGuitar}{Mi} Suspendida cuarta
\upchord{\FsusGuitar}{Fa} Suspendida cuarta
\upchord{\GsusGuitar}{Sol} Suspendida cuarta

\upchord{\FssusGuitar}{Fa} sostenido Suspendida cuarta
\upchord{\GssusGuitar}{Sol sostenido} Suspendida cuarta
\normalsize

\vskip 20pt
\textbf{Acodes Mayor S\'eptima Aumentada}
\vskip 25pt

Estos acordes tienen las siguientes notaciones:
Amaj7, A+7, AM7, $A^{+7}$, $A^{M7}$, $A\Delta7$, $A^{\Delta7}$\break
\vskip 20pt

\small
\upchord{\AMajGuitar}{La} Maj
\upchord{\CMajGuitar}{Do} Maj
\upchord{\DMajGuitar}{Re} Maj
\upchord{\FMajGuitar}{Fa} Maj
\upchord{\GMajGuitar}{Sol} Maj
\upchord{\AflatMajGuitar}{Ab} Maj
\upchord{\AsMajGuitar}{A\#} Maj
\upchord{\EflatMajGuitar}{Eb} Maj
\normalsize

\vskip 20pt
\textbf{Acodes Suspendida 2}
\vskip 25pt

\small
\upchord{\AsustwoGuitar}{La} Suspendida 2
\upchord{\CsustwoGuitar}{Do} Suspendida 2
\normalsize

\vskip 20pt
\textbf{Acodes Suspendida 9}
\vskip 25pt

\small
\upchord{\DsusnineGuitar}{Re} Suspendida novena
\normalsize

\vskip 20pt
\textbf{Acodes Aumentada 2}
\vskip 25pt

\small
\upchord{\AtwoGuitar}{La} Aumentada 2
\upchord{\CtwoGuitar}{Do} Aumentada 2
\upchord{\DtwoGuitar}{Re} Aumentada 2
\upchord{\EflattwoGuitar}{Eb} Aumentada 2
\upchord{\FtwoGuitar}{Fa} Aumentada 2
\upchord{\GtwoGuitar}{Sol} Aumentada 2
\normalsize

\vskip 20pt
\textbf{Acodes Novena}
\vskip 25pt

\small
\upchord{\AnineGuitar}{La} Novena
\upchord{\CnineGuitar}{Do} Novena
\upchord{\FnineGuitar}{Fa} Novena
\upchord{\GnineGuitar}{Sol} Novena
\normalsize

\vskip 20pt
\textbf{Acodes trecena}
\vskip 25pt

\small
\upchord{\AsthirteenGuitar}{A\#} 13
\normalsize

\vskip 20pt
\textbf{Acodes Menor Onceava}
\vskip 25pt

\small
\upchord{\EmelevenGuitar}{Mi} Menor 11
\normalsize

\vskip 20pt
\textbf{Acodes Disminuidos}
\vskip 25pt

\small
\upchord{\GsdimGuitar}{Sol} sostenido disminuido
\normalsize


\vskip 20pt
\textbf{Acodes Con Bajo cambiado}

\small
\upchord{\AAsGuitar}{La Mayor bajo Bb} \hfill
\upchord{\ACsGuitar}{La Mayor bajo C\#} \hfill
\upchord{\ADGuitar}{La Mayor bajo D} \hfill
\upchord{\AEGuitar}{La Mayor bajo E} \hfill\null\break
\vskip 20pt
\upchord{\AGGuitar}{La Mayor bajo G} \hfill
\upchord{\AsevenCsGuitar}{La} S\'eptima bajo C\# \hfill
\upchord{\AmCGuitar}{La Menor bajo C} \hfill
\upchord{\AmCsGuitar}{La Menor bajo C\#} \hfill\null\break
\vskip 20pt
\upchord{\AmFGuitar}{La Menor bajo F} \hfill
\upchord{\AmFsGuitar}{La Menor bajo F\#} \hfill
\upchord{\AmGGuitar}{La Menor bajo G} \hfill
\upchord{\AmGsGuitar}{La Menor bajo G\#} \hfill\null\break
\vskip 20pt
\upchord{\AmsevenFGuitar}{La Menor S\'eptima bajo F} \hfill
\upchord{\AflatFGuitar}{La bemol Mayor bajo F} \hfill
\upchord{\BDsGuitar}{Si Mayor bajo D\#} \hfill\null\break
\vskip 20pt
\upchord{\BflatFGuitar}{Si bemol} Mayor bajo F \hfill
\upchord{\BsevenDsGuitar}{Si S\'eptima bajo D\#} \hfill
\upchord{\BsevenFsGuitar}{Si S\'eptima bajo F\#} \hfill
\upchord{\CEGuitar}{Do} Mayor bajo E \hfill\null\break
\vskip 20pt
\upchord{\CGGuitar}{Do} Mayor bajo G \hfill
\upchord{\CmAflatGuitar}{Do} Menor bajo Ab \hfill
\upchord{\DflatFGuitar}{Re bemol bajo F} \hfill
\upchord{\DAGuitar}{Re Mayor bajo A} \hfill\null\break
\vskip 20pt
\upchord{\DCGuitar}{Re Mayor bajo C} \hfill
\upchord{\DEGuitar}{Re Mayor bajo E} \hfill
\upchord{\DmCGuitar}{Re Menor bajo C} \hfill
\upchord{\DmFGuitar}{Re Menor bajo F} \hfill\null\break
\vskip 20pt
\upchord{\DmsevenGGuitar}{Re Menor S\'eptima bajo G} \hfill
\upchord{\DFsGuitar}{Re Mayor bajo F\#} \hfill
\upchord{\DsevenFsGuitar}{Re S\'eptima bajo F\#} \hfill
\upchord{\EflatFGuitar}{Mi bemol bajo F} \hfill\null\break
\vskip 20pt
\upchord{\EDGuitar}{Mi Mayor Bajo D} \hfill
\upchord{\EFsGuitar}{Mi Mayor Bajo F\#} \hfill
\upchord{\EGGuitar}{Mi Mayor Bajo G} \hfill
\upchord{\EGsGuitar}{Mi Mayor Bajo G\#} \hfill\null\break
\vskip 20pt
\upchord{\EmDGuitar}{Mi Menor Bajo D} \hfill
\upchord{\EmGGuitar}{Mi Menor Bajo G} \hfill
\upchord{\FAGuitar}{Fa Mayor Bajo A} \hfill
\upchord{\GflatBflatGuitar}{Sol bemol} Bajo Si bemol \hfill\null\break
\vskip 20pt
\upchord{\GBGuitar}{Sol Mayor Bajo B} \hfill
\upchord{\GDGuitar}{Sol Mayor Bajo D} \hfill
\upchord{\GEGuitar}{Sol Mayor Bajo E} \hfill
\upchord{\GFGuitar}{Sol Mayor Bajo F} \hfill\null\break
\vskip 20pt
\upchord{\GnineDGuitar}{Sol} Mayor Novena Bajo D \hfill
\upchord{\GmBflatGuitar}{Sol Menor Bajo Bb} \hfill
\upchord{\AflatCGuitar}{La bemol Bajo C} \hfill\null\break
\vskip 20pt
\upchord{\AflatEflatGuitar}{La bemol Bajo Eb} \hfill
\upchord{\BmajsevenGGuitar}{Si} + bajo G \hfill\null\break
\vskip 20pt
\upchord{\BflatCGuitar}{Si bemol} bajo C \hfill
\upchord{\BflatDGuitar}{Si bemol} bajo D \hfill
\upchord{\BflatGGuitar}{Si bemol} bajo G \hfill\null\break
\vskip 20pt
\upchord{\DsusFsGuitar}{Re} Suspendida cuarta bajo F\# \hfill
\upchord{\CMajtwoDGuitar}{\qquad Do} maj7 aumentada 2 bajo D \hfill
\upchord{\FtwoAGuitar}{Fa}+2 Mayor bajo A \hfill\null\break
\vskip 20pt
\upchord{\BdimDGuitar}{Si} disminuido bajo D \hfill\null\break
\normalsize

\vskip 20pt
\textbf{Acodes mayor disminuida quinta}

\small
\upchord{\EdimfiveGuitar}{Mi} dim5
\normalsize

\vskip 20pt
\textbf{Acodes semidisminuidos}

\small
\upchord{\BmsevenbfiveGuitar}{Si} Menor s\'eptima semidisminuido
\upchord{\EdimfiveGuitar}{Mi} Menor s\'eptima semidisminuido
\normalsize

\vskip 20pt
\textbf{Acodes diada arm\'onica}

\small
\upchord{\EfiveGuitar}{Mi} 5
\normalsize

\vskip 20pt
\textbf{Acodes 13 suspendida cuarta}

\small
\upchord{\CsusthirteenGuitar}{Do} 13 suspendida cuarta
\normalsize

\clearpage
\fi

\ifpiano
\pdfbookmark[0]{Acordes~para~Piano}{acordespiano}
\lhead{\LHeadFont Acordes~para~Piano}
{\parindent 8pt
        {\myTitleFont --- Acordes para Piano ---}}\par
\vskip 20pt
\textbf{Acodes Mayores}
\vskip 25pt

%\small{El s\'imbolo \# significa sostenido y {\flat}~significa~bemol}
\small
\upchord{\APiano}{\qquad La Mayor} \qquad\qquad \upchord{\BPiano}{Si Mayor} \qquad\qquad \upchord{\CPiano}{\qquad Do Mayor} \qquad\qquad \upchord{\DPiano}{\qquad Re Mayor} \hfill \break
\vskip 25pt
\upchord{\EPiano}{\qquad Mi Mayor} \qquad\qquad  \upchord{\FPiano}{\qquad Fa Mayor} \qquad\qquad \upchord{\GPiano}{\qquad Sol Mayor}
\vskip 25pt
\upchord{\AsPiano}{A\#/$B\flat$ Mayor} \qquad\qquad \upchord{\CsPiano}{C\#/$D\flat$ Mayor} \qquad\qquad \upchord{\DsPiano}{D\#/$E\flat$ Mayor} \qquad\qquad \upchord{\FsPiano}{F\#/$G\flat$ Mayor} \hfill \break
\vskip 25pt
\upchord{\GsPiano}{G\#/$A\flat$ Mayor} \qquad\qquad \upchord{\AsPiano}{A\#/$B\flat$ Mayor}
\normalsize

\textbf{Acodes Menores}
\vskip 25pt

\small
\upchord{\AmPiano}{\qquad La} Menor \qquad\qquad \upchord{\BmPiano}{\qquad Si} Menor \qquad\qquad \upchord{\CmPiano}{\qquad Do} Menor \qquad\qquad \upchord{\DmPiano}{\qquad Re} Menor \hfill \break
\vskip 25pt
\upchord{\EmPiano}{\qquad Mi} Menor \qquad\qquad \upchord{\FmPiano}{\qquad Fa} Menor \qquad\qquad \upchord{\GmPiano}{\qquad Sol} Menor
\vskip 25pt
\upchord{\AsmPiano}{\small{A\#/$B\flat$ Menor}}  \qquad\qquad  \upchord{\CsmPiano}{C\#/$D\flat$ Menor}  \qquad\qquad  \upchord{\DsmPiano}{D\#/$E\flat$ Menor} \hfill \break
\vskip 25pt
\upchord{\FsmPiano}{F\#/$G\flat$ Menor} \qquad\qquad \upchord{\GsmPiano}{G\#/$A\flat$ Menor}  \qquad\qquad  \upchord{\AsmPiano}{A\#/$B\flat$ Menor}
\normalsize

\clearpage
%\vskip 20pt
\textbf{Acodes Mayores S\'eptima}
\vskip 25pt

\small
\upchord{\AsevenPiano}{La Mayor s\'eptima} \hfill
\upchord{\BsevenPiano}{Si Mayor s\'eptima} \hfill
\upchord{\CsevenPiano}{Do Mayor s\'eptima} \hfill\null\break
\vskip 25pt
\upchord{\DsevenPiano}{Re Mayor s\'eptima} \hfill
\upchord{\EsevenPiano}{Mi Mayor s\'eptima} \hfill
\upchord{\FsevenPiano}{Fa Mayor s\'eptima} \hfill\null\break
\vskip 25pt
\upchord{\GsevenPiano}{Sol Mayor s\'eptima} \hfill
\upchord{\BflatsevenPiano}{Si} bemol Mayor s\'eptima \hfill\null\break
\vskip 25pt
\upchord{\EflatsevenPiano}{Mi bemol Mayor s\'eptima} \hfill
\upchord{\FssevenPiano}{Fa sostenido Mayor s\'eptima} \hfill
\upchord{\AflatsevenPiano}{Sol sostenido Mayor s\'eptima} \hfill\null\break
\normalsize
\vskip 20pt

\textbf{Acodes Menores S\'eptima}
\vskip 25pt

\small
\upchord{\AmsevenPiano}{La} Menor s\'eptima \hfill
\upchord{\BmsevenPiano}{Si} Menor s\'eptima \hfill
\upchord{\CmsevenPiano}{Do} Menor s\'eptima \hfill\null\break
\vskip 25pt
\upchord{\DmsevenPiano}{Re} Menor s\'eptima \hfill
\upchord{\EmsevenPiano}{Mi} Menor s\'eptima \hfill
\upchord{\FmsevenPiano}{Fa} Menor s\'eptima \hfill\null\break
\vskip 25pt
\upchord{\GmsevenPiano}{Sol} Menor s\'eptima \hfill
\upchord{\BflatmsevenPiano}{Si bemol} Menor s\'eptima \hfill
\upchord{\FsmsevenPiano}{Fa sostenido} Menor s\'eptima \hfill\null\break
\vskip 25pt
\upchord{\CssevenPiano}{Do sostenido} Mayor s\'eptima \hfill
\upchord{\DsmsevenPiano}{Re} Sostenido Menor s\'eptima \hfill
\upchord{\GsmsevenPiano}{Sol} Sostenido Menor s\'eptima \hfill\null\break
\normalsize

\vskip 20pt

\textbf{Acodes Mayores Suspendido cuarta}
\vskip 25pt

\small
\upchord{\AsusPiano}{La} Suspendida cuarta \hfill
\upchord{\BsusPiano}{Si} Suspendida cuarta \hfill
\upchord{\CsusPiano}{Do} Suspendida cuarta \hfill\null\break
\vskip 25pt
\upchord{\DsusPiano}{Re} Suspendida cuarta \hfill
\upchord{\EsusPiano}{Mi} Suspendida cuarta \hfill
\upchord{\FsusPiano}{Fa} Suspendida cuarta \hfill\null\break
\vskip 25pt
\upchord{\FssusPiano}{Fa sostenido} Suspendida cuarta \hfill
\upchord{\GsusPiano}{Sol} Suspendida cuarta \hfill
\upchord{\GssusPiano}{Sol sostenido} Suspendida cuarta \hfill\null\break
\normalsize

\vskip 20pt
\textbf{Acodes Mayor S\'eptima Aumentada}
\vskip 25pt

Estos acordes tienen las siguientes notaciones:
Amaj7, A+7, AM7, $A^{+7}$, $A^{M7}$, $A\Delta7$, $A^{\Delta7}$\break
\vskip 20pt

\small
\upchord{\AMajPiano}{La} Maj \hfill
\upchord{\CMajPiano}{Do} Maj \hfill
\upchord{\DMajPiano}{Re} Maj \hfill\null\break
\vskip 20pt
\upchord{\FMajPiano}{Fa} Maj \hfill
\upchord{\GMajPiano}{Sol} Maj \hfill
\upchord{\AsMajPiano}{La\#/Sib} Maj \hfill\null\break
\vskip 20pt
\upchord{\EflatMajPiano}{Re\#/Mib} Maj \hfill\null\break
\normalsize

\vskip 20pt
\textbf{Acodes Suspendida 2}
\vskip 25pt

\small
\upchord{\AsustwoPiano}{La} Suspendida 2
\upchord{\CsustwoPiano}{Do} Suspendida 2
\normalsize

\vskip 20pt
\textbf{Acodes Suspendida 9}
\vskip 25pt

\small
\upchord{\DsusninePiano}{Re} Suspendida novena
\normalsize

\vskip 20pt
\textbf{Acodes Aumentada 2}
\vskip 25pt

\small
\upchord{\AtwoPiano}{La} Aumentada 2 \hfill
\upchord{\CtwoPiano}{Do} Aumentada 2 \hfill
\upchord{\DtwoPiano}{Re} Aumentada 2 \hfill\null\break
\vskip 20pt
\upchord{\EflattwoPiano}{Re b} Aumentada 2 \hfill
\upchord{\FtwoPiano}{Fa} Aumentada 2 \hfill
\upchord{\GtwoPiano}{Sol} Aumentada 2 \hfill\null\break
\normalsize


\vskip 20pt
\textbf{Acodes Novena}
\vskip 25pt

\small
\upchord{\AninePiano}{La} Novena \hfill
\upchord{\CninePiano}{Do} Novena \hfill
\upchord{\FninePiano}{Fa} Novena \hfill\null\break
\vskip 20pt
\upchord{\GninePiano}{Sol} Novena \hfill
\normalsize

\vskip 20pt
\textbf{Acodes trecena}
\vskip 25pt

\small
\upchord{\AsthirteenPiano}{A\#} 13
\normalsize

\vskip 20pt
\textbf{Acodes Menor Onceava}
\vskip 25pt

\small
\upchord{\EmelevenPiano}{Mi} Menor 11
\normalsize
\clearpage

%\vskip 20pt
\textbf{Acodes Disminuidos}
\vskip 25pt

\small
\upchord{\GsdimPiano}{Sol} sostenido disminuido
\normalsize


\vskip 20pt
\textbf{Acodes Con Bajo cambiado}
\vskip 25pt

En el caso del piano, es com\'un que la parte del bajo se toque octavado en la mano izquierda y en la derecha el acorde normal, aqu\'i se pone la transposici\'on para ayudar a encontrar la melod\'ia.
\vskip 25pt
\small
\upchord{\AAsPiano}{La Mayor bajo Bb} \hfill
\upchord{\ACsPiano}{La Mayor bajo C\#} \hfill\null\break
\vskip 20pt
\upchord{\ADPiano}{La Mayor bajo D} \hfill
\upchord{\AEPiano}{La} Mayor bajo E \hfill\null\break
\vskip 20pt
\upchord{\AGPiano}{La Mayor bajo G} \hfill
\upchord{\AsevenCsPiano}{La} Mayor S\'eptima bajo C\# \hfill\null\break
\vskip 20pt
\upchord{\AmCPiano}{La Menor bajo C} \hfill
\upchord{\AmCsPiano}{La Menor bajo C\#} \hfill\null\break
\vskip 20pt
\upchord{\AmFPiano}{La Menor bajo F} \hfill
\upchord{\AmFsPiano}{La Menor bajo F\#} \hfill
\upchord{\AmGPiano}{La Menor bajo G} \hfill\null\break
\vskip 20pt
\upchord{\AmGsPiano}{La Menor bajo G\#} \hfill
\upchord{\AmsevenFPiano}{La Menor S\'eptima bajo F} \hfill
\upchord{\AflatFPiano}{La bemol Mayor bajo F} \hfill\null\break
\vskip 20pt
\upchord{\BDsPiano}{Si Mayor bajo D\#} \hfill
\upchord{\BflatFPiano}{Si bemol} Mayor bajo F \hfill
\upchord{\BsevenDsPiano}{Si S\'eptima bajo D$\#$} \hfill\null\break
\vskip 20pt
\upchord{\BsevenFsPiano}{Si S\'eptima bajo F$\#$} \hfill\null\break
\vskip 20pt
\upchord{\CEPiano}{Do Mayor bajo E} \hfill
\upchord{\CmAflatPiano}{Do Menor bajo Ab} \hfill\null\break
\vskip 20pt
\upchord{\CGPiano}{Do Mayor bajo G} \hfill
\upchord{\DAPiano}{Re Mayor bajo A} \hfill
\upchord{\DCPiano}{Re Mayor bajo C} \hfill\null\break
\vskip 20pt
\upchord{\DmCPiano}{Re Menor bajo C} \hfill
\upchord{\DmFPiano}{Re Menor bajo F} \hfill
\upchord{\DmsevenGPiano}{Re Menor S\'eptima bajo G} \hfill\null\break
\vskip 20pt
\upchord{\DFsPiano}{Re Mayor bajo F\#} \hfill
\upchord{\DsevenFsPiano}{Re S\'eptima bajo F\#} \hfill
\upchord{\EflatFPiano}{Mi bemol bajo F} \hfill\null\break
\vskip 20pt
\upchord{\EDPiano}{\qquad Mi} Mayor con bajo D \hfill
\upchord{\EFsPiano}{\qquad Mi} Mayor con bajo F\# \hfill
\upchord{\EGPiano}{\qquad Mi} Mayor con bajo G \hfill\null\break
\vskip 20pt
\upchord{\EGsPiano}{\qquad Mi} Mayor con bajo G\# \hfill
\upchord{\EmDPiano}{\qquad Mi} Menor Bajo D \hfill
\upchord{\EmGPiano}{\qquad Mi} Menor Bajo G \hfill\null\break
\vskip 20pt
\upchord{\FAPiano}{Fa} Mayor bajo A \hfill
\upchord{\GflatBflatPiano}{Sol bemol Mayor bajo Si Bemol} \hfill
\upchord{\GDPiano}{Sol Mayor Bajo D} \hfill\null\break
\vskip 20pt
\upchord{\GEPiano}{Sol Mayor Bajo E} \hfill %long
\upchord{\GFPiano}{Sol Mayor Bajo F} \hfill\null\break
\vskip 20pt
\upchord{\GBPiano}{\qquad\qquad Sol}  Mayor Bajo B \hfill
\upchord{\GnineDPiano}{Sol} Mayor Novena Bajo D \hfill\null\break %long
\vskip 20pt
\upchord{\GmBflatPiano}{Sol Menor Bajo Bb} \hfill
\upchord{\AflatCPiano}{La bemol Bajo C} \hfill
\upchord{\AflatEflatPiano}{La bemol Bajo Eb} \hfill\null\break
\vskip 20pt
\upchord{\BflatCPiano}{Si bemol} bajo C \hfill
\upchord{\BflatDPiano}{Si bemol} bajo D \hfill\null\break
\vskip 20pt
\upchord{\BflatGPiano}{Si bemol} bajo G \hfill
\upchord{\BmajsevenGPiano}{Si} + bajo G \hfill
\upchord{\DsusFsPiano}{Re Suspendida} cuarta bajo F\# \hfill\null\break
\vskip 20pt
\upchord{\CMajtwoDPiano}{\qquad Do}+7+2 bajo D \hfill
\upchord{\FtwoAPiano}{Fa}+2 Mayor bajo A \hfill\null\break
\vskip 20pt
\upchord{\BdimDPiano}{Si} disminuido bajo D \hfill\null\break
\vskip 20pt
\normalsize

\clearpage
%\vskip 20pt
\textbf{Acodes mayor disminuida quinta}
\vskip 25pt

\small
\upchord{\EdimfivePiano}{Mi} dim5
\normalsize

\vskip 20pt
\textbf{Acodes medio disminuido s\'eptima}
\vskip 25pt

\small
\upchord{\BmsevenbfivePiano}{Si} medio disminuido s\'eptima
\normalsize

\vskip 20pt
\textbf{Acodes diada arm\'onica}
\vskip 25pt

\small
\upchord{\EfivePiano}{Mi} 5
\normalsize

\vskip 20pt
\textbf{Acodes 13 suspendida cuarta}
\vskip 25pt

\small
\upchord{\CsusthirteenPiano}{Do} 13 suspendida cuarta
\normalsize

\clearpage
\fi
%\end{document}
%\bye
        \include{infantilesAdx}
        %%%%%% rcsid = @(#)$Id: sampleKdx.tex,v 1.16 2010-04-12 18:04:30 rathc Exp $
%%%%%%
%%
%%      ================================
%%      Sample Key Index (sampleKdx.tex)
%%      ================================
%%
%%      Version 4.5, 30 April, 2010
%%
%%      Copyright 1992--2010 Christopher Rath <christopher@rath.ca>
%%
%%	This package is free software; you can redistribute it and/or
%%	modify it under the terms of version 2.1 of the GNU Lesser 
%%	General Public License as published by the Free Software
%%	Foundation.
%%
%%	This package is distributed in the hope that it will be
%%	useful, but WITHOUT ANY WARRANTY; without even the implied
%%	warranty of MERCHANTABILITY or FITNESS FOR A PARTICULAR
%%	PURPOSE.  See the GNU Lesser General Public License for more
%%	details.
%%
%%      This file is provided as a template for Song Key
%%      Index generation.
%%
%%%%%%
%%%%%%

%%%%%%%%%%%%%%%%%%%%%%%%%%%%%%%%%%%%%%%%%%%%%%%%%%%%%%%%%%
%%%%%%%%%%%%%%%%%%%%%%%%%%%%%%%%%%%%%%%%%%%%%%%%%%%%%%%%%%
%%                                                      %%
%%           P R E A M B L E   B E G I N S              %%
%%                                                      %%
%%%%%%%%%%%%%%%%%%%%%%%%%%%%%%%%%%%%%%%%%%%%%%%%%%%%%%%%%%
%%%%%%%%%%%%%%%%%%%%%%%%%%%%%%%%%%%%%%%%%%%%%%%%%%%%%%%%%%

%\documentclass[12pt,twocolumn]{book}
%\usepackage{latexsym,fancyhdr}
%\usepackage[wordbk]{songbook}

%\usepackage{tikz}

%%%
% Revision Date and Release Date definitions.
%
%       \RelDate - The last time this songbook was released.
%       \RevDate - The last time this file was revised in any way.
%%%
%\newcommand{\RelDate}{30~May'96}
%\newcommand{\RevDate}{\RelDate}

%%%
% Redefine fonts from SongBook style that I don't like, and define
% any extra fonts I require.
%%%
\font\myTinySF=cmss8    at  8pt
\font\myHugeSF=cmssbx10 at 25pt
\renewcommand{\CpyRtInfoFont}{\tiny\myTinySF}
%\newcommand{\myTitleFont}{\Huge\myHugeSF}
%\newcommand{\mySubTitleFont}{\large\sf}

%%%
% Define fonts to use in the headers and footers of the songbook.
%%%
%\newcommand{\LHeadFont}{\normalsize}            % = cmr12  at 12pt
%\newcommand{\CHeadFont}{\normalsize\rm}         % = cmr12  at 12pt
%\newcommand{\RHeadFont}{\normalsize}            % = cmr12  at 12pt
%\newcommand{\LFootFont}{\scriptsize}            % = cmr8   at  8pt
%\newcommand{\CFootFont}{\tiny\myTinySF}         % = cmss8  at  8pt
%\newcommand{\RFootFont}{\scriptsize}            % = cmr8   at  8pt

%%%
% Turn on and define fancy page heading/footing definition.
%%%
\pagestyle{fancy}
\pdfbookmark[0]{\'Indice~Tonal}{tonal}
%\addtolength{\headwidth}{\marginparsep}
%\addtolength{\headwidth}{\marginparwidth}
%\renewcommand{\footrulewidth}{0.4pt}
\lhead{\LHeadFont \'Indice~Tonal}
       \chead{\CHeadFont ({\rm\thepage})}
       \rhead{\RHeadFont\RelDate}

%\lfoot{\LFootFont Property of a Church}
%       \cfoot{\CFootFont Last Revised:  \RevDate}
%       \rfoot{\RFootFont Material used by permission.}


%%%
% Index entries command definition.
%%%
\renewcommand{\item}{\par\hangindent=40pt}
\renewcommand{\subitem}{\par\hangindent=40pt \hspace*{20pt}}
\renewcommand{\subsubitem}{\par\hangindent=40pt \hspace*{30pt}}


%%%%%%%%%%%%%%%%%%%%%%%%%%%%%%%%%%%%%%%%%%%%%%%%%%%%%%%%%%
%%%%%%%%%%%%%%%%%%%%%%%%%%%%%%%%%%%%%%%%%%%%%%%%%%%%%%%%%%
%%                                                      %%
%%           D O C U M E N T   B E G I N S              %%
%%                                                      %%
%%%%%%%%%%%%%%%%%%%%%%%%%%%%%%%%%%%%%%%%%%%%%%%%%%%%%%%%%%
%%%%%%%%%%%%%%%%%%%%%%%%%%%%%%%%%%%%%%%%%%%%%%%%%%%%%%%%%%
%\begin{document}

%%%
% Index begins.
%%%
{\parindent 8pt
  {\myTitleFont --- INDICE TONAL ---}}\par
\vskip 20pt

Con base en el c\'irculo de quintas, el modo mayor y el modo menor usar\'an las mismas notas si est\'an alineadas en la l\'{\i}nea dorada que va del centro hacia afuera.\hfill\null\break
Para facilitar al cantante la interpretaci\'on, se pueden elegir canciones en la misma escala o en su defecto hacia la derecha o izquierda ligeramente y as\'{\i} evitar que se desafine.

\begin{figure}[h]
    \centering
    \includegraphics[width=0.5\textwidth]{circulo_de_quintas}
    \caption{C\'irculo de quintas.\label{fig:circulo_de_quintas}}
\end{figure}

Para saber que nota lleva la alteraci\'on, si son bemoles, la secuencia va siguiendo la l\'{\i}nea azul, si son sostenidos, la secuencia va siguiendo la l\'{\i}nea roja.\hfill\null\break
%Para realizar la modulación, podemos usar un acorde pivote, que es cuando se comparten acordes entre las escalas.
%Un círculo armónico, consiste en una progresión de acordes formada por los grados I, VI, II, V, con la particularidad de que el VI y el II son menores, y el V es séptima.
%Para ello usaremos la secuencia ii-V-I, es decir los últimos dos acordes del círculo armónico, y pasaremos al primer acorde del círculo.

%! Author = Ruslan López
%! Date = 19/08/2025

% Preamble
% genera acordes de guitarra
\newif\ifguitarra

%genera acordes de piano
\newif\ifpiano

\guitarratrue
\pianotrue

% comandos para pintar acordes de guitarra
\newcommand{\AGuitar}{\chord{t}{x,o,f1p2,f2p2,f3p2,n}{A}}
\newcommand{\AsusGuitar}{\chord{t}{x,o,f1p2,f2p2,f3p3,n}{Asus4}}
\newcommand{\AsevenGuitar}{\chord{t}{x,o,f1p2,n,f3p2,n}{A7}}
\newcommand{\AMajGuitar}{\chord{t}{x,o,f2p2,f1p1,f3p2,n}{A+7}}
\newcommand{\ABGuitar}{\chord{t}{x,bf1p2,f2p2,f3p2,f4p2,n}{A/B}}
\newcommand{\ACsGuitar}{\chord{t}{x,bf4p3,f2p2,f3p2,f1p2,n}{A/C$\#$}}
\newcommand{\ADGuitar}{\chord{t}{x,x,o,f2p2,f3p2,n}{A/D}}
\newcommand{\AEGuitar}{\chord{t}{o,n,f1p2,f2p2,f3p2,n}{A/E}}
\newcommand{\AGGuitar}{\chord{t}{bf3p3,n,f1p2,n,f2p2,n}{A/G}}
\newcommand{\AAsGuitar}{\chord{t}{x,bf1p1,f2p2,f3p2,f4p2,n}{A/Bb}}
\newcommand{\AsevenCsGuitar}{\chord{t}{x,bf4p3,f1p2,n,f2p2,n}{A7/C\#}}
\newcommand{\AtwoGuitar}{\chord{t}{x,o,f1p2,f3p4,f2p2,n}{A2}}
\newcommand{\AsustwoGuitar}{\chord{t}{x,o,f1p2,f2p2,n,n}{Asus2}}
\newcommand{\AnineGuitar}{\chord{t}{x,o,f1p2,f3p4,f1p2,f2p3}{A9}}

\newcommand{\AmGuitar}{\chord{t}{x,o,f2p2,f3p2,f1p1,n}{Am}}
\newcommand{\AmsevenGuitar}{\chord{t}{x,n,f2p2,n,f1p1,n}{Am7}}
\newcommand{\AmCGuitar}{\chord{t}{x,bf4p3,f3p2,f2p2,f1p1,x}{Am/C}}
\newcommand{\AmCsGuitar}{\chord{t}{x,bf4p4,f3p2,f2p2,f1p1,x}{Am/C$\#$}}
\newcommand{\AmEGuitar}{\chord{t}{o,n,f2p2,f3p2,f1p1,n}{Am/E}}
\newcommand{\AmFGuitar}{\chord{t}{bf1p1,n,f3p2,f4p2,f2p1,x}{Am/F}}
\newcommand{\AmFsGuitar}{\chord{t}{bf2p2,n,f3p2,f4p2,f1p1,x}{Am/F$\#$}}
\newcommand{\AmGGuitar}{\chord{t}{p3,n,p2,p2,p1,x}{Am/G}}
\newcommand{\AmGsGuitar}{\chord{t}{p4,n,p2,p2,p1,x}{Am/G$\#$}}
\newcommand{\AmsevenFGuitar}{\chord{t}{p1,n,p2,n,p1,x}{Am7/F}}

\newcommand{\AsGuitar}{\chord{t}{x,bf1p1,f2p3,f3p3,f4p3,f1p1}{A$\#$}}
\newcommand{\AsMajGuitar}{\chord{t}{p1,p1,p3,p2,p3,p1}{A$\#$+7}}
\newcommand{\AsmGuitar}{\chord{t}{x,bf1p1,f3p3,f4p3,f2p2,f1p1}{A$\#$m}}
\newcommand{\AsthirteenGuitar}{\chord{t}{x,bf1p1,f1p1,f1p1,f4p3,f4p3}{A$\#$13}}
\newcommand{\BflatGuitar}{\chord{t}{x,bf1p1,f2p3,f3p3,f4p3,f1p1}{Bb}}
\newcommand{\BflatsevenGuitar}{\chord{t}{x,bf1p1,f3p3,f1p1,f4p3,f1p1}{Bb7}}
\newcommand{\BflatmGuitar}{\chord{t}{x,bf1p1,f3p3,f4p3,f2p2,f1p1}{Bbm}}
\newcommand{\BflatmsevenGuitar}{\chord{t}{x,bf1p1,f3p3,f1p1,f2p2,f1p1}{Bbm7}}
\newcommand{\BflatMajGuitar}{\chord{t}{p1,p1,p3,p2,p3,p1}{Bb+7}}
\newcommand{\BflatCGuitar}{\chord{t}{x,p3,n,p3,p1,p1}{Bb/C}}
\newcommand{\BflatDGuitar}{\chord{t}{p1,p1,p3,p3,p3,p1}{Bb/D}}
\newcommand{\BflatFGuitar}{\chord{t}{bf1p1,f1p1,f2p3,f3p3,f4p3,f1p1}{Bb/F}}
\newcommand{\BflatGGuitar}{\chord{t}{bf1p3,f2p1,n,n,f3p3,f4p3}{Bb/G}}

\newcommand{\BGuitar}{\chord{t}{x,bf1p2,f2p4,f3p4,f4p4,f1p2}{B}}
\newcommand{\BsusGuitar}{\chord{t}{x,bf1p2,f2p4,f3p4,f4p5,f1p2}{Bsus4}}
\newcommand{\BsevenGuitar}{\chord{t}{x,f1p2,f3p4,f1p2,f4p4,f1p2}{B7}}
\newcommand{\BsevenFsGuitar}{\chord{t}{f2p2,n,f1p1,f3p2,n,f4p2}{B7/F$\#$}}
\newcommand{\BsevenDsGuitar}{\chord{t}{x,x,p1,p2,n,p2}{B7/D$\#$}}
\newcommand{\BDsGuitar}{\chord{2}{x,bf4p4,f1p2,f2p2,n,x}{B/D$\#$}}
\newcommand{\BEflatGuitar}{\chord{2}{x,bf4p4,f1p2,f2p2,n,x}{B/Eb}}
\newcommand{\BmGuitar}{\chord{t}{x,bf1p2,f3p4,f4p4,f2p3,f1p2}{Bm}}
\newcommand{\BmsevenGuitar}{\chord{t}{x,bf1p2,f3p4,f1p2,f2p3,f1p2}{Bm7}}
\newcommand{\BmseveNGuitar}{\chord{t}{x,p2,p4,p3,p3,p2}{Bm7+}}
\newcommand{\BmsevenAGuitar}{\chord{t}{x,n,p4,p4,p3,n}{Bm7/A}}
\newcommand{\BmsevenbfiveGuitar}{\chord{t}{x,bf1p2,f3p3,f2p2,f4p3,n}{Bm7(b5)}} %si semidisminuido
\newcommand{\BmajsevenGGuitar}{\chord{t}{p3,x,p1,p3,n,p2}{B+7/G}}
\newcommand{\BdimDGuitar}{\chord{t}{x,x,o,p4,n,p1}{Bdim/D}}

\newcommand{\CGuitar}{\chord{t}{x,bf3p3,n,f2p2,f1p1,n}{C}}
\newcommand{\CsusGuitar}{\chord{t}{x,bf3p3,f4p3,n,f1p1,f2p1}{Csus4}}
\newcommand{\CsevenGuitar}{\chord{t}{x,bf3p3,f2p2,f4p3,f1p1,n}{C7}}
\newcommand{\CsevenBflatGuitar}{\chord{t}{x,bf1p1,f3p2,n,f2p1,n}{C7/Bb}}
\newcommand{\CmGuitar}{\chord{t}{x,bf3p3,f1p1,n,f2p1,f4p3}{Cm}}
\newcommand{\CmsevenGuitar}{\chord{2}{x,bf1p1,f3p3,f1p1,f2p2,f1p1}{Cm7}}
\newcommand{\CmAflatGuitar}{\chord{3}{x,x,p4,p3,p2,p1}{Cm/Ab}}
\newcommand{\CDGuitar}{\chord{t}{n,n,o,n,f1p1,n}{C/D}}
\newcommand{\CEGuitar}{\chord{t}{o,bf3p3,n,f2p2,f1p1,n}{C/E}}
\newcommand{\CGGuitar}{\chord{t}{bf3p3,n,f2p2,n,f1p1,n}{C/G}}
\newcommand{\CMajGuitar}{\chord{t}{f2p3,bf3p3,f1p2,n,n,n}{C+7}}
\newcommand{\CMajtwoDGuitar}{\chord{t}{x,x,o,p4,p1,n}{C+7+2/D}}
\newcommand{\CsustwoGuitar}{\chord{t}{x,bf2p3,n,n,f1p1,f4p3}{Csus2}}
\newcommand{\CtwoGuitar}{\chord{t}{x,bf2p3,f1p2,n,f3p3,n}{C2}}
\newcommand{\CnineGuitar}{\chord{t}{x,bf2p3,f1p2,f3p3,f4p3,n}{C9}}
\newcommand{\CsusthirteenGuitar}{\chord{t}{x,bf1p3,f1p3,f1p3,f1p3,f4p5}{C13sus4}}

\newcommand{\CsGuitar}{\chord{4}{n,o,f2p2,f3p2,f4p2,n}{C$\#$}}
\newcommand{\CsmGuitar}{\chord{2}{x,bf1p2,f3p4,f4p4,f2p3,f1p2}{C$\#$m}}
\newcommand{\CsmsevenGuitar}{\chord{t}{x,bf1p4,f2p2,f3p1,o,o}{C$\#$m7}}
\newcommand{\CssevenGuitar}{\chord{t}{x,p4,p3,p4,p2,x}{C$\#$7}}

\newcommand{\DflatGuitar}{\chord{4}{n,o,f2p2,f3p2,f4p2,n}{Db}}
\newcommand{\DflatmGuitar}{\chord{2}{x,bf1p2,f3p4,f4p4,f2p3,f1p2}{Dbm}}
\newcommand{\DflatFGuitar}{\chord{t}{x,x,p3,p5,p4,p4}{Db/F}}

\newcommand{\DGuitar}{\chord{t}{x,x,o,f1p2,f3p3,f2p2}{D}}
\newcommand{\DsevenGuitar}{\chord{t}{x,x,o,f2p2,f1p1,f3p2}{D7}}
\newcommand{\DmGuitar}{\chord{t}{x,x,o,f2p2,f3p3,f1p1}{Dm}}
\newcommand{\DmsevenGuitar}{\chord{t}{n,n,o,f2p2,f1p1,f1p1}{Dm7}}
\newcommand{\DMajGuitar}{\chord{t}{x,x,o,f1p2,f1p2,f1p2}{D+7}}
\newcommand{\DsusGuitar}{\chord{t}{x,x,o,f1p2,f3p3,f4p3}{Dsus4}}
\newcommand{\DsusnineGuitar}{\chord{t}{x,x,o,f2p2,f3p3,n}{Dsus9}}
\newcommand{\DmBGuitar}{\chord{t}{x,bf2p2,n,f3p2,f4p3,f1p1}{Dm/B}}
\newcommand{\DmCGuitar}{\chord{t}{x,bf3p3,n,f2p2,f4p3,f1p1}{Dm/C}}
\newcommand{\DmFGuitar}{\chord{t}{bf1p1,n,n,f2p2,f3p3,x}{Dm/F}}
\newcommand{\DtwoGuitar}{\chord{t}{n,bf4p5,f1p2,f1p2,f2p3,f1p2}{D2}}
\newcommand{\DsixGuitar}{\chord{t}{x,x,n,p2,n,p2}{D6}}
\newcommand{\DmBasBGuitar}{\chord{t}{x,p2,p3,p2,p3,x}{Dm/B}}
\newcommand{\DAGuitar}{\chord{t}{x,o,n,f1p2,f3p3,f2p2}{D/A}}
\newcommand{\DCGuitar}{\chord{t}{x,p3,n,p2,p3,p2}{D/C}}
\newcommand{\DEGuitar}{\chord{t}{n,n,bf1p2,f1p2,f2p3,f1p2}{D/E}}
\newcommand{\DFsGuitar}{\chord{t}{bf1p2,n,n,f2p2,f4p3,f3p2}{D/F$\#$}}
\newcommand{\DsevenFsGuitar}{\chord{t}{bf2p2,n,n,f3p2,f1p1,f4p2}{D7/F$\#$}}
\newcommand{\DmsevenGGuitar}{\chord{t}{p3,n,n,p2,p1,p1}{Dm7/G}}
\newcommand{\DsusFsGuitar}{\chord{t}{bf1p2,x,n,f2p2,f3p3,f4p3}{Dsus/F\#}}

\newcommand{\DsGuitar}{\chord{t}{n,n,bf1p1,f2p3,f4p4,f3p3}{D$\#$}}
\newcommand{\DssevenGuitar}{\chord{t}{n,n,bf1p1,f3p3,f1p4,f4p3}{D\#7}}
\newcommand{\DsmGuitar}{\chord{t}{x,x,bf1p1,f3p3,f4p4,f1p2}{D\#m}}
\newcommand{\DsmsevenGuitar}{\chord{t}{x,x,bf1p1,f4p3,f2p2,f3p2}{D\#m7}}
\newcommand{\EflatGuitar}{\chord{t}{n,n,bf1p1,f2p3,f4p4,f3p3}{E$\flat$}}
\newcommand{\EflattwoGuitar}{\chord{3}{x,bf3p4,f1p1,f1p1,f2p2,f1p1}{E$\flat$2}}
\newcommand{\EflatMajGuitar}{\chord{t}{n,n,bf1p1,f2p3,f3p3,f4p3}{E$\flat$+7}}
\newcommand{\EflatFGuitar}{\chord{t}{bf1p1,f1p1,f1p1,f2p3,f4p4,f3p3}{E$\flat$/F}}
\newcommand{\EflatsevenGuitar}{\chord{t}{n,n,bf1p1,f3p3,f1p4,f4p3}{7}}
\newcommand{\EflatmGuitar}{\chord{t}{x,x,bf1p1,f3p3,f4p4,f1p2}{E$\flat$m}}
\newcommand{\EflatmsevenGuitar}{\chord{t}{x,x,bf1p1,f4p3,f2p2,f3p2}{E$\flat$m7}}

\newcommand{\EGuitar}{\chord{t}{o,f2p2,f3p2,f1p1,n,n}{E}}
\newcommand{\EsusGuitar}{\chord{t}{o,f2p2,f3p2,f4p2,n,n}{Esus4}}
\newcommand{\EfiveGuitar}{\chord{t}{o,f1p2,f2p2,f4p4,n,n}{E5}}
\newcommand{\EdimfiveGuitar}{\chord{t}{o,f1p1,f2p2,f3p3,n,n}{E(b5)}}
\newcommand{\EsevenGuitar}{\chord{t}{o,f2p2,f3p2,f1p1,f4p3,n}{E7}}
\newcommand{\EseveNGuitar}{\chord{t}{o,p2,p2,p4,p3,p4}{E7}}
\newcommand{\EmGuitar}{\chord{t}{o,f1p2,f2p2,n,n,n}{Em}}
\newcommand{\EmsevenGuitar}{\chord{t}{o,f1p2,f2p2,n,f4p3,n}{Em7}}
\newcommand{\EmseventrGuitar}{\chord{t}{o,f1p2,f2p2,n,f3p3,f4p3}{Em7}}
\newcommand{\EmelevenGuitar}{\chord{t}{b,n,n,n,n,n}{Em11}}
\newcommand{\EmBGuitar}{\chord{t}{x,bf1p2,f2p2,n,n,n}{Em/B}}
\newcommand{\EmDGuitar}{\chord{t}{x,x,o,n,n,n}{Em/D}}
\newcommand{\EmGGuitar}{\chord{t}{bf3p3,f1p2,f2p2,n,n,n}{Em/G}}
\newcommand{\EsevenFourGuitar}{\chord{t}{n,p2,p2,p4,p3,p5}{E7,11}}
\newcommand{\EseveNNineGuitar}{\chord{t}{n,f1p2,f1p2,p4,p3,f1p2,}{E79}}
\newcommand{\EBGuitar}{\chord{t}{x,bf2p2,f3p2,f1p1,n,n}{E/B}}
\newcommand{\EDGuitar}{\chord{t}{x,x,o,f1p1,n,n}{E/D}}
\newcommand{\EFsGuitar}{\chord{t}{bf2p2,n,f3p2,f1p1,n,n}{E/F\#}}
\newcommand{\EGGuitar}{\chord{t}{bf3p3,n,f2p2,f1p1,n,n}{E/G}}
\newcommand{\EGsGuitar}{\chord{t}{bf4p4,n,f2p2,f1p1,n,n}{E/G\#}}

\newcommand{\FGuitar}{\chord{t}{bf1p1,f3p3,f4p3,f2p2,f1p1,f1p1}{F}}
\newcommand{\FsusGuitar}{\chord{t}{bf1p1,f3p3,f4p3,f2p3,f1p1,f1p1}{Fsus4}}
\newcommand{\FAGuitar}{\chord{t}{x,o,f3p3,f2p2,f1p1,f1p1}{F/A}}
\newcommand{\FtwoGuitar}{\chord{t}{x,x,bf3p3,f2p2,f1p1,f4p3}{F2}}
\newcommand{\FtwoAGuitar}{\chord{t}{x,o,f3p3,n,f1p1,f1p1}{F2/A}}
\newcommand{\FMajGuitar}{\chord{t}{x,x,bf3p3,f2p2,f1p1,n}{FMaj7}}
\newcommand{\FsevenGuitar}{\chord{t}{bf1p1,f3p3,f1p1,f2p2,f1p1,f1p1}{F7}}
\newcommand{\FnineGuitar}{\chord{t}{bf1p1,n,f2p1,n,f3p1,f4p3}{F9}}
\newcommand{\FmGuitar}{\chord{t}{bf1p1,f3p3,f4p3,f1p1,f1p1,f1p1}{Fm}}
\newcommand{\FmsevenGuitar}{\chord{t}{bf1p1,f3p3,f1p1,f1p1,f1p1,f1p1}{Fm7}}
\newcommand{\FmsixGuitar}{\chord{t}{bf1p1,f2p3,f3p3,f1p1,f4p3,f1p1}{Fm6}}
\newcommand{\FdimGuitar}{\chord{t}{x,x,f3p3,f1p1,n,f2p1}{Fdim}}

\newcommand{\FsGuitar}{\chord{t}{bf1p2,f3p4,f4p4,f2p3,f1p2,f1p2}{F\#}}
\newcommand{\FssevenGuitar}{\chord{t}{bf1p2,f3p4,f1p2,f2p3,f1p2,f1p2}{F\#7}}
\newcommand{\FssusGuitar}{\chord{t}{bf1p2,f1p2,f3p4,f4p4,f1p2,f1p2}{F\#sus}}
\newcommand{\FsmGuitar}{\chord{t}{f1p2,f3p4,f4p4,f1p2,f1p2,f1p2,}{F\#m}}
\newcommand{\FsmLightGuitar}{\chord{t}{x,x,f3p4,f1p2,f1p2,f1p2,}{F\#m}}
\newcommand{\FsmBasSeveNGuitar}{\chord{t}{x,x,f3p3,f1p2,f1p2,f1p2,}{F\#m/E\#}}
\newcommand{\FsmBasSevenGuitar}{\chord{t}{x,x,f2p2,f1p2,f1p2,f1p2,}{F\#m/E}}
\newcommand{\FsmsevenGuitar}{\chord{t}{f1p2,f2p4,f3p4,f1p2,f4p5,f1p2,}{F\#m7}}
\newcommand{\GflatGuitar}{\chord{t}{bf1p2,f3p4,f4p4,f2p3,f1p2,f1p2}{Gb}}
\newcommand{\GflatBflatGuitar}{\chord{t}{x,bp4,p4,p3,f1p2,f1p2}{Gb/Bb}}
\newcommand{\GflatmGuitar}{\chord{t}{f1p2,f3p4,f4p4,f1p2,f1p2,f1p2,}{Gbm}}

\newcommand{\GGuitar}{\chord{t}{bf2p3,f1p2,n,n,n,f3p3}{G}}
\newcommand{\GsevenGuitar}{\chord{t}{bf3p3,f2p2,n,n,n,f1p1}{G7}}
\newcommand{\GMajGuitar}{\chord{t}{bf3p3,f1p2,n,n,n,f2p2}{G+7}}
\newcommand{\GsusGuitar}{\chord{t}{bf3p3,f2p2,n,n,f1p1,f4p3}{Gsus4}}
\newcommand{\GBGuitar}{\chord{t}{n,bf1p2,n,n,f3p3,n}{G/B}}
\newcommand{\GDGuitar}{\chord{t}{x,x,x,o,f3p3,f4p3}{G/D}}
\newcommand{\GEGuitar}{\chord{t}{o,f1p2,f2p2,n,f3p3,f4p3}{G/E}}
\newcommand{\GFGuitar}{\chord{t}{bf1p1,f2p2,n,n,f3p3,f4p3}{G/F}}
\newcommand{\GnineGuitar}{\chord{t}{bf2p3,n,n,n,n,f1p1}{G9}}
\newcommand{\GnineDGuitar}{\chord{t}{n,n,o,n,n,f4p1}{G9/D}}
\newcommand{\GtwoGuitar}{\chord{t}{bf1p3,n,n,n,n,f4p3}{G2}}
\newcommand{\GmGuitar}{\chord{t}{f1p3,f3p5,f4p5,f1p3,f1p3,f1p3,}{Gm}}
\newcommand{\GmsevenGuitar}{\chord{t}{f1p3,f3p5,f1p3,f1p3,f1p3,f1p3,}{Gm7}}
\newcommand{\GmBflatGuitar}{\chord{t}{x,bf1p1,n,n,f3p3,f4p3}{Gm/Bb}}

\newcommand{\GsGuitar}{\chord{3}{x,x,f3p4,f2p3,f1p2,f1p2}{G\#}}
\newcommand{\AflatGuitar}{\chord{3}{x,x,f3p4,f2p3,f1p2,f1p2}{A$\flat$}}
\newcommand{\AflatsevenGuitar}{\chord{4}{bf1p1,f3p3,f1p1,f2p2,f1p1,f1p1}{A$\flat$7}}
\newcommand{\AflatmGuitar}{\chord{t}{p4,p2,p1,p1,n,x}{A$\flat$m}}
\newcommand{\AflatEflatGuitar}{\chord{3}{x,bf3p4,f4p4,f2p3,f1p2,f1p2}{A$\flat$/E$\flat$}}
\newcommand{\AflatFGuitar}{\chord{t}{bf1p1,f3p3,f1p1,f1p1,f1p1,f1p1}{A$\flat$/F}}
\newcommand{\AflatCGuitar}{\chord{t}{x,bf2p3,f1p1,f1p1,f1p1,f4p4}{A$\flat$/C}}
\newcommand{\AflatMajGuitar}{\chord{7}{p4,p3,x,o,p4,p4}{A$\flat$Maj7}}
\newcommand{\GsmGuitar}{\chord{t}{p4,p2,p1,p1,n,x}{G\#m}}
\newcommand{\GsmsevenGuitar}{\chord{t}{f2p4,x,f4p4,f4p4,f4p4,f4p4}{G\#m7}}
\newcommand{\GssusGuitar}{\chord{4}{bf1p1,f2p3,f3p3,f4p3,f1p1,f1p1}{G\#sus4}}
\newcommand{\GsdimGuitar}{\chord{3}{p1,p2,p3,p1,p1,p1}{G\#dim}}

% comandos para pintar acordes de piano
\newcommand{\APiano}{\keyboardf[Ao][Cso][Eo]}
\newcommand{\AsevenPiano}{\keyboardf[Ao][Cso][Eo][Go]}
\newcommand{\AMajPiano}{\keyboardf[Ao][Cso][Eo][Gso]}
\newcommand{\AmPiano}{\keyboardf[Ao][Co][Eo]}
\newcommand{\AmsevenPiano}{\keyboardf[Ao][Co][Eo][Go]}
\newcommand{\AmCPiano}{\keyboard[Co][Eo][Ao]}
\newcommand{\AmCsPiano}{\keyboardtwooctaves[Cso][Ao][Eo][Ct]}
\newcommand{\AmFPiano}{\keyboardf[Fo][Ao][Co][Eo]}
\newcommand{\AmFsPiano}{\keyboardf[Fso][Ao][Co][Eo]}
\newcommand{\AmGPiano}{\keyboardf[Go][Ao][Co][Eo]}
\newcommand{\AmGsPiano}{\keyboardf[Gso][Ao][Co][Eo]}
\newcommand{\AsevenCsPiano}{\keyboard[Cso][Eo][Ao][Go]}
\newcommand{\AmsevenFPiano}{\keyboardf[Fo][Go][Ao][Co][Eo]}
\newcommand{\AAsPiano}{\keyboard[Aso][Cso][Eo][Ao]}
\newcommand{\ABPiano}{\keyboardtwooctaves[Bo][Cst][Et][At]}
\newcommand{\ACsPiano}{\keyboard[Cso][Eo][Ao]}
\newcommand{\ADPiano}{\keyboardtwooctaves[Do][Eo][Ao][Cst]}
\newcommand{\AEPiano}{\keyboardtwooctaves[Eo][Ao][Cst]}
\newcommand{\AGPiano}{\keyboardf[Go][Ao][Cso][Eo]}
\newcommand{\AsusPiano}{\keyboardf[Ao][Do][Eo]}
\newcommand{\AtwoPiano}{\keyboardf[Ao][Bo][Cso][Eo]}
\newcommand{\AsustwoPiano}{\keyboardf[Ao][Bo][Eo]}
\newcommand{\AninePiano}{\keyboardf[Ao][Bo][Cso][Eo][Go]}

\newcommand{\AsPiano}{\keyboard[Do][Fo][Aso]}
\newcommand{\AsMajPiano}{\keyboard[Do][Fo][Ao][Aso]}
\newcommand{\AsmPiano}{\keyboard[Cso][Fo][Aso]}
\newcommand{\AsthirteenPiano}{\keyboard[Aso][Do][Fo][Gso][Co][Dso]}
\newcommand{\BflatPiano}{\keyboard[Do][Fo][Aso]}
\newcommand{\BflatsevenPiano}{\keyboard[Do][Fo][Aso][Gso]}
\newcommand{\BflatmPiano}{\keyboard[Cso][Fo][Aso]}
\newcommand{\BflatmsevenPiano}{\keyboard[Cso][Fo][Aso][Gso]}
\newcommand{\BflatMajPiano}{\keyboard[Do][Fo][Ao][Aso]}
\newcommand{\BflatCPiano}{\keyboard[Co][Do][Fo][Aso]}
\newcommand{\BflatDPiano}{\keyboard[Do][Fo][Aso]}
\newcommand{\BflatFPiano}{\keyboardf[Fo][Aso][Do]}
\newcommand{\BflatGPiano}{\keyboardf[Go][Aso][Do][Fo]}

\newcommand{\BPiano}{\keyboard[Dso][Fso][Bo]}
\newcommand{\BsusPiano}{\keyboard[Eo][Fso][Bo]}
\newcommand{\BDsPiano}{\keyboard[Dso][Fso][Bo]}
\newcommand{\BEflatPiano}{\keyboard[Dso][Fso][Bo]}
\newcommand{\BsevenPiano}{\keyboard[Dso][Fso][Bo][Ao]}
\newcommand{\BsevenFsPiano}{\keyboardf[Fso][Ao][Bo][Dso]}
\newcommand{\BsevenDsPiano}{\keyboard[Dso][Fso][Ao][Bo]}
\newcommand{\BmPiano}{\keyboard[Do][Fso][Bo]}
\newcommand{\BmsevenPiano}{\keyboard[Do][Fso][Bo][Ao]}
\newcommand{\BmsevenbfivePiano}{\keyboard[Bo][Do][Fo][Ao]}
\newcommand{\BmajsevenGPiano}{\keyboardf[Go][Bo][Dso][Fso]}
\newcommand{\BdimDPiano}{\keyboard[Do][Fo][Bo]}

\newcommand{\CPiano}{\keyboard[Co][Eo][Go]}
\newcommand{\CsusPiano}{\keyboard[Co][Fo][Go]}
\newcommand{\CDPiano}{\keyboardtwooctaves[Do][Eo][Go][Ct]}
\newcommand{\CEPiano}{\keyboardtwooctaves[Eo][Go][Ct]}
\newcommand{\CGPiano}{\keyboardf[Go][Co][Eo]}
\newcommand{\CsevenPiano}{\keyboard[Co][Eo][Go][Aso]}
\newcommand{\CsevenBflatPiano}{\keyboardtwooctaves[Aso][Ct][Et][Gt]}
\newcommand{\CmPiano}{\keyboard[Co][Dso][Go]}
\newcommand{\CmsevenPiano}{\keyboard[Co][Dso][Go][Aso]}
\newcommand{\CmAflatPiano}{\keyboardtwooctaves[Gso][Ct][Dst][Gt]}
\newcommand{\CtwoPiano}{\keyboard[Co][Do][Eo][Go]}
\newcommand{\CsustwoPiano}{\keyboard[Co][Do][Go]}
\newcommand{\CninePiano}{\keyboard[Co][Do][Eo][Go][Aso]}
\newcommand{\CMajPiano}{\keyboard[Co][Eo][Go][Bo]}
\newcommand{\CMajtwoDPiano}{\keyboardtwooctaves[Do][Eo][Go][Bo][Ct]}
\newcommand{\CsusthirteenPiano}{\keyboard[Co][Do][Fo][Go][Ao][Aso]}

\newcommand{\CsPiano}{\keyboard[Cso][Fo][Gso]}
\newcommand{\CssevenPiano}{\keyboard[Cso][Fo][Gso][Bo]}
\newcommand{\CsmPiano}{\keyboard[Cso][Eo][Gso]}
\newcommand{\CsmsevenPiano}{\keyboard[Cso][Eo][Gso][Aso]}
\newcommand{\CsFPiano}{\keyboardf[Fo][Gso][Cso]}

\newcommand{\DflatPiano}{\CsPiano}
\newcommand{\DflatmPiano}{\CsmPiano}
\newcommand{\DflatFPiano}{\CsFPiano}

\newcommand{\DPiano}{\keyboard[Do][Fso][Ao]}
\newcommand{\DsevenPiano}{\keyboard[Do][Fso][Ao][Co]}
\newcommand{\DmPiano}{\keyboard[Do][Fo][Ao]}
\newcommand{\DmBPiano}{\keyboard[Bo][Do][Fo][Ao]}
\newcommand{\DmCPiano}{\keyboard[Co][Do][Fo][Ao]}
\newcommand{\DmFPiano}{\keyboardf[Do][Fo][Ao]}
\newcommand{\DmsevenPiano}{\keyboard[Do][Fo][Ao][Co]}
\newcommand{\DmsevenGPiano}{\keyboardf[Go][Do][Fo][Ao][Co]}
\newcommand{\DtwoPiano}{\keyboard[Do][Eo][Fso][Ao]}
\newcommand{\DMajPiano}{\keyboard[Do][Fso][Ao][Cso]}
\newcommand{\DCPiano}{\keyboard[Co][Do][Fso][Ao]}
\newcommand{\DAPiano}{\keyboardf[Ao][Do][Fso]}
\newcommand{\DEPiano}{\keyboardtwooctaves[Eo][Fso][Ao][Dt]}
\newcommand{\DFsPiano}{\keyboardf[Fso][Ao][Do]}
\newcommand{\DsevenFsPiano}{\keyboardf[Fso][Ao][Co][Do]}
\newcommand{\DsusPiano}{\keyboard[Do][Go][Ao]}
\newcommand{\DsusninePiano}{\keyboard[Do][Go][Ao][Eo]}
\newcommand{\DsusFsPiano}{\keyboardf[Fo][Go][Ao][Do]}

\newcommand{\DsPiano}{\keyboard[Dso][Go][Aso]}
\newcommand{\DsmPiano}{\keyboard[Dso][Fso][Aso]}
\newcommand{\DsmsevenPiano}{\keyboard[Dso][Fso][Aso][Cso]}
\newcommand{\EflatPiano}{\keyboard[Dso][Go][Aso]}
\newcommand{\EflattwoPiano}{\keyboard[Dso][Fo][Go][Aso]}
\newcommand{\EflatMajPiano}{\keyboard[Dso][Go][Aso][Do]}
\newcommand{\EflatFPiano}{\keyboardf[Fo][Go][Aso][Dso]}
\newcommand{\EflatsevenPiano}{\keyboard[Dso][Go][Aso][Cso]}
\newcommand{\EflatmPiano}{\DsmPiano}
\newcommand{\EflatmsevenPiano}{\keyboard[Dso][Fso][Aso][Cso]}

\newcommand{\EPiano}{\keyboard[Eo][Gso][Bo]}
\newcommand{\EsusPiano}{\keyboard[Eo][Ao][Bo]}
\newcommand{\EfivePiano}{\keyboard[Eo][Bo]}
\newcommand{\EdimfivePiano}{\keyboard[Eo][Aso][Bo]}
\newcommand{\EsevenPiano}{\keyboard[Eo][Gso][Bo][Do]}
\newcommand{\EmPiano}{\keyboard[Eo][Go][Bo]}
\newcommand{\EmsevenPiano}{\keyboard[Eo][Go][Bo][Do]}
\newcommand{\EmelevenPiano}{\keyboard[Eo][Fso][Go][Ao][Bo][Do]}
\newcommand{\EmBPiano}{\keyboardtwooctaves[Bo][Et][Gt]}
\newcommand{\EmDPiano}{\keyboard[Do][Eo][Go][Bo]}
\newcommand{\EmGPiano}{\keyboardf[Go][Bo][Eo]}
\newcommand{\EBPiano}{\keyboardtwooctaves[Bo][Et][Gst]}
\newcommand{\EDPiano}{\keyboard[Do][Eo][Gso][Bo]}
\newcommand{\EFsPiano}{\keyboardf[Fso][Eo][Gso][Bo]}
\newcommand{\EGPiano}{\keyboardf[Go][Eo][Bo]}
\newcommand{\EGsPiano}{\keyboardf[Gso][Eo][Bo]}

\newcommand{\FPiano}{\keyboardf[Co][Fo][Ao]}
\newcommand{\FsevenPiano}{\keyboard[Co][Fo][Ao][Dso]}
\newcommand{\FninePiano}{\keyboardf[Fo][Go][Ao][Co][Dso]}
\newcommand{\FAPiano}{\keyboardf[Ao][Co][Ft]}
\newcommand{\FtwoPiano}{\keyboardf[Fo][Go][Ao][Co]}
\newcommand{\FtwoAPiano}{\keyboardtwooctaves[Ao][Ct][Ft][Gt]}
\newcommand{\FsusPiano}{\keyboardf[Aso][Co][Fo]}
\newcommand{\FMajPiano}{\keyboardf[Co][Fo][Ao][Eo]}
\newcommand{\FmPiano}{\keyboardf[Fo][Aso][Co]}
\newcommand{\FmsevenPiano}{\keyboard[Co][Fo][Aso][Dso]}
\newcommand{\FmsixPiano}{\keyboardf[Fo][Gso][Co][Do]}
\newcommand{\FdimPiano}{\keyboardf[Fo][Aso][Bo]}

\newcommand{\FsPiano}{\keyboard[Cso][Fso][Aso]}
\newcommand{\FssevenPiano}{\keyboardf[Fso][Aso][Cso][Eo]}
\newcommand{\FssusPiano}{\keyboard[Cso][Fso][Bo]}
\newcommand{\FsmPiano}{\keyboardf[Cso][Fso][Ao]}
\newcommand{\FsmsevenPiano}{\keyboard[Cso][Fso][Ao][Eo]}
\newcommand{\GflatPiano}{\keyboard[Cso][Fso][Aso]}
\newcommand{\GflatmPiano}{\keyboardf[Cso][Fso][Ao]}
\newcommand{\GflatBflatPiano}{\keyboardf[Aso][Cso][Fso]}

\newcommand{\GPiano}{\keyboard[Do][Go][Bo]}
\newcommand{\GsevenPiano}{\keyboard[Do][Fo][Go][Bo]}
\newcommand{\GsusPiano}{\keyboardf[Go][Co][Do]}
\newcommand{\GMajPiano}{\keyboardf[Do][Go][Bo][Fso]}
\newcommand{\GBPiano}{\keyboardtwooctaves[Bo][Dt][Gt]}
\newcommand{\GEPiano}{\keyboardtwooctaves[Eo][Go][Bo][Dt]}
\newcommand{\GFPiano}{\keyboardf[Fo][Go][Bo][Do]}
\newcommand{\GDPiano}{\keyboard[Do][Go][Bo]}
\newcommand{\GtwoPiano}{\keyboardf[Go][Ao][Bo][Do]}
\newcommand{\GninePiano}{\keyboardf[Go][Ao][Bo][Do][Fo]}
\newcommand{\GnineDPiano}{\keyboard[Do][Fo][Go][Ao][Bo]}
\newcommand{\GmPiano}{\keyboard[Do][Go][Aso]}
\newcommand{\GmsevenPiano}{\keyboard[Do][Go][Aso][Fo]}
\newcommand{\GmBflatPiano}{\keyboardf[Aso][Go][Do]}

\newcommand{\GsPiano}{\keyboard[Dso][Gso][Co]}
\newcommand{\GssevenPiano}{\keyboardf[Gso][Co][Dso][Fso]}
\newcommand{\GssusPiano}{\keyboardf[Gso][Cso][Dso]}
\newcommand{\GsmPiano}{\keyboardf[Gso][Bo][Dso]}
\newcommand{\GsmsevenPiano}{\keyboardf[Gso][Bo][Dso][Fso]}
\newcommand{\AflatmPiano}{\keyboardf[Gso][Bo][Dso]}
\newcommand{\GsdimPiano}{\keyboardf[Gso][Bo][Do]}
\newcommand{\AflatPiano}{\keyboard[Dso][Gso][Co]}
\newcommand{\AflatsevenPiano}{\GssevenPiano}
\newcommand{\AflatMajPiano}{\keyboardf[Gso][Dso][Co][Go]}
\newcommand{\AflatCPiano}{\keyboard[Co][Dso][Gso]}
\newcommand{\AflatEflatPiano}{\keyboardtwooctaves[Dso][Gso][Ct]}
\newcommand{\AflatFPiano}{\keyboardf[Fo][Gso][Co][Dso]}

%% impresión de nombres de acordes
\newcommand{\A}{\ifguitarra\AGuitar\fi\ifpiano\APiano\fi}
\newcommand{\Asus}{\ifguitarra\AsusGuitar\fi\ifpiano\AsusPiano\fi}
\newcommand{\Aseven}{\ifguitarra\AsevenGuitar\fi\ifpiano\AsevenPiano\fi}
\newcommand{\AMaj}{\ifguitarra\AMajGuitar\fi\ifpiano\AMajPiano\fi}
\newcommand{\AB}{\ifguitarra\ABGuitar\fi\ifpiano\ABPiano\fi}
\newcommand{\ACs}{\ifguitarra\ACsGuitar\fi\ifpiano\ACsPiano\fi}
\newcommand{\AD}{\ifguitarra\ADGuitar\fi\ifpiano\ADPiano\fi}
\newcommand{\AEg}{\ifguitarra\AEGuitar\fi\ifpiano\AEPiano\fi}
\newcommand{\AG}{\ifguitarra\AGGuitar\fi\ifpiano\AGPiano\fi}
\newcommand{\AAs}{\ifguitarra\AAsGuitar\fi\ifpiano\AAsPiano\fi}
\newcommand{\AsevenCs}{\ifguitarra\AsevenCsGuitar\fi\ifpiano\AsevenCsPiano\fi}
\newcommand{\Atwo}{\ifguitarra\AtwoGuitar\fi\ifpiano\AtwoPiano\fi}
\newcommand{\Asustwo}{\ifguitarra\AsustwoGuitar\fi\ifpiano\AsustwoPiano\fi}
\newcommand{\Anine}{\ifguitarra\AnineGuitar\fi\ifpiano\AninePiano\fi}

\newcommand{\Am}{\ifguitarra\AmGuitar\fi\ifpiano\AmPiano\fi}
\newcommand{\Amseven}{\ifguitarra\AmsevenGuitar\fi\ifpiano\AmsevenPiano\fi}
\newcommand{\AmC}{\ifguitarra\AmCGuitar\fi\ifpiano\AmCPiano\fi}
\newcommand{\AmCs}{\ifguitarra\AmCsGuitar\fi\ifpiano\AmCsPiano\fi}
\newcommand{\AmF}{\ifguitarra\AmFGuitar\fi\ifpiano\AmFPiano\fi}
\newcommand{\AmFs}{\ifguitarra\AmFsGuitar\fi\ifpiano\AmFsPiano\fi}
\newcommand{\AmG}{\ifguitarra\AmGuitar\fi\ifpiano\AmGPiano\fi}
\newcommand{\AmGs}{\ifguitarra\AmGsGuitar\fi\ifpiano\AmGsPiano\fi}
\newcommand{\AmsevenF}{\ifguitarra\AmsevenFGuitar\fi\ifpiano\AmsevenFPiano\fi}

\newcommand{\As}{\ifguitarra\AsGuitar\fi\ifpiano\AsPiano\fi}
\newcommand{\AsMaj}{\ifguitarra\AsMajGuitar\fi\ifpiano\AsMajPiano\fi}
\newcommand{\Asm}{\ifguitarra\AsmGuitar\fi\ifpiano\AsMPiano\fi}
\newcommand{\Asthirteen}{\ifguitarra\AsthirteenGuitar\fi\ifpiano\AsthirteenPiano\fi}
\newcommand{\Bflat}{\ifguitarra\BflatGuitar\fi\ifpiano\BflatPiano\fi}
\newcommand{\Bflatseven}{\ifguitarra\BflatsevenGuitar\fi\ifpiano\BflatsevenPiano\fi}
\newcommand{\Bflatm}{\ifguitarra\BflatmGuitar\fi\ifpiano\BflatmPiano\fi}
\newcommand{\Bflatmseven}{\ifguitarra\BflatmsevenGuitar\fi\ifpiano\BflatmsevenPiano\fi}
\newcommand{\BflatMaj}{\ifguitarra\BflatMajGuitar\fi\ifpiano\BflatMajPiano\fi}
\newcommand{\BflatC}{\ifguitarra\BflatCGuitar\fi\ifpiano\BflatCPiano\fi}
\newcommand{\BflatD}{\ifguitarra\BflatDGuitar\fi\ifpiano\BflatDPiano\fi}
\newcommand{\BflatF}{\ifguitarra\BflatFGuitar\fi\ifpiano\BflatFPiano\fi}
\newcommand{\BflatG}{\ifguitarra\BflatGuitar\fi\ifpiano\BflatGPiano\fi}

\newcommand{\B}{\ifguitarra\BGuitar\fi\ifpiano\BPiano\fi}
\newcommand{\Bsus}{\ifguitarra\BsusGuitar\fi\ifpiano\BsusPiano\fi}
\newcommand{\Bseven}{\ifguitarra\BsevenGuitar\fi\ifpiano\BsevenPiano\fi}
\newcommand{\BsevenFs}{\ifguitarra\BsevenFsGuitar\fi\ifpiano\BsevenFsPiano\fi}
\newcommand{\BsevenDs}{\ifguitarra\BsevenDsGuitar\fi\ifpiano\BsevenDsPiano\fi}
\newcommand{\BsevenBasDs}{\ifguitarra\BsevenBasDsGuitar\fi\ifpiano\BsevenBasDsPiano\fi}
\newcommand{\BDs}{\ifguitarra\BDsGuitar\fi\ifpiano\BDsPiano\fi}
\newcommand{\BEflat}{\ifguitarra\BEflatGuitar\fi\ifpiano\BEflatPiano\fi}
\newcommand{\Bm}{\ifguitarra\BmGuitar\fi\ifpiano\BmPiano\fi}
\newcommand{\Bmseven}{\ifguitarra\BmsevenGuitar\fi\ifpiano\BmsevenPiano\fi}
\newcommand{\BmseveN}{\ifguitarra\BmseveNGuitar\fi\ifpiano\BmseveNPiano\fi}
\newcommand{\BmsevenA}{\ifguitarra\BmsevenAGuitar\fi\ifpiano\BmsevenAPiano\fi}
\newcommand{\Bmsevenbfive}{\ifguitarra\BmsevenbfiveGuitar\fi\ifpiano\BmsevenbfivePiano\fi} %si semidisminuido
\newcommand{\BmajsevenG}{\ifguitarra\BmajsevenGGuitar\fi\ifpiano\BmajsevenGPiano\fi}
\newcommand{\BdimD}{\ifguitarra\BdimDGuitar\fi\ifpiano\BdimDPiano\fi}

\newcommand{\C}{\ifguitarra\CGuitar\fi\ifpiano\CPiano\fi}
\newcommand{\Csus}{\ifguitarra\CsusGuitar\fi\ifpiano\CsusPiano\fi}
\newcommand{\Cseven}{\ifguitarra\CsevenGuitar\fi\ifpiano\CsevenPiano\fi}
\newcommand{\CsevenBflat}{\ifguitarra\CsevenBflatGuitar\fi\ifpiano\CsevenBflatPiano\fi}
\newcommand{\Cm}{\ifguitarra\CmGuitar\fi\ifpiano\CmPiano\fi}
\newcommand{\Cmseven}{\ifguitarra\CmsevenGuitar\fi\ifpiano\CmsevenPiano\fi}
\newcommand{\CmAflat}{\ifguitarra\CmAflatGuitar\fi\ifpiano\CmAflatPiano\fi}
\newcommand{\CD}{\ifguitarra\CDGuitar\fi\ifpiano\CDPiano\fi}
\newcommand{\CE}{\ifguitarra\CEGuitar\fi\ifpiano\CEPiano\fi}
\newcommand{\CG}{\ifguitarra\CGuitar\fi\ifpiano\CGPiano\fi}
\newcommand{\CMaj}{\ifguitarra\CMajGuitar\fi\ifpiano\CMajPiano\fi}
\newcommand{\CMajtwoD}{\ifguitarra\CMajtwoDGuitar\fi\ifpiano\CMajtwoDPiano\fi}
\newcommand{\Csustwo}{\ifguitarra\CsustwoGuitar\fi\ifpiano\CsustwoPiano\fi}
\newcommand{\Ctwo}{\ifguitarra\CtwoGuitar\fi\ifpiano\CtwoPiano\fi}
\newcommand{\Cnine}{\ifguitarra\CnineGuitar\fi\ifpiano\CNinePiano\fi}
\newcommand{\Csusthirteen}{\ifguitarra\CsusthirteenGuitar\fi\ifpiano\CsusthirteenPiano\fi}

\newcommand{\Cs}{\ifguitarra\CsGuitar\fi\ifpiano\CsPiano\fi}
\newcommand{\Csm}{\ifguitarra\CsmGuitar\fi\ifpiano\CsmPiano\fi}
\newcommand{\Csmseven}{\ifguitarra\CsmsevenGuitar\fi\ifpiano\CsmsevenPiano\fi}
\newcommand{\Csseven}{\ifguitarra\CsmsevenGuitar\fi\ifpiano\CsmsevenPiano\fi}
\newcommand{\Dflat}{\ifguitarra\DflatGuitar\fi\ifpiano\DflatPiano\fi}
\newcommand{\Dflatm}{\ifguitarra\DflatmGuitar\fi\ifpiano\DflatmPiano\fi}
\newcommand{\DflatF}{\ifguitarra\DflatFGuitar\fi\ifpiano\DflatFPiano\fi}

\newcommand{\D}{\ifguitarra\DGuitar\fi\ifpiano\DPiano\fi}
\newcommand{\Dseven}{\ifguitarra\DsevenGuitar\fi\ifpiano\DsevenPiano\fi}
\newcommand{\Dm}{\ifguitarra\DmGuitar\fi\ifpiano\DmPiano\fi}
\newcommand{\Dmseven}{\ifguitarra\DmsevenGuitar\fi\ifpiano\DmsevenPiano\fi}
\newcommand{\DMaj}{\ifguitarra\DMajGuitar\fi\ifpiano\DMajPiano\fi}
\newcommand{\Dsus}{\ifguitarra\DsusGuitar\fi\ifpiano\DsusPiano\fi}
\newcommand{\Dsusnine}{\ifguitarra\DsusnineGuitar\fi\ifpiano\DsusninePiano\fi}
\newcommand{\DmB}{\ifguitarra\DmBGuitar\fi\ifpiano\DmBPiano\fi}
\newcommand{\DmC}{\ifguitarra\DmCGuitar\fi\ifpiano\DmCPiano\fi}
\newcommand{\DmF}{\ifguitarra\DmFGuitar\fi\ifpiano\DmFPiano\fi}
\newcommand{\Dtwo}{\ifguitarra\DtwoGuitar\fi\ifpiano\DtwoPiano\fi}
\newcommand{\Dsix}{\ifguitarra\DsixGuitar\fi\ifpiano\DsixPiano\fi}
\newcommand{\DmBasB}{\ifguitarra\DmBasBGuitar\fi\ifpiano\DmBasBPiano\fi}
\newcommand{\DA}{\ifguitarra\DAGuitar\fi\ifpiano\DAPiano\fi}
\newcommand{\DC}{\ifguitarra\DCGuitar\fi\ifpiano\DCPiano\fi}
\newcommand{\DE}{\ifguitarra\DEGuitar\fi\ifpiano\DEPiano\fi}
\newcommand{\DFs}{\ifguitarra\DFsGuitar\fi\ifpiano\DFsPiano\fi}
\newcommand{\DsevenFs}{\ifguitarra\DsevenFsGuitar\fi\ifpiano\DsevenFsPiano\fi}
\newcommand{\DmsevenG}{\ifguitarra\DmsevenGGuitar\fi\ifpiano\DmsevenGPiano\fi}
\newcommand{\DsusFs}{\ifguitarra\DsusFsGuitar\fi\ifpiano\DsusFsPiano\fi}

\newcommand{\Ds}{\ifguitarra\DsGuitar\fi\ifpiano\DsPiano\fi}
\newcommand{\Dsseven}{\ifguitarra\DsevenGuitar\fi\ifpiano\DsevenPiano\fi}
\newcommand{\Dsm}{\ifguitarra\DsmGuitar\fi\ifpiano\DsmPiano\fi}
\newcommand{\Dsmseven}{\ifguitarra\DsmsevenGuitar\fi\ifpiano\DsmsevenPiano\fi}
\newcommand{\Eflat}{\ifguitarra\EflatGuitar\fi\ifpiano\EflatPiano\fi}
\newcommand{\Eflattwo}{\ifguitarra\EflattwoGuitar\fi\ifpiano\EflattwoPiano\fi}
\newcommand{\EflatMaj}{\ifguitarra\EflatMajGuitar\fi\ifpiano\EflatMajPiano\fi}
\newcommand{\EflatF}{\ifguitarra\EflatFGuitar\fi\ifpiano\EflatFPiano\fi}
\newcommand{\Eflatseven}{\ifguitarra\EflatsevenGuitar\fi\ifpiano\EflatsevenPiano\fi}
\newcommand{\Eflatmseven}{\ifguitarra\EflatmsevenGuitar\fi\ifpiano\EflatmsevenPiano\fi}
\newcommand{\Eflatm}{\ifguitarra\EflatmGuitar\fi\ifpiano\EflatmPiano\fi}

\newcommand{\E}{\ifguitarra\EGuitar\fi\ifpiano\EPiano\fi}
\newcommand{\Esus}{\ifguitarra\EsusGuitar\fi\ifpiano\EsusPiano\fi}
\newcommand{\Efive}{\ifguitarra\EfiveGuitar\fi\ifpiano\EfivePiano\fi}
\newcommand{\Edimfive}{\ifguitarra\EdimfiveGuitar\fi\ifpiano\EdimfivePiano\fi}
\newcommand{\Eseven}{\ifguitarra\EsevenGuitar\fi\ifpiano\EsevenPiano\fi}
\newcommand{\EseveN}{\ifguitarra\EseveNineGuitar\fi\ifpiano\EseveNinePiano\fi}
\newcommand{\Em}{\ifguitarra\EmGuitar\fi\ifpiano\EmPiano\fi}
\newcommand{\Emseven}{\ifguitarra\EmsevenGuitar\fi\ifpiano\EmsevenPiano\fi}
\newcommand{\Emseventr}{\ifguitarra\EmseventrGuitar\fi\ifpiano\EmsevenPiano\fi}
\newcommand{\Emeleven}{\ifguitarra\EmelevenGuitar\fi\ifpiano\EmelevenPiano\fi}
\newcommand{\EmB}{\ifguitarra\EmBGuitar\fi\ifpiano\EmBPiano\fi}
\newcommand{\EmD}{\ifguitarra\EmDGuitar\fi\ifpiano\EmDPiano\fi}
\newcommand{\EmG}{\ifguitarra\EmGGuitar\fi\ifpiano\EmGPiano\fi}
\newcommand{\EsevenFour}{\ifguitarra\EsevenFourGuitar\fi\ifpiano\EsevenFourPiano\fi}
\newcommand{\EseveNNine}{\ifguitarra\EsevenNNineGuitar\fi\ifpiano\EsevenNNinePiano\fi}
\newcommand{\EB}{\ifguitarra\EBGuitar\fi\ifpiano\EBPiano\fi}
\newcommand{\ED}{\ifguitarra\EDGuitar\fi\ifpiano\EDPiano\fi}
\newcommand{\EFs}{\ifguitarra\EFsGuitar\fi\ifpiano\EFsPiano\fi}
\newcommand{\EG}{\ifguitarra\EGuitar\fi\ifpiano\EPiano\fi}
\newcommand{\EGs}{\ifguitarra\EGsGuitar\fi\ifpiano\EGsPiano\fi}

\newcommand{\F}{\ifguitarra\FGuitar\fi\ifpiano\FPiano\fi}
\newcommand{\FA}{\ifguitarra\FGuitar\fi\ifpiano\FPiano\fi}
\newcommand{\Ftwo}{\ifguitarra\FtwoGuitar\fi\ifpiano\FtwoPiano\fi}
\newcommand{\FtwoA}{\ifguitarra\FtwoAGuitar\fi\ifpiano\FtwoAPiano\fi}
\newcommand{\FMaj}{\ifguitarra\FMajGuitar\fi\ifpiano\FMajPiano\fi}
\newcommand{\Fseven}{\ifguitarra\FsevenGuitar\fi\ifpiano\FsevenPiano\fi}
\newcommand{\Fnine}{\ifguitarra\FnineGuitar\fi\ifpiano\FninePiano\fi}
\newcommand{\Fm}{\ifguitarra\FMajGuitar\fi\ifpiano\FMajPiano\fi}
\newcommand{\Fmseven}{\ifguitarra\FsevenGuitar\fi\ifpiano\FsevenPiano\fi}
\newcommand{\Fmsix}{\ifguitarra\FmsixGuitar\fi\ifpiano\FmsixPiano\fi}
\newcommand{\Fdim}{\ifguitarra\FdimGuitar\fi\ifpiano\FdimPiano\fi}

\newcommand{\Fs}{\ifguitarra\FsGuitar\fi\ifpiano\FsPiano\fi}
\newcommand{\Fsseven}{\ifguitarra\FssevenGuitar\fi\ifpiano\FssevenPiano\fi}
\newcommand{\Fssus}{\ifguitarra\FssevenGuitar\fi\ifpiano\FssevenPiano\fi}
\newcommand{\Fsm}{\ifguitarra\FsmGuitar\fi\ifpiano\FsmPiano\fi}
\newcommand{\FsmLight}{\ifguitarra\FsmLightGuitar\fi\ifpiano\FsmLightPiano\fi}
\newcommand{\FsmBasSeveN}{\ifguitarra\FsmBasSeveNineGuitar\fi\ifpiano\FsmBasSeveNinePiano\fi}
\newcommand{\FsmBasSeven}{\ifguitarra\FsmBasSevenGuitar\fi\ifpiano\FsmBasSevenPiano\fi}
\newcommand{\Fsmseven}{\ifguitarra\FsmsevenGuitar\fi\ifpiano\FsmsevenPiano\fi}
\newcommand{\Gflat}{\ifguitarra\GflatGuitar\fi\ifpiano\GflatPiano\fi}
\newcommand{\GflatBflat}{\ifguitarra\GflatBflatGuitar\fi\ifpiano\GflatBflatPiano\fi}
\newcommand{\Gflatm}{\ifguitarra\GflatmGuitar\fi\ifpiano\GflatmPiano\fi}

\newcommand{\G}{\ifguitarra\GGuitar\fi\ifpiano\GPiano\fi}
\newcommand{\Gseven}{\ifguitarra\GsevenGuitar\fi\ifpiano\GsevenPiano\fi}
\newcommand{\GMaj}{\ifguitarra\GMajGuitar\fi\ifpiano\GMajPiano\fi}
\newcommand{\Gsus}{\ifguitarra\GsusGuitar\fi\ifpiano\GsusPiano\fi}
\newcommand{\GB}{\ifguitarra\GBGuitar\fi\ifpiano\GPiano\fi}
\newcommand{\GD}{\ifguitarra\GDGuitar\fi\ifpiano\GPiano\fi}
\newcommand{\GE}{\ifguitarra\GEGuitar\fi\ifpiano\GPiano\fi}
\newcommand{\GF}{\ifguitarra\GFGuitar\fi\ifpiano\GPiano\fi}
\newcommand{\GnineD}{\ifguitarra\GnineDGuitar\fi\ifpiano\GnineDPiano\fi}
\newcommand{\Gtwo}{\ifguitarra\GtwoGuitar\fi\ifpiano\GPiano\fi}
\newcommand{\Gm}{\ifguitarra\GmGuitar\fi\ifpiano\GmPiano\fi}
\newcommand{\Gmseven}{\ifguitarra\GsevenGuitar\fi\ifpiano\GsevenPiano\fi}
\newcommand{\GmBflat}{\ifguitarra\GmBflatGuitar\fi\ifpiano\GmBflatPiano\fi}

\newcommand{\Gs}{\ifguitarra\GsGuitar\fi\ifpiano\GsPiano\fi}
\newcommand{\Aflat}{\ifguitarra\AflatGuitar\fi\ifpiano\AflatPiano\fi}
\newcommand{\Aflatseven}{\ifguitarra\AflatsevenGuitar\fi\ifpiano\AflatsevenPiano\fi}
\newcommand{\Aflatm}{\ifguitarra\AflatmGuitar\fi\ifpiano\AflatmPiano\fi}
\newcommand{\AflatEflat}{\ifguitarra\AflatEflatGuitar\fi\ifpiano\AflatEflatPiano\fi}
\newcommand{\AflatF}{\ifguitarra\AflatFGuitar\fi\ifpiano\AflatFPiano\fi}
\newcommand{\AflatC}{\ifguitarra\AflatCGuitar\fi\ifpiano\AflatCPiano\fi}
\newcommand{\AflatMaj}{\ifguitarra\AflatMajGuitar\fi\ifpiano\AflatMajPiano\fi}
\newcommand{\Gsm}{\ifguitarra\GsmGuitar\fi\ifpiano\GsmPiano\fi}
\newcommand{\Gsmseven}{\ifguitarra\GsmsevenGuitar\fi\ifpiano\GsmsevenPiano\fi}
\newcommand{\Gssus}{\ifguitarra\GssusGuitar\fi\ifpiano\GssusPiano\fi}
\newcommand{\Gsdim}{\ifguitarra\GsdimGuitar\fi\ifpiano\GsdimPiano\fi}
\documentclass[12pt, spanish]{book}

% Packages
\usepackage[T1]{fontenc} %pdflatex
%\usepackage{fontspec} %xelatex
\usepackage[utf8]{inputenc}
\usepackage{babel}
\usepackage{mypiano}
\usepackage{gchords}
\usepackage{latexsym,fancyhdr}
\usepackage{imakeidx}
\usepackage[pdfpagelabels,hyperindex,unicode=true,pdfusetitle, bookmarks=true,bookmarksnumbered=false,bookmarksopen=true,
    breaklinks=false,pdfborder={0 0 1},backref=false,colorlinks=true]{hyperref}
\usepackage[chordbk]{songbook} %% Words & Chords edition. Estribillero Musicos
\usepackage{endnotes}
%\usepackage{biblatex}
\usepackage{tikz}
%%\usepackage[compactallsongs,chordbk]{songbook}    %% Words & Chords edition.
%\usepackage[wordbk]{songbook}                 %% Words Only edition. Estribillero publicable
%\usepackage[overhead]{songbook}               %% Overhead Transparency edition. Estribillero Letras

%\newcommand\themarker[1]{\pdfbookmark[1]{#1}{\pdfmdfivesum{#1}}}


% Paso 1: Guardar el entorno original
\let\oldsong\song
\let\endoldsong\endsong


\newcommand{\strconcat}[2]{#1 - #2}

% Paso 2: Redefinir el entorno para insertar el pdfbookmark
%\renewenvironment{song}[7][YF]{%
\RenewDocumentEnvironment{song}{m m m m m m}{%
    \pdfbookmark[1]{\strconcat{#1}{#4}}{\pdfmdfivesum{\strconcat{#1}{#4}}}% usar el primer argumento como título
    \oldsong{#1}{#2}{#3}{#4}{#5}{#6}% llamar al entorno original con los mismos argumentos
    }{%
    \endoldsong
}
\renewcommand*{\UrlFont}{\ttfamily\smaller\relax}

\renewcommand{\SBChorusTag}{Coro:}
\renewcommand{\SBBridgeTag}{Puente:}
\renewcommand{\SBEndTag}{Fin:}
%\newcommand{\myTitleFont}{\Huge\myHugeSF}
\newcommand{\mySubTitleFont}{\large\sf}


%%%
% Turn on/off index-file generation.  Uncomment the \makeindex line to
% turn index generation on;  comment it out to turn index generation
% off.
%%%
\makeTitleIndex         %% Title and First Line Index.
\makeTitleContents      %% Table of Contents.
\makeKeyIndex           %% Index of song by key.
\makeArtistIndex        %% Index of song by artist.
\makeindex

\renewcommand{\notesname}{Notas}

%%%
% Revision Date and Release Date definitions.
%
%       \RelDate - The last time this songbook was released.  Set this
%                  date each time a new release/update of the songbook
%                  is generated.
%       \RevDate - The last time a particular song was revised in any
%                  way.  This command will be renewed inside every
%                  song.
%%%
\newcommand{\RelDate}{\today}
\newcommand{\RevDate}{\today}

%%%
% C.C.L.I. license number definition; for copyright licensing info.
% One of these macros will be manually inserted into the {CpyRt}
% parameter of the {song} environment.
%
%       \CCLInumber - The actual copyright license number.  Don't
%               insert this command in the {CpyRt} parameter, use one
%               of the others.
%       \CCLIed - Indicates a song falls under our CCLI license.
%       \NotCCLIed - Indicates a song doesn't fall under our CCLI
%               license.  Public Domain songs fall into this category.
%       \PGranted - We have received specific permission from the
%               copyright holder to use this song.
%       \PPending - We are in the process of obtaining permission to
%               use this song.
%%%
\newcommand{\CCLInumber}{Your CCLI Number}
\newcommand{\CCLIed}{{\CpyRtInfoFont (CCLI \CCLInumber)}}
\newcommand{\NotCCLIed}{\relax}
\newcommand{\PGranted}{\relax}
\newcommand{\PPending}{{\CpyRtInfoFont (Permission Pending)}}

%%%
% Title page information.
%%%
\title{Cuaderno de Cantos Infantiles}
\author{Ruslan L\'opez}
\date{\'Ultima Revisi\'on:  \RevDate}

%%%
% Redefine fonts from SongBook style that I don't like.
%%%
\font\myTinySF=cmss8 at 8pt
\renewcommand{\CpyRtInfoFont}{\tiny\myTinySF}

%%%
% Define fonts to use in the headers and footers of the songbook.
%%%
\newcommand{\LHeadFont}{\normalsize}            % = cmr12  at 12pt
\newcommand{\CHeadFont}{\normalsize\rm}         % = cmr12  at 12pt
\newcommand{\RHeadFont}{\normalsize}            % = cmr12  at 12pt
\newcommand{\LFootFont}{\scriptsize}            % = cmr8   at  8pt
\newcommand{\CFootFont}{\tiny\myTinySF}         % = cmss8  at  8pt
\newcommand{\RFootFont}{\scriptsize}            % = cmr8   at  8pt

%%%
% Turn on and define fancy page heading/footing definition.
%%%
\pagestyle{fancy}

\ifChordBk
% It's a words & chords songbook...
\headsep=7mm
\oddsidemargin=1in
\evensidemargin=1.2in
\footskip=0.2in
\addtolength{\headwidth}{\marginparsep}
\addtolength{\headwidth}{\marginparwidth}
\renewcommand{\headrulewidth}{0.4pt}
\renewcommand{\footrulewidth}{0.4pt}
\fancyhead[LE,RO]{\LHeadFont\emph{\leftmark\/}\SBContinueMark}
\fancyhead[CE,CO]{\CHeadFont\thepage}
\fancyhead[RE,LO]{\RHeadFont\RelDate}
\else\ifOverhead
% It's an overhead...
\renewcommand{\footrulewidth}{0pt}
\renewcommand{\headrulewidth}{0pt}
\fancyhead[LE,RO]{}
\fancyhead[CE,CO]{}
\fancyhead[RE,LO]{}
\else\ifWordBk
% It's a words only songbook...
\addtolength{\headwidth}{\marginparsep}
\addtolength{\headwidth}{\marginparwidth}
\renewcommand{\headrulewidth}{0.4pt}
\renewcommand{\footrulewidth}{0.4pt}
\fancyhead[LE,RO]{\LHeadFont Estribillero}
\fancyhead[CE,CO]{\CHeadFont\thepage}
\fancyhead[RE,LO]{\RHeadFont\RelDate}
\fi\fi\fi

\fancyfoot[LE,RO]{\LFootFont Transcripciones}
\ifSongEject
\fancyfoot[CE,CO]{\CFootFont \RevDate}
\else
\fancyfoot[CE,CO]{\CFootFont}
\fi
\fancyfoot[RE,LO]{\RFootFont Todo el material son transcripciones personales.}


% Document
\begin{document}
    \hypersetup{pageanchor=false}

%%%
% Custom titlepage for Christian church songbook
%%%
    \begin{titlepage}
        \thispdfpagelabel{Portada}
        \centering

        \vspace*{1cm}

        {\Huge\bfseries\sffamily{$\dagger$}\par}

        \vspace{1.5cm}

        {\Huge\bfseries\sffamily Cuaderno de\par}
        \vspace{0.5cm}
        {\Huge\bfseries\sffamily Cantos Infantiles\par}

        \vspace{2cm}

        \begin{center}
            \Large\itshape ``Cantad alegres a Dios, habitantes de toda la tierra.\\
            Servid a Jehov\'a con alegr\'ia;\\
            Venid ante su presencia con regocijo.''\\
            \vspace{0.3cm}
            --- Salmos 100:1-2
        \end{center}

        \vspace{2cm}

        {\Large\bfseries Compilado por:\par}
        \vspace{0.5cm}
        {\Large Ruslan L\'opez\par}

        \vfill

        {\large \'Ultima Revisi\'on: \RevDate\par}

        \vspace{1cm}

    \end{titlepage}

    \hypersetup{pageanchor=true}
    \pdfbookmark[0]{Piezas musicales}{piezas}
%    \mainmatter
    \ifWordBk
    \twocolumn
    \fi
%%%
% Songbook begins.
%%%

    \begin{song}{Una tortuguita}{C}
    {\SBPubDom} %copyright \SBPubDom
    {Dalberto Gomez Perez}
    {} %pasaje
    {} %\NotCCLIed
        \ifChordBk
        \paragraph{\mbox{}\hfill\protect\href{https://open.spotify.com/intl-es/track/0XgjeE9k0MrDisfz7PPtPf}{Escuchar}\mbox{}\hfill}
        \fi

        \begin{SBOpGroup}
            \Ch{C}{U}na tortuguita saca la cabeza

            es\Ch{G}{ti}ra sus manitas se le quita la pe\Ch{C}{re}za
        \end{SBOpGroup}

        \begin{SBOpGroup}
            \Ch{C}{Di}ce el perezoso, me duele la cabeza

            me \Ch{G}{due}le la cintura tengo ganas de \Ch{C}{dor}mir\Ch{C7}{}
        \end{SBOpGroup}

        \begin{SBChorus}
            \Ch{F}{Es} un buen ejemplo, \Ch{C}{pa}ra los cristianos

            \Ch{G}{que} de mala gana le \Ch{C}{sir}ven al Señor.
        \end{SBChorus}

        \begin{SBOpGroup}
            \Ch{C}{Eso} pasará cuando Cristo venga

            \Ch{G}{los} que estén dormidos aquí se queda\Ch{C}{rán}
        \end{SBOpGroup}

        \ifChordBk
        \begin{SBOpGroup}
            Acordes:
            \upchord{\C}{\qquad Do} Mayor  \hfill
            \upchord{\G}{\qquad Sol} Mayor \hfill\null\break
            \upchord{\Csmseven}{Do Sostenido} Menor S\'eptima \hfill
            \upchord{\F}{\qquad Fa} Mayor \hfill\null\break
        \end{SBOpGroup}
        \fi
    \end{song}

    \begin{song}{Zaqueo}{G}
    {} %copyright \SBPubDom
    {Manuel Bonilla}
    {} %pasaje
    {} %\NotCCLIed
        \ifChordBk
        \paragraph{\mbox{}\hfill\protect\href{https://open.spotify.com/intl-es/track/08vuoQqGMlE0HLEe1GG8ui}{Escuchar}\mbox{}\hfill}
        \fi

%  \SBRef{No puedo parar de alabarte}{2006}%fuente \#

        \begin{SBOpGroup}
Zaqueo era un chaparrito así


Que vivía en Jericó


Y cuando Jesús pasó por allí


A un sicomoro subió
        \end{SBOpGroup}

        \begin{SBChorus}
El Salvador le vio allí


Y le hablo así


Zaqueo, bájate de allí


Porque a tu casa voy a ir


A tu casa voy a ir
        \end{SBChorus}

        \begin{SBOpGroup}
Si tú también chiquitito estás


Y a Cristo quieres ver


Tú puedes hoy venir a Él


No hay nada que temer
        \end{SBOpGroup}

        \begin{SBOpGroup}
Decídete a invitarle hoy


Y dile al Salvador


Cristo, pasa por aquí


Porque contigo quiero ir


Que contigo quiero ir
        \end{SBOpGroup}

        \ifChordBk
        \begin{SBOpGroup}
            Acordes:
            \upchord{\Em}{\qquad Mi} Menor \hfill
            \upchord{\Bseven}{\qquad Si} S\'eptima \hfill\null\break
            \upchord{\C}{\qquad Do} Mayor \hfill
            \upchord{\D}{Re} Mayor \hfill\null\break
            \upchord{\G}{\qquad Sol} Mayor \hfill
            \upchord{\Am}{\qquad La} Menor \hfill\null\break
        \end{SBOpGroup}
        \fi
    \end{song}

    \ifChordBk
        \pdfbookmark[0]{Pads~Para~Yamaha}{pads}
\section*{Pads Para Yamaha E-473}

Estos pads son compatibles con varios teclados de Yamaha, por ejemplo el E-463

Para guardar una configuraci\'on en el registration memory.

Apretamos la tecla bank hasta que aparezca en pantalla el n\'umero de banco en que lo queramos guardar, normalmente hay 8.

Si hay que guardar m\'as de uno, tenemos que tener cuidado porque por defecto guarda el \'ultimo que hemos guardado.
Observe que en registration tiene alg\'un candadito, debemos desactivarla para poder guardar los cambios.

Buscamos la funci\'on VoiceFrz y lo modificamos con la rueda a off para poder guardar los sonidos.
apretar la tecla voice para configurar la voz principal M.Voice.
para cambiar los par\'ametros oprima la tecla function, busque la configuraci\'on a modificar, oprima enter, modifique el valor mediante la rueda que se ubica a la derecha
y oprima la tecla enter para guardar los cambios.
Finalmente oprima al mismo tiempo la tecla bank y una de las 4 teclas numeradas que est\'an a su derecha. Deber\'a aparecer la leyenda Memory OK en el visor.
Acabando de modificar y guardar las voces debe volver a ponerlo en On.

Para cargar la voz basta con oprimir la t�cla del n\'umero a la derecha de la tecla bank para que esta se cargue.

\textbf{Pad adoraci\'on 1}
\vskip 25pt

Touch res: Medium
M. Voice: 002 GrandPno
M. Volume: 115
M. Octave: 0
M.Pan: R30
M. Reverb: 100
M. Chorus: 64
M. Attack: 64
M. Release: 64
M. Cutoff: 64
M. Reso: 64
D. Voice: 661 WarmPad
D. Volume: 100
D. Octave: 0
D. Pan: L30
D. Reverb: 100
D. Chorus: 0
D. Attack: 64
D. Release: 115
D. Cutoff: 64
D. Reso: 64
\vskip 25pt

\textbf{Brass Coritos}
\vskip 25pt

Touch res: Medium
M. Voice: 002 GandPno
M. Volume: 115
M. Octave: 0
M. Pan: R10
M. Reverb: 28
M. Chorus: 0
M. Attack: 64
M. Release: 64
M. Cutoff: 64
M. Reso: 64
D. Voice: 133 OctBrass
D. Volume: 100
D. Octave: 1
D. Pan: L10
D. Reverb: 80
D. Chorus: 0
D. Attack: 64
D. Release: 72
D. Cutoff: 75
D. Reso: 64
\vskip 25pt

\textbf{Acorde\'on coritos}
\vskip 25pt

Touch res: Medium
M. Voice: 370 El.Grand
M. Volume: 90
M. Octave: 0
M. Pan: C
M. Reverb: 22
M. Chorus: 0
M. Attack: 64
M. Release: 64
M. Cutoff: 64
M. Reso: 64
D. Voice: 56 ReedOrgn
D. Volume: 64
D. Octave: 1
D. Pan: C
D. Reverb: 51
D. Chorus: 38
D. Attack: 61
D. Release: 29
D. Cutoff: 64
D. Reso: 64
\vskip 25pt

\textbf{Dyno piano}
\vskip 25pt

Touch res: Medium
M. Voice: 389 DXPhase
M. Volume: 100
M. Octave: 0
M. Pan: C
M. Reverb: 50
M. Chorus: 127
M. Attack: 64
M. Release: 64
M. Cutoff: 64
M. Reso: 64
D. Voice: 388 DXLegend
D. Volume: 90
D. Octave: 0
D. Pan: C
D. Reverb: 40
D. Chorus: 127
D. Attack: 64
D. Release: 64
D. Cutoff: 64
D. Reso: 104
\vskip 25pt

\textbf{Modificaciones simples a instrumentos}

\textbf{Saxof\'on}
\vskip 25pt

Touch res: Medium
M. Voice: 159 S!Soprn
M. Volume: 96
M. Octave: 0
M. Pan: C
M. Reverb: 90
M. Chorus: 0
M. Attack: 64
M. Release: 64
M. Cutoff: 64
M. Reso: 100

\textbf{Synthetizador electr\'onica}
\vskip 25pt

Touch res: Medium
M. Voice: 185 HandsUp!
M. Volume: 100
M. Octave: 0
M. Pan: C
M. Reverb: 100
M. Chorus: 127
M. Attack: 64
M. Release: 105
M. Cutoff: 64
M. Reso: 64

\textbf{Piano bolero sudam}
\vskip 25pt

Touch res: Medium
M. Voice: 241 Bright
M. Volume: 100
M. Octave: -1
M. Pan: C
M. Reverb: 100
M. Chorus: 64
M. Attack: 64
M. Release: 64
M. Cutoff: 64
M. Reso: 64

        \theendnotes
        %\begin{document}
\newcommand{\myTitleFont}{\Huge\myHugeSF}
\ifguitarra
\pdfbookmark[0]{Acordes~para~Guitarra}{acordesguitarra}
\lhead{\LHeadFont Acordes~para~Guitarra}
\chead{\CHeadFont ({\rm\thepage})}
\rhead{\RHeadFont\RelDate}
{\parindent 8pt
        {\myTitleFont --- Acodes para Guitarra ---}}\par
\vskip 20pt
\textbf{Acodes Mayores}

%\small{El s\'imbolo \# significa sostenido y {\flat}~significa~bemol}
\small
\upchord{\AGuitar}{La Mayor} \upchord{\BGuitar}{Si Mayor} \upchord{\CGuitar}{Do Mayor} \upchord{\DGuitar}{Re Mayor} \upchord{\EGuitar}{Mi Mayor} \upchord{\FGuitar}{Fa Mayor} \upchord{\GGuitar}{Sol Mayor}

\upchord{\AsGuitar}{A\#/$B\flat$ Mayor} \upchord{\CsGuitar}{C\#/$D\flat$ Mayor} \upchord{\DsGuitar}{D\#/$E\flat$ Mayor}  \upchord{\FsGuitar}{F\#/$G\flat$ Mayor} \upchord{\GsGuitar}{G\#/$A\flat$ Mayor} \upchord{\AsGuitar}{A\#/$B\flat$ Mayor}
\normalsize

\vskip 20pt
\textbf{Acodes Menores}

Estos acordes tienen las siguientes notaciones:
A-, Amin, Am, Aminor\break
\vskip 20pt

\small
\upchord{\AmGuitar}{La} Menor \upchord{\BmGuitar}{Si} Menor \upchord{\CmGuitar}{Do} Menor \upchord{\DmGuitar}{Re} Menor \upchord{\EmGuitar}{Mi} Menor \upchord{\FmGuitar}{Fa} Menor \upchord{\GmGuitar}{Sol} Menor

\upchord{\AsmGuitar}{\small{A\#/$B\flat$ Menor}} \upchord{\CsmGuitar}{\small{C\#/$D\flat$ Menor}} \upchord{\DsmGuitar}{\small{D\#/$E\flat$ Menor}}  \upchord{\FsmGuitar}{\small{F\#/$G\flat$ Menor}} \upchord{\GsmGuitar}{\small{G\#/$A\flat$ Menor}} \upchord{\AsmGuitar}{\small{A\#/$B\flat$ Menor}}
\normalsize

\vskip 20pt
\textbf{Acodes Mayores S\'eptima}

\upchord{\AsevenGuitar}{La} Mayor s\'eptima
\upchord{\BflatsevenGuitar}{Si} bemol Mayor s\'eptima
\upchord{\BsevenGuitar}{Si} Mayor s\'eptima
\upchord{\CsevenGuitar}{\small{Do Mayor s\'eptima}}
\vskip 20pt
\upchord{\CssevenGuitar}{\small{Do sostenidoMayor s\'eptima}}
\upchord{\DsevenGuitar}{\small{Re Mayor s\'eptima}}
\upchord{\EflatsevenGuitar}{\small{Mi bemol Mayor s\'eptima}}
\upchord{\EsevenGuitar}{\small{Mi Mayor s\'eptima}}
\upchord{\FsevenGuitar}{\small{Fa Mayor s\'eptima}}
\upchord{\FssevenGuitar}{\small{Fa sostenido Mayor s\'eptima}}
\vskip 20pt
\upchord{\GsevenGuitar}{\small{Sol Mayor s\'eptima}}
\upchord{\AflatsevenGuitar}{\qquad La bemol S\'eptima}

\textbf{Acodes Menores S\'eptima}

\small
\upchord{\AmsevenGuitar}{La} Menor s\'eptima
\upchord{\BmsevenGuitar}{Si} Menor s\'eptima
\upchord{\CmsevenGuitar}{Do} Menor s\'eptima
\upchord{\CsmsevenGuitar}{Do} Menor s\'eptima
\upchord{\DmsevenGuitar}{Re} Menor s\'eptima
\vskip 20pt
\upchord{\DsmsevenGuitar}{Re} Sostenido Menor s\'eptima
\upchord{\EmsevenGuitar}{Mi} Menor s\'eptima
\upchord{\EmseventrGuitar}{Mi} Menor s\'eptima
\upchord{\FmsevenGuitar}{Fa} Menor s\'eptima
\vskip 20pt
\upchord{\FsmsevenGuitar}{Fa sostenido} Menor s\'eptima
\upchord{\GmsevenGuitar}{Sol} Menor s\'eptima
\upchord{\GsmsevenGuitar}{Sol} Sostenido Menor s\'eptima
\upchord{\BflatmsevenGuitar}{Si bemol} Menor s\'eptima
\normalsize

\vskip 20pt
\textbf{Acodes Mayores Suspendido cuarta}
\vskip 25pt

\small
\upchord{\AsusGuitar}{La} Suspendida cuarta
\upchord{\BsusGuitar}{Si} Suspendida cuarta
\upchord{\CsusGuitar}{Do} Suspendida cuarta
\upchord{\DsusGuitar}{Re} Suspendida cuarta
\upchord{\EsusGuitar}{Mi} Suspendida cuarta
\upchord{\FsusGuitar}{Fa} Suspendida cuarta
\upchord{\GsusGuitar}{Sol} Suspendida cuarta

\upchord{\FssusGuitar}{Fa} sostenido Suspendida cuarta
\upchord{\GssusGuitar}{Sol sostenido} Suspendida cuarta
\normalsize

\vskip 20pt
\textbf{Acodes Mayor S\'eptima Aumentada}
\vskip 25pt

Estos acordes tienen las siguientes notaciones:
Amaj7, A+7, AM7, $A^{+7}$, $A^{M7}$, $A\Delta7$, $A^{\Delta7}$\break
\vskip 20pt

\small
\upchord{\AMajGuitar}{La} Maj
\upchord{\CMajGuitar}{Do} Maj
\upchord{\DMajGuitar}{Re} Maj
\upchord{\FMajGuitar}{Fa} Maj
\upchord{\GMajGuitar}{Sol} Maj
\upchord{\AflatMajGuitar}{Ab} Maj
\upchord{\AsMajGuitar}{A\#} Maj
\upchord{\EflatMajGuitar}{Eb} Maj
\normalsize

\vskip 20pt
\textbf{Acodes Suspendida 2}
\vskip 25pt

\small
\upchord{\AsustwoGuitar}{La} Suspendida 2
\upchord{\CsustwoGuitar}{Do} Suspendida 2
\normalsize

\vskip 20pt
\textbf{Acodes Suspendida 9}
\vskip 25pt

\small
\upchord{\DsusnineGuitar}{Re} Suspendida novena
\normalsize

\vskip 20pt
\textbf{Acodes Aumentada 2}
\vskip 25pt

\small
\upchord{\AtwoGuitar}{La} Aumentada 2
\upchord{\CtwoGuitar}{Do} Aumentada 2
\upchord{\DtwoGuitar}{Re} Aumentada 2
\upchord{\EflattwoGuitar}{Eb} Aumentada 2
\upchord{\FtwoGuitar}{Fa} Aumentada 2
\upchord{\GtwoGuitar}{Sol} Aumentada 2
\normalsize

\vskip 20pt
\textbf{Acodes Novena}
\vskip 25pt

\small
\upchord{\AnineGuitar}{La} Novena
\upchord{\CnineGuitar}{Do} Novena
\upchord{\FnineGuitar}{Fa} Novena
\upchord{\GnineGuitar}{Sol} Novena
\normalsize

\vskip 20pt
\textbf{Acodes trecena}
\vskip 25pt

\small
\upchord{\AsthirteenGuitar}{A\#} 13
\normalsize

\vskip 20pt
\textbf{Acodes Menor Onceava}
\vskip 25pt

\small
\upchord{\EmelevenGuitar}{Mi} Menor 11
\normalsize

\vskip 20pt
\textbf{Acodes Disminuidos}
\vskip 25pt

\small
\upchord{\GsdimGuitar}{Sol} sostenido disminuido
\normalsize


\vskip 20pt
\textbf{Acodes Con Bajo cambiado}

\small
\upchord{\AAsGuitar}{La Mayor bajo Bb} \hfill
\upchord{\ACsGuitar}{La Mayor bajo C\#} \hfill
\upchord{\ADGuitar}{La Mayor bajo D} \hfill
\upchord{\AEGuitar}{La Mayor bajo E} \hfill\null\break
\vskip 20pt
\upchord{\AGGuitar}{La Mayor bajo G} \hfill
\upchord{\AsevenCsGuitar}{La} S\'eptima bajo C\# \hfill
\upchord{\AmCGuitar}{La Menor bajo C} \hfill
\upchord{\AmCsGuitar}{La Menor bajo C\#} \hfill\null\break
\vskip 20pt
\upchord{\AmFGuitar}{La Menor bajo F} \hfill
\upchord{\AmFsGuitar}{La Menor bajo F\#} \hfill
\upchord{\AmGGuitar}{La Menor bajo G} \hfill
\upchord{\AmGsGuitar}{La Menor bajo G\#} \hfill\null\break
\vskip 20pt
\upchord{\AmsevenFGuitar}{La Menor S\'eptima bajo F} \hfill
\upchord{\AflatFGuitar}{La bemol Mayor bajo F} \hfill
\upchord{\BDsGuitar}{Si Mayor bajo D\#} \hfill\null\break
\vskip 20pt
\upchord{\BflatFGuitar}{Si bemol} Mayor bajo F \hfill
\upchord{\BsevenDsGuitar}{Si S\'eptima bajo D\#} \hfill
\upchord{\BsevenFsGuitar}{Si S\'eptima bajo F\#} \hfill
\upchord{\CEGuitar}{Do} Mayor bajo E \hfill\null\break
\vskip 20pt
\upchord{\CGGuitar}{Do} Mayor bajo G \hfill
\upchord{\CmAflatGuitar}{Do} Menor bajo Ab \hfill
\upchord{\DflatFGuitar}{Re bemol bajo F} \hfill
\upchord{\DAGuitar}{Re Mayor bajo A} \hfill\null\break
\vskip 20pt
\upchord{\DCGuitar}{Re Mayor bajo C} \hfill
\upchord{\DEGuitar}{Re Mayor bajo E} \hfill
\upchord{\DmCGuitar}{Re Menor bajo C} \hfill
\upchord{\DmFGuitar}{Re Menor bajo F} \hfill\null\break
\vskip 20pt
\upchord{\DmsevenGGuitar}{Re Menor S\'eptima bajo G} \hfill
\upchord{\DFsGuitar}{Re Mayor bajo F\#} \hfill
\upchord{\DsevenFsGuitar}{Re S\'eptima bajo F\#} \hfill
\upchord{\EflatFGuitar}{Mi bemol bajo F} \hfill\null\break
\vskip 20pt
\upchord{\EDGuitar}{Mi Mayor Bajo D} \hfill
\upchord{\EFsGuitar}{Mi Mayor Bajo F\#} \hfill
\upchord{\EGGuitar}{Mi Mayor Bajo G} \hfill
\upchord{\EGsGuitar}{Mi Mayor Bajo G\#} \hfill\null\break
\vskip 20pt
\upchord{\EmDGuitar}{Mi Menor Bajo D} \hfill
\upchord{\EmGGuitar}{Mi Menor Bajo G} \hfill
\upchord{\FAGuitar}{Fa Mayor Bajo A} \hfill
\upchord{\GflatBflatGuitar}{Sol bemol} Bajo Si bemol \hfill\null\break
\vskip 20pt
\upchord{\GBGuitar}{Sol Mayor Bajo B} \hfill
\upchord{\GDGuitar}{Sol Mayor Bajo D} \hfill
\upchord{\GEGuitar}{Sol Mayor Bajo E} \hfill
\upchord{\GFGuitar}{Sol Mayor Bajo F} \hfill\null\break
\vskip 20pt
\upchord{\GnineDGuitar}{Sol} Mayor Novena Bajo D \hfill
\upchord{\GmBflatGuitar}{Sol Menor Bajo Bb} \hfill
\upchord{\AflatCGuitar}{La bemol Bajo C} \hfill\null\break
\vskip 20pt
\upchord{\AflatEflatGuitar}{La bemol Bajo Eb} \hfill
\upchord{\BmajsevenGGuitar}{Si} + bajo G \hfill\null\break
\vskip 20pt
\upchord{\BflatCGuitar}{Si bemol} bajo C \hfill
\upchord{\BflatDGuitar}{Si bemol} bajo D \hfill
\upchord{\BflatGGuitar}{Si bemol} bajo G \hfill\null\break
\vskip 20pt
\upchord{\DsusFsGuitar}{Re} Suspendida cuarta bajo F\# \hfill
\upchord{\CMajtwoDGuitar}{\qquad Do} maj7 aumentada 2 bajo D \hfill
\upchord{\FtwoAGuitar}{Fa}+2 Mayor bajo A \hfill\null\break
\vskip 20pt
\upchord{\BdimDGuitar}{Si} disminuido bajo D \hfill\null\break
\normalsize

\vskip 20pt
\textbf{Acodes mayor disminuida quinta}

\small
\upchord{\EdimfiveGuitar}{Mi} dim5
\normalsize

\vskip 20pt
\textbf{Acodes semidisminuidos}

\small
\upchord{\BmsevenbfiveGuitar}{Si} Menor s\'eptima semidisminuido
\upchord{\EdimfiveGuitar}{Mi} Menor s\'eptima semidisminuido
\normalsize

\vskip 20pt
\textbf{Acodes diada arm\'onica}

\small
\upchord{\EfiveGuitar}{Mi} 5
\normalsize

\vskip 20pt
\textbf{Acodes 13 suspendida cuarta}

\small
\upchord{\CsusthirteenGuitar}{Do} 13 suspendida cuarta
\normalsize

\clearpage
\fi

\ifpiano
\pdfbookmark[0]{Acordes~para~Piano}{acordespiano}
\lhead{\LHeadFont Acordes~para~Piano}
{\parindent 8pt
        {\myTitleFont --- Acordes para Piano ---}}\par
\vskip 20pt
\textbf{Acodes Mayores}
\vskip 25pt

%\small{El s\'imbolo \# significa sostenido y {\flat}~significa~bemol}
\small
\upchord{\APiano}{\qquad La Mayor} \qquad\qquad \upchord{\BPiano}{Si Mayor} \qquad\qquad \upchord{\CPiano}{\qquad Do Mayor} \qquad\qquad \upchord{\DPiano}{\qquad Re Mayor} \hfill \break
\vskip 25pt
\upchord{\EPiano}{\qquad Mi Mayor} \qquad\qquad  \upchord{\FPiano}{\qquad Fa Mayor} \qquad\qquad \upchord{\GPiano}{\qquad Sol Mayor}
\vskip 25pt
\upchord{\AsPiano}{A\#/$B\flat$ Mayor} \qquad\qquad \upchord{\CsPiano}{C\#/$D\flat$ Mayor} \qquad\qquad \upchord{\DsPiano}{D\#/$E\flat$ Mayor} \qquad\qquad \upchord{\FsPiano}{F\#/$G\flat$ Mayor} \hfill \break
\vskip 25pt
\upchord{\GsPiano}{G\#/$A\flat$ Mayor} \qquad\qquad \upchord{\AsPiano}{A\#/$B\flat$ Mayor}
\normalsize

\textbf{Acodes Menores}
\vskip 25pt

\small
\upchord{\AmPiano}{\qquad La} Menor \qquad\qquad \upchord{\BmPiano}{\qquad Si} Menor \qquad\qquad \upchord{\CmPiano}{\qquad Do} Menor \qquad\qquad \upchord{\DmPiano}{\qquad Re} Menor \hfill \break
\vskip 25pt
\upchord{\EmPiano}{\qquad Mi} Menor \qquad\qquad \upchord{\FmPiano}{\qquad Fa} Menor \qquad\qquad \upchord{\GmPiano}{\qquad Sol} Menor
\vskip 25pt
\upchord{\AsmPiano}{\small{A\#/$B\flat$ Menor}}  \qquad\qquad  \upchord{\CsmPiano}{C\#/$D\flat$ Menor}  \qquad\qquad  \upchord{\DsmPiano}{D\#/$E\flat$ Menor} \hfill \break
\vskip 25pt
\upchord{\FsmPiano}{F\#/$G\flat$ Menor} \qquad\qquad \upchord{\GsmPiano}{G\#/$A\flat$ Menor}  \qquad\qquad  \upchord{\AsmPiano}{A\#/$B\flat$ Menor}
\normalsize

\clearpage
%\vskip 20pt
\textbf{Acodes Mayores S\'eptima}
\vskip 25pt

\small
\upchord{\AsevenPiano}{La Mayor s\'eptima} \hfill
\upchord{\BsevenPiano}{Si Mayor s\'eptima} \hfill
\upchord{\CsevenPiano}{Do Mayor s\'eptima} \hfill\null\break
\vskip 25pt
\upchord{\DsevenPiano}{Re Mayor s\'eptima} \hfill
\upchord{\EsevenPiano}{Mi Mayor s\'eptima} \hfill
\upchord{\FsevenPiano}{Fa Mayor s\'eptima} \hfill\null\break
\vskip 25pt
\upchord{\GsevenPiano}{Sol Mayor s\'eptima} \hfill
\upchord{\BflatsevenPiano}{Si} bemol Mayor s\'eptima \hfill\null\break
\vskip 25pt
\upchord{\EflatsevenPiano}{Mi bemol Mayor s\'eptima} \hfill
\upchord{\FssevenPiano}{Fa sostenido Mayor s\'eptima} \hfill
\upchord{\AflatsevenPiano}{Sol sostenido Mayor s\'eptima} \hfill\null\break
\normalsize
\vskip 20pt

\textbf{Acodes Menores S\'eptima}
\vskip 25pt

\small
\upchord{\AmsevenPiano}{La} Menor s\'eptima \hfill
\upchord{\BmsevenPiano}{Si} Menor s\'eptima \hfill
\upchord{\CmsevenPiano}{Do} Menor s\'eptima \hfill\null\break
\vskip 25pt
\upchord{\DmsevenPiano}{Re} Menor s\'eptima \hfill
\upchord{\EmsevenPiano}{Mi} Menor s\'eptima \hfill
\upchord{\FmsevenPiano}{Fa} Menor s\'eptima \hfill\null\break
\vskip 25pt
\upchord{\GmsevenPiano}{Sol} Menor s\'eptima \hfill
\upchord{\BflatmsevenPiano}{Si bemol} Menor s\'eptima \hfill
\upchord{\FsmsevenPiano}{Fa sostenido} Menor s\'eptima \hfill\null\break
\vskip 25pt
\upchord{\CssevenPiano}{Do sostenido} Mayor s\'eptima \hfill
\upchord{\DsmsevenPiano}{Re} Sostenido Menor s\'eptima \hfill
\upchord{\GsmsevenPiano}{Sol} Sostenido Menor s\'eptima \hfill\null\break
\normalsize

\vskip 20pt

\textbf{Acodes Mayores Suspendido cuarta}
\vskip 25pt

\small
\upchord{\AsusPiano}{La} Suspendida cuarta \hfill
\upchord{\BsusPiano}{Si} Suspendida cuarta \hfill
\upchord{\CsusPiano}{Do} Suspendida cuarta \hfill\null\break
\vskip 25pt
\upchord{\DsusPiano}{Re} Suspendida cuarta \hfill
\upchord{\EsusPiano}{Mi} Suspendida cuarta \hfill
\upchord{\FsusPiano}{Fa} Suspendida cuarta \hfill\null\break
\vskip 25pt
\upchord{\FssusPiano}{Fa sostenido} Suspendida cuarta \hfill
\upchord{\GsusPiano}{Sol} Suspendida cuarta \hfill
\upchord{\GssusPiano}{Sol sostenido} Suspendida cuarta \hfill\null\break
\normalsize

\vskip 20pt
\textbf{Acodes Mayor S\'eptima Aumentada}
\vskip 25pt

Estos acordes tienen las siguientes notaciones:
Amaj7, A+7, AM7, $A^{+7}$, $A^{M7}$, $A\Delta7$, $A^{\Delta7}$\break
\vskip 20pt

\small
\upchord{\AMajPiano}{La} Maj \hfill
\upchord{\CMajPiano}{Do} Maj \hfill
\upchord{\DMajPiano}{Re} Maj \hfill\null\break
\vskip 20pt
\upchord{\FMajPiano}{Fa} Maj \hfill
\upchord{\GMajPiano}{Sol} Maj \hfill
\upchord{\AsMajPiano}{La\#/Sib} Maj \hfill\null\break
\vskip 20pt
\upchord{\EflatMajPiano}{Re\#/Mib} Maj \hfill\null\break
\normalsize

\vskip 20pt
\textbf{Acodes Suspendida 2}
\vskip 25pt

\small
\upchord{\AsustwoPiano}{La} Suspendida 2
\upchord{\CsustwoPiano}{Do} Suspendida 2
\normalsize

\vskip 20pt
\textbf{Acodes Suspendida 9}
\vskip 25pt

\small
\upchord{\DsusninePiano}{Re} Suspendida novena
\normalsize

\vskip 20pt
\textbf{Acodes Aumentada 2}
\vskip 25pt

\small
\upchord{\AtwoPiano}{La} Aumentada 2 \hfill
\upchord{\CtwoPiano}{Do} Aumentada 2 \hfill
\upchord{\DtwoPiano}{Re} Aumentada 2 \hfill\null\break
\vskip 20pt
\upchord{\EflattwoPiano}{Re b} Aumentada 2 \hfill
\upchord{\FtwoPiano}{Fa} Aumentada 2 \hfill
\upchord{\GtwoPiano}{Sol} Aumentada 2 \hfill\null\break
\normalsize


\vskip 20pt
\textbf{Acodes Novena}
\vskip 25pt

\small
\upchord{\AninePiano}{La} Novena \hfill
\upchord{\CninePiano}{Do} Novena \hfill
\upchord{\FninePiano}{Fa} Novena \hfill\null\break
\vskip 20pt
\upchord{\GninePiano}{Sol} Novena \hfill
\normalsize

\vskip 20pt
\textbf{Acodes trecena}
\vskip 25pt

\small
\upchord{\AsthirteenPiano}{A\#} 13
\normalsize

\vskip 20pt
\textbf{Acodes Menor Onceava}
\vskip 25pt

\small
\upchord{\EmelevenPiano}{Mi} Menor 11
\normalsize
\clearpage

%\vskip 20pt
\textbf{Acodes Disminuidos}
\vskip 25pt

\small
\upchord{\GsdimPiano}{Sol} sostenido disminuido
\normalsize


\vskip 20pt
\textbf{Acodes Con Bajo cambiado}
\vskip 25pt

En el caso del piano, es com\'un que la parte del bajo se toque octavado en la mano izquierda y en la derecha el acorde normal, aqu\'i se pone la transposici\'on para ayudar a encontrar la melod\'ia.
\vskip 25pt
\small
\upchord{\AAsPiano}{La Mayor bajo Bb} \hfill
\upchord{\ACsPiano}{La Mayor bajo C\#} \hfill\null\break
\vskip 20pt
\upchord{\ADPiano}{La Mayor bajo D} \hfill
\upchord{\AEPiano}{La} Mayor bajo E \hfill\null\break
\vskip 20pt
\upchord{\AGPiano}{La Mayor bajo G} \hfill
\upchord{\AsevenCsPiano}{La} Mayor S\'eptima bajo C\# \hfill\null\break
\vskip 20pt
\upchord{\AmCPiano}{La Menor bajo C} \hfill
\upchord{\AmCsPiano}{La Menor bajo C\#} \hfill\null\break
\vskip 20pt
\upchord{\AmFPiano}{La Menor bajo F} \hfill
\upchord{\AmFsPiano}{La Menor bajo F\#} \hfill
\upchord{\AmGPiano}{La Menor bajo G} \hfill\null\break
\vskip 20pt
\upchord{\AmGsPiano}{La Menor bajo G\#} \hfill
\upchord{\AmsevenFPiano}{La Menor S\'eptima bajo F} \hfill
\upchord{\AflatFPiano}{La bemol Mayor bajo F} \hfill\null\break
\vskip 20pt
\upchord{\BDsPiano}{Si Mayor bajo D\#} \hfill
\upchord{\BflatFPiano}{Si bemol} Mayor bajo F \hfill
\upchord{\BsevenDsPiano}{Si S\'eptima bajo D$\#$} \hfill\null\break
\vskip 20pt
\upchord{\BsevenFsPiano}{Si S\'eptima bajo F$\#$} \hfill\null\break
\vskip 20pt
\upchord{\CEPiano}{Do Mayor bajo E} \hfill
\upchord{\CmAflatPiano}{Do Menor bajo Ab} \hfill\null\break
\vskip 20pt
\upchord{\CGPiano}{Do Mayor bajo G} \hfill
\upchord{\DAPiano}{Re Mayor bajo A} \hfill
\upchord{\DCPiano}{Re Mayor bajo C} \hfill\null\break
\vskip 20pt
\upchord{\DmCPiano}{Re Menor bajo C} \hfill
\upchord{\DmFPiano}{Re Menor bajo F} \hfill
\upchord{\DmsevenGPiano}{Re Menor S\'eptima bajo G} \hfill\null\break
\vskip 20pt
\upchord{\DFsPiano}{Re Mayor bajo F\#} \hfill
\upchord{\DsevenFsPiano}{Re S\'eptima bajo F\#} \hfill
\upchord{\EflatFPiano}{Mi bemol bajo F} \hfill\null\break
\vskip 20pt
\upchord{\EDPiano}{\qquad Mi} Mayor con bajo D \hfill
\upchord{\EFsPiano}{\qquad Mi} Mayor con bajo F\# \hfill
\upchord{\EGPiano}{\qquad Mi} Mayor con bajo G \hfill\null\break
\vskip 20pt
\upchord{\EGsPiano}{\qquad Mi} Mayor con bajo G\# \hfill
\upchord{\EmDPiano}{\qquad Mi} Menor Bajo D \hfill
\upchord{\EmGPiano}{\qquad Mi} Menor Bajo G \hfill\null\break
\vskip 20pt
\upchord{\FAPiano}{Fa} Mayor bajo A \hfill
\upchord{\GflatBflatPiano}{Sol bemol Mayor bajo Si Bemol} \hfill
\upchord{\GDPiano}{Sol Mayor Bajo D} \hfill\null\break
\vskip 20pt
\upchord{\GEPiano}{Sol Mayor Bajo E} \hfill %long
\upchord{\GFPiano}{Sol Mayor Bajo F} \hfill\null\break
\vskip 20pt
\upchord{\GBPiano}{\qquad\qquad Sol}  Mayor Bajo B \hfill
\upchord{\GnineDPiano}{Sol} Mayor Novena Bajo D \hfill\null\break %long
\vskip 20pt
\upchord{\GmBflatPiano}{Sol Menor Bajo Bb} \hfill
\upchord{\AflatCPiano}{La bemol Bajo C} \hfill
\upchord{\AflatEflatPiano}{La bemol Bajo Eb} \hfill\null\break
\vskip 20pt
\upchord{\BflatCPiano}{Si bemol} bajo C \hfill
\upchord{\BflatDPiano}{Si bemol} bajo D \hfill\null\break
\vskip 20pt
\upchord{\BflatGPiano}{Si bemol} bajo G \hfill
\upchord{\BmajsevenGPiano}{Si} + bajo G \hfill
\upchord{\DsusFsPiano}{Re Suspendida} cuarta bajo F\# \hfill\null\break
\vskip 20pt
\upchord{\CMajtwoDPiano}{\qquad Do}+7+2 bajo D \hfill
\upchord{\FtwoAPiano}{Fa}+2 Mayor bajo A \hfill\null\break
\vskip 20pt
\upchord{\BdimDPiano}{Si} disminuido bajo D \hfill\null\break
\vskip 20pt
\normalsize

\clearpage
%\vskip 20pt
\textbf{Acodes mayor disminuida quinta}
\vskip 25pt

\small
\upchord{\EdimfivePiano}{Mi} dim5
\normalsize

\vskip 20pt
\textbf{Acodes medio disminuido s\'eptima}
\vskip 25pt

\small
\upchord{\BmsevenbfivePiano}{Si} medio disminuido s\'eptima
\normalsize

\vskip 20pt
\textbf{Acodes diada arm\'onica}
\vskip 25pt

\small
\upchord{\EfivePiano}{Mi} 5
\normalsize

\vskip 20pt
\textbf{Acodes 13 suspendida cuarta}
\vskip 25pt

\small
\upchord{\CsusthirteenPiano}{Do} 13 suspendida cuarta
\normalsize

\clearpage
\fi
%\end{document}
%\bye
        \include{infantilesAdx}
        %%%%%% rcsid = @(#)$Id: sampleKdx.tex,v 1.16 2010-04-12 18:04:30 rathc Exp $
%%%%%%
%%
%%      ================================
%%      Sample Key Index (sampleKdx.tex)
%%      ================================
%%
%%      Version 4.5, 30 April, 2010
%%
%%      Copyright 1992--2010 Christopher Rath <christopher@rath.ca>
%%
%%	This package is free software; you can redistribute it and/or
%%	modify it under the terms of version 2.1 of the GNU Lesser 
%%	General Public License as published by the Free Software
%%	Foundation.
%%
%%	This package is distributed in the hope that it will be
%%	useful, but WITHOUT ANY WARRANTY; without even the implied
%%	warranty of MERCHANTABILITY or FITNESS FOR A PARTICULAR
%%	PURPOSE.  See the GNU Lesser General Public License for more
%%	details.
%%
%%      This file is provided as a template for Song Key
%%      Index generation.
%%
%%%%%%
%%%%%%

%%%%%%%%%%%%%%%%%%%%%%%%%%%%%%%%%%%%%%%%%%%%%%%%%%%%%%%%%%
%%%%%%%%%%%%%%%%%%%%%%%%%%%%%%%%%%%%%%%%%%%%%%%%%%%%%%%%%%
%%                                                      %%
%%           P R E A M B L E   B E G I N S              %%
%%                                                      %%
%%%%%%%%%%%%%%%%%%%%%%%%%%%%%%%%%%%%%%%%%%%%%%%%%%%%%%%%%%
%%%%%%%%%%%%%%%%%%%%%%%%%%%%%%%%%%%%%%%%%%%%%%%%%%%%%%%%%%

%\documentclass[12pt,twocolumn]{book}
%\usepackage{latexsym,fancyhdr}
%\usepackage[wordbk]{songbook}

%\usepackage{tikz}

%%%
% Revision Date and Release Date definitions.
%
%       \RelDate - The last time this songbook was released.
%       \RevDate - The last time this file was revised in any way.
%%%
%\newcommand{\RelDate}{30~May'96}
%\newcommand{\RevDate}{\RelDate}

%%%
% Redefine fonts from SongBook style that I don't like, and define
% any extra fonts I require.
%%%
\font\myTinySF=cmss8    at  8pt
\font\myHugeSF=cmssbx10 at 25pt
\renewcommand{\CpyRtInfoFont}{\tiny\myTinySF}
%\newcommand{\myTitleFont}{\Huge\myHugeSF}
%\newcommand{\mySubTitleFont}{\large\sf}

%%%
% Define fonts to use in the headers and footers of the songbook.
%%%
%\newcommand{\LHeadFont}{\normalsize}            % = cmr12  at 12pt
%\newcommand{\CHeadFont}{\normalsize\rm}         % = cmr12  at 12pt
%\newcommand{\RHeadFont}{\normalsize}            % = cmr12  at 12pt
%\newcommand{\LFootFont}{\scriptsize}            % = cmr8   at  8pt
%\newcommand{\CFootFont}{\tiny\myTinySF}         % = cmss8  at  8pt
%\newcommand{\RFootFont}{\scriptsize}            % = cmr8   at  8pt

%%%
% Turn on and define fancy page heading/footing definition.
%%%
\pagestyle{fancy}
\pdfbookmark[0]{\'Indice~Tonal}{tonal}
%\addtolength{\headwidth}{\marginparsep}
%\addtolength{\headwidth}{\marginparwidth}
%\renewcommand{\footrulewidth}{0.4pt}
\lhead{\LHeadFont \'Indice~Tonal}
       \chead{\CHeadFont ({\rm\thepage})}
       \rhead{\RHeadFont\RelDate}

%\lfoot{\LFootFont Property of a Church}
%       \cfoot{\CFootFont Last Revised:  \RevDate}
%       \rfoot{\RFootFont Material used by permission.}


%%%
% Index entries command definition.
%%%
\renewcommand{\item}{\par\hangindent=40pt}
\renewcommand{\subitem}{\par\hangindent=40pt \hspace*{20pt}}
\renewcommand{\subsubitem}{\par\hangindent=40pt \hspace*{30pt}}


%%%%%%%%%%%%%%%%%%%%%%%%%%%%%%%%%%%%%%%%%%%%%%%%%%%%%%%%%%
%%%%%%%%%%%%%%%%%%%%%%%%%%%%%%%%%%%%%%%%%%%%%%%%%%%%%%%%%%
%%                                                      %%
%%           D O C U M E N T   B E G I N S              %%
%%                                                      %%
%%%%%%%%%%%%%%%%%%%%%%%%%%%%%%%%%%%%%%%%%%%%%%%%%%%%%%%%%%
%%%%%%%%%%%%%%%%%%%%%%%%%%%%%%%%%%%%%%%%%%%%%%%%%%%%%%%%%%
%\begin{document}

%%%
% Index begins.
%%%
{\parindent 8pt
  {\myTitleFont --- INDICE TONAL ---}}\par
\vskip 20pt

Con base en el c\'irculo de quintas, el modo mayor y el modo menor usar\'an las mismas notas si est\'an alineadas en la l\'{\i}nea dorada que va del centro hacia afuera.\hfill\null\break
Para facilitar al cantante la interpretaci\'on, se pueden elegir canciones en la misma escala o en su defecto hacia la derecha o izquierda ligeramente y as\'{\i} evitar que se desafine.

\begin{figure}[h]
    \centering
    \includegraphics[width=0.5\textwidth]{circulo_de_quintas}
    \caption{C\'irculo de quintas.\label{fig:circulo_de_quintas}}
\end{figure}

Para saber que nota lleva la alteraci\'on, si son bemoles, la secuencia va siguiendo la l\'{\i}nea azul, si son sostenidos, la secuencia va siguiendo la l\'{\i}nea roja.\hfill\null\break
%Para realizar la modulación, podemos usar un acorde pivote, que es cuando se comparten acordes entre las escalas.
%Un círculo armónico, consiste en una progresión de acordes formada por los grados I, VI, II, V, con la particularidad de que el VI y el II son menores, y el V es séptima.
%Para ello usaremos la secuencia ii-V-I, es decir los últimos dos acordes del círculo armónico, y pasaremos al primer acorde del círculo.

\input{infantiles.kdx}
%
%\end{document}
%\bye
%
%%%
% Document ends.
%%%

% Local Variables:
%   LaTeX-item-indent:     -1
%   LaTeX-indent-level:     2
%   TeX-brace-indent-level: 2
%   TeX-auto-untabify:      nil
%   TeX-style-local:        style/
% End:

    \fi

    \include{infantilesTdx}
\end{document}
%
%\end{document}
%\bye
%
%%%
% Document ends.
%%%

% Local Variables:
%   LaTeX-item-indent:     -1
%   LaTeX-indent-level:     2
%   TeX-brace-indent-level: 2
%   TeX-auto-untabify:      nil
%   TeX-style-local:        style/
% End:

    \fi

    \include{infantilesTdx}
\end{document}